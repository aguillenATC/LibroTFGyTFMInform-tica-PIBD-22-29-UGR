\section{TFG de revisión del estado del arte}
\label{appendix:revisionestado}

Existe una modalidad de TFG que es en sí una revisión del estado del arte. Para que nos entendamos, es como la sección de revisión del estado del arte vista en el capítulo \ref{cap:RevisionEstadoDelArte}, pero a lo grande, en la que la revisión ocupa toda la memoria del TFG.

Si vas a realizar este tipo de TFG, en primer lugar te recomendamos que leas ese capítulo con objeto de obtener una idea general de qué es una revisión del estado del arte y, por supuesto, el capítulo \ref{cap:bibliografia} con el objetivo de tener claro cómo gestionar la bibliografía del TFG, pues, si ya es importante este asunto, en este tipo de proyecto fin de carrera, las referencias bibliográficas se configuran como algo vital.

El objetivo de esta sección será explicar brevemente cómo llevar a cabo la elaboración de la memoria de un trabajo de revisión del estado del arte. En este caso, hay mucho material publicado que puede serte de utilidad, por lo que haremos una revisión poco somera del mismo con objeto de ofrecerte algún material de inicio para que, al menos, comiences la tarea. De cualquier forma, ponte de acuerdo con la persona que te tutoriza sobre la forma de enfocar el trabajo, pues es un tipo muy especial de TFG que necesita ser definido y desarrollado de forma muy precisa.

\subsection{Tipos de revisiones}

Lluis Codina en \cite{codina2024lluis} establece dos grandes grupos de tipos de revisiones: las \textit{tradicionales} o \textit{narrativas} y las de tipo sistemático. Las primeras son más bien ensayos y carecen de validez científica; las segundas, sí que tienen esta validez, y se clasifican en \textit{sistemáticas}, que se emplean para determinar el impacto de intervenciones, y \textit{de alcance}, usadas para describir hasta dónde (alcance) llega un conocimiento dado. En este trabajo, el autor explica claramente cuándo se debería elegir un tipo u otro de revisión.

Pero esta es una simplificación que el profesor Codina ha realizado porque en realidad existe una gran cantidad de tipos diferentes de revisiones tal y como Grant y Booth indican en \cite{grant2009maria}, la mayoría ampliamente usados en el campo de las ciencias de la salud y con un componente estadístico muy fuerte:

\begin{itemize}
    \item Revisiones tradicionales. 
    \item Síntesis de conocimiento: de forma genérica se podrían definir como aquellas revisiones que contextualizan e integran los hallazgos de investigación de estudios individuales dentro del cuerpo más amplio de conocimiento sobre el tema que tengas entre manos. Algunos de los más conocidos y usados pueden ser los siguientes:
    \begin{itemize}
        \item Revisiones sistemáticas: identifican, evalúan y sintetizan todas las pruebas empíricas que cumplen unos criterios de elegibilidad previamente especificados. Las revisiones sistemáticas deben ser lo más exhaustivas e imparciales posibles.
        
        \item Metanálisis: subconjunto de revisiones sistemáticas que combina estadísticamente los resultados de estudios cuantitativos encontrados, con objeto de ofrecer un efecto más preciso de los resultados.
        
        \item Revisiones de alcance: abordan una pregunta de investigación exploratoria destinada a extraer conceptos clave, tipos de evidencia y nichos en la investigación relacionada con un área o campo definido, mediante la búsqueda sistemática, selección y síntesis del conocimiento existente.
        
        \item Revisiones rápidas: un tipo de síntesis de conocimiento en el cual los procesos de revisión sistemática se aceleran y los métodos se simplifican para completar la revisión más rápidamente que en el caso de las revisiones sistemáticas típicas, que vienen a tener un tiempo de realización de un año, reduciendo el tiempo de confección de cinco a doce semanas. Se emplean cuando no se dispone de mucho tiempo para realizarlas.
        
        \item Revisiones realistas: comprenden y desentrañan los mecanismos por los que una intervención funciona (o no funciona), proporcionando así una explicación, en lugar de un juicio sobre cómo funciona.

        \item Revisiones cualitativas: aquellas que integran o comparan los hallazgos de estudios cualitativos.

        \item Revisiones mixtas: combinación de los hallazgos de estudios cualitativos y cuantitativos dentro de una sola revisión sistemática para abordar las mismas preguntas de revisión superpuestas o complementarias.
        
        \item Síntesis narrativas: se basan en el uso de palabras y texto para resumir y explicar los hallazgos de la síntesis más que en resultados estadísticos. Básicamente usan la palabra para ``contar la historia'' de los hallazgos en los estudios incluidos.

        \item Revisiones tipo paraguas: se refiere a una revisión que recopila evidencia de múltiples revisiones en un documento accesible y utilizable. 
        
    \end{itemize}
\end{itemize}

En la web de la {biblioteca de la Universidad de Melbourne}\footnote{\url{https://unimelb.libguides.com/whichreview}} dispones de una definición muy detallada de los diferentes tipos de revisiones de literatura así como bibliografía de cada una de ellas para que las consultes. Si estás pensando realizar tu TFG en este contexto, te recomendamos que conozcas previamente los diferentes tipos de revisión existentes y que, junto a quien te dirige el trabajo, decidáis cuál es la que mejor se ajusta a los objetivos del mismo.

\subsection{Revisiones sistemáticas}

Una de las más ampliamente usadas es la revisión sistemática, sobre todo en el ámbito de la salud, aunque su uso se ha generalizado a todos los campos del conocimiento. Tal y como indica Codina en \cite{codina2018lluis} habría que diferenciar entre las revisiones sistemáticas y las sistematizadas, ya que estas primeras están centradas en conocer la eficacia de una intervención basándose en el análisis de estudios científicos que se han realizado sobre ella, como hemos dicho en el campo de la salud, y las segundas están enfocadas en explorar campos de conocimiento e investigación específicos, identificando  tendencias y corrientes dominantes, y detectando vacíos y posibles oportunidades para futuras investigaciones. Esta segunda definición sí que podría ser aplicada a cualquier campo de conocimiento y, por tanto, si somos estrictos con el lenguaje, en el campo de la informática deberíamos realizar una revisión sistematizada. Pero... por abuso del lenguaje se habla de forma general, independientemente del campo, de revisión sistemática. 

De cualquier forma una revisión (sistemática o sistematizada) de este tipo está compuesta de varias fases muy bien pautadas, que pasamos a describir seguidamente, cuyas descripciones detalladas podrás encontrar en  \cite{booth2021a}, un clásico en el campo de las revisiones de literatura:

\begin{enumerate}
\item Formulación de la pregunta(s) de investigación: define claramente el objetivo de la revisión y las preguntas de investigación que se abordarán. Es importante que estas preguntas sean específicas, claras y relevantes para el tema de estudio.

\item Búsqueda de literatura: se realiza una búsqueda exhaustiva y sistemática de la literatura relevante utilizando bases de datos académicas, bibliotecas digitales y otros recursos. En nuestro campo de la informática también puede interesarte realizar búsquedas en repositorios de código abierto, por ejemplo.

\item Selección de estudios (en inglés, \textit{screening}): se aplican criterios de inclusión y exclusión para seleccionar los estudios que cumplen con los criterios de la revisión. Esta selección se suele realizar en varias etapas, comenzando con la revisión de títulos y resúmenes para realizar un primer filtrado, seguida de la revisión de los textos completos de los estudios potencialmente relevantes que han pasado esta primera criba. Por ejemplo, se pueden descartar los estudios que lleguen a conclusiones con pocos datos o que usen métodos o tecnologías no actuales o no apropiados. 

\item Extracción de datos: se recopilan los datos relevantes de cada estudio seleccionado, como características del estudio, métodos utilizados y resultados obtenidos. En nuestro caso, información específica sobre tecnologías, detalles técnicos, prestaciones de aplicaciones, algoritmos empleados, lenguajes, etc.

\item Evaluación de la calidad de los estudios: se realiza una evaluación crítica de la calidad metodológica de los estudios incluidos en la revisión. 

\item Análisis y síntesis de los datos: se analizan los datos extraídos de los estudios y se realiza una síntesis para identificar patrones, tendencias, inconsistencias o nichos.

\item Interpretación de los resultados: se interpretan los hallazgos de la revisión en el contexto de la pregunta de investigación y se discuten sus implicaciones.

\item Escritura de la revisión: se redacta un informe detallado que describe el proceso de revisión, los métodos utilizados, los resultados obtenidos y las conclusiones alcanzadas. La memoria de tu TFG podría seguir una estructura estándar en este tipo de trabajos, que podría ser algo así:

\begin{enumerate}
\item Introducción: contextualización del tema y justificación de su importancia. Establecimiento de los objetivos.
\item Metodología: descripción de los métodos usados (estrategia de búsqueda, bases de datos, criterios de inclusión y exclusión, procedimientos de selección de estudios, evaluación de la calidad de los mismos, técnicas de análisis de datos, si corresponde).
\item Resultados: exposición de los resultados conseguidos. Tablas y gráficas ayudarán a visibilizarlos.
\item Discusión: interpretación de los resultados teniendo en cuenta los objetivos de la revisión, implicaciones prácticas o teóricas, y también es importante que se establezcan las limitaciones del proceso de revisión. Finalmente, la exposición de los resultados finales y recomendaciones. 
\item Conclusiones: resumen de los principales hallazgos, conclusiones finales y recomendaciones en base a los resultados obtenidos. También es habitual meter aquí una serie de líneas de trabajo futuras. 
\end{enumerate}
\end{enumerate}

Ten en cuenta que este proceso no es estático en el sentido de que en cualquier momento vas a poder obtener nuevos trabajos y tendrás que decidir si son relevantes para tu estudio e incorporarlos al mismo en caso afirmativo, con los cambios que conllevará en el análisis que has realizado hasta el momento. Con la persona que te tutoriza tendréis que decidir cuándo parar de incorporar más estudios para revisar.

Ni que decir tiene que todas las referencias deben estar correctamente citadas y dispuestas en una sección final de bibliografía.

El aporte realmente relevante de tu revisión vendrá de la mano del contenido de la sección de discusión, pues es ahí donde vas a mostrar tu capacidad de análisis y descubrimiento de hallazgos a partir de los trabajos analizados. La calidad de la interpretación de estos y las conclusiones harán o no valioso tu trabajo de revisión. Es por esto que te recomendamos que te esfuerces especialmente en esta parte de tu TFG. 

Para finalizar, indicarte que en las referencias \cite{carrera2022angela,kofod2022anders,silva2016rodrigo} tienes algunos ejemplos en los que los autores realizan una adaptación de las revisiones sistemáticas al campo de la informática. Échales un vistazo porque pueden serte de mucha utilidad.

