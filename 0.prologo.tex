\chapter*{Prólogo}

Un libro, un verdadero libro, no es algo que nos habla, es algo que nos escucha y sabe responder a nuestras preguntas. No es fácil para un estudiante de Ingeniería Informática abordar el desarrollo, la escritura y la defensa de su Trabajo Fin de Grado (TFG) y no lo es por muchas razones: porque no siempre existen pautas claras sobre cómo hacerlo, por una falta de costumbre a la hora de redactar memorias y sobre todo porque los estudiantes de Ingeniería son muchas veces más de hacer y no tanto de explicar lo que hacen y cómo lo han hecho.

Un TFG es un trabajo original realizado  individualmente en el que el estudiante desarrolla un proyecto en el ámbito de las tecnologías específicas de la Ingeniería en Informática y que suele ser de naturaleza profesional en el que, y ahí está la clave, se sintetizan e integran las competencias adquiridas a lo largo de toda su titulación de Grado. Los estudiantes de Informática tienen un histórico respeto al principio de entropía o economía del lenguaje, que se basa en expresar la mayor cantidad de información utilizando el menor número de palabras, pero el problema es que no siempre saben sintetizar en una memoria de forma simple y comprensible para todos la complejidad de aquello que han desarrollado durante arduos meses de trabajo y es que escribir con sencillez es tan difícil como escribir bien. 

Un empresario que conozco y que tenía una empresa en Silicon Valley, me comentaba que le encantaban los ingenieros españoles, aparte de por su excelente formación,  especialmente por su creatividad de la que carecían muchos de los ingenieros de allí, pero que les veía un problema, algo que les sobraba a los suyos y que los nuestros no tenían, y era capacidad para explicar y vender su producto, aquello que hacían. Ellos se educaban desde que entraban en la Universidad, en la mentalidad de que cualquier cosa que crearan o inventaran  había que saber venderla. Esa educación aún nos falta a nosotros y debemos potenciarla. Cada aplicación que se desarrolle hay que ponerla en valor, darla a conocer para que preste un servicio a la sociedad y sea útil para todos. Una memoria y una defensa de un TFG son una antesala de esa idea. Un estudiante debe saber redactarla y defenderla para que quede plasmado su trabajo en todas sus dimensiones: planificación, desarrollo, elaboración de la memoria y exposición pública. Quizá sea lo más complicado para el estudiante cara a que su trabajo sea apreciado a través de una defensa amena y sintética, lo que no está en contradicción con que a la vez sea científica y técnicamente impecable. 

Internet está lleno de tutoriales sobre cómo debe desarrollarse, redactarse y defenderse un TFG. Son un abanico de consejos genéricos muchas veces deslavazados y centrados normalmente en la algorítmica de la redacción de los diferentes apartados de la memoria y personalmente cuando leo esas guías, siempre he echado de menos consejos basados en la experiencia con un lenguaje ameno y centrado en hacer del proceso de escritura algo de lo que se pueda disfrutar. Y de eso trata este libro, de aprender a desarrollar, escribir y defender una memoria de TFG disfrutando especialmente de ese proceso de escritura y haciendo de cada paso, no un muro que paralice al estudiante, sino una puerta al siguiente paso teniendo muy claro cuál es el objetivo de cada etapa y cuando la ha de dar el estudiante por cumplida.

Estamos, además, en pleno auge de la IA generativa. Muchos estudiantes hacen uso de ella para casi todo, incluyendo evidentemente la posibilidad de que la IA les (casi) redacte una memoria. El libro no elude esta irrupción y no elude la posibilidad de usarla, pero defiende que hay algo para lo que no les es útil: ordenar sus propias ideas (que  a veces se mezclan y agolpan), sobre algo que solo ellos han desarrollado sin hacer uso más que de sus habilidades, capacidades y su esfuerzo en ese desarrollo.  Solo ellos conocen los detalles más profundos y, por tanto, que solo ellos y ninguna IA pueden contar al mundo qué han hecho y cómo lo han hecho.  El truco, en mi opinión, está en escribir la memoria para uno mismo, sabiendo ordenar las ideas y plasmar lo relevante que se ha hecho, no solo para satisfacer a un tribunal, que aunque sea importante, no debe ser lo más relevante.

El poeta romano Lucano decía que el hombre que se esfuerza y se atreve a subir a los hombros de gigantes ve más claro y más lejos que los propios gigantes. Esta máxima de Lucano es una hermosa metáfora que explica el interés y utilidad que tienen libros como este, redactado por profesoras y profesores de ámbitos muy diferentes, pero a los que une su preocupación por la innovación, por ofrecer a nuestros estudiantes una docencia de calidad, y en este caso para que aprovechen su experiencia acumulada durante muchos años de dirigir y ayudar a redactar y defender memorias de TFG. Subidos a esos hombros, aprovechando sus enseñanzas, los estudiantes verán más claro el horizonte, que a veces se les difumina, de la escritura y defensa de su proyecto. 

Ningún libro podrá nunca dar una receta definitiva a un estudiante sobre cómo debe desarrollar un TFG. La escritura, la forma de redactar, la consideración sobre qué es importante y qué no, es muy personal y muy complicada de moldear, pero sí pueden darse pautas basadas en la experiencia para que el estudiante aprenda a distinguir lo que es relevante escribir y lo que no, lo que es relevante destacar y lo que no, lo que es relevante dejar y lo que se debe eliminar. Nadie conoce mejor que el propio estudiante su proyecto, pero una cosa es hacerlo y otra muy diferente es saber transmitir lo que se ha hecho, y eso a veces tiene más de arte que de ciencia.  Este libro aporta esa pincelada de arte. Es un libro donde se condensa la experiencia de muchos años dirigiendo TFG y ayudando a los estudiantes a mejorar sus habilidades a la hora de redactar  y defender una memoria.

Este libro tiene una estructura y redacción impecables, es ameno, instructivo, irreverente a veces, pero lleno de consejos de calado. Completar el libro con vídeos dónde destacar de forma resumida las ideas allí plasmadas y el hecho de liberarlo para convertirlo en colaborativo, hacen de esta obra algo diferente, innovador y sobre todo útil para la comunidad de estudiantes de Ingeniería Informática que están en fase de desarrollo de su proyecto.  

El texto no elude ningún aspecto relacionado con la elaboración de un TFG, desde el proceso para la elección del mismo incluyendo un análisis de las diferentes tipologías de proyectos, su planificación temporal, la búsqueda de información, el estilo de redacción y maquetación, los diferentes apartados que debe tener la memoria, las cuestiones éticas y legales tan importantes para nuestros ingenieros, planes de gestión de riesgos y presupuesto, bibliografía a la que no se le suele prestar atención, pero que para el estudiante es extremadamente relevante, porque debe saber por qué citar, qué citar y como hacerlo, sin dejar finalmente de lado la presentación del proyecto ante el tribunal dando unas pautas muy claras sobre como planificar y hacer esa exposición pública.

El libro tiene además una característica interesante: hace del lector, cómplice necesario de quien lo escribe. Parafraseando a Hemingway, un libro puede y debe mostrar solo la punta y dejar al lector, imaginar cómo será el resto del iceberg. No es, por tanto, un compendio de recetas, como lo son los numerosos manuales que uno pueda encontrar dispersos por la inmensidad de Internet o en las fauces de una IA generativa. Es algo más. Consigue atraparte en su lectura, adaptada al estudiante de Ingeniería Informática que no siempre está acostumbrado a que le reduzcan ideas complejas a esquemas simples y sobre todo consigue cuando te sumerges en él, que te queden muy claros los conceptos que expresan y que establecen unas pautas nítidas para el desarrollo, redacción y defensa de una memoria de TFG.

Termino mis breves palabras con un agradecimiento a los autores por su invitación para redactar este prólogo. Coincido con Jorge Luis Borges en que publicamos nuestros libros para librarnos de ellos, para no pasar el resto de nuestras vidas corrigiendo borradores. Sé que este libro ha pasado por muchos borradores y que lo que estamos leyendo ahora es fruto de muchas horas de trabajo y de discusión sobre los contenidos. Alberto, Eugenio, Juanma, Manolo, M. José, Pablo, Rocío, debéis sentiros orgullosos de esta versión final. Espero y confío en que distintas generaciones de estudiantes aprovechen el libro, disfruten de su lectura y aprendan lo que significa escribir y defender una memoria de TFG. Que aprendan en esencia, que no la escriben porque quieran decir algo, sino porque tienen algo que decir y aprendan a expresarlo. En esa tarea tan diversa y compleja encontrarán en este libro un poco de luz.



\begin{flushright}
Dr. Joaquín Fdez-Valdivia \\                            
Granada, 5 de febrero de 2025
\end{flushright}