\chapter{La introducción del TFG}
\label{cap:IntroducciónTFG}

\section{Introducción}

En este capítulo nos centramos en describir las pautas para redactar el capítulo de introducción, que normalmente será el primero de la memoria y que también será el contenga cierta entidad dentro de la misma, pues es vital para que el lector entienda de qué trata tu TFG.

En síntesis, el capítulo de introducción debe destinarse a ofrecer una panorámica general del trabajo llevado a cabo, sobre todo en lo referente al contexto y razones que lo han motivado, de forma que se acerque al lector a los temas tratados en el mismo. Como veremos más adelante, esto ocupará dos secciones específicas del capítulo.

Por otro lado, como con cualquier otra sección de la memoria, tener presente el punto de vista del lector y facilitarle su comprensión resulta especialmente relevante. Ello fomentará que se anime a profundizar y seguir leyendo el resto de secciones. Si además se trata de alguien interesado en trabajar en líneas iguales o parecidas, se le motivará a que pueda consultar el resto del trabajo y continuarlo en el punto en el que se dejó en la memoria, o lo revise y hasta mejore.

Finalmente, desde un punto de vista pragmático, hemos de tener en cuenta que, por el mero hecho de enviar el TFG para someterlo a su defensa pública, ya tenemos un conjunto de lectores asegurado que, además, van a leer y consultar la memoria elaborada desde un punto de vista crítico. En efecto, dichos lectores serán las personas que formen parte del tribunal de evaluación del TFG, o al menos, el tutor o tutores, según venga establecido en la guía docente correspondiente.

Si el trabajo lo evalúa un tribunal, es muy probable que alguna o todas las personas que lo formen, no estén tan familiarizados como tú en el tema del trabajo, o no sepan por qué es importante o pertinente trabajar en el mismo. Es posible incluso que para alguno sea la primera vez que lee algo acerca de las temáticas principales del TFG. El capítulo de introducción sirve para aclarar todos estos aspectos y ayudar al tribunal a comprender el problema entre manos y valorar mejor las contribuciones del trabajo.

Por esto mismo, siendo el primer capítulo de la memoria, es uno de en los que conviene poner especial cuidado y esmero en la redacción, de forma que sea especialmente ágil, concisa y clara. También es importante seguir cierto orden y estructura a la hora de presentar (\textit{introducir}) los contenidos, siguiendo un patrón que atienda al \textit{Qué}, para describir el contexto); \textit{por qué}, para dar razón o motivar el trabajo, y \textit{por tanto}, para definir objetivos consecuentes con la motivación y el contexto del trabajo.

Podría ocurrirnos que aun tratando de ser breves, nos cueste ajustarnos a esta forma de estructurar, bien porque no encontremos muchas cosas que decir (mucho contexto), o porque de forma natural queramos conectar enseguida el contexto con la motivación. También, como se trata del primer capítulo y no estamos acostumbrados a documentar trabajos tan extensos, experimentemos cierto bloqueo a la hora de escribir. Esto mismo es lo que nos han reportado estudiantes como vosotros en un estudio previo \colorbox{yellow}{[REFERENCIAR ENCUESTAS]}. Como con el resto de secciones de la memoria del TFG, en las siguientes secciones daremos algunas pautas o pistas, con ejemplos, sobre cómo abordar o con qué tipo de contenidos completar el capítulo de Introducción. De forma similar a lo que ocurre con el primer capítulo de una serie de televisión o un libro cualquiera, el capítulo de introducción del TFG debe servir para conectar y enganchar con el lector, atrayendo su atención y provocando su curiosidad por el trabajo. De hecho, las productoras de televisión suelen invertir cantidades de dinero por minuto de metraje en los primeros minutos o el tráiler de una producción muy superiores a las que invierten en otros minutos de capítulos ordinarios. Saben que ese primer contacto con el espectador es fundamental para despertar su interés.

Seguramente hayas experimentado una situación similar al comenzar a leer otros libros o incluso otras memorias de TFG: si encuentras las primeras páginas difíciles de entender o pesadas, es más fácil abandonar la lectura de las siguientes o pasar por alto detalles relevantes, ya que cuesta más mantener la atención por el sobreesfuerzo requerido. Ponte especialmente en el lugar de tus potenciales lectores cuando escribas este capítulo. Como dice el refrán, \textit{para la primera impresión no hay una segunda oportunidad}.

Ten en cuenta también que este primer capítulo, junto con el de las conclusiones, será también uno de en los que en más se fijen quienes evalúen el trabajo. Esto se debe, por un lado, a que se encuentra al principio de la memoria, y lógicamente aparece y se lee antes; y por otro lado, a que en este capítulo deben reflejarse en una sección separada los objetivos del TFG. Por tanto, lo normal será que quien lo evalúe, acuda con frecuencia a dicha sección de objetivos y la contraste con las aportaciones reflejadas en la memoria, las conclusiones del trabajo y lo que se presente durante la defensa. 

Por último, este capítulo no deberá ser muy extenso (con 5 o 6 páginas puede ser más que suficiente), ya que se trata simplemente de dar un anticipo de la información que se detallará más adelante en el resto de secciones de la memoria para quien esté realmente interesado en el trabajo. Lo que sí es importante es tratar de respetar la estructura del capítulo, ya que suele ser bastante universal. Es decir, una especie de estándar o protocolo relativo a la forma de estructurar un libro, que en este caso es la memoria del TFG. Como ocurre con los estándares o los protocolos, quienes se adhieren a ellos, se obligan a respetar una serie de formatos y estructuras para que sea posible la comunicación o interacción entre diferentes actores. Análogamente, respetar una misma estructura de documento, en este caso, la memoria de un TFG, y un mismo tipo de contenidos en cada una de ellas, hará que tutores y revisores experimentados puedan recorrer y consultar la memoria de forma más eficaz, ya que podrán encontrar el tipo de información que buscan en las secciones previstas para ello.

Por la importancia de este capítulo, a veces se redacta al final del todo, cuando el resto de la memoria ha sido completada. La ventaja de esta aproximación es que ya tienes una visión global de tu TFG y su memoria, lo que posibilita que puedas centrarte en este capítulo y ser capaz de plasmar la esencia del TFG mucho mejor.

A continuación pasaremos a presentar las secciones, junto con sus contenidos y pequeños ejemplos de los mismos, que deben aparecer en el capítulo de introducción de un TFG. Normalmente, dichas secciones serán las siguientes: contexto, motivación, objetivos del TFG y estructura de la memoria.

% [Autores: Manolo]
\section{Contexto}\label{Contexto}
Esta sección está destinada a responder a la pregunta "¿Qué?". Qué existe, qué es o en qué consiste, qué se puede hacer, cuál es el tema de estudio o del trabajo, en qué dominio se sitúa, cómo se viene abordando una determinada actividad (es decir, de qué forma, qué enfoques, propuestas o aproximaciones se vienen defendiendo), qué propiedades, qué mercado, qué permite, etc., son ejemplos de cuestiones que debemos responder a la hora de presentar el contenido de esta sección. Dejamos por aquí algunas pistas o sugerencias de contenidos de que deben aparecer.

\subsection{Qué de qué. Definiciones}

En conjunto, la respuesta a los diferentes \textit{"qués"}, nos dará como resultado el contexto del trabajo y orientarán al lector con respecto al contenido de la memoria. Por ejemplo, si nuestro TFG se encuadra en el ámbito de los \textit{asistentes inteligentes}, una buena forma de comenzar a escribir el contexto es decir qué es un \textit{asistente inteligente} o qué tipo de sistemas software o hardware vamos a entender por tales.

Esto es especialmente importante cuando un mismo término se use de forma diferente en varios contextos. Para ello, es bueno comenzar con una definición o introducir alguna frase aclaratoria del tipo \textit{"... Si bien existen diferentes acepciones para el término} \textless término\textgreater, \textit{en este trabajo vamos a entender por tales aquellos que..."}. Lo mismo también es aplicable cuando nos vayamos a referir con diferentes términos a una misma idea o concepto. Por ejemplo, siguiendo con el caso de los asistentes inteligentes, si nos vamos a referir a ellos también como \textit{sistemas conversacionales}, podemos incluir una frase aclaratoria del tipo \textit{...Es frecuente encontrar referencias a los asistentes inteligentes denominándolos } sistemas conversacionales. \textit{En esta memoria utilizaremos indistintamente ambos términos para referirnos a aquellos sistemas que...}.

También podemos explicar brevemente la diferencia entre términos, para demostrar que hemos tomado la decisión conscientes de la existencia de matices, pero que no son relevantes en el contexto de nuestro TFG. Por ejemplo, si nuestro trabajo se enmarca en el ámbito \textit{Ubitquitous Computing} (Computación Ubicua) y nos vamos referir a estos sistemas también como del ámbito del \textit{Pervasive Computing}, podríamos decir \textit{``...En Computación Ubicua, el objetivo principal es proporcionar a los usuarios la capacidad de acceder a servicios y recursos en todo momento e independientemente de su ubicación, mientras que en Pervasive Computing, el objetivo principal es proporcionar servicios emergentes y espontáneos creados sobre la marcha por dispositivos móviles que interactúan mediante conexiones ad hoc. Existiendo esta diferencia, la realidad es que en muchas situaciones, ambos paradigmas de computación ofrecen las mismas soluciones y podemos referirnos a ellas como formas de Computación Ubicua o Pervasive Computing, indistintamente...''}.

\subsection{No despistes al lector}
Por otro lado, debemos tratar de seguir un enfoque bastante sintético, tratando de ir directamente al grano, sin digresiones sobre temas (otros "qué") que  se alejen del núcleo del trabajo. En efecto, es muy importante evitar incluir información o hacer referencia a cuestiones que no se hayan abordado con cierta profundidad, ya que pueden despertar falsas expectativas en el lector acerca de lo que va a encontrar. Como ya hemos comentado, conviene ponernos en el lugar de quien va a leer la memoria. Pensemos en nuestra propia experiencia como lectores de otros libros o manuales técnicos (una memoria de TFG es en parte eso). Si cuando consultamos el índice, la sinopsis de la contraportada o la introducción se nos hablaba de una tecnología, y por esto mismo nos animamos a sacar el libro de la biblioteca, y luego esta se usaba de una forma muy marginal en el resto del documento, nos sentimos decepcionados y que hemos perdido el tiempo.

Así, imaginemos la siguiente situación. En nuestro proyecto o desarrollo hemos utilizado unas técnicas bastantes complejas de aprender y con una curva de aprendizaje bastante plana al comienzo, siendo esta una dificultad conocida de dichas técnicas. A pesar de ello, hemos conseguido dominarla y resolver un problema concreto e interesante. Sin embargo, en nuestro trabajo no hemos realizado ninguna aportación que facilite el uso de dichas técnicas. En este caso, tendríamos que evitar hacer una referencia reiterada a la dificultad de manejar dichas técnicas, ya que el lector o revisor podría esperar que el trabajo terminará abordando ese problema, debido a que  se menciona varias veces y parece importante.

Esto ocurre con frecuencia con problemas conocidos: ruido en los datos en los mismos \textit{outliers}, usabilidad de una aplicación, rendimiento de un algoritmo, consumo energético, etc. Estos problemas podrían ser desafíos vigentes de la tecnología que utilicemos en nuestro TFG, pero si no hemos realizado ninguna aportación para paliarlos, lo mejor es no mencionarlos más de una vez en nuestro capítulo de introducción. Así, si decimos que en ciertos tipos de conjuntos de datos existe mucho ruido, que actualmente se está trabajando en técnicas que al generarlos reduzcan dicho ruido y que una línea de trabajo interesante en relación con esos conjuntos de datos es la eliminación del ruido, será porque en nuestro TFG, el ruido y su gestión se habrán abordado de alguna forma novedosa. Si, por el contrario, pero en nuestro TFG no hemos hecho o aportado ninguna solución para gestión del ruido, estaremos creando cierta confusión en la persona que lea el trabajo.

La experiencia con muchos trabajos que hemos supervisado o evaluado en tribunales de defensa en el pasado nos enseña que buenos trabajos de fin de grado se ven penalizados o deslucidos por referencias a aspectos o cuestiones en las memorias, que no eran el objeto o contribución nuclear del trabajo.

%\subsection{Respeta la estructura}
%Por último, indicar que una tendencia habitual al presentar una tecnología, un estado de la cuestión sobre algún tema, describir un dominio de aplicación, etc., es comentar a la vez los problemas o deficiencias con los que vengan aparejados. Por ejemplo, podemos haber realizado un TFG sobre un sistema de monitorización de la actividad física que sincroniza datos que recibe de sensores colocados en el cuerpo y graba en vídeo sesiones de entrenamiento. Hasta aquí habríamos descrito un \textit{qué}, es decir, monitorizamos la actividad física sincronizando vídeo y datos de sensores. Nuestro sistema es novedoso porque hasta la fecha no se dispone en el mercado de soluciones de bajo coste o abiertas que satisfagan dicha funcionalidad. En este caso, una tentación frecuente es conectar ambas ideas con una descripción como la siguiente: \textit{En los últimos años se han popularizado los sistemas y plataformas de monitorización remota de la actividad física a través de sensores corporales para estudiar el rendimiento durante el ejercicio físico. Estos estudios pueden complementarse con el análisis de vídeos o grabaciones de los sujetos mientras realizaban la actividad física registrada con dichos sensores} (hasta aquí un \textit{qué}).\textit{Sin embargo, no existen abundan soluciones de bajo coste que, de forma integrada, permitan gestionar la información obtenida desde ambos tipos de fuentes, es decir, vídeo y datos de sensores...} (habríamos comenzado a describir una \textit{deficiencia} o a dar un motivo, es decir, responder a \textit{por qué} hemos realizado el trabajo que se presenta.

%La conexión de ambas ideas en el resumen sería razonable, ya que por limitaciones de espacio, no habría opción a otras alternativas. En cambio, en el caso de la Introducción conviene cierta disciplina y estructura a la hora de presentar los contenidos. Como veremos en la siguiente sección, las razones que justifican la pertinencia del trabajo desarrollado deben describirse en la siguiente sección de \textit{motivación}.

\subsection{Estructura del introducción: algunas sugerencias}
A lo largo de toda la memoria, es esencial respetar la estructura del documento. Debemos respetar que tras un \textit{¿Qué} debe aparecer un \textit{porqué}.
Si encontramos dificultades para ajustarnos la estructura presentada porque necesitemos dar enseguida argumentos o motivaciones para el trabajo (que vendrán en la siguiente sección \ref{Motivation}, pensando que nos queda demasiado pobre si no lo hacemos, o no sabemos qué más comentar acerca de lo que ya existe, podemos completar proporcionando breves datos acerca de cuestiones como usuarios actuales o potenciales beneficiarios de una tecnología, orígenes de la misma, tamaño de mercado o volumen de facturación, iniciativas políticas, legislación, etc. Por ejemplo, si nuestro TFG versa sobre una plataforma para la promoción de hábitos nutricionales saludables en población infantil, para describir el contexto podemos apoyarnos de datos como \textit{"Un estudio reciente llevado a cabo en el ámbito del programa conocido como Estrategia de Promoción de una Vida Saludable en Andalucía (2024-2030), dependiente de la Consejería de Salud y Consumo de la misma comunidad, estima que la prevalencia del exceso de peso en la población infantil (de 2 a 17 años) andaluza se sitúa en el 33,40\%, situándose de desde el comienzo de estos estudios en varios puntos por encima de la media nacional"}\footnote{https://www.juntadeandalucia.es/organismos/saludyconsumo/areas/planificacion/estrategia-promocion-vida-saludable-andalucia.html}. Como vemos en este texto de ejemplo, hemos aprovechado para explicar indirectamente qué vamos a entender por población infantil (la que se encuentra entre 2 y 17 años de edad). Asimismo, hemos proporcionado también una referencia para demostrar que nos hemos documentado y acudido a fuentes oficiales para proporcionar el contexto de nuestro trabajo. En el capítulo  \ref{bibliografia} Bibliografía profundizaremos más en cómo citar los distintos trabajos.

Por último, si nos ocurre lo contrario, es decir, que tenemos exceso de contenidos con los que documentar el contexto del trabajo, trataremos de ser breves, describiendo o reservando los detalles para el capítulo de revisión del estado de la cuestión. Recordemos que en este capítulo hemos de tratar de ser concisos y presentar la información de forma atractiva para facilitar al lector la revisión de contenidos posteriores, además de despertar su curiosidad por leer el resto de la memoria.

\section{Motivación}\label{Motivation}
Esta sección debe justificar la razón de ser del trabajo realizado. Si la primera sección \ref{Contexto} Contexto, buscaba responder a la pregunta \textit{"Qué"}, esta sección debe más bien tratar de responder a preguntas como \textit{qué falta}, \textit{qué necesidades existen}, \textit{qué problemas tiene} o \textit{cómo se puede potenciar}, aquello que presentamos en dicha sección de contexto. Si, por ejemplo, nuestro trabajo trata sobre el análisis de la facilidad de navegación de un sitio web y conocemos una tecnología o una técnica que pensamos que podría aplicarse para monitorizar las distintas páginas que visitan los usuarios, pero que aún nadie la ha probado, podemos justificar nuestro trabajo indicando que el uso de dicha tecnología o técnica aún no se ha explorado para el análisis de la usabilidad de un sitio web. Como es plausible que su aplicación facilite dicha tarea, queda justificado hacer un TFG que lo investigue. La misma argumentación puede seguirse cuando describimos problemas, carencias, o funcionalidades por desarrollar y que decidimos abordar en nuestro TFG.

Por otro lado, si nuestro TFG ha consistido en la simulación de un encargo profesional relacionado con nuestro grado, es interesante indicar \textit{qué competencias} hemos adquirido o mejorado al realizar el mismo.

Se trata, de justificar por qué hemos llegado hasta aquí, o por qué es bueno o conveniente trabajar de la manera y en el ámbito que lo hemos hecho. En definitiva, hay que responder a la pregunta \textit{"Por qué"}: porque faltaba una determinada funcionalidad que hemos desarrollado; porque no se ha explorado la aplicación de una técnica en un determinado ámbito y podemos demostrar que se pueden obtener resultados interesantes haciéndolo; porque atendiendo una determinada necesidad se mejora la vida de ciertas personas; porque aplicando algo de distinta manera se obtienen mejores resultados que los que se vienen obteniendo en un determinado ámbito, etc.

Así, dependiendo del trabajo, podremos presentar unas motivaciones u otras, pero no debe resultarnos difícil encontrar un buen puñado de ellas. Algunos ejemplos podrían ser: 

\begin{itemize}
  \item Atención a un determinado colectivo en relación a una tecnología. Por ejemplo, nuestro trabajo podría consistir en el diseño de unas etiquetas en Braille y una aplicación para un dispositivo móvil capaz de interpretarlas para colocarlas junto a las obras de arte de un museo de forma que las personas con discapacidad visual puedan localizarlas y acceder a información enlazada que les proporcione información acerca de dichas obras. Nuestro TFG viene justificado o motivado por la necesidad de atender a ciertas personas y la ausencia de algo parecido.
  \item Problemas de usabilidad. Por ejemplo, los dispositivos para interactuar con agentes conversacionales inteligentes a veces tienen dificultades para captar correctamente las intenciones de sus usuarios, sin embargo tampoco existen métodos o herramientas para facilitar el análisis de por qué la interacción falla y nuestro trabajo ha consistido en desarrollar un sistema que demuestra que haciendo uso de una determinada tecnología es posible depurar las interacciones con dichos sistemas.
  \item Ausencia de una funcionalidad específica que demande un determinado tipo de usuarios o colectivo sobre ciertos conjuntos de datos. Por ejemplo, sabemos de las plataformas de compartición y reproducción de música bajo demanda. Algunas de estas ofrecen APIs (\textit{Application Programming Interface}) que ofrecen datos generalistas sobre artistas y descargas adaptadas a sus consumidores (este sería el \textit{qué}), pero conocemos que dentro del colectivo de los propios músicos o creadores de contenidos, estarían más interesados en disponer de herramientas que ofrecieran un análisis más pormenorizado de sus obras y evolución en ciertos períodos. Entonces, decidimos hacer un TFG para cubrir esta carencia desarrollando las funcionalidades pertinentes.
  \item Si nuestro trabajo va a consistir en ejecutar un encargo profesional\footnote{Que podrá ser simulado o real.}, como por ejemplo, el desarrollo de un sistema de información para la gestión de un gimnasio, podremos enumerar distintas competencias o habilidades que queramos adquirir y/o reforzar. De hecho, su adquisición ya representa por sí misma una motivación o justifica el desarrollo del TFG.
\end{itemize}

Por otro lado, como con todas las secciones de la memoria, debemos mantener el hilo y presentar un relato coherente. Por tanto, Como vemos, en esta sección concretamos un poco más el ámbito en el que hemos trabajado (para los trabajos más aplicados o que simulen encargos profesionales) o donde hemos aportado algo novedoso (para aquellos trabajos con alguna componente de investigación o innovación), en línea con la idea de centrar el tema del trabajo y no divergir o distraer a los potenciales lectores de la memoria.

\section{Objetivos del TFG}
A partir de los \textit{"qué"} y los \textit{"por qué"}, esto es, del contexto y motivación de las dos secciones anteriores, la sección de objetivos debe representar un \textit{"por tanto", "en consecuencia"} de lo que ya se ha explicado. Por ejemplo, como falta esta funcionalidad en este ámbito o dominio particular, nos proponemos abordar un desarrollo que cubra dicho vacío, o bien, como cierta tecnología presenta los problemas que ya se han descrito, proponemos un objetivo que resuelva o mitigue dichos problemas. Como venimos insistiendo, alinear los objetivos con el contexto y la motivación del trabajo, también ayudará a revisar mejor el trabajo y no distraer al lector.

Lo normal será también que los objetivos se presenten describiendo en primer lugar un objetivo principal\footnote{También podemos denominarlo como objetivo general (OG).} que sea lo más comprensivo posible, de forma que englobe las tareas o resultados más destacables de nuestro TFG y,  a continuación, una serie de objetivos más particulares o concretos, también denominados como específicos, sobre resultados o tareas más sencillas. Esto demostrará que hemos analizado el problema a abordar y que hemos sido capaces de descomponer y estructurar una tarea compleja en otras más sencillas para conseguir un alcance mayor. Como siempre, pero especialmente en esta sección, conviene que nos expresemos con concisión. Así, una buena frase para comenzar esta sección puede ser:

\textit{El objetivo principal de este TFG es } \textless nuestro principal objetivo\textgreater. \textit{Este objetivo principal se ha articulado en torno a la consecución de los siguientes objetivos específicos}:
\begin{itemize}
  \item \textit{Objetivo específico 1}
  \item \textit{Otro objetivo específico}
  \item \textit{Y otro más}
  \item \textit{...}
\end{itemize}

Como pista, podríamos decir que el objetivo principal debería ir alineado con el título del proyecto, para dotar de coherencia a la memoria. Asimismo, en la lista o enumeración anterior de objetivos específicos, deberán aparecer primero aquellos que sirvan de base o deban alcanzarse antes de abordar otros, secuenciando los pasos que deben darse para alcanzar gradualmente el objetivo principal.

Cabe indicar en este punto que los objetivos deben estar formulados en infinitivo: estudiar, analizar, desarrollar, comparar, evaluar, etc.  

Por otro lado, las características principales que debemos buscar al fijar (y expresar) los distintos objetivos en el contexto de un proyecto, en general, y de un TFG en particular, es que sean realistas y concretos. Sin ánimo de seguirlas exhaustivamente, podemos guiarnos por las orientaciones del marco de trabajo SMART \cite{doran1981there}, en particular, las revisiones más modernas y orientadas a proyectos tecnológicos, como las que podemos encontrar en el sitio web de \href{https://www.atlassian.com/blog/productivity/how-to-write-smart-goals}{Atlassian}. SMART es el acrónimo de \textit{Specific}, \textit{Measurable}, \textit{Achievable}, \textit{Relevant}, y \textit{Time-Bound}, es decir, los objetivos que fijemos deben abordar un área concreta de trabajo o mejora (esto es, ser \textit{específicos}); incluir alguna métrica o valor objetivo para seguir y evaluar su grado de progreso o cumplimiento (para que sean \textit{medibles}); es factible (realista) conseguirlos (es decir, son \textit{alcanzables}); son pertinentes o necesarios para el TFG (por tanto, \textit{relevantes}); y finalmente, puede anticiparse un plazo de tiempo para su ejecución (es decir, son \textit{planificables} o acotarse en el tiempo).

Finalmente, indicar que, aunque idealmente los objetivos de un trabajo se establecen (al menos informalmente) al comienzo del mismo y se van abordando conforme a una determinada planificación, la realidad y el posible desconocimiento de algunas tecnologías que estemos utilizando, nos lleven a modificar nuestros planes iniciales y revisar dichos objetivos, bien porque vemos más interesante algunas posibilidades que hemos descubierto, o bien porque nos encontramos con algún imprevisto que nos impida alcanzarlos.

Por ello, desde un punto de vista estratégico y siendo también realistas, podemos dejar para el momento de finalización de la memoria la redacción concreta de los objetivos, según el trabajo que ya conoceremos con seguridad que hemos completado. La mayoría de las veces infravaloramos el tiempo que vamos a necesitar para realizar una tarea y sobrestimamos nuestras capacidades, pero digamos que no hace falta que lo explicitemos en nuestra memoria. Al fin y al cabo, ordinariamente tendremos que hacer una \textit{defensa} del mismo, y si indicáramos posteriormente que no hemos alcanzado los objetivos que nos fijamos, además de resultar extraño, nos estaríamos despojando de argumentos para defender o justificar nuestra capacidad de análisis o planificación.  

También indicar que se suelen poner, además de los objetivos del proyecto, algunos objetivos personales que se desean alcanzar con la elaboración del mismo. Estas metas reflejan ciertos aprendizajes o puesta en práctica de metodologías o tecnologías que, aprovechado que estás haciendo el TFG quieres conseguir. Un par de ejemplos podrían ser "aprender y aplicar en un proyecto real la metodología SCRUM" o "aprender el \textit{framework} Django y ponerlo en práctica en un proyecto real". 

\section{Estructura de la memoria}
Esta sección está destinada a comentar cómo hemos organizado el resto de la memoria, por lo que simplemente debemos describir qué bloques principales o capítulos hemos utilizado y, en una frase o dos, dar una breve idea de qué contiene cada uno. La idea es mostrar cómo queda estructurado el texto y qué se va a describir en cada una de las partes del mismo.

Desde un punto de vista visual, podríamos utilizar una enumeración en forma de lista de elementos separados por puntos.

Por ejemplo, podríamos escribir algo como: \textit{"En el capítulo 1 se han presentado el contexto y motivación de este trabajo, así como los principales objetivos que nos hemos propuesto para abordar la problemática asociada a \textless X\textgreater. El resto de la memoria se ha estructurado como sigue:}

\begin{itemize}
  \item \textit{En el capítulo 2 se describe el estado de la cuestión en relación a las principales tecnologías/métodos/técnicas/estudios} (escribir lo que corresponda)\textit{ sobre \textless lo-que-hayamos-usado-en-nuestro-TFG \textgreater}.
  \item \textit{En el capítulo 3 se presenta la propuesta que hemos realizada para} \textless abordar-el-problema-u-objeto-principal-de-nuestro-TFG \textgreater.
  \item \textit{El capítulo 4 contiene la especificación y arquitectura de los principales módulos/interfaces/servicios que hemos desarrollado}.
  \item ...
  \item \textit{En el último capítulo se presentan las conclusiones y trabajos futuros a partir del presente TFG}.
  \item \textit{Finalmente se presentan las referencias bibliográficas usadas en este trabajo}.
\end{itemize}

Como es lógico, lo más práctico será escribir esta sección cuando hayamos completado todos los demás capítulos y conozcamos la estructura definitiva que vamos a dar a la memoria de TFG.
