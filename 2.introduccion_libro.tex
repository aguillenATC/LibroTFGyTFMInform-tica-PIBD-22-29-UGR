\chapter{Introducción}
\label{cap:Introducción}
%Juanma



Este libro queda dividido en los siguientes capítulos: en el capítulo \ref{cap:Recomendaciones} se ofrece, para empezar, información general sobre el proceso de asignación del TFG, contexto general que debes conocer antes de comenzar a trabajar, así como pautas y recomendaciones de trabajo para ayudarte en el desarrollo del proyecto. El capítulo \ref{cap:EstructuraMemoria} establece una propuesta de estructura general de la memoria, indicando cada una de las partes que la compone, para que tengas una idea general de cómo organizar este documento. El siguiente capítulo, el \ref{cap:IntroducciónTFG}, se centra en describir el contenido del capítulo de introducción, ofreciendo tanto una posible estructura del mismo como consejos para su elaboración. El capítulo \ref{cap:RevisionEstadoDelArte} pasa a describir cómo se debe hacer una revisión del estado del arte, parte fundamental en cualquier memoria de TFG. Un elemento importante en todo proyecto es la planificación y el presupuesto, y en el capítulo \ref{cap:PlanificacionPresupuesto} damos algunos consejos para su elaboración. Seguidamente, en el capítulo \ref{cap:Tipologías} se presentan los cuatro tipos de TFG principales: de desarrollo, experimental, investigación y de revisión. Recomendaciones sobre cómo abordar la confección de las conclusiones y los trabajos futuros se incluyen en el capítulo \ref{cap:Conclusiones}. El capítulo \ref{cap:bibliografia} aborda asuntos relacionados con la bibliografía y cómo referenciarla en la memoria y el siguiente, el capítulo \ref{cap:anexos} hace lo propio con los anexos que puedes incluir en la memora.  Una vez que la esta está terminada, en el capítulo \ref{cap:Revisión} se muestra una lista de comprobaciones que deberías hacer para estar seguros de que la memoria alcanza el mínimo de calidad exigible. Los dos capítulos siguientes, y últimos, el \ref{cap:elaboraciónPresentación} y \ref{cap:defensa}, se centran en la presentación y en la defensa, respectivamente, aconsejando sobre cómo montar la primera y cómo afrontar la segunda. Y como no podía ser de otra manera, este libro finaliza con un capítulo de anexos 