\appendix
\chapter{Licencias} \label{anexo:licencias}
\subsection{¿Qué son las licencias?}

Una licencia no es más que un contrato que indica qué se puede y qué no se puede hacer con tu obra. Por ejemplo, quiénes son los autores, en qué territorios puede usarse, qué derechos das a las personas que van a usar tu obra, o incluso qué fechas de uso les puedes dar. 

En esta sección vamos a ver los tipos de licencias que existen sobre una obra, lo cual, en nuestro caso incluye software y recursos digitales (textos, imágenes, vídeos, código, etc.) que sean obras tuyas o de terceros. El reconocimiento de la propiedad intelectual de una obra es importante para tener argumentos que permitan ganar un litigio en caso de plagio, protegiendo a la persona que ha creado la obra, pero también otorgarles derechos de uso a otras personas. Aunque haya licencias ya creadas y ampliamente extendidas (como CC o GNU) tú también puedes crear tu tipo de licencia desde cero, aunque te recomendamos que uses las que ya existen.
\begin{naranja}
Comenzamos con los tipos de licencias existentes según los derechos de autor de una obra. Las licencias pueden abiertas o cerradas. De entre las licencias abiertas, se especifica principalmente si se pueden hacer cambios sobre el original y si se permite o no la distribución del original o copias (incluyendo copia y venta) por parte de otros. Así, las licencias abiertas pueden ser permisivas (se puede modificar el original y hay flexibilidad en la distribución), robustas débiles (se puede modificar el original pero ni distribuirlo) o robustas fuertes (no se puede modificar el original ni distribuirlo, pero si usar tal cual). Por ejemplo, el software de Apache tiene licencia abierta permisiva, y el de Eclipse tiene licencia abierta robusta fuerte.

\subsection{Licencias para texto, datasets e imágenes}

Las licencias de Creative Commons (CC) \url{https://creativecommons.org/} regulan bastante bien estos aspectos de las licencias abiertas y otros a la hora de crear recursos propios o usar recursos de terceros. Tienen en cuenta:
\begin{itemize}
    \item Reconocimiento al autor (BY): Obligación de nombrar al autor cuando se use su obra.
    \item Compartir Igual (SA): las obras derivadas deben mantener la misma licencia que e original al ser distribuidas
    \item Sin Obra Derivada (ND): No se puede transformar una obra original, por ejemplo, no se puede traducir y distribuir la traducción.
    \item NoComercial(NC): Solo el autor puede distribuir la obra con fines comerciales
    \item Dominio Público: El autor renuncia a sus derechos. Cualquiera puede copiar, modificar y distribuir la obra, incluso con fines comerciales, sin pedir permiso
    \item Zero (0): Puedes usar el original, generar una obra derivada, distribuirlo con uso comercial o no comercial. Es como el dominio público pero no tienes ni siquiera que mencionar al autor.
\end{itemize}

Todas las licencias de CC, excepto la \textit{zero}, tienen obligación BY, de reconocimiento del autor de la obra.

Vamos a ver los tipos más comunes de licencia CC que hay en función de la combinación de algunos de los aspectos de la lista.
\begin{itemize}
    \item (BY-NC) Reconocimiento - No Comercial: puedes generar obras derivadas siempre que no hagas un uso comercial de las mismas ni del original. 
    \item (BY-NC-SA) Reconocimiento - No Comercial - Compartir Igual: igual que by-nc, pero además, la distribución de las obras derivadas se debe hacer con una licencia igual a la que regula la obra original. Por ejemplo, si alguien usa tu imagen con esta licencia para hacer un collage, este collage debe tener ese mismo tipo de licencia. Es frecuente que la licencia tenga un número de versión, que debes conservar en este caso, por ejemplo, CC BY-NC-SA 4.0.
    \item (BY-NC-ND) Reconocimiento - No Comercial - Sin Obra Derivada:  no puedes generar obras derivadas ni hacer un uso comercial de la obra original. Es la licencia más restrictiva de CC. Por ejemplo, si una imagen, esquema o dataset tienen esta licencia, no puedes hacer cambios sobre ellos, pero sí usarlos tal cual están.
    \item (BY-SA) Reconocimiento - Compartir Igual: Puedes crear obras derivadas y hacer uso comercial de éstas y del original, siempre que sea bajo el mismo tipo de licencia de la original.
    \item  (BY-ND) Reconocimiento - Sin Obra Derivada: No puedes generar obras derivadas, pero sí puedes hacer uso comercial del original. 
\end{itemize}

Existe otra licencia más restrictiva aún que es la del Copyright (c). Si una obra tiene una licencia con Copyright, no puedes usarlas sin pedir permiso y no puedes generar obras derivadas, ni distribuirla, aunque sea para uso no comercial. Por ejemplo, no puedes poner en tu TFG una imagen con Copyright, aunque nombres al autor. 

\textcolor{orange}{Para que no infrinjas la ley, te recomendamos que si necesitas usar recursos digitales en tu memoria o código, busques en bancos de imágenes, sonidos y vídeos de dominio público o con licencia Creative Commons. También tienes disponibles varios repositorios de recursos educativos abiertos (REA) que puedes usar para la introducción y estado del arte de tu TFG. En la siguiente tabla te proporcionamos algunos enlaces a sitios desde donde puedes descargarlos usando filtros y palabras clave para buscar lo que deseas:}


\begin{table}[t]
\begin{center}
\resizebox{13cm}{!}{
\begin{tabular}{| l | c | c | c | c |}
\hline
Enlace & Imagen & Sonido & Vídeo & REA \\ \hline
\url{https://biblioteca.uoc.edu/es/biblioguias/biblioguia/Bancos-de-imagenes/} & x &  &  & \\ \hline

\url{https://biblioteca.uoc.edu/es/biblioguias/biblioguia/Bancos-de-audiovisuales/} &  &  & x & \\ \hline

\url{https://www.pexels.com/es-es/videos}&  &  & x & \\ \hline

\url{https://edpuzzle.com/}&  &  & x & \\ \hline

\url{http://recursostic.educacion.es} & x & x &  & \\ \hline

\url{http://stock.adobe.com} & x & x & & \\ \hline

\url{https://www.flaticon.es} & x &  &  & \\ \hline

\url{https://unsplash.com/es} & x &  &  & \\ \hline

\url{https://freesound.org} &  & x &  & \\ \hline

\url{cedec.intef.es}  &  &  &  & x\\ \hline

\url{procomun.educalab.es}   &  &  &  & x\\ \hline

\url{oerworldmap.org}    &  &  &  & x\\ \hline
\end{tabular}
}
\caption{Sitios con recursos digitales libres}
\end{center}
\end{table}

\textcolor{orange}{
En la OUC hacen un recopilatorio de webs que proporcionan imágenes de este tipo (\url{https://biblioteca.uoc.edu/es/biblioguias/biblioguia/Bancos-de-imagenes/}( y vídeos (\url{https://biblioteca.uoc.edu/es/biblioguias/biblioguia/Bancos-de-audiovisuales/}). Otro sitio que con un buscador de imágenes o sonidos para usar en educación es \href{http://recursostic.educacion.es}{Recursos TIC Educación}. También puedes descargar imágenes o sonidos gratuitos desde \url{http://stock.adobe.com}, buscando en aquellas colecciones que estén libres de derechos. Si necesitas iconos con licencia abierta, gratuitos, puedes descargarlos desde \url{https://www.flaticon.es}. También hay imágenes gratuitas para descargar en \url{https://unsplash.com/es}.
También puedes descargar sonidos desde \url{https://freesound.org}. Además, puedes encontrar recursos educativos  (materiales de enseñanza, aprendizaje e investigación) abiertos en repositorios como \href{cedec.intef.es}{Proyecto EDIA}, \href{procomun.educalab.es}{PROCOMUN} y \href{oerworldmap.org}{OER WORLD MAP}. 
}
\textcolor{orange}{
Otra opción que tienes si necesitas una imagen, es la de usar el buscador de Google, seleccionando imágenes y luego en herramientas, en licencias de uso, elegir Creative Commons. Por ejemplo, si te bajas imágenes de Internet para usarlas en la memoria o el programa sin filtrar antes, verás que te indican con una marca superpuesta las que están bajo licencia. 
}

Si utilizas un generador automático de imágenes o texto con IA generativa, tú serás el autor o tendrás la propiedad intelectual de los prompts o entradas, y del output o salida, ya que los generadores no son personas que puedan poseer derechos de autor \url{https://universoabierto.org/2023/11/10/quien-posee-los-derechos-de-autor-del-contenido-generado-por-inteligencia-artificial-ia/}. Eso sí, si en los prompts das información específica de fuentes con derechos de autor o utilizas contenido protegido, es tu responsabilidad el garantizar que cumples con las leyes de derechos de autor al citar adecuadamente esas fuentes.

Recuerda que aunque obtengas un recurso de forma gratuita, esto no quiere decir que lo puedas usar de forma abierta. Por ello, y y a no ser que tenga licencia \textit{zero} (0), asegúrate siempre de mencionar en tu TFG al autor y/o la web desde donde te has descargado el recurso, y el tipo de licencia que tiene. Tienes ejemplos de cómo citar la fuentes de imágenes originales o adaptadas con licencia CC \href{https://wiki.creativecommons.org/wiki/Recommended_practices_for_attribution}{aquí}. Lo habitual es poner detrás del título de una imagen el título original, el autor y el tipo de licencia, así como indicar si has modificado la imagen y la nueva licencia que le atribuyes, si es el caso.

 
En cuanto a tu TFG, si deseas que tu memoria final tenga licencia CC, bastará con que uses el \href{https://chooser-beta.creativecommons.org/}{generador de licencias de CC}, para indicar qué tipos de licencia deseas asignar a tu TFG. Este generador te proporciona un icono para insertar en la memoria, o un código html para incrustar en una web. Se suele insertar el icono en la primera página. También puedes poner el texto legal completo en el que se basa la licencia (extrayéndolo de la página de CC). Puedes optar entre (BY-NC), (BY-NC-SA) si deseas que alguien use tu memoria y haga adaptaciones de ella, o (BY-NC-ND) si das permiso para usarla tal cual, sin adaptaciones. También puedes añadir el enlace URL de tu memoria, si está disponible de forma abierta en un blog, portal web o similar para facilitar la reutilización por parte de otros. 

Te aconsejamos que subas tu memoria a un repositorio oficial como \href{https://digibug.ugr.es/password-login}{digibug}, previa solicitud de claves. Al subir un documento a este repositorio, se licencia automáticamente como CC (BY-NC-ND), a no ser que desees que sea menos restrictiva, por ejemplo, solo BY-NC, permitiendo a otros que creen obras derivadas con tu trabajo. Si es así, debes indicarlo en la primera página del documento. 

Estas recomendaciones para licenciar la memoria del TFG puedes aplicarlas a otros trabajos que realices dentro o fuera del ámbito universitario, y no solo en formato de texto, también para imágenes o vídeos creados por tí. CC facilita la descarga de elementos que identifiquen los diferentes tipos de licencias en los diferentes formatos, así, en \url{https://creativecommons.org/mission/downloads/} puedes descargarte logos, iconos, stickers, gifts animados, y en \url{https://wiki.creativecommons.org/wiki/CC_video_bumpers} bumpers para colocar en tus vídeos.

\subsection{Licencias de código fuente}

El código creado por ti durante el desarrollo del proyecto también podrías licenciarlo como CC, pero desde la versión 4.0, no se recomienda esta licencia para código fuente, así que lo más habitual es usar licencias específicas de software, que están pensadas para las especificidades del código (por ejemplo, también tratan su  ejecución). 

Si deseas licenciar código, bastará que en las primeras líneas de cada fichero, dentro de un comentario, indiques el tipo de licencia que le atribuyes. Algunas licencias también requieren un fichero LICENSE o README con la licencia, que deberás compartir con tu código, por ejemplo, en el directorio raíz de tu proyecto. Hay dos tipos de licencias: las privativas y las abiertas/libres.

Respecto a las licencias de código abierto/libre hay dos corrientes: el Software Libre (Free Software) y el Código Abierto (Open Source). De hecho se refiere al conjunto de las dos como FLOSS (Free/Libre Open Source Software). El ``Libre'' del acrónimo es porque en inglés ``gratis'' y ``libre'' se escriben igual, así que le han añadido la palabra en español para que quede claro el componente de libertad, no de gratuidad (``\textit{Free as a bird, not as a beer}'').

Dependiendo de la comunidad de desarrollo, se prefieren algunos tipos de licencias a otras. Por ejemplo los paquetes de Node.js suelen tener MIT o ISC, los crates de Rust usan MIT o Licencia Apache, y los plugins de Wordpress deben ser GNU.

Aunque no es obligatorio que liberes el código de tu TFG, puedes obtener una serie de beneficios si lo haces: te obliga a seguir buenas prácticas de desarrollo, sirve para crear comunidad, genera ejemplo, y no menos importante, te servirá como portfolio que te ayudará a encontrar un buen trabajo.

\subsubsection{Licencias de Software Libre}
La corriente del Software Libre es digamos, más ``filosófica'', y está orientada a la Libertad de la persona que usa el Software: libertad para usarlo, para estudiarlo, distribuirlo y mejorarlo. Para conseguir estos objetivos es necesario tener acceso al código siempre, por lo que las licencias de Software Libre tienen como requisito compartir las modificaciones (como en CC-SA). La Free Software Foundation (FSF, \url{http://www.fsf.org}) es la entidad que se encarga de mantener licencias como la GNU. Su página web tiene muchísima información sobre cómo usar sus licencias explicada con un lenguaje claro \cite{FSFfaq}.

\begin{itemize}

\item \textbf{GNU General Public License (GPL)} La última versión es la V3. Te permite el uso comercial del software que la tiene (es decir, venderlo y ganar dinero), distribuirlo, modificarlo, y las personas que contribuyen a él tienen derecho a patente. Como condiciones, si vas a distribuir el programa o una versión a partir de él (a un cliente, o ponerlo en internet) tienes que revelar el código, mantener el copyright original y el archivo de licencia y documentar los cambios respecto al original. Digamos que es una licencia vírica, puedes usar mi código para lo que quieras, pero tiene que seguir siendo GPL.

\item \textbf{GNU Affero General Public License (AGLP)}. En los últimos años ya no usamos el software en nuestro ordenador, a veces accedemos a él a través de internet. Esta licencia en concreto extiende la GPL con una cláusula para que si el software ofrece servicios a través de una red, entonces también hay que liberar el código.

\item \textbf{GNU Lesser General Public License (LGLP)}. Esta es más flexible que la GPL. Si tu software se usa vía interfaces (por ejemplo, accediendo a los métodos de una librería), el software que llama a esas funciones puede usar cualquier otra licencia (incluso privativa), pero si modificas la librería entonces tienes que liberar el código modificado.

\end{itemize}

\subsection{Código Abierto}
El Open Source digamos que está más orientado al mundo empresarial, y las licencias no exigen liberar los cambios. La entidad que promueve estas licencias es la Open Source Initiative \url{https://opensource.org/}. Las licencias de código abierto más extendidas son las siguientes (aunque hay más de 100 en la web de la OSI):

\begin{itemize}
   
\item \textbf{Licencia Apache}. Tiene los mismos permisos que la GPL (uso comercial, distribución, modificación, derecho a patente), pero a diferencia de GPL no es necesario liberar el código ni usar la misma licencia, pero sí documentar los cambios que se han añadido.

\item \textbf{Licencia MIT}. Es mucho más permisiva que la anterior: no hace falta ni documentar el cambio, se pueden generar patentes sin añadir a todos las personas que han contribuido, y no hay que documentar los cambios.

\end{itemize}

\end{naranja}
\subsection{Código generado por IA o copiado de páginas web}

Por último, indicarte que si has usado alguna una herramienta de IA generativa para obtener el código, o lo has tomado de foros como Stack Overflow o Reddit, por ejemplo, debes tener en cuenta también lo relativo a la LPDGDD y respetar las licencias que tienen los códigos originales que hayas copiado. Es decir, solo debes copiar o crear un código derivado como modificación del copiado si la licencia del original te lo permite. En caso de que no encuentres licencia asociada, por omisión tiene copyright, así que actúa en consecuencia.


\subsection{Otras consideraciones sobre las licencias y cómo obtenerlas}

En el caso de que utilices varios paquetes o librerías con licencia deberás tener mucho cuidado para que sean compatibles entre sí. Puedes usar webs como \href{https://joinup.ec.europa.eu/collection/eupl/soluti on/joinup-licensing-assistant/jla-compatibility-ch ecker}{JoinUP} para ver el tipo de licencia que tendrías que ponerle a tu proyecto si las combinas. Si no sabes qué licencia escoger, puedes usar el test de \url{http://choosealicense.org}.

Por último, nos gustaría comentarte otra opción que es la de obtener un registro de propiedad intelectual, que equivale a tener una licencia de Copyright, en la que el autor puede dar derechos a otras personas previa solicitud, por ejemplo de uso o copia. Para obtener ese registro, debes ir físicamente a una oficina provincial de registro de propiedad intelectual o bien hacer el registro en una web oficial como la del \href{https://www.juntadeandalucia.es/organismos/turismoculturaydeporte/servicios/procedimientos/detalle/297.html}{Registro Territorial de Andalucía}. Para hacer el registro debes completar unos formularios en los que se piden datos relativos a la obra y los autores, y pagar unas tasas que oscilan desde los 15 euros si se hace de forma presencial, a los 8 euros si el registro es online. Al hacer un registro, obtienes un documento que te reconoce como autor de la obra. En España los derechos de autor expiran a los 50 años de la muerte del autor, en cuyo caso, la obra queda como dominio público.

Tanto con un registro de propiedad, como indicando tipo de licencia en tu memoria de TFG y código, si en el futuro alguien plagia tu TFG o parte de él, no te cita, o se lucra con tu trabajo sin tu autorización y sin respetar la licencia que has indicado, podrás tener un respaldo de tus derechos para denunciar.

Finalmente, recuerda que en la UGR puedes consultar con la OSL (Oficina de Software Libre) o con la OTRI (Oficina de Transferencia de Resultados de Investigación) para resolver dudas sobre licencias libres o normativa de transferencia y propiedad intelectual.
