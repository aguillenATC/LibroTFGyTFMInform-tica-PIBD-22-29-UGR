\chapter{La revisión del estado del arte} \label{cap:RevisionEstadoDelArte}

% [Autores: Pablo, Juanma]

\section{Introducción}
¿Qué es una revisión el estado del arte? Básicamente un análisis del conocimiento actual sobre un tema de tal forma que te permita identificar qué hay hecho sobre el mismo (qué se ha investigado o desarrollado -- visión general del conocimiento) y qué lagunas existen. Es el paso previo para que tú puedas realizar una propuesta que mejore lo que hay ya hecho y, sobre todo, que sitúes tú trabajo en un contexto. La idea es que busques recursos bibliográficos (fundamentalmente literatura relevante en forma de artículos científicos o técnicos) y los estudies, comprendiendo el problema y las soluciones propuestas para resolverlo, entendiendo los métodos aplicados y que, como consecuencia de ese estudio, seas capaz de tener una imagen general y clara del tema y de su evolución e identifiques situaciones o elementos de mejora. En lo que se refiere al TFG no es resumir los recursos que has encontrado, sino analizarlos, sintetizarlos, clasificarlos, contextualizarlos y evaluarlos de forma crítica para obtener una imagen clara del estado del conocimiento.

Por tanto, ¿para qué te sirve realizar una revisión del estado del arte en tu TFG? Es una de las primeras tareas que tienes que llevar a cabo una vez asignado el TFG ya que te permitirá inicialmente familiarizarte con el problema y con todo lo que se ha hecho en el campo de tu TFG. Y también de forma práctica, como se aborda en una fase inicial, para evitar que hagas algo que ya está hecho. También sirve para que identifiques limitaciones, lagunas o brechas en el conocimiento o problemas no resueltos donde tu TFG puede aportar. El que la memoria de tu TFG conste de una buena sección del estado del arte es importante ya que está demostrando que conoces lo que hay hecho en el contexto de tu TFG y también muestra tu capacidad de síntesis y análisis. Además, te va a permitir desarrollar un marco teórico y metodológico para tu proyecto y posicionarte con respecto a las metodologías, teorías o desarrollos existentes, con objeto de diferenciar tu trabajo. 

Veamos un par de ejemplos. En el primero, tienes entre mano un TFG en el que deseas aplicar inteligencia artificial a la gestión de invernaderos. Lo primero que tienes que hacer, por tanto, es preguntarte qué se ha hecho en este campo, qué problemas se han abordado, qué técnicas se han aplicado, qué resultados se han obtenido. Y esto sólo puedes hacerlo mediante una búsqueda bibliográfica, leyendo todos los recursos que obtengas y analizándolos. De esta forma puedes darte cuenta que el campo en general está muy trillado, pues es algo donde se ha trabajado mucho, se han producido muchas aportaciones y con buenos resultados, y quizá que tu aportación sería insignificante o poco relevante. Pero con este análisis, además de llegar a conocer bien el tema y de ser capaz de describir cuáles han sido sus avances, te has dado cuenta que en el área del riego inteligente por goteo los métodos que se ha aplicado no funcionan del todo bien y no se adaptan correctamente a las circunstancias específicas de los invernaderos (fundamentalmente por los diferentes tipos de suelo, de plantaciones y condiciones ambientales, por ejemplo). En este caso acabas de detectar una posible línea de trabajo y, por tanto, un lugar por donde poder orientar tu TFG y realizar una aportación metodológica o práctica: la aplicación de algoritmos genéticos a la gestión eficiente del riego (por decir algo). 

Imagina una segunda situación en la que vas a desarrollar una aplicación móvil para ayudar a las personas mayores en caso de necesidad. ¿Qué otras aplicaciones hay en el mercado que estén en este ámbito? En este caso, los recursos bibliográficos no sólo serán artículos científicos donde se presentan esas aplicaciones y se evalúan, sino también  sitios web de aplicaciones de este estilo, en la que cuentan sus prestaciones y funcionalidades. En este caso debes hacer una búsqueda exhaustiva y obtener todas las aplicaciones, analizar su funcionamiento en el caso de que te las puedas descargar y probar, o estudiar las prestaciones publicadas y clasificarlas por funcionalidades. Así conocerás qué hace la ``competencia'' y qué puntos fuertes y débiles tiene cada una. Con esta información establecerás las prestaciones que tendrá que tener tu aplicación (las habituales que todas tienen, por ejemplo, el botón de petición de ayuda) y podrás contribuir con aquellas novedosas que no has visto en ninguna (la conexión automática con familiares en caso de necesidad o la monitorización de constantes vitales y el envío de un médico de forma automática cuando se detecta un problema en estas, por decir algo) y, por tanto, que aporten un valor añadido a la tuya.

El objetivo de este capítulo es dar unas consideraciones generales que te permitan realizar un estudio del estado del arte en tu TFG y plasmarlo correctamente en la memoria. Para tal fin, este capítulo tiene como objetivo ayudarte a conocer cómo realizar y organizar tu revisión del estado del arte en la memoria del TFG. Así, en primer lugar tendrás que buscar información, realizar un análisis y plasmarlo en la memoria. En las secciones siguientes te indicamos cómo llevar a cabo este proceso.

\section{Búsqueda de información}

Una revisión del estado del arte tiene una primera fase que es la búsqueda de los recursos que posteriormente pasarás a analizar. Veamos algunos elementos importantes en esta etapa.

\subsection{Identificar las preguntas}

 %añadido por María José en forma más o menos telegráfica (obtenido del curso de la UNED de redacción de TFGs):  Para hacer una revisión del estado del arte, antes tenemos que delimitar el tema del TFG y en base a ello hacer una revisión bibiográfica con la que  podemos ver qué hay publicado o realizado sobre el mismo tema o temas cercanos, qué aspectos se han analizado en esas publicaciones, qué discusiones o polémicas han suscitado, etc. 

A estas alturas de la película tienes una idea más o menos clara sobre de qué va a tratar tu proyecto. La idea ahora es ver qué hace tu proyecto especial y no ser la enésima aplicación CRUD. Pero también puedes aprender de lo que ya existe. ¿Te suena la expresión ``a hombros de gigantes''? Pues vamos a subirnos a los hombros de los gigantes y hacernos una serie de preguntas:

\begin{itemize}
    \item \textit{¿Alguien ha hecho antes lo mismo que yo?}
    
Posiblemente tu idea no sea 100\% original. Y no pasa nada. De hecho, es muy raro que lo sea: para empezar cada año se leen cientos de TFGs solo en tu escuela y cada vez es más difícil realizar algo totalmente novedoso. Pero es importante saber qué han hecho los demás para no caer en los mismos errores, sino también encontrar sus fortalezas.

\item \textit{¿Qué han hecho los demás que puede serme útil?}

Mira las funcionalidades que ofrecen. ¿Qué te parece interesante? ¿Qué te parece que no aporta nada? ¿Qué te parece que está bien pero se podría mejorar?

\item \textit{¿Cómo lo han hecho?} 

¿Qué tecnologías / metodologías han usado? ¿Por qué esas y no otras? ¿Qué les han ofrecido a sus creadores? Podemos incluso enfocarnos un poco en estas preguntas en caso de un TFG de desarrollo: ¿Es mejor una aplicación web o una aplicación de escritorio? ¿Es mejor usar C o Python? ¿Es mejor un algoritmo evolutivo o una red neuronal para este problema?

\item \textit{¿Qué ofrece mi proyecto que no ofrece el resto?} 

Esta es quizás la pregunta más importante a resolver y que hará que tu proyecto brille sobre el resto. Quizás tu proyecto es el primero que es Software Libre, y con ello ayudarás a la comunidad. Quizás tu proyecto sea el primero para Android. O quizás tu proyecto sea el primero que utiliza un algoritmo de explicabilidad. Por pequeña que sea tu propuesta, ya habrás hecho algo nuevo y mejorado el estado del arte.

\end{itemize}

Para resolver estas preguntas obviamente necesitas ver y entender lo que ha hecho el resto. Y para ello necesitas información de calidad.

\subsection{Creación de la consulta}

Antes de comenzar a buscar recursos debes definir el alcance y los criterios de la revisión. 

En primer lugar debes establecer una pregunta que resuma la intención que tienes con la revisión. El título de tu TFG te dará una pista bastante importante para tal fin. 

El siguiente paso será determinar las palabras que vas a emplear para realizar la consulta, o términos de búsqueda. Puedes usar las de la pregunta anterior o incluso las palabras que aparecen en el título de tu TFG o las palabras clave del mismo. Asegúrate que usas los términos adecuados y que te ayudarán a encontrar material relevante para poder responder a la pregunta de la revisión. Parte de ahí y refina la consulta, añadiendo o quitando términos, hasta que creas que cubre el ámbito que quieres alcanzar. Este proceso lo deberás hacer previamente a la búsqueda en sí e iterativamente después de consultar, analizando los resultados, porque a la luz de los recursos que obtengas puede ser que cambies los términos empleados o añadas algunos más que centren mejor la consulta sobre lo que quieres buscar.

\subsection{Selección de las fuentes y recursos}

Las fuentes concretas dependerán de cada proyecto pero de forma genérica sí podemos concluir que las principales fuentes bibliográficas son bases de datos académicas. Estos repositorios nos permitirán acceder a una gran cantidad de material de forma sencilla y eficiente. 

Por supuesto la web de la biblioteca de tu universidad (por ejemplo, \url{http://biblioteca.ugr.es}, para la de la UGR) es una fuente magnífica en la que apoyarte para hacer la búsqueda y encontrar recursos relevantes. Suelen tener convenios con editoriales que nos permiten acceder a recursos a los cuales de otra forma no podríamos acceder sin tener que pagar). Utilízala, selecciona recursos interesantes y no tengas miedo a irte a la biblioteca de tu centro y consultar físicamente el libro, por ejemplo. Las bibliotecas están llenas de recursos inesperados. No todo lo encontrarás en la Web. Estamos acostumbrados a hacerlo todo digitalmente y, muchas veces, hacerlo de forma presencial nos puede aportar sensaciones diferentes. No hay nada como coger un libro en una sala de la biblioteca y consultarlo con tranquilidad. 

Algunas fuentes interesantes que puedes emplear para comenzar son las siguientes:

\begin{itemize}
    \item \href{www.scopus.com}{Scopus}
    \item \href{https://scholar.google.es}{Google Scholar}
    \item \href{https://www.microsoft.com/en-us/research/project/academic}{Microsoft Academic}
\end{itemize}

En estas bases de datos podrás consultar mediante un formulario, obtener una lista de resultados, normalmente ordenados por relevancia a tu consulta y descargar los recursos, sus metadatos y referencias bibliográficas.

También podrás emplear otras bases de datos bibliográficas específicas según la temática de tu TFG. Por ejemplo, en informática podrás usar la \href{https://dl.acm.org}{biblioteca digital de la ACM}, de la \href{https://ieeexplore.ieee.org}{IEEE}, entre otros, o en temas biomédicos, el conocido \href{https://pubmed.ncbi.nlm.nih.gov/}{PubMed}, por ejemplo. Editoriales como \href{https://link.springer.com}{Springer} o \href{https://www.sciencedirect.com}{Elsevier}, por citar dos de las más importantes, tienen también sus propias bibliotecas digitales para la búsqueda de artículos en sus revistas y congresos.

Otra fuente de información pueden ser las bases de datos de patentes, como la \href{https://worldwide.espacenet.com}{europea} y la \href{https://ppubs.uspto.gov/pubwebapp/static/pages/landing.html}{estadounidense}.

Bases de datos como \href{https://www.educacion.gob.es/teseo}{Teseo}, de tesis doctorales, también pueden ser una fuente interesante para que la consideres según el tipo de TFG que estés haciendo.

Pero no sólo debes tirar de los recursos encontrados en las bases de datos bibliográficas, sino que también debes de tener en cuenta las referencias bibliográficas citadas en un recurso, ya que son una fuente de información muy preciada para obtener información relevante del tema. Por tanto, analiza las referencias de los recursos para ampliar la búsqueda.

Y ahora llega la pregunta del millón: ¿uso la Wikipedia como fuente? Y la respuesta es clara y contundente: no. ¿Por qué? La primera razón es que es está abierta a que cualquier persona sea una autora de sus artículos, sea experta o no de un tema, y por tanto los artículos pueden incorporar información incorrecta o poco precisa. La segunda es que puede ofrecer información no actual sobre una temática. Si quieres usarla, puedes emplearla como punto de partida para tener una idea general sobre tu temática o como punto de partida siguiendo las referencias de los artículos, pero nunca para citarla en tu revisión del estado del arte.

En relación a los recursos que obtienes de estas bases de datos, los principales a considerar son publicaciones académicas: artículos en congresos, revistas, libros, capítulos de libros, memorias de tesis doctorales, de trabajos fin de máster o de grado, informes de proyectos, etc. 

Los recursos deben ser creíbles y confiables. ¿Y cómo detectarlo? Puedes ver si el lugar de publicación (editorial, revista, congreso, sitio web) son reputados en el campo y centrarte seguidamente en estudiar a los autores del artículo y determinar si son expertos en el tema y con una trayectoria avalada. También en si se plasma en ella información objetiva, contrastada y demostrada, o simplemente opiniones personales. Descarta entonces aquellas que sean subjetivas y de personas poco conocidas o con poca experiencia en el tema entre manos.  Un recurso tiende a ser fiable si ha sido publicado en las actas de congresos o en revistas nacionales o internacionales de prestigio. Se supone que el propio proceso de evaluación por pares de las publicaciones es una garantía de que el artículo cumple unos estándares mínimos de calidad. Aún así tendrás que filtrar recursos porque consideres que no son de calidad (por ejemplo, porque no aportan detalles para reproducir los experimentos, o el análisis de resultados es poco somero, las conclusiones muy vagas, etc.). Ten cuidado con las revistas o congresos que publican todo lo que les cae en las manos y no tienen procesos de calidad que aseguren que lo publicado tiene unos mínimos asegurados de rigor.

También se pueden considerar publicaciones en la web. En ese caso deberás tener en cuenta si el autor está cualificado para hablar sobre el tema y si está publicado bajo un dominio de una institución reconocida (universidad, institución, organización, etc.). También tenemos los casos de artículos publicados en sitios como \href{www.medium.com}{Medium} o \href{www.researchgate.net}{ResearchGate}, entre otros muchos. De nuevo debes comprobar la reputación del autor o si se apoya en referencias bibliográficas, por ejemplo. Lo que está claro que no debes citar ninguna respuesta de sitios como StackOverflow o Reddit porque son opiniones y respuestas a preguntas, que aunque pueden ser correctas muchas, sólo deben servir para ayudarte a solucionar algún inconveniente surgido (una cosa es que introduzcas una referencia a la solución de un problema que has tenido en el proceso de desarrollo de la aplicación de tu TFG y que has tomado de alguna entrada de estos sitios y otra es que la emplees como recurso para el estado del arte).

Hay que tener especial cuidado con las publicaciones web que vienen de empresas o compañías pues pueden estar sesgadas por temas comerciales y publicitarios. Aún así, no es lo mismo una descripción de un software en el sitio web de la compañía que lo desarrolla, que en alguna otra página de alguien que no conoces, por ejemplo, y que simplemente está dando opiniones sin fundamento. Ten cuidado con estas cosas.  

Otro aspecto importante es el temporal, ya que en ocasiones querrás seleccionar aquellos recursos que sean actuales y en otras te sirvan todos los recursos que encuentres, independientemente del momento en que hayan sido escritos. 

Los criterios de inclusión/exclusión se suelen aplicar en esta fase. Son condiciones que deben cumplir los recursos para poder considerarlos en las fases siguientes o descartarlos, respectivamente. Suelen estar compuestos por periodos temporales, idiomas, restricciones en cuanto a metodologías usadas o tipos de publicaciones. Todos los recursos que no los cumplan no serán tenidos en cuenta para su análisis y los que sí pasarán a la siguiente fase de análisis.

\subsection{Evaluación de la calidad de la información, análisis y síntesis}

En la sección anterior has visto que existen unas fuentes más fiables que otras. Un artículo reciente revisado por pares en una revista de calidad puede que sea más importante más que un post de Reddit de hace 10 años (ojo, no caigamos en falacias de autoridad tampoco). Tampoco te fíes de qué te dice ChatGPT.

Para hacer un buen TFG debes convertirte en la persona más experta en el tema que vas a desarrollar. O por lo menos, acercarse.

Lo primero que recomendamos es que puedas separar lo importante de lo que no lo es. ¿Y qué es lo importante?  Pues resulta que no lo sabrás hasta que no hayas leído lo suficiente y empieces a ser consciente de ello. Es decir, vas a empezar a trabajar sin saber en qué centrarte. En este momento comenzarás a cribar recursos, eliminando los que consideres que no son interesantes para tu trabajo y seleccionando los relevantes para su estudio en profundidad.

Empieza a tomar notas de todo lo que leas y te parezca interesante. Pero cuidado, no te interesa tener un montón de notas enormes una detrás de otra con un montón de datos vomitados, porque quizás la mitad no sirva de nada. Pero tampoco algo muy críptico, porque cuando la cabeza te haga ``clic'' y descubras qué es Lo Importante\texttrademark  te va a tocar releer cosas que ya pensabas que habías revisado. 

Quizás lo suyo es un término medio. Un truco que nos funciona muy bien es abrir una hoja de cálculo en tu suite favorita (LibreOffice, MS Office, Notion, Obsidian...) y crear las siguientes columnas (para empezar):
\begin{itemize}
    \item Año,
    \item título de la referencia,
    \item autores,
    \item enlace,
    \item justificación para el proyecto.
\end{itemize}

Las primeras 4 columnas son autodescriptivas y objetivas, datos de la fuente, pero la última es lo que la referencia aporta a tu estado del arte. Intenta resumirlo a una o dos frases como mucho.

Mientras vayas leyendo a lo mejor descubres que tienes que crear una nueva columna. Por ejemplo, puede interesarte una columna ``Algoritmo'', ``Lenguaje usado'', ``Framework'', ``Sistema operativo compatible'' o ``Licencia''. A lo mejor, si tu trabajo es de investigación en redes neuronales, la quinta referencia que estás leyendo de repente te inspira para crear ``Librería usada'', ``Número de capas'' y ``Dataset utilizado''. Vaya, ahora vas a tener que releer las cuatro referencias anteriores que ya habías anotado antes de crear la columna y completar sus huecos. Pero no pasa nada, eso es que te estás haciendo una persona más experta en el tema ahora que cuando leías esas cuatro primeras referencias. Y no solo eso, sino que también conforme vayas rellenando esa tabla podrás agrupar las referencias según algunos criterios relevantes para tí: metodologías, algoritmos, problemas, soluciones, etc.

Cuando termines tendrás una visión general muy fácil de comprender de un sólo vistazo. Quizás veas que la gran mayoría de software parecido al tuyo solo es compatible con Windows. O que casi todo el mundo utiliza el algoritmo de explicabilidad de Shapley. O que la mayoría de proyectos como el tuyo se han desarrollado en los últimos dos años. A partir de toda esta información podrás extraer conocimiento que podrás plasmar en la memoria de una manera más fácil: tu proyecto es el primero en ser multiplataforma, tu proyecto es el primero en ofrecer esta funcionalidad, tu proyecto es el primero en aplicar este dataset, tu proyecto es el primero en aplicar esta metodología... Tu proyecto es el primero en algo.

De hecho, es muy buena idea poner una tabla-resumen en tu TFG. Permitirá a la persona que lea tu capítulo tener también una visión muy rápida de cómo está el asunto.

Actualmente con el desarrollo que están teniendo los LLMs, existen herramientas como \href{https://notebooklm.google}{NoteBookLM}, de Google, que te pueden ayudar a realizar un estudio comparativo de varios recursos que subas a esta aplicación. Utiliza este recurso si lo ves conveniente, pero no olvides revisar la salida y entender bien lo que te dicen.

\section{Cómo plasmarlo en la memoria}

Una vez que haz hecho la búsqueda, la selección de material relevante, su lectura y análisis, queda plasmarlo en la memora. Esta sección de la revisión del estado del arte podría tener tres partes claramente diferenciadas:

\begin{itemize}
    \item Introducción. Tras dar una breve descripción del contexto entre manos o del problema para el cual vas a hacer la revisión, debes seguir con la exposición de los objetivos de la misma.  Tienes que dejar claro por qué es necesario hacer una revisión y qué pretendes conseguir con ella. En esta sección también tienes que explicar la metodología seguida: cuál es tu pregunta, cómo has generado la consulta, qué fuentes bibliográficas has consultado, qué criterios son los de aceptación y de exclusión y por qué y cualquier otra cosa que consideres relevante para describir el proceso de búsqueda y confección de la revisión.

    \item La revisión propiamente dicha. Esta parte es donde debes plasmar la revisión que has realizado. Podrías hacerlo mediante estas estrategias:

    \begin{itemize}
        \item Cronológicamente: simplemente ir mostrando la evolución en la temática a lo largo del tiempo. Pero esto no es simplemente un listado, sino que tienes que indicar qué se va aportando en cada fuente analizada, es decir, cuál es el avance.

        \item Temáticamente: como habrás identificado ciertas temáticas, conceptos, enfoques o áreas, puedes organizar la revisión entorno a esos campos, quizá en forma de secciones, y dentro de cada una la presentación de los recursos y su discusión.

        \item Metodológicamente: también puedes haber distinguido diferentes metodologías o aportaciones teóricas. En ese caso, al igual que el punto anterior, puedes estructurar tu análisis agrupando los recursos según las que apliquen y analizando la forma de aplicarlas y los resultados obtenidos.
    \end{itemize}
    
    También puedes combinar estas estrategias. Por ejemplo, puedes hacer una organización temática de primer nivel y seguidamente dentro de cada sección hacer el análisis cronológico.

    Es útil, para resumir y ofrecer información relevante de un vistazo, que incluyas una tabla con las referencias en las filas y en las columnas características de interés para tu estudio y los diferentes valores que aporta cada recurso analizado, al estilo de lo explicado en el apartado anterior.

    De cualquier forma, recuerda que debes ofrecer los elementos principales de cada recurso, interpretarlos y discutir sus aportaciones de forma individual, pero también de forma global. Y todo con tus palabras. Realiza una labor crítica estableciendo ventajas, inconvenientes, puntos fuertes o débiles y situaciones de mejora. También es muy importante que como resultado de este análisis encuentres lagunas de conocimiento, que pueden manifestarse como áreas poco estudiadas o aspectos que han recibido poca atención en la investigación previa, preguntas sin respuesta o temas que requieren una exploración más profunda. Identificar y señalar estos vacíos en la literatura es importante porque puede orientar investigaciones futuras y proporcionar oportunidades para contribuir de manera significativa al campo de estudio.

    Todos los recursos deben estar citados convenientemente y sus detalles bibliográficos deben aparecer en la sección de bibliografía, tal y como se indica en el Capítulo \ref{cap:bibliografia}.

    \item Conclusión. Es la sección donde debes resumir los hallazgos de ese proceso de revisión crítica de los recursos que has encontrado y analizado. Debes dar importancia a dichos hallazgos y sobre todo enfatizar las lagunas que has encontrado y cómo tu propuesta de proyecto se enmarca en alguna de ellas y cómo puede ayudar a mejorar la situación actual de conocimiento.
\end{itemize}

\section{Recomendaciones generales}

Seguidamente te damos algunas recomendaciones generales para realizar la revisión del estado del arte y confeccionar la sección correspondiente en la memoria de tu TFG:

\begin{itemize}
    \item Haz la revisión al principio. No lo dejes para el final pues puedes llevarte sorpresas de que lo que tú hayas hecho en tu proyecto ya esté hecho.
    \item Dale importancia a la escritura de la revisión pues es una magnífica carta de presentación tuya ya que estás mostrando tu capacidad de comprensión, análisis y síntesis.
    \item Como ya hemos dicho, no copies texto. Eso es plagio. Entiéndelo y plásmalo con tus palabras.
    \item Un argumento es como una katana, no puedes sacarla sin hacer sangre. Así que cualquier argumento que escribas debería ir citado. Y no solo deberías añadir citas en el capítulo del estado del arte, sino a lo largo de toda la memoria.
    \item No copies texto literalmente de un recurso. Eso es plagio. Entiéndelo y redáctalo con tus propias palabras.
    \item Analiza la documentación de que dispones y plasma ese análisis. No pongas simples resúmenes.
    \item Compara enfoques y aproximaciones, establece ventajas e inconvenientes, puntos fuertes y débiles.
    \item Habla con la persona que te tutoriza sobre la orientación del estudio. 
    \item Discute el estudio la persona que te supervisa conforme lo vas realizando.
    \item Busca lagunas y limitaciones de lo que hay hecho. Ahí es donde podrás enmarcar y construir tu propuesta.
    \item Dedícale tiempo a buscar y a leer. Hazlo con tranquilidad y con tiempo suficiente.  
    \item Incluye figuras que ilustren la metodogía que has seguido en el estudio.
    \item Incluye figuras y gráficos que, de forma visual, describan en análisis que has hecho.
    \item Incluye tablas comparativas.
    \item Ve al grano. No es necesario que escribas la biblia en verso en este capítulo. Recuerda que lo bueno si breve dos veces bueno.
\end{itemize}
