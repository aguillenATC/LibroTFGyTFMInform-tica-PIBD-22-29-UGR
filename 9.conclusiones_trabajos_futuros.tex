\chapter{Las conclusiones y los trabajos futuros}
\label{cap:Conclusiones}

% [Autores: María José Rodríguez Fórtiz]
El capítulo de \textit{Conclusiones y trabajos futuros} es muy importante pues recoge qué se ha realizado en el TFG, los principales resultados y qué puede hacerse a partir de este momento. Muchas personas leen la introducción y luego las conclusiones antes de leerse el resto de la memoria, para así conocer bien la motivación, objetivos y los resultados del trabajo. Eso significa que debes tener especial cuidado al redactar este capítulo para que quede muy claro y sea muy completo.

Habitualmente incluye dos secciones, la de conclusiones y la de trabajos futuros.
 
 \section{Conclusiones}

Debes empezar las conclusiones con una frase inicial a modo de resumen sobre el objetivo general y el problema que se ha abordado indicando una valoración positiva sobre los resultados de tu trabajo (en caso de que todo haya ido bien). Si por algún motivo, no se ha satisfecho el objetivo general, lo puedes indicar pero justificando el porqué, y añadiendo que a pesar de ello, se han cumplido algunos de los objetivos específicos, obteniendo resultados favorables. 

A continuación debes hacer un repaso uno a uno de los objetivos específicos, indicando: (1) el porcentaje de realización, (2) un resumen de lo que se ha hecho para cumplir ese objetivo (dos o tres líneas explicando las tareas realizadas asociadas a ese objetivo, y los resultados obtenidos deben bastar), y (3) una indicación de dónde pueden verse las evidencias de ese objetivo en la memoria, en qué capítulo o sección.

También puedes mencionar en esta parte, cómo tu formación previa en materias concretas del grado te ha sido de ayuda para el TFG y los retos nuevos que has tenido que afrontar para resolver cuestiones que no hubieras visto antes durante tu formación.

 En cuanto a la redacción de este repaso de objetivos, te ponemos un ejemplo. Suponiendo que estás abordando un objetivo específico que has redactado como ``Revisar aplicaciones similares para comparar con la propuesta'', puedes indicar que ese objetivo se ha cumplido completamente, explicando por ejemplo que has revisado 6 aplicaciones similares y que has realizado una tabla comparando 8 características básicas de cada una, la cual puede consultarse en el capítulo o sección X de la memoria. También puedes añadir que al elaborar esta tabla se demuestran tus capacidades de análisis y síntesis de información. Si este objetivo no se hubiera cumplido completamente, porque, por ejemplo solo hayas revisado 2 aplicaciones y tenías previsto revisar más, pues dices lo que sí has hecho pero solo un 30\%, y argumentas porqué es insuficiente, por ejemplo porque solo hay 2 de libre acceso que has podido consultar con profundidad, o porque has priorizado terminar la tarea X, que habéis considerado que era más importante para el TFG. 

 %De cara a la redacción de esta sección puedes tener en cuenta el registro de marcas propuesto en \cite{meza2019comunicacion} \textcolor{orange}{, que sugiere verbos que puedes utilizar (en este caso para explicar en las conclusiones cuál ha sido el alcance de cada objetivo), como son: "se ha abordado", "hemos hecho un recorrido por", "podemos afirmar que", "esto evidencia que", "confirmamos que", "confirma nuestras hipótesis/ideas", "hemos propuesto/obtenido/identificado/revisado/observado/descubierto/utilizado/demostrado/explicado/desarrollado ...",  "no hemos podido demostrar/confirmar/revisar/identificar ... porque ...", etc. Como marcadores discursivos, podemos usar conectores como los siguientes: "por tanto", "sin embargo", "en consecuencia", "por el contrario", "a pesar de", "gracias a", "entendemos que", o "de acuerdo/según todo lo anterior".}
 
Es importante que en las conclusiones añadas un párrafo final como valoración personal, escrito esta vez en primera persona. En esa valoración debes mencionar cómo te has sentido al realizar el TFG y en base a sus resultados. Puedes indicar que te sientes orgulloso/a, contento/a, satisfecho/a, encantado/a, etc. por lo que has aprendido, por cómo te has organizado en el tiempo, por cómo has redactado la memoria, por la calidad del código desarrollado, por cómo te has comunicado con tu tutor, etc. Si tienes alguna valoración negativa, debes mencionarla también, pero te recomendamos que la redactes de forma positiva, aportando que has aprendido de ello. Por ejemplo, "No estoy satisfecho/a con cómo he organizado el trabajo temporalmente porque he dejado muchas tareas para el último mes y eso me ha saturado, con lo cual he aprendido que en un futuro debo hacer una mejor planificación temporal desde el principio.". En tu valoración personal, y si no lo has hecho al revisar los objetivos, también puedes mencionar cómo has aplicado y mejorado tus habilidades blandas o \textit{soft skills}, como son organización de trabajo, pensamiento crítico, creatividad, adaptación, resolución de problemas y comunicación. En esta valoración personal también es un buen momento para recapitular todo lo que has aprendido y qué competencias de las que se mencionan en el plan de estudios has desarrollado con el TFG.

 \section{Trabajo futuro}

 En esta sección se enumeran:
 \begin{itemize}
     \item Tareas que que estaban previstas y no se han hecho o han quedado incompletas, de las mencionadas en las conclusiones.
     \item Requisitos de desarrollo que tenías previsto abordar pero que al final no has tratado.
     \item Nuevos requisitos que hayan surgido durante el desarrollo del TFG, que no se habían previsto y por tanto no se han planificado ni abordado.
     \item Nuevos objetivos e ideas para dar continuidad al TFG en futuros TFGs, desarrollos o investigaciones.
 \end{itemize} 

 Para cada una de estas tareas, requisitos u objetivos conviene añadir un pequeño párrafo que explique porqué se propone y cómo se abordaría, de forma muy resumida. Por ejemplo: ``En un futuro se puede desarrollar una versión en iOS del prototipo realizado en el TFG. Esto ayudaría a que más personas pudieran utilizar la aplicación. Para ello, se podría utilizar un \textit{framework} de desarrollo como Flutter o Ionic, que permiten esta portabilidad y el desarrollo híbrido de aplicaciones móviles. Habría que valorar si el código actual o parte de éste puede reutilizarse''. Otro ejemplo de párrafo: ``Sería necesario completar la gestión de usuarios en la aplicación desarrollada, ya que por el momento solo pueden hacerse altas y modificaciones. Bastaría para ello diseñar e incluir funciones e interfaces para el borrado de usuarios de la misma forma que se ha hecho para las otras operaciones. Esto no supondría ningún cambio en la base de datos''. 

 En el caso de ser una tarea del primer tipo, justifica bien la razón por la que no se ha podido realizar íntegra o parcialmente. 

 

 