\chapter{La revisión del proyecto y la memoria}
\label{cap:Revisión}

% Comentarios:
% ¿Meter una lista de ítems a comprobar?
% ¿Aconsejar que se la confeccione el estudiante?
% Esto lo tendremos que ordenar por grupos.

\begin{itemize}
  \item Sobre la estructura o formato de la memoria:

  \begin{todolist}
    \item Seguir la normativa de portadas (logos, colores, etc.) y prólogos (autorización, resumen, etc.).
    \item Los índices de contenidos, tablas y figuras deben estar actualizados, generándolos automáticamente.
    \item No quedan tablas cortadas al final de una página.
    \item Si hay títulos de secciones al final de una página, debajo de ellos debe haber al menos un párrafo.
    \item Utiliza los mismos estilos y tipo de letra en toda la memoria, a no ser que quieras resaltar algo como citas literales, fórmulas, ecuaciones o código.
    \item Las páginas en blanco que dejes deben ser intencionales, por ejemplo para que cada capítulo empiece en página impar.
  \end{todolist}
  
  \item Sobre la bibliografía:

  \begin{todolist}
    \item Todas las referencias de la bibliografía deberían aparecer citadas en el texto.
    \item Todas las citas bibliográficas del texto deben tener una referencia asociada.
    \item Las referencias de recursos de Internet deben indicar también la fecha de la última consulta.
  \end{todolist}

  \item Sobre las figuras y tablas:

  \begin{todolist}
    \item Todas las figuras y tablas deben estar numeradas secuencialmente y referenciadas/citadas en el texto.
    \item Todas las figuras y tablas deben tener un título descriptivo.
    \item Los títulos de figuras y tablas deben estar en la misma página de la figura o tabla.
    \item Si se usan figuras hechas por otra persona deberían ser citadas/usadas correctamente, por ejemplo en el pie de figura "Extraído de [x]" o "Fuente: [x].
    \item Las imágenes y las tablas están dentro del espacio del cuerpo de la página y ninguna se desborda hacia los márgenes. 
  \end{todolist}

  \item Por último, no olvides:

  \begin{todolist}
    \item Revisa la ortografía. 
  \end{todolist}

\end{itemize}