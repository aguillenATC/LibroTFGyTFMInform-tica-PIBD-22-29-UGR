\documentclass{book}
\usepackage[a4paper]{geometry}
\usepackage{xcolor}
\usepackage{color}
\usepackage[main=spanish]{babel}
\usepackage[utf8]{inputenc}
\usepackage[T1]{fontenc}
\usepackage{graphicx}
\usepackage{hyperref}
\usepackage{booktabs}
\usepackage{varwidth}
\usepackage{caption}
\usepackage{multirow}
\usepackage{textcomp}
\usepackage{url}
\usepackage{amsmath}
\usepackage[inline]{enumitem}
\usepackage{todonotes}
\usepackage{enumitem,amssymb}
\newlist{todolist}{itemize}{2}
\setlist[todolist]{label=$\square$}
\usepackage{color}
\usepackage{rotating}
\usepackage{charter} % Palatino como fuente principal
\usepackage{emptypage} % Evita encabezados y pies de página en páginas vacías
\usepackage{microtype}
\usepackage{pdfpages}
\usepackage[fixlanguage]{babelbib}

\hypersetup{
    pdftitle={Cómo escribir la memoria de tu TFG del Grado en Ingeniería Informática y presentarlo sin morir en el intento}
    pdflang={es-ES}
}
%\usepackage[tagged,highstructure]{accessibility}


% Cambiar "Cuadro" por "Tabla"
\addto\captionsspanish{\renewcommand{\tablename}{Tabla}}

\newcommand{\emc}[1]{\textcolor{blue}{#1}}
\newenvironment{naranja}{\par\color{orange}}{\par}

\title{Cómo escribir la memoria de tu TFG del \\ Grado en Ingeniería Informática\\ y presentarlo sin morir en el intento}
\author{
    \makebox[\textwidth][c]{%
        \begin{minipage}{\textwidth}
            \centering
            Alberto Guillén Perales, Eugenio Martínez Cámara, \\
            Juan Manuel Fernández Luna, Pablo García Sánchez, \\
            Manuel Noguera García, María José Rodríguez Fórtiz, \\
            Rocío Romero Zaliz \\[1ex]
           \vspace{2ex}
            Escuela Técnica Superior de Ingenierías\\ Informática y de Telecomunicación\\
            Universidad de Granada\\ 
           \vspace{5ex}
            %Depósito legal: XXXXXXX
        \end{minipage}%
    }
}
\date{\today}

\begin{document}

% Página de portada con imagen
\clearpage

\thispagestyle{empty} % Sin numeración
\includepdf[pages={1},width=1.45\textwidth]{images/portada.pdf}
%\maketitle

\thispagestyle{empty} % Sin numeración

\textit{Cómo escribir la memoria de tu TFG del Grado en Ingeniería Informática y presentarlo sin morir en el intento} \textcopyright 2025 por Alberto Guillén Perales, Eugenio Martínez Cámara, Juan Manuel Fernández Luna, Pablo García Sánchez,  Manuel Noguera García, María José Rodríguez Fórtiz,  Rocío Romero Zaliz está licenciado bajo CC BY-NC-SA 4.0. Para ver una copia de esta licencia visite \url{https://creativecommons.org/licenses/by-nc-sa/4.0/}

\begin{figure}[h!]
    \centering
    \includegraphics[width=2cm]{images/by-nc-sa.png}
\end{figure}

\vspace{5ex}
Depósito legal: GR 242-2025

%ISBN: XXXXXXXXXX


\chapter*{Agradecimientos}
%\addcontentsline{toc}{chapter}{Agradecimientos}
Queremos expresar nuestro más profundo agradecimiento a los estudiantes egresados del Doble Grado de Ingeniería Informática y Matemáticas, Javier Granados López, Mario García Márquez, al egresado del Doble Grado de Ingeniería Informática y ADE, Néstor Martínez Sáez, y a Álvaro Jesús Baena Rosino, alumno en este curso 2024/2025 de este doble grado, y también a los compañeros de la ETSIIT, Javier Medina Quero y José Manuel Soto Hidalgo, que aceptaron nuestra propuesta de revisión del manuscrito. Sus certeros comentarios y correcciones sin duda han mejorado la calidad de este libro. Muchas gracias por vuestra colaboración. También queremos agradecer a nuestros compañeros y compañeras de la ETS. de Ingenierías Informática y de Telecomunicación por todas esas fructíferas reuniones, tribunales compartidos y comisiones, pero sobre todo por esos cafés y charlas informales en pasillos que nos han hecho mejorar en nuestra labor docente como tutores y tutoras de TFG. Y, por último, pero no menos importante, damos las gracias a Joaquín Fernández Valdivia por aceptar nuestra invitación y escribir el prólogo de este libro.

Y no podemos olvidar a quienes hemos tenido el privilegio de tutorizar en estos años. Esperamos que hayáis aprendido una fracción de todo lo que hemos aprendido gracias a vuestro trabajo, esfuerzo y dedicación. Esta es nuestra forma de agradecerlo.

Este manual ha sido escrito dentro del marco del Proyecto de Innovación y Buenas Prácticas Docentes Avanzados/Coordinados 2022-2024 con código 22-29 y titulado ``Cómo escribir tu TFG o TFM de ingeniería informática y no morir en el intento: dificultades, retos y elaboración de materiales docentes'' y financiado por el Plan FIDO UGR, Plan de Formación e Innovación Docente 2022-2023, Programa de Innovación y Buenas Prácticas Docentes, Unidad de Calidad, Innovación Docente y Prospectiva de la Universidad de Granada. 

% \chapter*{Prólogo}

% Todavía no está terminado. Terminarlo al final de la revisión completa del libro.

% % [Autores: lo podemos hacer al final todos]
% \textit{Aviso 1}: en este libro se refleja la  “opinión” de un conjunto de profesores a partir de su experiencia y conocimientos académicos, así como su experiencia, tanto como tutores como evaluadores. Es un conjunto de recomendaciones, sugerencias y buenas prácticas que en ningún momento garantizan que, tras seguirlas, se obtengan buenos resultados y además puede haber efectos secundarios ;-) 

% \textit{Aviso 2}: al igual que las distribuciones normales tienen una media representativa, las colas de la distribución siguen siendo valores correctos de la distribución. En este documento hemos intentado comentar la media con una amplia desviación, no obstante, la tupla formada por (TFG, alumno, profesor, tema) puede caer en los extremos de la distribución y seguir siendo normal. En otras palabras, si consideras que tu TFG no se acomoda a lo aquí presentado, no te preocupes, y si sabes responder el porqué, seguro que estará en el extremo de los excelentes. 

% \textit{Aviso 3}: este libro ha intentado realizar un uso no sexista del lenguaje, haciendo uso indistinto de  tutor/tutora, profesor/profesora y alumno/alumna, empleando sustantivos genéricos como estudiante, estudiantado, alumnado o profesorado, pronombres sin marcas de género (quienes), entre otras alternativas al lenguaje no sexista. Si algo se nos ha pasado, pedimos disculpas.



\tableofcontents
\listoftables
\listoffigures

% Incluye los archivos de los capítulos
%\include{1.prologo}
\chapter{Introducción}
\label{cap:Introducción}
%Juanma
%Iniciativas previas: \url{https://drive.google.com/drive/folders/1WMPwX1v7IhG88bJYLXDgm4Zp6a_pCCHh}

Estás cursando el cuarto curso del Grado en Ingeniería Informática o a punto de hacerlo. Si estás leyendo esta introducción, es porque el desarrollo del Trabajo Fin de Grado (TFG) es algo que te ronda la cabeza o ya lo tienes entre manos.

Estás terminando tus estudios de grado y aparece esta asignatura obligatoria de doce créditos en el segundo semestre de cuarto, que constituye un desafío fundamental y monumental en tu proceso formativo como estudiante de Ingeniería Informática. En ella, pondrás en práctica muchos de los conocimientos adquiridos en el grado y otros muchos que obtendrás por cuenta propia; te enfrentarás a retos técnicos y/o metodológicos que no imaginarías. Y lo mejor es que, aunque todavía no lo sabes y casi ni te lo imaginas, porque ahora estás cag... eh, perdón, temeroso/a, vas a salir airoso/a de ellos proponiendo soluciones creativas y mostrando tus habilidades y capacidades. En definitiva, es un paso más en tu formación como ingeniero/a y debes aprovecharlo para convertirte en un mejor profesional.

Normalmente, en este tiempo aplicarás metodologías de desarrollo y escribirás mucho código. Como informáticos, nos gusta esta última tarea más que cualquier otra cosa, especialmente, e infinitamente, más que escribir informes y memorias. Disfrutamos programando y nuestro humor cambia radicalmente cuando se nos dice que hay que explicar lo que hemos hecho por escrito. ¿Por qué? Claramente porque nos suele costar trabajo: bien porque no nos gusta esta tarea, bien porque no sabemos cómo hacerla. Y normalmente no le damos la importancia que se merece. Piensa que la memoria de un TFG no es sólo un requisito administrativo para conseguir tu título de graduado, sino que es el medio principal que emplearás para comunicar y justificar el (buen) trabajo que has llevado a cabo en el tiempo que has dedicado al TFG. Es, de alguna forma, como una especie de carta de presentación tuya ante un tribunal evaluador que describe todo lo que has hecho y por qué lo has hecho. También es una carta de presentación para potenciales empleadores, ya que podrán conocer tus conocimientos y habilidades de forma práctica. Piensa también que lo habitual en cualquier trabajo es describir tus tareas para que los compañeros y compañeras entiendan lo que has hecho y para que tú te acuerdes unos meses más tarde. Por tanto, la confección de la memoria del TFG no es más que un ejercicio de entrenamiento en este sentido.

Este libro tiene como objetivo acompañarte en el desarrollo de tu TFG, de forma general, pero muy especialmente en una de las actividades más importantes, y precisamente en la que más dudas y miedos aparecen: la escritura de la memoria de tu proyecto. En este manual nos enfocamos en guiarte a través del desafiante pero gratificante proceso de elaborar una memoria de TFG en Informática. Para ello, te proporcionamos herramientas prácticas, consejos y ejemplos concretos que abarcan desde la elección de la temática hasta la defensa, pasando por todo el proceso de escritura de la memoria. Pero también somos conscientes de que cada (estudiante, tutor/a, proyecto) es único, por lo que este libro no pretende ser una guía rígida, sino un recurso flexible y totalmente adaptable por ti y la persona que te tutoriza, que facilite la comunicación de lo que has conseguido en tu TFG.

¿Y por qué este libro? Los autores somos profesores de informática y llevamos mucho tiempo tutorizando a estudiantes y evaluando TFG. Hemos visto que la tarea que más os cuesta en todo el desarrollo del TFG es la escritura de la memoria. Si bien durante la carrera habéis ejecutado numerosos proyectos en las prácticas de las asignaturas y redactado informes técnicos de los mismos, partimos del hecho de que no se os ha formado en redacción, ni general ni técnica, en vuestro paso por la Universidad y que no tenéis la experiencia necesaria para escribir una memoria de este calibre porque nunca habéis documentado un proyecto de esta envergadura. Así las cosas, cuántos casos existen de estudiantes que hacen un buen TFG pero que se bloquean con la memoria, que escriben un informe final muy escueto o hecho con prisas por dejarlo para el final, emborronado, sin una organización adecuada, escrito con desgana, con problemas de redacción (gramaticales y ortográficos), sin saber cómo presentar las ideas o resultados, sin apoyarlo en referencias bibliográficas, etc. Podréis tener el mejor proyecto del mundo, pero si lo acompañáis de una mala memoria, el fracaso está garantizado. Y eso es lo que queremos evitar. O dicho de otro modo: una memoria bien estructurada y redactada puede marcar la diferencia entre un TFG notable y uno mediocre, independientemente de la calidad técnica del proyecto desarrollado.

Este documento es el resultado de un proyecto de innovación docente (PIBD 22-29 -- ``Cómo escribir tu TFG o TFM de Ingeniería Informática y no morir en el intento: dificultades, retos y elaboración de materiales docentes'') de la Universidad de Granada, donde los participantes nos propusimos crear un recurso flexible para que tengáis apoyo en esta fase de escritura del informe final del TFG y procurar evitar caer en todos los problemas citados anteriormente. Con él queremos animarte a ver la redacción de la memoria como una parte integral de tu proyecto, tan importante como el análisis, el diseño o la implementación, y que tengas herramientas para conseguir un producto de calidad. Pero el objetivo no es sólo ofrecer un material de apoyo finalista, sino que con los consejos que te damos, su lectura y práctica te permitan afrontar cualquier otra redacción técnica con mucha mayor claridad y soltura, ya sea de un hipotético Trabajo Fin de Máster (TFM)\footnote{Podríamos decir que todo lo indicado en este libro sobre la redacción de la memoria sería extensible a la escritura del informe final de un TFM.} o de encargos profesionales.

También queremos ofrecer este recurso a la comunidad de docentes que tutorizan TFG. Muchas veces somos nosotros, los docentes, los que también necesitamos algo de orientación para poder, a su vez, orientar a los estudiantes. Y disponer de recursos de referencia nos resulta muy útil en nuestra labor de enseñanza. Por tanto, este libro pretende mostrar a nuestras compañeras y compañeros cómo los autores pensamos que podría ser una memoria de TFG y el proceso para su confección. Cada cual que tome lo que vea interesante para el ejercicio de su tutoría.

Llegado este punto de la introducción, donde hemos expresado nuestra motivación e intenciones, creemos que es el momento de hacer dos descargos de responsabilidad:

\begin{itemize}
    \item {\textit Disclaimer} I: este libro refleja la ``opinión'' consensuada de un conjunto de profesores y profesoras de Universidad a partir de su experiencia y conocimientos académicos, así como su experiencia en el rol de tutores y de evaluadores de TFG. Es un conjunto de recomendaciones, sugerencias y buenas prácticas que en ningún momento garantizan que, tras seguirlas, se obtengan buenos resultados y además puede haber efectos secundarios ;-) 

    \item {\textit Disclaimer} II: al igual que las distribuciones normales matemáticas tienen una media representativa, las colas de la distribución siguen siendo valores correctos de la distribución. En este documento hemos intentado comentar la media con una amplia desviación, no obstante, la tupla formada por (TFG, alumno, profesor, tema) puede caer en los extremos de la distribución y seguir siendo normal. En otras palabras, si consideras que tu TFG no se acomoda a lo aquí presentado, no te preocupes, y si sabes responder el porqué, seguro que estará en el extremo de los excelentes. 
    
\end{itemize}

Conocedores de cómo el estudiantado va evolucionando en cuanto a la forma en que realiza su aprendizaje, y con objeto de facilitárselo, hemos grabado varios vídeos cortos de cada capítulo con el objetivo de ofrecer las ideas principales de los mismos en un formato complementario y resumido al libro. Los vídeos están alojados en el canal de YouTube \url{https://www.youtube.com/@ComoescribirtuTFGenInformatica}.

Pretendemos que este texto sea un documento vivo, en el que todos los miembros de la comunidad académica del Grado en Ingeniería Informática puedan participar de forma colaborativa en la mejora del mismo. Por ello, hemos subido a GitHub los fuentes del libro y te invitamos a que hagas aportaciones al texto, detectando erratas, añadiendo cosas que se nos han pasado y que consideras importantes, proponiendo nuevos temas sobre los que hablar, etc. También puedes crear ``issues'' con sugerencias o comentarios. La URL de GitHub es \url{https://github.com/aguillenATC/LibroTFGyTFMInform-tica-PIBD-22-29-UGR}. Y, por supuesto, también puedes pasar a dar las gracias si consideras que te hemos ayudado. No te lo pedimos por que sí sino para saber si hemos conseguido nuestro propósito de ayudarte a escribir el TFG sin morir en el intento.

Este libro ha intentado realizar un uso no sexista del lenguaje, haciendo uso indistinto de tutor/tutora, profesor/profesora y alumno/alumna, empleando sustantivos genéricos como estudiante, estudiantado, alumnado o profesorado, docente, pronombres sin marcas de género (quienes), entre otras alternativas al lenguaje no sexista. Si algo se nos ha pasado, pedimos disculpas.

Y por último, vamos a describirte cómo hemos organizado el libro, su estructura y contenidos, para que conozcas los diferentes temas que vamos a tratar en este texto:

\begin{itemize}
\item Todo lo que siempre quisiste saber sobre el TFG y que nunca te atreviste a preguntar. $\rightarrow$ Capítulo \ref{cap:Recomendaciones}. Recomendaciones generales.
\item Sobre la construcción de un castillo de naipes sin que se te derrumbe. $\rightarrow$ Capítulo \ref{cap:EstructuraMemoria}. La estructura general de la memoria.
\item Ese agujero negro donde un estudiante pasa tres días delante de una página en blanco para justificar su proyecto. $\rightarrow$ Capítulo \ref{cap:IntroducciónTFG}. La introducción del TFG.
\item Donde descubres que tu brillante idea ya se hizo en los años 80. $\rightarrow$ Capítulo \ref{cap:RevisionEstadoDelArte}. Revisión del estado del arte.
\item Cómo no demostrar mis dotes de ciencia ficción (I) con un diagrama de Gantt que parece sacado de un universo paralelo con días de 48 horas y presupuestos donde mi sueldo como 'ingeniero junior' es tan optimista que hasta mi madre se ha reído al verlo. $\rightarrow$ Capítulo \ref{cap:PlanificacionPresupuesto}. La planificación y el presupuesto.
\item Cómo no demostrar mis dotes de ciencia ficción (II) resumiendo todo como si hubiera sido planeado y proponiendo trabajos futuros que, sinceramente, esperas que haga otro. $\rightarrow$ Capítulo \ref{cap:Conclusiones}. Las conclusiones y los trabajos futuros.
\item Cómo organizar un cajón con un millón de calcetines desparejados. $\rightarrow$ Capítulo \ref{cap:bibliografia}. La bibliografía.
\item Cómo intentar darle los últimos retoques a una obra de arte mientras el reloj corre y el público espera ansioso su presentación. $\rightarrow$ Capítulo \ref{cap:Revisión}. La revisión del proyecto y la memoria.
\item Cómo teniendo una buena moto ser capaz de venderla (I). $\rightarrow$ Capítulo \ref{cap:elaboraciónPresentación}. La elaboración de la presentación de la defensa.
\item Cómo teniendo una buena moto ser capaz de venderla (II). $\rightarrow$ Capítulo \ref{cap:Defensa}. La defensa. 
\item Cómo ser un mago que protege tu obra con un hechizo, y permitir que otros la disfruten sin que se rompa. $\rightarrow$ Apéndice \ref{anexo:licencias}. Las licencias. 
\item Cómo organizar los diferentes mundos del videojuego con un poco de orden. $\rightarrow$ Apéndice \ref{anexo:Tipologías}. Las tipologías de TFG y su desarrollo.
\item La experiencia de otros/as compañeros/as $\rightarrow$ Apéndice \ref{anexo:Experiencias}. Experiencias y consejos del estudiantado. 
 \end{itemize}

\subsubsection*{Manual de uso}

Te recomendamos las siguientes formas de interactuar con este libro y sus vídeos:

\begin{itemize} \item Ha llegado el verano de tercero, tengo mucho tiempo, estoy aburrido y no se me va de la cabeza el TFG $\rightarrow$ entonces aprovecha el tiempo y léete de tirón el libro.

\item Ha llegado el verano de tercero, tengo mucho tiempo, estoy aburrido y no se me va de la cabeza el TFG, pero me da pereza ponerme a leer en las vacaciones $\rightarrow$ entonces échale un vistazo a los vídeos entre baño y baño.

\item Estoy en el comienzo de cuarto y sigue sin írseme de la cabeza el tema del TFG $\rightarrow$ léete el capítulo \ref{cap:Recomendaciones} de recomendaciones generales para saber cómo proceder y tener una idea general del TFG.

\item Ya tengo tema y tutor y estoy empezando a trabajar en el TFG $\rightarrow$ visualiza cada vídeo conforme te haga falta durante el desarrollo de tu TFG y léete cada capítulo correspondiente si quieres profundizar en algo concreto.

\item Usa el libro y los vídeos como te venga en gana, pero... ¡úsalos! 

\item ¡Por fin he terminado el TFG! $\rightarrow$ sé generoso y comparte tu experiencia y comentarios, que será muy útil para quien viene detrás tuya.

\end{itemize}

Y por último, agradecerte la lectura del libro, esperando que te sea útil ya seas tanto estudiante como docente. Esperamos tus comentarios.

%Este libro queda dividido en los siguientes capítulos: en el capítulo \ref{cap:Recomendaciones} se ofrece, para empezar, información general sobre el proceso de asignación del TFG, contexto general que debes conocer antes de comenzar a trabajar, así como pautas y recomendaciones de trabajo para ayudarte en el desarrollo del proyecto. El capítulo \ref{cap:EstructuraMemoria} establece una propuesta de estructura general de la memoria, indicando cada una de las partes que la compone, para que tengas una idea general de cómo organizar este documento. El siguiente capítulo, el \ref{cap:IntroducciónTFG}, se centra en describir el contenido del capítulo de introducción, ofreciendo tanto una posible estructura del mismo como consejos para su elaboración. El capítulo \ref{cap:RevisionEstadoDelArte} pasa a describir cómo se debe hacer una revisión del estado del arte, parte fundamental en cualquier memoria de TFG. Un elemento importante en todo proyecto es la planificación y el presupuesto, y en el capítulo \ref{cap:PlanificacionPresupuesto} damos algunos consejos para su elaboración. Seguidamente, en el capítulo \ref{cap:Tipologías} se presentan los cuatro tipos de TFG principales: de desarrollo, experimental, investigación y de revisión. Recomendaciones sobre cómo abordar la confección de las conclusiones y los trabajos futuros se incluyen en el capítulo \ref{cap:Conclusiones}. El capítulo \ref{cap:bibliografia} aborda asuntos relacionados con la bibliografía y cómo referenciarla en la memoria y el siguiente, el capítulo \ref{cap:anexos} hace lo propio con los anexos que puedes incluir en la memora.  Una vez que la esta está terminada, en el capítulo \ref{cap:Revisión} se muestra una lista de comprobaciones que deberías hacer para estar seguros de que la memoria alcanza el mínimo de calidad exigible. Los dos capítulos siguientes, y últimos, el \ref{cap:elaboraciónPresentación} y \ref{cap:Defensa}, se centran en la presentación y en la defensa, respectivamente, aconsejando sobre cómo montar la primera y cómo afrontar la segunda. Y como no podía ser de otra manera, este libro finaliza con un capítulo de anexos 
\include{2.recomendaciones_generales}
\chapter{La estructura general de la memoria}
\label{cap:EstructuraMemoria}
% [Autores: Pablo]
% Introducción general a cada una de las partes de la memoria.
% (este quizás sea de los últimos capítulos en escribir)
%Pablo revisa/copia/quita lo que vea oportuno de las recomendaciones generales -> Sobre la redacción de la memoria
% Lo siguiente añadido por María José

En este capítulo te mostramos de forma breve cómo estructurar la memoria de tu TFG. Esta estructura debería ser lo primero que deberías abordar, ya que te marcará un esquema mental del trabajo a realizar. Además es muy importante que esté consensuada con la persona que te tutoriza. Por tanto, bien puedes tener una tutoría para acordar dicha estructura o bien pensar tú en ella y hacer una propuesta. Para tal fin, puedes buscar memorias de TFG en las que basarte y confeccionar tú la tuya propia teniendo en cuenta el ámbito o temática de tu proyecto.

\section{Estructura general de un TFG}

La estructura depende del tipo de TFG que realices (ver Apéndice \ref{anexo:Tipologías}). En general, e independientemente del tipo de proyecto, todas las memorias deberían tener capítulos de \textit{Introducción}, \textit{Estado del Arte}, \textit{Conclusiones} y \textit{Bibliografía}. Aquí te damos una sugerencia de una posible organización de capítulos y secciones dentro de cada capítulo para un TFG genérico:

\begin{enumerate}
    \item Introducción.
        \begin{itemize}
            \item Contexto/Antecedentes.
            \item Justificación/Motivación.
            \item Objetivos/Hipótesis.
            \item Estructura de la memoria.
        \end{itemize}
    \item Estado del arte.
        \begin{itemize}
            \item Descripción de dominio del problema.
            \item Metodologías potenciales a aplicar.
            \item Tecnologías potenciales para usar.
            \item Trabajos relacionados.
        \end{itemize}
    \item \textit{Distintos capítulos sobre la propuesta, que cubrirán:}
    \begin{itemize}
            \item Descripción de la propuesta.
            \item Metodología.
            \item Planificación temporal.
            \item Presupuesto.
            \item Otros capítulos y secciones según la metodología y tipo de proyecto.
        \end{itemize}
    \item Conclusiones y trabajos futuros.
    \begin{itemize}
            \item Conclusiones.
            \item Trabajos Futuros.
        \end{itemize}
    \item Bibliografía.
    \item Anexos.
\end{enumerate}

En el capítulo \ref{cap:IntroducciónTFG} podrás ver en profundidad las descripciones de cada una de las secciones de la Introducción. Es una sección muy importante pues da una visión global del TFG y de sus objetivos, es decir, de qué trata el TFG, por qué el problema que resuelve es relevante y qué metas te marcas. Por tanto, te recomendamos que te esfuerces en su confección. 

En el capítulo del Estado del Arte de tu trabajo se pueden incluir tantas secciones como se desee para agrupar bien los tipos de revisiones realizadas. Si la envergadura de estas secciones es muy grande, pueden incluso separarse en capítulos aparte, aunque te recomendamos que sólo ocupe uno. Tal y como se indica en el capítulo \ref{cap:RevisionEstadoDelArte}, cada sección del estado del arte que se desee incluir deberá constar primero de una introducción explicando la metodología seguida para la revisión, luego la revisión concreta y al final unas conclusiones. Este capítulo es imprescindible en todo TFG que se precie ya que establecerá lo que hay ya hecho en la temática del mismo y dónde se enmarca la contribución de tu trabajo.

%Es muy aconsejable incluir tablas con aspectos comparativos, ya que son más visuales y sirven de resumen. Por ejemplo, se recomienda incluir comparativas entre las potenciales tecnologías o metodologías, para luego en el capítulo de la propuesta justificar cuáles de ellas son las elegidas para la solución. También es común comparar los trabajos relacionados entre sí y según nuestros objetivos o requisitos.

Las secciones de los capítulos en los que desarrollamos nuestra propuesta están muy relacionadas con el tipo de proyecto y metodología. En las siguientes secciones de este capítulo te damos unas guías de cómo podría estructurarse según ello, aunque tienes información más detallada en el Anexo \ref{anexo:Tipologías} para cada tipo de proyecto.

Finalmente, las dos secciones del capítulo de Conclusiones y Trabajo Futuro de tu TFG se explican con más detalle en el capítulo \ref{cap:Conclusiones}.

\section{Proyectos de desarrollo}

Los proyectos de desarrollo suelen ser iterativos, y dependen de la metodología seguida (Ver Apéndice \ref{apendice:desarrollo}). La importancia de este capítulo reside en el hecho de que en él recae todo el peso de la descripción de la metodología que se ha seguido, de tal forma que un lector pueda entender claramente cómo has construido la solución de tu TFG. A continuación mostramos cómo podría estructurarse este capítulo según las dos  metodologías principales, ágiles y clásica. En caso de no seguir ninguna de éstas, deberás plasmar las etapas que compongan la que has seguido en diferentes secciones del capítulo. 

\subsection{Metodologías ágiles}

Cuando usamos una metodología ágil, como Scrum, podría ser buena idea crear la siguiente estructura:
\begin{itemize}
    \item \textit{Product backlog}: listado de historias de usuario priorizadas, y su descripción detallada, incluyendo las pruebas a realizar. Y no te olvides de las historias técnicas. 
    \item \textit{Sprint backlog}: una sección por cada iteración o \textit{sprint}, describiendo las tareas de cada historia e incluyendo gestión de riesgos si procede, así como la planificación en el tiempo de esas tareas y sus resultados.
    \item Implementación: descripción de la implementación de cada iteración así como de las pruebas correspondientes. 
    
\end{itemize}
            
\subsection{Ciclo de vida clásico}
En este caso podemos crear un capítulo por cada una de las etapas.
                \begin{itemize}
                    \item Análisis: requisitos funcionales y no funcionales, de datos y de información, casos de uso, diagramas de los casos de uso, presupuesto, planificación, etc.
                    \item Diseño: diagramas de clases, de secuencia, de la arquitectura, diseño de bases de datos y de interfaces de usuario, etc.
                    \item Implementación: elección de tecnologías, detalles concretos de la implementación de cada componente.
                    \item Pruebas: descripción de las pruebas unitarias, de integración, etc.
                \end{itemize}
%\subsection{Otros ciclos de desarrollo}
%Si usamos otro tipo de desarrollo deberemos adaptar los capítulos a sus etapas.

\section{Proyectos de investigación}
Los proyectos de investigación tienen su propia estructura (ver Apéndice \ref{appendix:investigacion}), en la que se le da especial importancia a la discusión de los resultados. 

Para empezar en un proyecto de investigación, la introducción también debería contar con las siguientes secciones:
    \begin{itemize}
                \item Preguntas de investigación: identificación de problemas o carencias que se van a abordar en el proyecto.
                \item Hipótesis: enunciados a demostrar o probar (suele ser uno de los objetivos el demostrar una hipótesis concreta).
                
    \end{itemize}

%\item \textbf{Evaluación de las hipótesis}: Elección de un método de investigación, determinación de experimentaciones a realizar, elección de instrumentos de recogida de datos y de herramientas análisis de datos y de evaluación.
El resto de capítulos (además de la \textit{Introducción}, \textit{Estado del arte} y \textit{Conclusiones}, claro) pueden ser los siguientes:
\begin{itemize}
    \item Metodología: explicación de los pasos a seguir, algoritmos a implementar, ecuaciones, variables a medir, etc.
    \item Experimentos: definir los parámetros exactos que se van a utilizar y explicaciones más específicas sobre los experimentos a realizar.
    \item Resultados: variables y valores obtenidos. Tablas de análisis, matrices de correlación, etc.
    \item Discusión: interpretación de resultados, referenciando estado del arte e hipótesis u objetivos.
\end{itemize}
    
Si alguno de estos capítulos se te queda corto se puede combinar con otros: por ejemplo, \textit{Metodología experimental} o \textit{Resultados y discusión}.
        
\section{Proyectos de revisión de Estado del Arte} 

En este tipo de proyectos (Ver Apéndice  \ref{appendix:revisionestado}) la mayor envergadura la tendrá el capítulo (o capítulos) del Estado del Arte. Sin embargo, justo después de la \textit{Introducción} deberías incluir un capítulo llamado \textit{Metodología}, donde expliques el proceso seguido para la revisión (dónde buscar, qué palabras clave utilizar, cómo filtrar la búsqueda, qué características revisar/comparar, etc.). A este capítulo le seguirían otros de \textit{Resultados} y \textit{Discusión}, finalizando con las \textit{Conclusiones}. 


 Como última consideración, las metodologías de proyectos no son exclusivas entre sí.  Por ejemplo, un proyecto de tipo desarrollo podría incluir una parte de validación que implique aplicar la metodología de investigación. Igualmente, un proyecto de investigación puede necesitar del desarrollo de software. Por ello, se pueden añadir secciones de un tipo en otro tipo. Por otro lado, un proyecto más relacionado con hardware también puede seguir una metodología en cascada, quizá adaptada a sus características.

En resumen \footnote{Esto no lo ha escrito chatGPT ;-)}, la estructura no está escrita en piedra y puede variar dependiendo de lo que vayas a hacer. Por eso es muy importante consensuarla con la persona que te tutoriza y también tener en cuenta que puede cambiar durante el desarrollo del proyecto. Por ejemplo, en algunos casos se aconseja incluir el cronograma y presupuesto en la introducción, en lugar de en el capítulo de la propuesta. También hay algunos tutores que prefieren que los trabajos relacionados estén en un capítulo aparte del estado del arte. En los siguientes capítulos de este libro te explicamos con más detalle cómo redactar cada uno de los capítulos y secciones de la memoria de tu proyecto.


\chapter{La introducción del TFG}
\label{cap:IntroducciónTFG}

\section{Introducción}

En este capítulo nos centramos en ofrecerte ciertos consejos que consideramos que pueden ser de tu interés para que redactes el capítulo de introducción, que normalmente será el primero de la memoria y que también será el que contenga cierta entidad dentro de la misma, pues es vital para que el lector entienda de qué trata tu TFG.

En síntesis, el capítulo de introducción debes destinarlo a ofrecer una panorámica general del trabajo llevado a cabo, sobre todo en lo referente al contexto y razones que lo han motivado, de forma que acerques al lector a los temas tratados en el mismo. Como veremos más adelante, esto ocupará dos secciones específicas del capítulo.

Por otro lado, como con cualquier otra sección de la memoria, debes tener presente el punto de vista del lector y facilitarle su comprensión resulta especialmente relevante. Cuando redactes esta sección procura que sea entendible por cualquier persona ajena a la informática, incluso tus abuelos y amigos. Ello fomentará que se anime a profundizar y seguir leyendo el resto de secciones. Si además se trata de alguien interesado en trabajar en líneas iguales o parecidas, le motivarás a que pueda consultar el resto del trabajo y continuarlo en el punto en el que se dejó en la memoria, o lo revise y hasta mejore.

Finalmente, desde un punto de vista pragmático, debes tener en cuenta que, por el mero hecho de entregar el TFG para su evaluación y defensa pública, ya tienes un conjunto de lectores asegurado que, además, van a leer y consultar la memoria elaborada desde un punto de vista crítico. En efecto, dichos lectores serán las personas que formen parte del tribunal de evaluación del TFG, o al menos, las personas que te han tutorizado, según venga establecido en la guía docente correspondiente.

Recuerda que es muy probable que alguna de las personas que formen el tribunal de evaluación, si no todas, no estén tan familiarizados como tú en el tema del trabajo, o no sepan por qué es importante o pertinente trabajar en el mismo. Es posible incluso que para alguien sea la primera vez que lee algo acerca de las temáticas principales del TFG. El capítulo de introducción sirve para aclarar todos estos aspectos y ayudar al tribunal a comprender y valorar mejor las contribuciones del trabajo.

Por esto mismo, siendo el primer capítulo de la memoria, es uno de los que te conviene poner especial cuidado y esmero en la redacción, de forma que sea especialmente ágil, conciso y claro. También es importante que sigas cierto orden y estructura a la hora de presentar (\textit{introducir}) los contenidos, siguiendo un patrón que atienda al \textit{qué}, para describir el contexto; al \textit{por qué}, para dar razón o motivar el trabajo, y al \textit{por tanto}, para definir objetivos consecuentes con la motivación y el contexto del trabajo.

Podría ocurrirte que aun tratando de ser breve, te cueste ajustarte a esta forma de estructurar, bien porque no encuentres muchas cosas que decir (mucho contexto), o porque de forma natural quieras conectar enseguida el contexto con la motivación. También, como se trata del primer capítulo, y no estás acostumbrado a documentar trabajos tan extensos, experimentes cierto bloqueo a la hora de escribir. Como con el resto de secciones de la memoria del TFG, en las siguientes secciones te daremos algunas pautas o pistas, con ejemplos, sobre cómo abordar o con qué tipo de contenidos podrías completar el capítulo de introducción. De forma similar a lo que ocurre con el primer capítulo de una serie de televisión o un libro cualquiera, el capítulo de introducción del TFG debe servir para conectar y enganchar al lector, atrayendo su atención y provocando su curiosidad por el trabajo. De hecho, las productoras de televisión suelen invertir cantidades de dinero por minuto de metraje en los primeros minutos o el tráiler de una producción muy superiores a las que invierten en otros minutos de capítulos ordinarios. Saben que ese primer contacto con el espectador es fundamental para despertar su interés.

% Esto mismo es lo que nos han reportado estudiantes como tú en un estudio previo \colorbox{yellow}{[REFERENCIAR ENCUESTAS]}.

Seguramente hayas experimentado una situación similar al comenzar a leer otros libros o incluso otras memorias de TFG: si encuentras las primeras páginas difíciles de entender o pesadas, es más fácil que abandones la lectura de las siguientes o pases por alto detalles relevantes, ya que cuesta más mantener la atención por el sobreesfuerzo requerido. Ponte especialmente en el lugar de tus potenciales lectores cuando escribas este capítulo y ten en cuenta que, como dice el refrán, \textit{para la primera impresión no hay una segunda oportunidad}.

Considera también que este primer capítulo será en el que más se fijen quienes evalúen el trabajo (junto con el de las conclusiones). Esto se debe, por un lado, a que se encuentra al principio de la memoria, y lógicamente aparece y se lee antes; y por otro lado, a que en este capítulo deben reflejarse en una sección separada los objetivos del TFG. Por tanto, lo normal será que quien lo evalúe, acuda con frecuencia a dicha sección de objetivos y la contraste con las aportaciones reflejadas en la memoria, las conclusiones del trabajo y lo que se presente durante la defensa. 

Por último, este capítulo no deberá ser muy extenso (con cinco o seis páginas puede ser más que suficiente), ya que se trata simplemente de dar un anticipo de la información que se detallará más adelante en el resto de secciones de la memoria para quien esté realmente interesado en el trabajo. Lo que sí es importante es tratar de respetar la estructura del capítulo, ya que suele ser bastante universal. Es decir, una especie de estándar o protocolo relativo a la forma de estructurar un libro, que en este caso es la memoria del TFG. Como ya te comentamos cuando hablamos de la estructura del documento, respetar una organización estándar y un mismo tipo de contenidos en cada uno de sus apartados, hará que las personas que te tutoricen y revisen puedan recorrer y consultar la memoria de forma más eficaz, ya que podrán encontrar el tipo de información que buscan en las secciones previstas para ello.

Por la importancia de este capítulo, a veces se redacta al final del todo, cuando el resto de la memoria ha sido completada. La ventaja de esta aproximación es que ya tienes una visión global de tu TFG y su memoria, lo que posibilita que puedas centrarte en este capítulo y ser capaz de plasmar la esencia del TFG mucho mejor.

A continuación pasaremos a presentar las secciones, junto con sus contenidos y pequeños ejemplos de los mismos, que deben aparecer en el capítulo de introducción de un TFG. Normalmente, dichas secciones serán las siguientes: contexto, motivación, objetivos del TFG y estructura de la memoria.

% [Autores: Manolo]
\section{Contexto}\label{Contexto}
Esta sección está destinada a responder a la pregunta ``¿qué?''. Qué existe, qué es o en qué consiste, qué se puede hacer, cuál es el tema de estudio o del trabajo, en qué dominio se sitúa, cómo se viene abordando una determinada actividad (es decir, de qué forma, qué enfoques, propuestas o aproximaciones se vienen defendiendo), qué propiedades, qué mercado, qué permite, etc., son ejemplos de cuestiones que debes responder a la hora de presentar el contenido de esta sección. Dejamos por aquí algunas pistas o sugerencias de contenidos de que deben aparecer.

\subsection{Qué de qué: definiciones}

En conjunto, la respuesta a los diferentes \textit{``qués''}, nos dará como resultado el contexto del trabajo y orientarán al lector con respecto al contenido de la memoria. Por ejemplo, si tu TFG se encuadra en el ámbito de los asistentes inteligentes, una buena forma de que comiences a escribir el contexto es decir qué es un asistente inteligente o qué tipo de sistemas software o hardware vamos a entender por tales.

Esto es especialmente importante cuando un mismo término se use de forma diferente en varios contextos. Para ello, es bueno comenzar con una definición o introducir alguna frase aclaratoria del tipo \textit{``... Si bien existen diferentes acepciones para el término} \textless término\textgreater, \textit{en este trabajo vamos a entender por tales aquellos que...''}. Lo mismo también es aplicable cuando te vayas a referir con diferentes términos a una misma idea o concepto. Por ejemplo, siguiendo con el caso de los asistentes inteligentes, si nos vamos a referir a ellos también como sistemas conversacionales, podemos incluir una frase aclaratoria del tipo \textit{...Es frecuente encontrar referencias a los asistentes inteligentes denominándolos sistemas conversacionales}. \textit{En esta memoria utilizaremos indistintamente ambos términos para referirnos a aquellos sistemas que...}.

También puedes explicar brevemente la diferencia entre términos, para demostrar que has tomado la decisión de omitir ciertos matices que no son relevantes en el contexto de tu TFG. Por ejemplo, si tu trabajo se enmarca en el ámbito \textit{ubitquitous computing} (Computación Ubicua) y te vas a referir a estos sistemas también como del ámbito del \textit{pervasive computing}, podrías decir \textit{``... En la computación ubicua, el objetivo principal es proporcionar a los usuarios la capacidad de acceder a servicios y recursos en todo momento e independientemente de su ubicación, mientras que en pervasive computing, el objetivo principal es proporcionar servicios emergentes y espontáneos creados sobre la marcha por dispositivos móviles que interactúan mediante conexiones ad hoc. Existiendo esta diferencia, la realidad es que en muchas situaciones, ambos paradigmas de computación ofrecen las mismas soluciones y podemos referirnos a ellas como formas de computación ubicua o pervasive computing, indistintamente...''}.

\subsection{No despistes al lector}
Por otro lado, debes tratar de seguir un enfoque bastante sintético, tratando de ir directamente al grano, sin digresiones sobre temas (otros ``qué'') que se alejen del núcleo del trabajo. En efecto, es muy importante que evites incluir información o hacer referencia a cuestiones que no se hayan abordado con cierta profundidad, ya que pueden despertar falsas expectativas en el lector acerca de lo que va a encontrar. Como ya hemos comentado, conviene que te pongas en el lugar de quien va a leer la memoria. Pensemos en nuestra propia experiencia como lectores de otros libros o manuales técnicos (una memoria de TFG es en parte eso). Si cuando consultamos el índice, la sinopsis de la contraportada o la introducción se nos hablaba de una tecnología, y por esto mismo nos animamos a sacar el libro de la biblioteca, y luego esta se usaba de una forma muy marginal en el resto del documento, nos sentimos decepcionados y que hemos perdido el tiempo.

Así, imagina la siguiente situación: en tu proyecto o desarrollo has utilizado unas técnicas bastantes complejas de aprender y con una curva de aprendizaje bastante plana al comienzo, siendo esta una dificultad conocida de dichas técnicas. A pesar de ello, has conseguido dominarla y resolver un problema concreto e interesante. Sin embargo, en tú trabajo no has realizado ninguna aportación que facilite el uso de dichas técnicas. En este caso, tendrías que evitar hacer una referencia reiterada a la dificultad de manejar dichas técnicas, ya que el lector o revisor podría esperar que el trabajo terminará abordando ese problema, debido a que se menciona varias veces y parece importante.

% PROPONGO ELIMINAR ESTE PÁRRAFO Esto ocurre con frecuencia con problemas conocidos: ruido en los datos en los mismos \textit{outliers}, usabilidad de una aplicación, rendimiento de un algoritmo, consumo energético, etc. Estos problemas podrían ser desafíos vigentes de la tecnología que utilicemos en nuestro TFG, pero si no hemos realizado ninguna aportación para paliarlos, lo mejor es no mencionarlos más de una vez en nuestro capítulo de introducción. Así, si decimos que en ciertos tipos de conjuntos de datos existe mucho ruido, que actualmente se está trabajando en técnicas que al generarlos reduzcan dicho ruido y que una línea de trabajo interesante en relación con esos conjuntos de datos es la eliminación del ruido, será porque en nuestro TFG, el ruido y su gestión se habrán abordado de alguna forma novedosa. Si, por el contrario, pero en nuestro TFG no hemos hecho o aportado ninguna solución para gestión del ruido, estaremos creando cierta confusión en la persona que lea el trabajo.

La experiencia con muchos trabajos que hemos supervisado o evaluado en tribunales de defensa en el pasado nos enseña que buenos trabajos de fin de grado se ven penalizados o deslucidos por referencias a aspectos o cuestiones en las memorias que no eran el objeto o contribución nuclear del trabajo.

%\subsection{Respeta la estructura}
%Por último, indicar que una tendencia habitual al presentar una tecnología, un estado de la cuestión sobre algún tema, describir un dominio de aplicación, etc., es comentar a la vez los problemas o deficiencias con los que vengan aparejados. Por ejemplo, podemos haber realizado un TFG sobre un sistema de monitorización de la actividad física que sincroniza datos que recibe de sensores colocados en el cuerpo y graba en vídeo sesiones de entrenamiento. Hasta aquí habríamos descrito un \textit{qué}, es decir, monitorizamos la actividad física sincronizando vídeo y datos de sensores. Nuestro sistema es novedoso porque hasta la fecha no se dispone en el mercado de soluciones de bajo coste o abiertas que satisfagan dicha funcionalidad. En este caso, una tentación frecuente es conectar ambas ideas con una descripción como la siguiente: \textit{En los últimos años se han popularizado los sistemas y plataformas de monitorización remota de la actividad física a través de sensores corporales para estudiar el rendimiento durante el ejercicio físico. Estos estudios pueden complementarse con el análisis de vídeos o grabaciones de los sujetos mientras realizaban la actividad física registrada con dichos sensores} (hasta aquí un \textit{qué}).\textit{Sin embargo, no existen abundan soluciones de bajo coste que, de forma integrada, permitan gestionar la información obtenida desde ambos tipos de fuentes, es decir, vídeo y datos de sensores...} (habríamos comenzado a describir una \textit{deficiencia} o a dar un motivo, es decir, responder a \textit{por qué} hemos realizado el trabajo que se presenta.

%La conexión de ambas ideas en el resumen sería razonable, ya que por limitaciones de espacio, no habría opción a otras alternativas. En cambio, en el caso de la Introducción conviene cierta disciplina y estructura a la hora de presentar los contenidos. Como veremos en la siguiente sección, las razones que justifican la pertinencia del trabajo desarrollado deben describirse en la siguiente sección de \textit{motivación}.

\subsection{Estructura de la introducción: algunas sugerencias}
A lo largo de toda la memoria, es esencial que respetes la estructura del documento. Debes respetar que tras un \textit{¿qué?} debe aparecer un \textit{¿por qué?}.
Puedes encontrar dificultades para ajustarte a la estructura presentada porque necesites dar enseguida argumentos o motivaciones para el trabajo (que vendrán en la sección \ref{Motivation} Motivación), pensando que te queda demasiado pobre si no lo haces. También puede darse el caso de que no sepas qué más comentar acerca de lo que ya existe. En estos casos puedes completar el capítulo de introducción proporcionando breves datos acerca de cuestiones como usuarios actuales o potenciales beneficiarios de una tecnología, orígenes de la misma, tamaño de mercado o volumen de facturación, iniciativas políticas, legislación, etc. Por ejemplo, si tu TFG versa sobre una plataforma para la promoción de hábitos nutricionales saludables en población infantil, para describir el contexto puedes apoyarte de datos como \textit{``Un estudio reciente llevado a cabo en el ámbito del programa conocido como Estrategia de Promoción de una Vida Saludable en Andalucía (2024-2030), dependiente de la Consejería de Salud y Consumo de la misma comunidad, estima que la prevalencia del exceso de peso en la población infantil andaluza (de dos a diecisiete años)  se sitúa en el 33,40\%, situados desde el comienzo de estos estudios en varios puntos por encima de la media nacional''}\footnote{\url{https://www.juntadeandalucia.es/organismos/saludyconsumo/areas/planificacion/estrategia-promocion-vida-saludable-andalucia.html}}. Como vemos en este texto de ejemplo, hemos aprovechado para explicar indirectamente qué vamos a entender por población infantil (la que se encuentra entre 2 y 17 años de edad). Asimismo, hemos proporcionado también una referencia para demostrar que nos hemos documentado y acudido a fuentes oficiales para proporcionar el contexto de nuestro trabajo. En el capítulo  \ref{cap:bibliografia} (Bibliografía) profundizaremos más en cómo citar los distintos trabajos.

Por último, si te ocurre lo contrario, es decir, que tienes exceso de contenidos con los que documentar el contexto del trabajo, trata de ser breve, de ir al grano, describiendo o reservando los detalles para el capítulo de revisión del estado del arte. Recordemos que en este capítulo de introducción,  tienes que tratar de ser conciso y presentar la información de forma atractiva para facilitar al lector la revisión de contenidos posteriores, además de despertar su curiosidad por leer el resto de la memoria.

\section{Motivación}\label{Motivation}
Esta sección debe justificar la razón de ser de tu trabajo. Si la sección \ref{Contexto} Contexto, buscaba responder a la pregunta \textit{``¿qué?''}, esta sección debe más bien tratar de responder, en dicho contexto, a preguntas como:
\begin{todolist}
    \item \textit{¿Qué falta?}
    \item \textit{¿Qué necesidades existen?}
    \item \textit{¿Qué problemas tiene?}
    \item \textit{¿Cómo se puede potenciar?}
    \item \textit{¿Por qué  se puede potenciar?}
    \item \textit{¿Por qué merece la pena que yo trabaje en esto?}
    \item \textit{¿Por qué debe hacer esa tarea necesariamente un ingeniero o ingeniera informáticos?}
\end{todolist} 

Si, por ejemplo, tu trabajo trata sobre el análisis de la facilidad de navegación de un sitio web y conoces una tecnología o una técnica que piensas que podría aplicarse para monitorizar las distintas páginas que visitan los usuarios, pero que aún nadie la ha probado, puedes justificar tu trabajo indicando que el uso de dicha tecnología o técnica aún no se ha explorado para el análisis de la usabilidad de un sitio web. Como es plausible que su aplicación facilite dicha tarea, queda justificado hacer un TFG que lo investigue. La misma argumentación puedes seguir cuando describas problemas, carencias, o funcionalidades por desarrollar y que decides abordar en tu TFG.

Por otro lado, si tu TFG ha consistido en la simulación de un encargo profesional relacionado con nuestro grado, es interesante responder a \textit{¿qué competencias} has adquirido o mejorado al realizar el mismo?

Se trata de justificar por qué has llegado hasta aquí, o por qué es bueno o conveniente trabajar de la manera y en el ámbito que lo has hecho. En definitiva, tienes que responder a la pregunta \textit{``¿por qué?''}: porque faltaba una determinada funcionalidad que has desarrollado; porque no se ha explorado la aplicación de una técnica en un determinado ámbito y puedes demostrar que se pueden obtener resultados interesantes haciéndolo; porque atendiendo una determinada necesidad se mejora la vida de ciertas personas; porque aplicando algo de distinta manera se obtienen mejores resultados que los que se vienen obteniendo en un determinado ámbito, etc.

Así, dependiendo del trabajo, podrás presentar unas motivaciones u otras, pero no debe resultarte difícil encontrar un buen puñado de ellas. Algunos ejemplos podrían ser: 

\begin{itemize}
  \item Atención a un determinado colectivo en relación a una tecnología. Por ejemplo, tu trabajo podría consistir en el diseño de unas etiquetas en Braille y una aplicación para un dispositivo móvil capaz de interpretarlas para colocarlas junto a las obras de arte de un museo de forma que las personas con discapacidad visual puedan localizarlas y acceder a información enlazada acerca de dichas obras. Tu TFG viene justificado o motivado por la necesidad de atender a ciertas personas y la ausencia de algo parecido.
  \item Problemas de usabilidad. Por ejemplo, los dispositivos para interactuar con agentes conversacionales inteligentes a veces tienen dificultades para captar correctamente las intenciones de sus usuarios, sin embargo tampoco existen métodos o herramientas para facilitar el análisis de por qué la interacción falla y tu trabajo ha consistido en desarrollar un sistema que demuestra que haciendo uso de una determinada tecnología es posible depurar las interacciones con dichos sistemas.
  \item Ausencia de una funcionalidad específica que demande un determinado tipo de usuarios o colectivo sobre ciertos conjuntos de datos. Por ejemplo, conoces varias plataformas de compartición y reproducción de música bajo demanda. Algunas de estas ofrecen APIs (\textit{Application Programming Interface}) que permite acceder a datos generalistas sobre artistas y descargas adaptadas a sus consumidores (este sería el \textit{qué}), pero sabes que dentro del colectivo de los propios músicos o creadores de contenidos, estarían más interesados en disponer de herramientas que ofrecieran un análisis más pormenorizado de sus obras y evolución en ciertos períodos. Entonces, decides hacer un TFG para cubrir esta carencia desarrollando las funcionalidades pertinentes.
  \item Si tu trabajo va a consistir en ejecutar un encargo profesional (que podrá ser simulado o real), como, por ejemplo, el desarrollo de un sistema de información para la gestión de un gimnasio, podrás enumerar distintas competencias o habilidades que quisieras adquirir y/o reforzar. De hecho, su adquisición ya representa por sí misma una motivación o justifica el desarrollo de tu TFG.
\end{itemize}

Por otro lado, como con todas las secciones de la memoria, debes mantener el hilo y presentar un relato coherente. Por tanto, en esta sección tienes que concretar un poco más el ámbito en el que has trabajado (para los trabajos más aplicados o que simulen encargos profesionales) o donde has aportado algo novedoso (para aquellos trabajos con alguna componente de investigación o innovación), en línea con la idea de centrar el tema del trabajo y no divergir o distraer a los potenciales lectores de la memoria.

\section{Objetivos del TFG}
A partir de los \textit{``qué''} y los \textit{``porqué''}, esto es, del contexto y motivación de las dos secciones anteriores, la sección de objetivos debe representar un \textit{``por tanto'', ``en consecuencia''} de lo que ya se ha explicado. Por ejemplo, como falta esta funcionalidad en este ámbito o dominio particular, te propones abordar un desarrollo que cubra dicho vacío, o bien, como cierta tecnología presenta los problemas que ya se han descrito, propones un objetivo que resuelva o mitigue dichos problemas. Como venimos insistiendo, alinear los objetivos con el contexto y la motivación del trabajo, también te ayudará a revisar mejor el trabajo y no distraer al lector.

Lo normal será también que los objetivos los presentes describiendo en primer lugar un objetivo principal (también denominado objetivo general (OG)) que sea lo más comprensivo posible, de forma que englobe las tareas o resultados más destacables de tu TFG y,  a continuación, una serie de objetivos más particulares o concretos (también denominados específicos (OE)), sobre resultados o tareas más sencillas. Esto demostrará que has analizado el problema a abordar y que has sido capaz de descomponer y estructurar una tarea compleja en otras más sencillas para conseguir un alcance mayor. Como siempre, pero especialmente en esta sección, conviene que te expreses con concisión. Así, una buena frase para comenzar esta sección puede ser:

\textit{El objetivo principal de este TFG es } \textless nuestro principal objetivo\textgreater. \textit{Este objetivo se ha articulado en torno a la consecución de los siguientes objetivos específicos}:
\begin{itemize}
  \item \textit{OE1: objetivo específico 1}
  \item \textit{OE2: otro objetivo específico}
  \item \textit{...}
  \item \textit{OEn: y otro más...}
\end{itemize}

Como pista, podríamos decir que el objetivo principal debería ir alineado con el título del proyecto, para dotar de coherencia a la memoria. Asimismo, en la lista o enumeración anterior de objetivos específicos, deberán aparecer primero aquellos que sirvan de base o deban alcanzarse antes de abordar otros, secuenciando los pasos que deben darse para alcanzar gradualmente el objetivo principal.

Cabe indicar en este punto que los objetivos deben estar formulados en infinitivo: estudiar, analizar, desarrollar, comparar, evaluar, etc.  

Por otro lado, las características principales que debes buscar al fijar (y expresar) los distintos objetivos en el contexto de un proyecto, en general, y de un TFG en particular, es que sean realistas y concretos. Sin ánimo de seguirlas exhaustivamente, puedes guiarte por las orientaciones del marco de trabajo SMART \cite{doran1981there}. En particular, por las revisiones más modernas y orientadas a proyectos tecnológicos, como las que podemos encontrar en el sitio web de {Atlassian}\footnote{\url{https://www.atlassian.com/blog/productivity/how-to-write-smart-goals}}. SMART es el acrónimo de \textit{Specific}, \textit{Measurable}, \textit{Achievable}, \textit{Relevant}, y \textit{Time-Bound}, es decir, los objetivos que fijes deben abordar un área concreta de trabajo o mejora (esto es, ser \textit{específicos}); incluir alguna métrica o valor objetivo para seguir y evaluar su grado de progreso o cumplimiento (para que sean \textit{medibles}); es factible (realista) conseguirlos (es decir, son \textit{alcanzables}); son pertinentes o necesarios para el TFG (por tanto, \textit{relevantes}); y finalmente, puede anticiparse un plazo de tiempo para su ejecución (es decir, son \textit{planificables} o acotarse en el tiempo).

Finalmente, indicar que, aunque idealmente los objetivos de un trabajo se establecen (al menos informalmente) al comienzo del mismo y se van abordando conforme a una determinada planificación, la realidad y el posible desconocimiento de algunas tecnologías que estés utilizando, te lleven a modificar tus planes iniciales y revisar dichos objetivos, bien porque ves más interesante algunas posibilidades que has descubierto, o bien porque te encuentras con algún imprevisto que te impida alcanzarlos.

Por ello, desde un punto de vista estratégico y siendo también realistas, puedes dejar para el momento de finalización de la memoria la redacción concreta de los objetivos, según el trabajo que ya habrás completado. La mayoría de las veces infravaloramos el tiempo que vamos a necesitar para realizar una tarea y sobrestimamos nuestras capacidades, pero digamos que no hace falta que lo explicites en tu memoria. Al fin y al cabo, ordinariamente tendrás que hacer una defensa del mismo, y si indicaras posteriormente que no has alcanzado los objetivos que te fijaste, además de resultar extraño, te estarías despojando de argumentos para defender o justificar tu capacidad de análisis o planificación.  

También indicar que se suelen poner, además de los objetivos del proyecto, algunos objetivos personales que se desean alcanzar con la elaboración del mismo. Estas metas reflejan ciertos aprendizajes o puesta en práctica de metodologías o tecnologías que, aprovechado que estás haciendo el TFG quieres conseguir. Un par de ejemplos podrían ser ``aprender y aplicar en un proyecto real la metodología Scrum'' o ``aprender el \textit{framework} Django y ponerlo en práctica en un proyecto real''. 

\section{Estructura de la memoria}
Esta sección está destinada a comentar cómo has organizado el resto de la memoria, por lo que simplemente debes describir qué bloques principales o capítulos has utilizado y, en una frase o dos, dar una breve idea de qué contiene cada uno. La idea es mostrar cómo queda estructurado el texto y qué se va a describir en cada una de las partes del mismo.

Desde un punto de vista visual, puedes utilizar una enumeración en forma de lista de elementos separados por puntos.

Por ejemplo, podrías escribir algo como: \textit{``En el capítulo 1 se han presentado el contexto y motivación de este trabajo, así como los principales objetivos que nos hemos propuesto para abordar la problemática asociada a \textless X\textgreater. El resto de la memoria se ha estructurado como sigue:}

\begin{itemize}
  \item \textit{En el capítulo 2 se describe el estado de la cuestión en relación a las principales tecnologías/métodos/técnicas/estudio (escribir lo que corresponda) sobre} \textless lo-que-hayamos-usado-en-nuestro-TFG \textgreater.
  \item \textit{En el capítulo 3 se presenta la propuesta que hemos realizada para} \textless abordar-el-problema-u-objeto-principal-de-nuestro-TFG \textgreater.
  \item \textit{El capítulo 4 contiene la especificación y arquitectura de los principales módulos/interfaces/servicios que hemos desarrollado}.
  \item ...
  \item \textit{En el último capítulo se presentan las conclusiones y trabajos futuros a partir del presente TFG}.
  \item \textit{Finalmente se presentan las referencias bibliográficas usadas en este trabajo''}.
\end{itemize}

Como es lógico, lo más práctico será que escribieras esta sección cuando hayas completado todos los demás capítulos y conozcas la estructura definitiva que vas a dar a la memoria de TFG.

\include{5.revision_estado_arte}
\include{6.planificacion_presupuesto}
\chapter{Las conclusiones y los trabajos futuros}
\label{cap:Conclusiones}

% [Autores: María José Rodríguez Fórtiz]
El capítulo de \textit{Conclusiones y trabajos futuros} es muy importante pues recoge qué se ha realizado en el TFG, los principales resultados y qué puede hacerse a partir de este momento. Muchas personas leen la introducción y luego las conclusiones antes de leerse el resto de la memoria, para así conocer bien la motivación, objetivos y los resultados del trabajo. Eso significa que debes tener especial cuidado al redactar este capítulo para que quede muy claro y sea muy completo.

Habitualmente se incluyen dos secciones en este capítulo, la de conclusiones y la de trabajos futuros.
 
 \section{Conclusiones}

Debes empezar las conclusiones con una frase inicial a modo de resumen sobre el objetivo general y el problema que se ha abordado, dando una valoración positiva sobre los resultados de tu trabajo (en caso de que todo haya ido bien). Si por algún motivo, no se ha satisfecho el objetivo general, lo puedes indicar pero justificando el porqué, y añadiendo que a pesar de ello, se han cumplido algunos de los objetivos específicos, obteniendo resultados favorables. 

A continuación debes hacer un repaso uno a uno de los objetivos específicos, indicando: (1) el porcentaje de realización, (2) un resumen de lo que se ha hecho para cumplir ese objetivo (dos o tres líneas explicando las tareas realizadas asociadas a ese objetivo y los resultados obtenidos deben bastar), y (3) una indicación de dónde pueden verse las evidencias de ese objetivo en la memoria, en qué capítulo o sección.

También puedes mencionar en esta parte, añadiendo un párrafo final donde expliques cómo tu formación previa en materias concretas del grado te ha sido de ayuda para el TFG y los retos nuevos que has tenido que afrontar para resolver cuestiones que no hubieras visto antes durante tu formación (qué cosas has tenido que aprender que no has visto en tu grado, por ejemplo).

 En cuanto a la redacción del  repaso de objetivos, te ponemos un ejemplo. Suponiendo que estás abordando un objetivo específico que has redactado como ``Revisar aplicaciones similares para comparar con la propuesta'', puedes indicar que ese objetivo se ha cumplido completamente, explicando por ejemplo que has revisado seis aplicaciones similares y que has realizado una tabla comparando ocho características básicas de cada una, la cual puede consultarse en el capítulo o sección X de la memoria. También puedes añadir que al elaborar esta tabla se demuestran tus capacidades de análisis y síntesis de información. Si este objetivo no se hubiera cumplido completamente, porque, por ejemplo solo hayas revisado dos aplicaciones y tenías previsto revisar más, indica que sí lo has hecho pero solo un 30\%, y argumentas por qué es insuficiente, por ejemplo, porque solo hay dos de libre acceso que has podido consultar con profundidad, o porque has priorizado terminar la tarea X, que habéis considerado que era más importante para el TFG. 

 Si has tenido en cuenta algún aspecto ético o has dado solución relacionada con algún ODS, menciónalo también en un párrafo resumiendo cómo lo has abordado. Esto es algo que se valora en la rúbrica de evaluación del tribunal, y conviene mencionarlo para que se tenga en cuenta.

 %De cara a la redacción de esta sección puedes tener en cuenta el registro de marcas propuesto en \cite{meza2019comunicacion} \textcolor{orange}{, que sugiere verbos que puedes utilizar (en este caso para explicar en las conclusiones cuál ha sido el alcance de cada objetivo), como son: "se ha abordado", "hemos hecho un recorrido por", "podemos afirmar que", "esto evidencia que", "confirmamos que", "confirma nuestras hipótesis/ideas", "hemos propuesto/obtenido/identificado/revisado/observado/descubierto/utilizado/demostrado/explicado/desarrollado ...",  "no hemos podido demostrar/confirmar/revisar/identificar ... porque ...", etc. Como marcadores discursivos, podemos usar conectores como los siguientes: "por tanto", "sin embargo", "en consecuencia", "por el contrario", "a pesar de", "gracias a", "entendemos que", o "de acuerdo/según todo lo anterior".}
 
Por último, es importante que en las conclusiones añadas un párrafo final como valoración personal, escrito esta vez en primera persona. En esa valoración debes mencionar cómo te has sentido al realizar el TFG y según sus resultados. Puedes indicar que te sientes orgulloso/a, contento/a, satisfecho/a, encantado/a, etc. por lo que has aprendido, por cómo te has organizado en el tiempo, por cómo has redactado la memoria, por la calidad del código desarrollado, por cómo te has comunicado con tu tutor, etc. Si tienes alguna valoración negativa, debes mencionarla también, pero te recomendamos que la redactes de forma positiva, aportando qué has aprendido de ello. Por ejemplo,``No estoy satisfecho/a con cómo he organizado el trabajo temporalmente porque he dejado muchas tareas para el último mes y eso me ha saturado, con lo cual he aprendido que en un futuro debo hacer una mejor planificación temporal desde el principio.''. En tu valoración personal, y si no lo has hecho al revisar los objetivos, también puedes mencionar cómo has aplicado y mejorado tus habilidades blandas o \textit{soft skills}, como son organización de trabajo, pensamiento crítico, creatividad, adaptación, resolución de problemas y comunicación. El momento en el que haces esta valoración personal también es bueno para recapitular todo lo que has aprendido y qué competencias de las que se mencionan en el plan de estudios has desarrollado con el TFG.

 \section{Trabajo futuro}

 En esta sección se indican qué tareas se pueden realizar en un futuro para completar, mejorar o continuar con el trabajo que tú has desarrollado en tu TFG. Por ejemplo:
 \begin{itemize}
     \item Tareas que que estaban previstas y no se han hecho o han quedado incompletas, de las mencionadas en las conclusiones.
     \item Requisitos de desarrollo que tenías previsto abordar pero que al final no has tratado.
     \item Nuevos requisitos que hayan surgido durante el desarrollo del TFG, que no se habían previsto y por tanto no se han planificado ni abordado.
     \item Nuevos objetivos e ideas para dar continuidad al TFG en futuros TFG, desarrollos o investigaciones.
 \end{itemize} 

 Para cada una de estas tareas, requisitos u objetivos conviene añadir un pequeño párrafo que explique por qué se propone y cómo se abordaría, de forma muy resumida. Por ejemplo: ``En un futuro se puede desarrollar una versión en iOS del prototipo realizado en el TFG. Esto ayudaría a que más personas pudieran utilizar la aplicación. Para ello, se podría utilizar un \textit{framework} de desarrollo como \textit{Flutter} o \textit{Ionic}, que permiten esta portabilidad y el desarrollo híbrido de aplicaciones móviles. Habría que valorar si el código actual o parte de éste puede reutilizarse''. Otro ejemplo de párrafo: ``Sería necesario completar la gestión de usuarios en la aplicación desarrollada, ya que por el momento solo pueden hacerse altas y modificaciones. Bastaría para ello diseñar e incluir funciones e interfaces para el borrado de usuarios de la misma forma que se ha hecho para las otras operaciones. Esto no supondría ningún cambio en la base de datos''.  En el caso de ser una tarea del primer tipo, justifica bien la razón por la que no se ha podido realizar íntegra o parcialmente. 

 Finalmente, dependiendo del tipo de TFG y de tus intereses sobre la comercialización del producto desarrollado, podrías poner una sección que hablara del modelo de negocio y la propuesta de valor del resultado del proyecto. Esta podría ser una subsección de los trabajos futuros, pero según su importancia en el TFG, quizá podría tener cabida como capítulo propio en la memoria.
 
\chapter{La bibliografía\label{cap:bibliografia}} % [Autores: Rocío Romero Zaliz] 

% Definición y propósito
% Importancia de la bibliografía en un TFG/TFM
La bibliografía es un componente esencial de un trabajo fin de carrera, tanto de grado como de posgrado, ya que no solo justifica y respalda tus argumentos, sino que también enriquece tu trabajo al proporcionar acceso a fuentes adicionales de información. Por tanto, es importante que dediques el tiempo necesario para elaborar una bibliografía completa y correctamente formateada.

\section{¿Por qué citar?}

Citar es una práctica fundamental en cualquier trabajo académico. Se trata de mencionar las fuentes de información que se han utilizado en tu trabajo ya que, además de constituir un reconocimiento hacia ellos/ellas, las citas académicas permiten distinguir claramente cuáles son tus aportaciones y en qué se sustentan. Citar el trabajo de otros/as autores/as te permite:

\begin{itemize}
    \item Contextualizar el tema de estudio, situándolo dentro de un marco teórico que ayude al lector a comprender mejor tu trabajo y su relevancia en ese contexto.
    \item Demostrar dominio del tema que abordas gracias a todo el trabajo de documentación realizado, aumentando la credibilidad en tu trabajo.
    \item Permitir la verificación de la información utilizada, algo indispensable para  garantizar la transparencia y la confiabilidad de tu trabajo.
    \item Respaldar tus argumentos mediante citas de fuentes confiables y relevantes, aumentando la credibilidad en un trabajo.
    \item Ampliar la información con recursos adicionales que ofrezcan al lector la posibilidad de profundizar en el tema de estudio, enriqueciendo tu trabajo y volviéndolo más completo e informativo.
\end{itemize}

\section{¿Qué citar?}

Es importante citar siempre que se utilice información de otra fuente. Esta información puede ser:

\begin{itemize}
    \item Una {\em cita textual} cuando se copia una frase o párrafo palabra por palabra.
    \item Una {\em paráfrasis} cuando se presenta una idea de otra fuente con tus propias palabras.
    \item Datos o estadísticas, tanto en forma de números individuales como de tablas completas o parciales.
    \item Material gráfico que no hayas generado tú, incluso en caso de rehacer una imagen de otro/a autor/a por querer cambiarle los colores o el tipo de letra, debes indicar cuál es la fuente original en qué te has basado. 
    \item Software o recurso en línea.
    \item Código fuente reutilizado de otros/as autores/as.
\end{itemize}

\section{¿Cómo citar?}

En caso de realizar una cita textual debes colocar ese texto entre comillas y en cursiva. En caso de citar una frase o proverbio, del cual no se tenga una referencia bibliográfica, se puede indicar en el texto, antes o después, quién es su autor/a. Esto también es válido para comunicaciones personales:

\begin{quote}
\begin{it}
    No olvides la famosa frase de Confucio ``Aprende a vivir y sabrás morir bien''.
\end{it}
\end{quote}

En caso de hacer referencia a una cita textual de la cual sí se tiene conocimiento del origen de ese texto, es necesario citarlo colocando una referencia a la bibliografía tras el entrecomillado. Las referencias a la bibliografía pueden tener distintos formatos, en el siguiente ejemplo puedes verlo indicado por un número entre corchetes que se enlaza a la bibliografía donde se incluyen todas las referencias bibliográficas:

\begin{quote}
\begin{it}
     Es importante mencionar un estudio que ``...revela que cualitativamente las guías docentes incluyen competencias que posteriormente no están consideradas...'' \cite{fernandez2023evaluacion}.
\end{it}
\end{quote}

Cuando realices alguna paráfrasis simplemente escribe tu texto y al final de la frase o párrafo colocas la referencia a la bibliografía:

\begin{quote}
\begin{it}
     Un estudio indica que las guías docentes consideran competencias luego se ignoran \cite{fernandez2023evaluacion}.
\end{it}
\end{quote}

Si escribes varios párrafos basados en el trabajo de otro/a autor/a coloca la referencia en el último párrafo:

\begin{quote}
\begin{it}
    El segundo hallazgo se centra en el análisis de la correspondencia entre las competencias descritas en las guías docentes y las rúbricas de evaluación. Este estudio pone de manifiesto que, cualitativamente, las guías docentes incluirán competencias que posteriormente no se reflejan en los ítems específicos de las rúbricas.
    
    Sorprendentemente, este fenómeno no es una excepción, sino más bien una tendencia generalizada a nivel nacional. La discrepancia entre las competencias propuestas y las evaluadas plantea interrogantes sobre la coherencia y la eficacia de los procesos de evaluación en los Trabajos Fin de Grado (TFG) en el ámbito de la Ingeniería Informática en España \cite{fernandez2023evaluacion}.
\end{it}
\end{quote}

En el caso de tener que hacer referencia a una figura o tabla puedes colocar la referencia bibliográfica en la leyenda de la misma. En caso de que tu figura o tabla no esté copiada textualmente y haya servido de inspiración o bien has recogido un subconjunto de la información puedes indicarlo en la leyenda:

\begin{table}[!ht]
    \begin{varwidth}[b]{0.45\linewidth}
        \centering
            \begin{tabular}{c c}
            \toprule
            \textbf{Pregunta} & \textbf{\%} \\
            \midrule
            ¿Las rúbricas consideran & \\
            la evaluación del tutor? & 59\% \\
            ¿Las rúbricas establecen & \\
            la ponderación de los ítems? &  66\% \\
            ¿Las rúbricas detalla los & \\
            rangos de calificaciones? & 34\%  \\
            \bottomrule
        \end{tabular}
        \caption{Resumen de hallazgos en las rúbricas tomados de \cite{fernandez2023evaluacion}.}
        \label{tab:rubricas}
       \end{varwidth}\hfill
    \begin{minipage}[b]{0.5\linewidth}
        \centering
        \includegraphics[scale=0.5, trim={0 5cm 5cm 5cm}, clip]{images/Mapa_prov.png}
        \captionof{figure}{Mapa de rúbricas basado en \cite{fernandez2023evaluacion}.}
        \label{fig:image}
    \end{minipage}
\end{table}

En caso de querer referenciar una página web, puedes hacerlo directamente en la bibliografía con el resto de elementos. En caso de tener pocas referencias a páginas web puedes optar por incluirlas directamente como un pie de página o entre paréntesis. Ten en cuenta que dependiendo de la página web a la que hagas referencia puede ser necesario indicar el momento en que se ha accedido a esta información, especialmente si ésta se actualiza frecuentemente:

\begin{figure*}[!ht]
    \begin{minipage}{.45\textwidth}
        \begin{it}
        Para mas información puedes consultar la página web del Instituto Nacional de Estadística (https://ine.es/index.htm).
        \end{it}
    \end{minipage}
    \hfill
    \begin{minipage}{.45\textwidth}
        \begin{it}
        Para mas información puedes consultar la página web del Instituto Nacional de Estadística$^1$.\\
        --\\
        $^1$ https://ine.es/index.htm. Accedido el 14 de febrero de 2024.
        \end{it}
    \end{minipage}
\end{figure*}

En caso de tener muchas referencias se recomienda tener un apartado de bibliografía exclusivo para los enlaces web y simplemente usar una referencia en el texto:

\begin{quote}
\begin{it}
    Para mas información puedes consultar la página web del Instituto Nacional de Estadística \cite{INE}.
\end{it}
\end{quote}

\begin{table}[!hbt]
    \centering
    \begin{minipage}{0.48\linewidth}
        \centering    
        \begin{tabular}{r|l}
            \toprule
            Referencia a & Información \\
            \midrule
            Libro & Autores/as \\
            & Título \\
            & Editorial \\
            & Año de publicación \\
            & {\it ISBN} \\
            & {\it Edición} \\
            \midrule
            Capítulo & Autores/as\\
            de Libro & Título del libro\\
            & Título del capítulo\\
            & Editorial\\
            & Año de publicación\\
            & {\it ISBN} \\
            & {\it Edición} \\
            \midrule
            Artículo & Autores/as\\
            científico & Título\\
            en revista & Nombre de la revista\\
            & Volumen \\
            & Año de publicación\\
            & {\it Número} \\  
            & {\it Páginas} \\
            & {\it DOI} \\
            \bottomrule
        \end{tabular}
    \end{minipage}%
    \hfill
    \begin{minipage}{0.48\linewidth}
        \centering
        \begin{tabular}{r|l}
            \toprule
            Referencia a & Información \\
            \midrule
            Artículo & Autores/as\\
            científico & Título\\
            en congreso & Nombre del congreso\\
            & Año de publicación\\
            & {\it Lugar} \\
            & {\it Fechas} \\
            & {\it Páginas} \\
            \midrule
            TFG & Autor/a\\
            TFM & Título \\
            Tesis doctoral & Universidad \\
            & Tipo \\
            & Año de publicación \\
            & {\it Tutor} \\
            \midrule
            Software & Nombre \\
            & Versión \\
            & {\it Enlace web} \\
            \midrule
            Página web & Título \\
            & Autores \\
            & Enlace web \\
            & Fecha de último acceso \\
            \bottomrule
        \end{tabular}
    \end{minipage}
    \caption{Información mínima y opcional (en itálica) para cada tipo de referencia a citar.}
    \label{tab:citar}
\end{table}

Si bien tienes estas tres opciones para referenciar contenido en línea, no debes usar más de una en tu memoria, elige la más conveniente y usa ese estilo para todas las referencias.

Si necesitas citar un software específico recuerda indicar la versión utilizada. Puedes adicionalmente agregar si quieres una referencia a la página web de la empresa o proyecto relacionado:

\begin{quote}
\begin{it}
    Para este trabajo se ha utilizado el lenguaje de programación Julia v1.10.0 (https://julialang.org/) \cite{bezanson2017julia}.
\end{it}
\end{quote}

En caso de reutilizar código fuente de otras personas indícalo tanto en el texto de la memoria, enlazando con la página web de donde has descargado esa información, como en tu propio código fuente mediante un comentario en la cabecera de la función o paquete donde se encuentre.

\begin{quote}
\begin{it}
    En este trabajo se ha reutilizado parte del código fuente del proyecto TSFEDL (\url{https://github.com/ari-dasci/S-TSFE-DL}), más detalles en el repositorio GitHub de este trabajo fin de grado (\url{https://github.com/mitfg/}).
\end{it}
\end{quote}

Para el resto de referencias, sean libros, artículos científicos u otros trabajos fin de carrera, la referencia bibliográfica deberá tener unos u otros componentes básicos dependiendo del tipo de referencia. En la Tabla \ref{tab:citar} puedes ver la información mínima y opcional (en itálica) para cada tipo de cita posible.
    
Existen muchos formatos de citas, algunas usan números entre corchetes para referenciarlos, otros utilizan el nombre del primer autor y el año entre paréntesis, etc. Luego, dependiendo del formato se colocarán las referencias en la bibliografía en un cierto orden: orden alfabético, en el orden en que fueron citadas, etc. Estos formatos siguen distintas normativas o estilos. Las más conocidas son:

\begin{itemize}
    \item APA (American Psychological Association): en este formato, las referencias en el texto utilizan un sistema de citación por autor y fecha. Todas las citas que aparecen en el texto deberán luego ordenarse alfabéticamente. Este estilo se diseñó originalmente para trabajos en psicología, pero se han extendido a otros campos como la educación y las ciencias sociales debido a su claridad y facilidad de uso. Un ejemplo de este formato es:
    \begin{quote}
        \includegraphics[scale=0.6, trim={2cm 20cm 3cm 3cm}, clip]{images/apa.pdf}
    \end{quote}
    \item MLA (Modern Language Association): en este formato, las citas dentro del texto no incluyen la fecha como en otros estilos, solo llevan el nombre del primer autor entre paréntesis. Es ampliamente utilizado en el ámbito de las humanidades, lengua y literatura. Un ejemplo de este formato es:
    \begin{quote}
        \includegraphics[scale=0.6, trim={2cm 20cm 3cm 3cm}, clip]{images/mla.pdf}
    \end{quote}
    \item Chicago: este estilo tiene la posibilidad de formatear las citas de dos maneras diferentes: utilizando autor y fecha o con una nota al pie. Debe su nombre a la Universidad de Chicago donde fue creado y se emplea para citar en publicaciones científicas y académicas de distintos campos de humanidades, ciencias sociales y naturales. Un ejemplo de este formato es:
    \begin{quote}
        \includegraphics[scale=0.6, trim={2cm 20cm 3cm 3cm}, clip]{images/chicagoA.pdf}
    \end{quote}
    \item Harvard: aunque este formato tiene su origen en la zoología y la biología en general, también es utilizado por los/las investigadores/as de distintas disciplinas como las ciencias sociales, la historia y las humanidades. Este sistema utiliza la información del autor-fecha para identificar una referencia bibliográfica. Un ejemplo de este formato es:
    \begin{quote}
        \includegraphics[scale=0.6, trim={2cm 20cm 3cm 3cm}, clip]{images/harvard.pdf}
    \end{quote}
    \item IEEE: en el estilo del Instituto de Ingenieros Eléctricos y Electrónicos (IEEE) las citas están numeradas entre corchetes. Toda la información bibliográfica se incluye exclusivamente en la lista de referencias al final del documento, junto al número de cita respectivo. Es el formato más utilizado en las ingenierías. Un ejemplo de este formato es:
    \begin{quote}
        \includegraphics[scale=0.6, trim={2cm 19cm 3cm 3cm}, clip]{images/ieee.pdf}
    \end{quote}
\end{itemize}

Tienes libertad para elegir el estilo que más te guste, siempre y cuando utilices el mismo formato para toda la memoria. No puedes combinar distintos formatos en un solo documento.

\section{Herramientas de gestión de referencias y citas}

Organizar las citas y la bibliografía a mano no es nada recomendable. Imagina que cada vez que agregas una nueva cita tienes que ponerla en la bibliografía en la posición correcta y revisar que el formato sea el adecuado al estilo de citas que has elegido. Es por ello que existen muchas herramientas para poder tener las referencias controladas y accesibles. Dependiendo de que programa uses para escribir la memoria (por ejemplo, Microsoft Word\texttrademark, \LaTeX\ u OpenOffice) tendrás disponibles unas u otras herramientas. La mayoría de los sistema de gestión de referencias y citas, sin embargo, suelen funcionar para cualquiera gracias a la posibilidad de importar y exportar las referencias en distintos formatos.

Los gestores de referencias y citas más populares son: EndNote (\url{https://endnote.com/es/}), Zotero (\url{https://www.zotero.org/}) y Mendeley (\url{https://www.mendeley.com/}). EndNote es una de las herramienta más completas y cómodas de usar, pero es una herramienta propietaria y es de pago. Mendeley es gratuita y te permite exportar conjuntos de bibliografías rápidamente. Zotero es la única de esta lista que es de código abierto. Existen muchas otras herramientas disponibles para descargar en Internet, con lo que te animo a explorar un poco más y elegir aquella que más cómoda te resulte. Verás la diferencia que hay entre utilizarla o tener que pasar horas poniendo las referencias y citas a mano.

En el caso de \LaTeX\ lo más apropiado es utilizar BibTeX. BibTeX es una herramienta y un formato de archivo para gestionar listas de referencias en documentos \LaTeX. Su función principal es generar automáticamente las citas y la lista de referencias en el formato seleccionado (e.g., APA, MLA, IEEE). También puedes utilizar herramientas independientes para la gestión de referencias y citas, como Mendeley, y combinarlas con \LaTeX\ y BibTeX.

\section{Consejos prácticos}

Como ya mencionamos, citar otras fuentes te permite respaldar tus argumentos. Es por ello que es imprescindible que utilices fuentes de información actualizadas, relevantes y confiables. Evita lo más posible citar la Wikipedia o Blogs de Internet no fiables. Intenta en su lugar cambiarlos por citas a libros o artículos científicos.

%No utilices generadores automáticos de texto, como ChatGPT, para buscar referencias. Muchos de ellos no están conectados a Internet, e incluso los que lo hacen pueden inventarse las referencias. Debes mantener el rigor científico en las citas que selecciones para tu memoria.

No incluyas bibliografía que nunca cites en el texto. Si bien el sistema de BibTeX controla esto, no todos los gestores lo hacen. Recuerda revisar tu memoria una vez finalizada para detectar estos problemas antes de entregarla.

Es fundamental citar las fuentes que has utilizado en tu trabajo. Si no lo haces, tu texto podría considerarse plagio y acarrear consecuencias graves.

Mantén un registro organizado de las referencias usadas desde el comienzo de tu trabajo y actualiza la bibliografía a medida que avanzas en tu proyecto. No lo dejes para el último momento ya que es posible que llegado a ese punto te hayas olvidado de las fuentes de donde hayas sacado la información y tengas que perder el tiempo volviéndolas a buscar.

Salvo que quieras poner algo literal de una cita, que ya te hemos enseñado cómo hacerlo, te recomendamos que leas el texto, entiendas la idea y lo escribas con tus propias palabras. De esta forma no habrá ningún problema y nunca estarás incurriendo en plagio. Esto es un tema muy serio, que puede incurrir en problemas legales, si no tienes cuidado. 

Por otro lado, existe una herramienta muy interesante, Turnitin, a la que está suscrita la UGR \footnote{\url{https://biblioteca.ugr.es/servicios/herramientas/turnitin}}, que permite analizar un texto y determinar la existencia de un posible plagio (con indicación de la fuente potencialmente plagiada y el correspondiente texto resaltado). Te recomendamos encarecidamente que pases tu memoria por ella y cambies los textos que puedan dar lugar a alguna duda. De esta forma entregarás tu memoria asegurándote de que no hay ningún problema.

Por último recordarte que uno de los aspectos a valorar en el baremo de los trabajos fin de grado y máster incluyen un apartado relacionado con la búsqueda y tratamiento de la información. En particular se tiene en cuenta la calidad, cantidad y variedad de las fuentes, pero también su adecuación y fiabilidad, por tanto, esfuérzate en que este aspecto quede bien.

\include{X.anexostfg}
\chapter{La revisión del proyecto y la memoria}
\label{cap:Revisión}

% Comentarios:
% ¿Meter una lista de ítems a comprobar?
% ¿Aconsejar que se la confeccione el estudiante?
% Esto lo tendremos que ordenar por grupos.

Una vez que la memoria está terminada es el momento de comprobar que está correcta y que contiene todo lo que debe contener. ¿Por qué? Porque sencillamente se suele ir con prisas en esta parte final y es necesario revisarla para asegurarnos de que no nos falta nada. Para tal fin, te aconsejamos que hagas una lista de ítems ({\it checklist}) para comprobar y que determines si cumples con todos ellos. Nosotros te proponemos aquí algunas cosas que deberías comprobar para que tú y la persona que te tutoriza añadáis lo que consideréis pertinente con objeto de no dejar nada relevante.


\begin{itemize}
  \item Sobre el formato de la memoria:

  \begin{todolist}
    \item Se sigue la normativa de portadas (logos, colores, etc.) y prólogos (autorización, resumen, etc.).
    \item La portada contiene el título de TFG, el nombre autor y el de la persona que te tutoriza, así como la fecha de entrega o, al menos, el curso académico en el que se entrega.
    \item Los índices de contenidos para capítulos, secciones, tablas y figuras deben estar actualizados. Lo más sencillo es que los generes de forma automática.
    \item No quedan tablas cortadas al final de una página.
    \item Si hay títulos de secciones al final de una página, debajo de ellos debe haber al menos un párrafo.
    \item Se utilizan los mismos estilos y tipo de letra en toda la memoria, a no ser que quieras resaltar algo como citas literales, fórmulas, ecuaciones o código.
    \item Todos los márgenes deben tener las mismas dimensiones, salvo casos excepcionales.
    \item Las páginas en blanco que dejes deben ser intencionales, por ejemplo, para que cada capítulo empiece en página impar.
    \item Las páginas están numeradas (correctamente).
    \item Existen encabezados y pies de páginas y están correctos.
 \end{todolist}

  \item Sobre la estructura:
  \begin{todolist}
    \item Están todas las secciones que, según el tipo de TFG, deben aparecer. 
    \item El resumen claramente describe de forma concisa qué se ha realizado en el TFG.
    \item La introducción establece claramente el contexto y la motivación necesarias para entender el problema entre manos.
    \item Están los objetivos. Son claros y concisos.
    \item Están las conclusiones, donde se muestran los resultados y contribuciones del TFG, y los trabajos futuros.
    \item Está la bibliografía.
    \item Están todos los anexos necesarios.
    \item En caso de que en el TFG se utilice código fuente, éste está enlazado en la memoria (por ejemplo, GitHub, GitLab).
    \item Es interesante incluir un anexo con el glosario de términos y abreviaturas empleados en la memoria.
  \end{todolist}
  
  \item Sobre la bibliografía:

  \begin{todolist}
    \item Todas las referencias de la bibliografía deberían aparecer citadas en el texto.
    \item Todas las citas bibliográficas del texto deben tener una referencia asociada. No puede haber referencias sin citar.
    \item Las referencias bibliográficas deben estar completas y también todas en el mismo formato.
    \item Las referencias de recursos de Internet deben indicar también la fecha de la última consulta.
  \end{todolist}

  \item Sobre las figuras y tablas:

  \begin{todolist}
    \item Todas las figuras y tablas deben estar numeradas secuencialmente y referenciadas/citadas en el texto.
    \item Todas las figuras y tablas deben tener un título descriptivo ({\it{caption}}), que en las primeras suele situado ir debajo y en las segundas arriba.
    \item Los títulos de figuras y tablas deben estar en la misma página de la figura o tabla.
    \item Si se usan figuras hechas por otra persona deberían ser citadas/usadas correctamente, por ejemplo en el pie de figura ``Tomado de [x]'',``Extraído de [x]'' o ``Fuente: [x]'', si no, indicar ``Elaboración propia''.
    \item Las imágenes y las tablas están dentro del espacio del cuerpo de la página y ninguna se desborda hacia los márgenes. 
  \end{todolist}

  \item Sobre las licencias:

  \begin{todolist}
    \item He incluido en la portada o primera página el tipo de licencia que deseo para la memoria.
    \item He incluido en el código realizado por mí el tipo de licencia que deseo.
    \item Si he usado figuras, código o documentos multimedia de otros, he citado las fuentes y me he asegurado de podía usarlas con su licencia original.
  \end{todolist}

  \item Y por último:

  \begin{todolist}
    \item Uso de un lenguaje adecuado y 
    \item que no haya errores ortográficos ni gramaticales, por lo que más quieras (¡¡¡Pásale el corrector ortográfico, por favor!!!).
  \end{todolist}

\end{itemize}

Además, también te recomendamos que te descargues las rúbricas de evaluación del tutor y de la comisión y compruebes que tratas, de una u otra forma, todos los puntos que aparecen en ella. Esto es importante porque si te dejas alguno sin tratar, los miembros de la comisión pueden preguntarte por él. Quien evita la ocasión, evita el peligro ;-)
\chapter{La elaboración de la presentación en la defensa} \label{cap:elaboraciónPresentación}

% [Autores: Pablo, Alberto]

¡Ya queda menos para poder presentar el proyecto! Has llegado a los objetivos que te habías propuesto (tu programa ya funciona, tus resultados experimentales son geniales, etc), has entregado la memoria en tiempo y forma y ya estás listo para enseñar tus avances al mundo.

Después de todo el esfuerzo, de todas las horas, de todos los cabezazos contra la pantalla, resulta que solo tienes veinte o veinticinco minutos ante un tribunal para contarles lo que has hecho. Además, seamos sinceros, muchas veces el tribunal se va a leer la memoria de pasada mientras estas haciendo la presentación, así que te juegas mucho dependiendo de cómo lo hagas en esos veinte minutos y lo que enseñes y no enseñes. En este capítulo te enseñaremos cómo preparar la presentación, es decir el archivo (el PDF, el ODP, el PPTX) que proyectarás en el aula, mientras que en el siguiente capítulo nos centraremos en cómo preparar la defensa de esa presentación.

\section{Centrarte en lo importante}

El primer consejo que te vamos a dar es que el tribunal tiene que mirarte a ti, no a las diapositivas. Tú eres la estrella de la función, y la presentación es un apoyo a lo que estás diciendo. El segundo consejo, y más importante todavía, es: \textbf{no te pases del tiempo}. Alguien del tribunal tiene un cronómetro encendido y no hay nada que quede peor en una defensa que decirle al estudiante ``lo siento, tienes que cortar''. Así que no prepares cincuenta diapositivas y quieras contarlo todo con miles de pelos y señales. Por ejemplo, entre otras muchas cosas, puedes ahorrarte diapositivas de bibliografía, que aportan poco en la defensa.

Aunque dependerá del tipo de TFG que realices, una posible estructura puede ser la siguiente:
\begin{itemize}
\item Título del proyecto, fecha, tu nombre y tu correo electrónico y el nombre de tu director o directora. Logos de la universidad y escuela, para dejarlo más profesional.
\item La segunda diapositiva debe ser un índice numerado de las secciones de la presentación.
\item Una o dos diapositivas de introducción y contexto.
\item Una diapositiva definiendo los objetivos.
\item Una diapositiva resumiendo el estado del arte.
\item Una o dos diapositivas de planificación y metodología.
\item Si tu trabajo es de desarrollo:
    \begin{itemize}
    \item Dos o tres diapositivas del diseño e implementación, diagramas sobre todo.
    \item Una diapositiva hablando de pruebas.
    \item Capturas de la UI, aunque mejor haz una demo.
    \end{itemize}
\item Si tu trabajo es experimentación o investigación:
    \begin{itemize}
        \item Dos o tres diapositivas describiendo el método y mostrando los resultados (gráficas sobre todo). Dependiendo del número de experimentos que realices quizás necesites más. Pero recuerda, casi siempre menos es más.
    \end{itemize}
\item Una diapositiva de conclusiones, incluyendo enlace a repositorio del código, si lo hubiera.
\item Una diapositiva de despedida, con un texto parecido a ``Muchas gracias por su atención''.
\end{itemize}

Esto nos lleva a una presentación con quince a veinte diapositivas aproximadamente. Si te centras un minuto en cada una, y créenos, un minuto pasa muy rápido, ya lo tienes listo.

Respecto a los títulos de las diapositivas, mucha gente cambia el título de cada una para que sea como un titular de periódico, que le da un toque más innovador.  Es decir, en vez de poner títulos genéricos como ``Introducción (I)'' o ``Introducción (II)'' puedes poner una diapositiva (que no usaremos en el conteo) con el texto centrado ``Introducción'' y pasar rápidamente a las siguientes tituladas ``El problema de los tres cuerpos es muy difícil de resolver'' y ``Se han utilizado algunas cosas sin éxito''. Fíjate como parecen titulares de periódico, pero seguimos siendo conscientes de que estamos en la introducción. De hecho, como curiosidad, fíjate también en los nombres de las secciones de este capítulo, no hace falta leer el texto para sacar la idea principal de cada una. De esta forma, estamos ofreciendo una especie de idea principal del contenido de la diapositiva que el espectador agradecerá enormemente.

\section{La presentación no es un karaoke}

Si vas a leer lo que pone en las diapositivas envíala por correo y nos ahorramos tiempo. Y recuerda que menos es más. Así que quita la broza. En tu memoria quizás hayas escrito algo como 

``\textit{El objetivo de este proyecto es demostrar que, bajo ciertas condiciones, utilizar el algoritmo Williamsito, creado por Williams [11] permite obtener mejor rendimiento para resolver el problema de los tres cuerpos [12] reduciendo el tiempo de computación}''. 

Pues eso, en la presentación quita la broza: 

``\textit{Williamsito reduce el tiempo para resolver el problema de los tres cuerpos}''. 

Ya está, dicen exactamente lo mismo, pero más rápido. El tribunal lo habrá pillado enseguida y podemos pasar a otra cosa.

Especialmente importante será la diapositiva de conclusiones. Es la última carta que tienes en la manga para que el tribunal vea que has hecho un buen trabajo.

\section{El poder de lo visual}
Vamos a mejorar el texto anterior: 

``\textit{Williamsito \textbf{reduce el tiempo} para resolver el problema de los tres cuerpos}''. 

Fíjate cómo hemos usado un componente visual (la negrita) para ir directamente a la idea de la frase. No te cortes en usar técnicas tipográficas para facilitar la lectura (negrita, cursiva, color), pero tampoco te pases. Resaltar de una a tres palabras por frase es más que suficiente. Piensa que lo que no haya que resaltar quizá es que directamente no debería estar en la diapositiva.

A veces incluso se puede sustituir el texto con una imagen o un pictograma que represente la idea. Por ejemplo, en vez de poner una lista de los componentes de tu sistema y lo que hacen, utiliza un diagrama. De un solo vistazo vemos que tiene 5 componentes y se comunican usando MQTT. Perfecto, todo claro. 

Pero igual que antes, no pongas diagramas que no aporten y te pongas a explicarlos. El diagrama de clases o el de E/R, por ejemplo, generalmente no aportan absolutamente nada en la presentación, ya has aprobado las asignaturas que te lo evaluaban. Tampoco pongas código fuente, a menos que sea realmente necesario.

Y ojo con los colores. Utiliza paletas ya establecidas. Existen webs que te permiten coger un grupo de cuatro colores que combinan bien. Por ejemplo Coolors.co \footnote{\url{https://coolors.co}}, pero hay otras muchas. A menos que sepas de diseño gráfico y teoría del color no te fíes de tu criterio artístico. Esto es especialmente importante si estás visualizando datos. Además, puede haber personas daltónicas en el tribunal.

Además, texto oscuro en fondo claro hace que el público mire a la diapositiva, pero texto claro sobre fondo oscuro hace que miren al orador.

Y para terminar un consejo muy fácil de aplicar y que queda muy bien: añade el número de diapositiva actual y el total de diapositivas en una esquina (Por ejemplo, \textit{4 de 20}). De este modo el tribunal podrá orientarse y saber cuánto te queda o anotar la diapositiva sobre la que hacerte una pregunta. Si, por ejemplo, ven que te pasas un poco de tiempo pero solo te queda una diapositiva quizás no te interrumpan. También servirá para que en el turno de preguntas puedas moverte a una diapositiva concreta si te lo piden.

\subsection{Memes y gifs}

Actualmente, aunque pueda resultar poco ortodoxo, hemos acumulado una gran cantidad de información a través de memes y gifs. Si encuentras uno que se ajuste a lo que quieres decir, no dudes en incluirlo en tu presentación. Eso sí, no abuses de ellos, ya que pueden hacer que tu presentación parezca poco seria y considera que el salto generacional entre el tribunal y tú puede hacer que no entiendan la referencia. 

\section{Show. Don't tell.}

Esto es un aforismo que se usa mucho en el guión cinematográfico. En vez de escribir qué hace tu aplicación prepara una demo de dos o tres minutos en la que veamos cómo funciona. No hace falta que muestres todo, por ejemplo, cómo crear usuarios, que es algo trivial, sino la parte interesante.

Si tienes miedo de que algo falle (por ejemplo, si el servidor está en tu casa), prepara un vídeo, pero no grabes tu voz explicándolo, da la explicación en directo mientras que visualizas el vídeo. En el turno de preguntas puedes ofrecer al tribunal probar la aplicación o hacer la demo en vivo para evitar sospechas y actos de fe sobre el trabajo mostrado en el vídeo.

Consensúa esta presentación con la persona que te tutoriza el TFG hasta que estéis ambos contentos con ella. Confía en su criterio, ya que ha visto muchas presentaciones pero recuerda que eres tú el que va a hacer la presentación así que procura estar cómodo con ella aunque no sigas todos los consejos que te hemos dado o te ha comentado el tutor.



\chapter{La defensa}
\label{cap:Defensa}
% [Autores: Juanma, Alberto]

\section{La estructura del acto}

Una vez que finalice el periodo de entrega del TFG, el centro publicará las comisiones de evaluación y los TFG asignados a cada una. La comisión está compuesta por tres miembros. A saber: un presidente, un secretario y un vocal. Una vez nombrada, el presidente, normalmente, convocará a los estudiantes, informándoles de la fecha, hora y lugar del acto de defensa. Este se divide, como bien hemos indicado anteriormente, en dos fases: la presentación del trabajo y la discusión con los miembros de la comisión.

El acto es público, por lo que puedes entrar a ver a tus compañeros. El que lo hagas o no dependerá de ti, de si te pones nervioso viendo a otros o no, o de si quieres estar tranquilo y concentrado antes. Si te quedas fuera de la sala, quédate cerca de ella. El secretario saldrá a llamarte cuando sea tu turno. En ese momento, entras en la sala, saludas al tribunal (buenos días, buenas tardes, hola) y te diriges al lugar donde expondrás, conectas tu portátil (no, no hay disponible ninguno para tu uso, tienes que llevarte el tuyo) al proyector y preparas la presentación. 

El presidente, en ese momento, dirá algo así como que se va a dar comienzo a la defensa del TFG titulado X realizado por el estudiante Y. Y te informará de la estructura del acto, indicándote explícitamente el tiempo que tienes disponible para la exposición (normalmente veinte minutos) y el tiempo, aproximado éste, de la ronda de preguntas y comentarios (otros veinte minutos habitualmente). Tras la presentación, que comenzarás sólo cuando el presidente te dé la palabra, éste dará paso al secretario y al vocal, los cuales te plantearán sus cuestiones, participando finalmente el propio presidente en dicha ronda. Una vez finalizada la segunda fase, éste te dirá que se ha concluido el acto, te dirá que se te comunicará la nota en uno o dos días y, finalmente, te dará las gracias. Lo habitual es que tú también des las gracias. Y... se acabó lo que se daba. Ya has terminado, recoges y abandonas la sala, o bien te quedas a oír a los compañeros siguientes.

Como ya hemos indicado, es un acto público, por lo que también pueden asistir tus familiares y amigos. El que lo hagan o no, dependerá también de ti: si tus padres y hermanos o amigos quieren ir a verte y tú no te pones nervioso con su presencia, es un momento bonito para que te vean en acción y lo compartan contigo. A veces te puedes sentir más tranquilo/a con su presencia. Si no es así, mejor les dices que prefieres que no vayan y no hay ningún problema. De cualquier forma, el público en general y los familiares y amigos en particular, deben permanecer en silencio, sin intervenir en ningún momento y sin realizar comentarios de ningún tipo a las preguntas de la comisión ni a tus respuestas (vamos que no se lleven pancartas ni pompones, ni pongan caretos ni resoplen cuando algún docente haga algún comentario). Tampoco es normal que se aplauda cuando finalices. Los abrazos y felicitaciones déjalos para cuando estés fuera de la sala.

\section{La importancia de la defensa}

%https://www.youtube.com/watch?v=AK_xGgGSdCo

Una vez entregado el TFG, la siguiente y última fase en el proceso es la defensa de tu trabajo. Es conveniente que descanses y desconectes uno o dos días antes de ponerte a trabajar en la confección del material que usarás en la defensa. Te permitirá hacer esas tareas mucho más tranquilamente, con más concentración, perspectiva y ganas. Has trabajado muy duro durante todo el curso y en especial en las últimas semanas previas a la entrega. El cansancio y el estrés se acumulan. Y es oportuno hacerlo porque este acto es el momento de la verdad donde tu TFG será evaluado y tienes que estar fresco/a y en plenitud de condiciones.

La defensa y su preparación tienes que tomártelas en serio. Es el principal acto académico que vas a realizar en tus estudios y de su resultado dependerá la nota de tu proyecto. Tu trabajo ya está terminado y en este acto lo que tienes que hacer es convencer a los miembros de la comisión de que es un buen trabajo. Puedes haber realizado un trabajo excepcional pero si no lo ``vendes'' bien, la nota que tengas no será la que verdaderamente refleje la calidad de tu TFG (y esto es un hecho que, lamentablemente ocurre muy frecuentemente). Imagina que eres un vendedor de motos y tienes la mejor del mercado. La moto no se vende por sí sola, tienes que hacerle ver al cliente todas las magníficas prestaciones que hacen que sea una de las mejores del mercado, si no la mejor, y conseguir que la compre. Si no lo haces así, el cliente no se va a enterar y no vas a poder ``venderle la moto''. Por tanto, tan importante es, al elaborar un buen TFG, el producto y la memoria, como comunicarlo correctamente a los miembros de la comisión. 

Pero ojo, ``vender la moto'' no es ``vender humo''. Una cosa es que emplees todas las técnicas de comunicación a tu alcance para mostrar la calidad del producto y el trabajo desarrollado, y otra es que las utilices para tratar de engañar a la comisión, exagerando o engordando tu trabajo. Ten cuidado porque estas cosas son muy fáciles de detectar y te pueden poner en un aprieto, sobre todo porque los miembros de la comisión han leído tu memoria y saben lo que has hecho y además disponen de un informe del tutor sobre tu trabajo.

Por tanto, tu capacidad de comunicación jugará un papel importante en esta fase. Mediante una comunicación efectiva serás capaz de expresar de forma clara, concisa y efectiva, tus ideas, facilitando su comprensión por parte de la comisión. Las habilidades de comunicación podemos dividirlas en dos: la verbal y la no verbal. La primera implica al lenguaje hablado (uso de vocabulario, gramática, volumen y tono de voz); la segunda, tiene que ver con la forma de comunicar sin hablar, es decir, la parte más física: las posturas, gestos, miradas, etc. Estos dos tipos los describiremos con más profundidad en la siguiente sección, pero sé consciente que ambos son importantes en la defensa. Esta comunicación la apoyarás en una presentación clara, concisa y bien estructurada, como se ha indicado en el capítulo anterior, en la fase de exposición de tu trabajo. También tendrás que hacerlo sin apoyo en la fase de discusión con la comisión, en la que te realizarán preguntas y comentarios sobre tu trabajo, y tendrás que responder de forma clara, concisa y convincente.

Por tanto, si comunicas de forma efectiva, no vas a tener problema en hacer ver el trabajo que has realizado y, por tanto, la comisión será capaz de entender lo bueno que es. 

Algunos consejos para preparar la defensa son los siguientes:

\begin{itemize}

    \item Prepara la defensa con la persona que te tutoriza. Esto es un proceso iterativo en el cual inicialmente te indicará las líneas generales con las que debes hacer la presentación. Seguidamente, prepara un primer borrador y se lo pasas al docente para que lo evalúe. Te hará los comentarios pertinentes. Si tienes dudas sobre ellos, habladlo y llegad a acuerdos. Refléjalos en la presentación y vuelve a enviárselos. Este proceso se repetirá hasta que haya una estabilidad en la presentación y no haya cambios. En ese momento puedes decir que tienes lista la presentación y puedes comenzar a ensayar.

    \item No sólo prepara una presentación sino también una demostración, si el tipo de TFG que has desarrollado se presta a esto. Puedes hacer un vídeo mostrando las prestaciones de tu software o una demostración en vivo del mismo. De cualquier forma, acuerda con la persona que te tutoriza lo más relevante de esta demostración, el tiempo que le vas a dedicar y el momento en que lo harás, y prepárala concienzudamente también. 

    \item Normalmente afrontamos este acto con inseguridad y miedo. Para evitarlo, la única receta que hay es ensayar la presentación una y otra vez hasta que tengas seguridad y confianza, y sepas qué decir en cada momento. También imagina, visualiza, cómo va a ser la defensa, la sala, los miembros, dónde estarás tú situado, los gestos que harás, cómo te moverás. Este entrenamiento te dará confianza en ti mismo.

    \item Cronométrate en los ensayos. No puedes pasarte del tiempo indicado para la exposición. Si lo haces, el presidente te podrá decir que te has excedido del tiempo asignado y retirarte la palabra. Esto implica que puede haber cosas importantes que te dejes sin comentar, con el consiguiente problema para que los docentes que te evalúen comprendan la dimensión, dificultad, alcance, etc. de tu trabajo. Además, aunque te dejaran más tiempo, dejarías mala impresión en los miembros de la comisión y hay un ítem de la rúbrica que evalúa esto. Por tanto, cronométrate y ajústate al tiempo. Esto puede implicar que elimines diapositivas, que las hagas más breves, que contengan menos texto, que simplemente pases por encima de algunas diciendo apenas una frase. Cualquier cosa para ajustarte al tiempo pero, ojo, sin que el mensaje pierda contenido importante.

    \item Piensa en posibles preguntas que los miembros de la comisión podrían hacerte y prepara las respuestas. Estas suelen ser del tipo:

    \begin{itemize}
        \item ¿Qué es lo que has aprendido?
        \item ¿Por qué has hecho esto de esa forma (y no de esta otra)?
        \item ¿Por qué no has tenido en cuenta esto?
        \item ¿Cuáles son las limitaciones de tu trabajo?
        \item ¿Qué aporta tu trabajo?
        \item ¿Por qué has elegido esta opción y no esta otra?
        \item ¿Por qué no has liberado tu código? 
        \item ¿Por qué has escogido esta licencia y no otra?
        \item ¿Por qué no has incluido X en tu revisión del estado del arte?
        \item ¿Por qué no has mencionado ni valorado el uso de esta tecnología, metodología, método o herramienta?
        \item ¿Cómo abordarías la propuesta X de tus trabajos futuros? 
        \item ¿Qué es lo que más te ha costado?
        \item ¿Qué desafíos técnicos encontraste y cómo los superaste?
        \item ¿Qué métricas utilizaste para medir el éxito de tu proyecto?
        \item ¿Cómo podría aplicarse tu proyecto en un entorno real?
        \item ¿Qué escalabilidad tendría el software que has desarrollado?
       
    \end{itemize}

    Ten en cuenta que te podrán preguntar sobre cualquier cosa que no esté clara en la memoria o no hayas incluido. Por eso es tan importante revisarla bien antes de entregarla.

     \item También te pueden hacer comentarios sobre la memoria y presentación del tipo: `No está bien explicado el propósito del TFM''. ``Faltan objetivos relacionados con tu formación''. `Deberías haber añadido una tabla comparativa en el estado del arte''. ``Hubiera sido mejor utilizar la tecnología X'', `En el diagrama de clases hay estos errores...''. En esos casos, dales la razón si la tienen e indica que lo tendrás en cuenta para futuros trabajos. Si te dan pie para responder, explica o completa lo que te están indicando. Por ejemplo, `La tecnología X también es óptima pero he escogido Y porque además me permite hacer Z, cosa que con X es más costoso, o porque estoy más familiarizado con ella''.  

    \item Como ya se ha explicado en el capítulo anterior, la presentación es un mero elemento de apoyo para presentar tu proyecto. Es un guion para que tú sepas qué tienes que decir. Por tanto, no leas su contenido. Y para que puedas decir todo lo que debes decir, cuando diseñes la presentación hazte un documento en el que escribas las cosas que tienes que comunicar en cada diapositiva y apréndetelo. Pero en la exposición no hables como un loro que suelta el texto de memoria, ve con tranquilidad, haciendo pausas, dirigiéndote a los miembros de la comisión, mirándolos, y explicando las cosas. No las recites. Lo de aprenderse todo lo que tienes que decir es para evitar dejarte algo que sea importante. 

    \item Practica la exposición con la persona que te tutoriza, si es posible. Él o ella te indicará aspectos a mejorar de la presentación y de la comunicación. Además podréis simular el turno de preguntas y te hará algunas que son habituales en las comisiones y otras específicas que pueden surgir tras exponer la presentación. Y, sabiendo de antemano quiénes serán los examinadores, también te podrá indicar algunas típicas de estos profesores. También te hará comentarios sobre las respuestas que das, tanto en forma como en contenido.

    \item Practica la exposición con tus familiares y amigos. Ellos no van a entender la parte técnica, pero sí que te podrán dar una realimentación muy valiosa sobre la forma de expresarte, moverte, gestos, muletillas o cualquier otra cosa que les llame la atención. Haz caso a sus comentarios y ensaya de nuevo, ya sin ellos, procurando evitar las situaciones negativas que te han indicado. Si no tienes posibilidad de tener espectadores en los ensayos, grábate tú con el móvil y luego analiza minuciosamente toda la exposición para ver qué cosas tienes que mejorar.

    \item Si es posible, por las fechas y horarios, asiste a alguna otra defensa. De esta manera podrás ver en directo todo el proceso, la exposición que hace el estudiante, las preguntas y cómo responde. Esto te servirá para entender el acto y para quitarte el miedo, pues es una situación que, aunque importante y algo más relevante, ya habrás realizado en múltiples ocasiones en tu grado.

    \item Procura preparar la presentación, llevar a cabo las correspondientes modificaciones, y los ensayos pertinentes con tiempo, de tal manera que no necesites nada más que dar un pequeño repaso a la misma el día anterior de tu defensa. Estarás más tranquilo/a ya que llegar al día de antes sin tenerla bien preparada sólo es una fuente de nervios y tensión que te perjudicará. Ese día descansa y come especialmente bien. 
    
\end{itemize}

Y algunos otros para el momento de la defensa:

\begin{itemize}
    \item Imaginamos que en el momento de comenzar tendrás nervios. Eso es normal. No te preocupes. Siempre es mejor tener un poco de tensión que ir súper relajado. Esos nervios se irán disipando conforme avances. Al comenzar, respira hondo y empieza con tu exposición. Céntrate en contar qué y cómo lo has hecho de una manera lo más efectiva posible.

    \item Ha sido un periodo largo de trabajo duro y ahora llega el momento que lo culmina. Nadie sabe más que tú del trabajo que has hecho, del cual debes sentirte orgulloso/a, cosa que debe infundirte seguridad, por tanto, disfruta de la presentación. 

    \item Intenta captar continuamente la atención de los miembros de la comisión empleando tanto un buen diseño de las diapositivas, tal y como se ha explicado en el capítulo \ref{cap:elaboraciónPresentación}, como técnicas verbales, como, por ejemplo, usando preguntas que lanzas a la comisión, pero que no esperas que respondan, cambios de volumen y tono, etc.

    \item Llévate un guión que será de ayuda si en algún momento te quedas en blanco o no te acuerdas de por dónde seguir, pero no lo tengas en la mano. Déjalo en la mesa a un lado.

    \item Si los miembros del tribunal no te están mostrando en algún momento atención, no te preocupes. No significa que no le interese tu presentación ni que hayan dejado de escucharte. Simplemente estarán mirando algo en la memoria o anotando alguna cuestión a hacerte. Tú no te pongas nervioso/a ni pienses que no les está gustando tu presentación y sigue con ella, mirándolos como si ellos te estuvieran mirando y escuchando también.

    \item Tanto en la presentación como en las respuestas sé claro/a y ve al grano. No hay mucho tiempo y si te vas por las ramas te vas a quedar sin tiempo para explicar cosas importantes. Y sé honesto/a y sincero/a. Si algo no lo has hecho, por ejemplo, y te preguntan, di que no lo has hecho y el porqué. No te inventes nada porque los miembros de la comisión se darán cuenta y te meterás en un lío. 

    \item Ve con tranquilidad al acto de defensa. En la presentación los nervios de durarán el primer minuto, luego disfruta del momento. En la ronda de preguntas, también debes estar tranquilo/a porque nadie sabe más que tú de tu proyecto, porque lo has hecho tú.

    \item Pon el móvil en modo avión y no lo sitúes a la vista. La única excepción a esto último es que lo quieras emplear como temporizador para ver el tiempo de exposición, aunque si la tienes bien ensayada, tampoco es necesario el móvil y ofreces una mejor imagen que si estás consultándolo continuamente para ver cuánto tiempo te queda.
\end{itemize}

En \cite{vallejo2009defensa} tienes algunos otros consejos que te podrán ser útiles en este acto de la defensa.

\section{La comunicación con la comisión}

La forma de interactuar con los miembros de la comisión es también muy importante en el acto de la defensa. En esta sección te hacemos varias recomendaciones sobre este asunto.

Si el presidente de la comisión no te ha presentado, entonces, tras darte la palabra y tu agradecerla, preséntate tú mismo e indica seguidamente que vas a presentar tu trabajo fin de grado, con el título correspondiente. Si ya te ha presentado, entonces da las gracias y di que comienzas con la presentación de tu TFG (y no vuelvas a presentarte). 

Cuando finalices tu presentación, puedes decir algo así como ``y con esto finalizo la presentación de este TFG y quedo a disposición de la comisión para resolver cualquier duda o comentario que deseen realizar'', y esperas a que tome la palabra el presidente para organizar la segunda fase del acto. En ese momento comenzará la ronda de preguntas. 

Lo habitual es dirigirse a los miembros de la comisión de usted y siempre con el máximo respeto. Invócalos mediante la denominación de profesor y su apellido (profesora García). Seguramente conoces a los docentes que estén en la comisión. Alguno te habrá dado clase en alguna asignatura del grado y tienes cierta confianza. En ese caso, puedes bajar algo el nivel de formalidad a la hora de dirigirte a ellos aunque nunca entrando en el colegueo. En ese caso puedes llamarlos por su nombre de pila. Si el presidente te habla de Usted, lo normal es que hables tú también en ese grado. Si lo hace de tú, decide si hablarles también de tú o de Usted. 

En relación a las preguntas y comentarios, como ya hemos indicado anteriormente, responde siempre yendo al grano y de forma clara. No des rodeos ni respondas otra cosa si no sabes qué decir o no quieres decir algo. Habla con sinceridad. Si no entiendes una pregunta di que no has comprendido qué quieren decirte y que, por favor, te la repitan. Si no estás de acuerdo con algún comentario, siempre con educación y de forma cortés, rebátelo y ofrece las razones por las que discrepas. No tengas miedo a hacerlo así. Eso es una discusión académica y es un punto a tu favor, ya que estás dejando claro a la comisión que tienes los conocimientos y la capacidad para defender lo que has hecho y el porqué.

Sé receptivo/a a los comentarios, críticas y sugerencias. No te los tomes a mal ni de forma personal. En algunas situaciones puedes responder que gracias por el comentario y que lo tendrás en cuenta para mejorar la aplicación, por ejemplo. 

%Y si en las preguntas aparece una donde la respuesta implica que ha sido la persona que te tutoriza la que ha tomado una decisión concreta, dilo también sin problema alguno. 

Si hay alguna respuesta que pueda encajar con ``porque me lo dijo o indicó mi tutor'', no tengas reparo en expresarlo de ese modo. No obstante, muestra tu opinión personal fundamentada al respecto, tanto si coincide, como si no, con lo que indicó la persona que te tutoriza. 

No tengas miedo de preguntar cualquier cuestión o pedirles ayuda en algún tema (por ejemplo, si al conectar el ordenador al proyector hay algún problema). Los docentes de la comisión te responderán con toda amabilidad y te ayudarán en la medida de sus posibilidades. También son conscientes de que se pasan algunos nervios en este tipo de actos y, por tanto, serán comprensivos y te intentarán tranquilizar.

Ten también presente que los miembros del tribunal no van a preguntarte con mala intención, simplemente están haciendo su trabajo y deben plantear ciertas cuestiones para entender detalles que no les han quedado claros o conocer tu capacidad para desenvolverte en estas situaciones. En definitiva, tienen que tener toda la información disponible para poder evaluar objetivamente tu trabajo.

\section{Cuestiones de protocolo}

La principal cuestión en este apartado es la vestimenta. ¿Qué me pongo? ¿Voy muy arreglado/a? ¿Voy como normalmente visto? La respuesta es sencilla pero a la vez complicada: irás vestido/a de la forma adecuada para ofrecer la imagen que quieres dar. Por simplificar, recomendamos evitar camisetas, tirantes, bermudas, pantalones cortos y ropa deportiva, ya que es un atuendo demasiado informal para una presentación pública de este calibre. ¿Irías como vistes normalmente a una entrevista de trabajo? No, ¿verdad? Pues esa es la idea en este acto. 

Uno de los aspectos que da más información sobre ti es tu forma de vestir, ya que refleja tu personalidad y esta también permite a los demás construir una primera impresión sobre ti en muy poco tiempo. Y lo que quieres es caer bien a los miembros de la comisión y que se lleven una primera impresión de que eres  profesional.

En general, simplemente debes ir vestido/a acorde al acto en el que vas a participar. Si fuera el de defensa de una tesis doctoral, claramente deberías ir de traje, pero esta presentación pública, aunque importante, podríamos decir que no tiene la categoría de la defensa de un trabajo doctoral. Por tanto, debes ir arreglado/a pero no es necesario súper arreglado. Un pantalón de vestir y una camisa podrían ser suficientes. Evalúa también una falda y una blusa o camisa, o un vestido como posibilidades. En definitiva algo con lo que te sientas cómodo/a y que sea uno o dos puntos más formal de la forma en que normalmente vistes. Procura no usar ropa llamativa o accesorios que puedan distraerte a ti o a los miembros del tribunal. Y ni qué decir tiene que debes acudir bien aseado/a.




%0) La importancia de la defensa -> Juanma y Alberto
%1) Estructura del acto -> Juanma	
%   mencionar que es un acto público (jugador 12? o motivo de nervios extra)

% https://www.researchgate.net/publication/301587876_Writing_and_Presenting_a_Dissertation_on_Linguistics_Applied_Linguistics_and_Culture_Studies_for_Undergraduates_and_Graduates_in_Spain

%2) Comunicación con el tribunal -> Juanma
%   “intro, si no la hay la haces tú”
%   Turno de preguntas -> cómo responder a las preguntas
% 3) Comunicación oral  -> Alberto
%   lectura de hojas vs memorización, etc.
% 4) Comunicación no verbal -> Alberto
% 5) Duración -> respetar los límites -> ALberto
% 6) Demos y vídeos -> postura común de demo + vídeo -> Alberto	
%   comentar brevemente ventajas/desventajas de demo + vídeo vs vídeo + demo
%  7) Cuestiones de protocolo -> Juanma
%   vestimenta adecuada, aseo, [qué imagen proyectar de nosotros]

%https://ucm.es/data/cont/media/www/pag-135806/Defensa%20del%20Trabajo%20Acade%CC%81mico.pdf



\bibliographystyle{bababbrv-fl}
\bibliography{bibliografia_libro}

%\appendix
\appendix
\chapter{Propiedad intelectual y licencias}\label{anexo:licencias}

En este anexo discutiremos sobre la propiedad intelectual de tu TFG y los distintos tipos de licencias que puedes utilizar para su desarrollo.

\section{Propiedad intelectual}
\label{sec:propiedadinte}
En esta sección comentaremos algunos aspectos importantes de cara a llegar a explotar el producto o servicio desarrollado en el TFG. Es especialmente útil para aquellos TFG que se quieran desarrollar con la colaboración de empresas o terceros.

\subsection{Normativa de la universidad}

Cada universidad tiene su propia normativa respecto a la propiedad intelectual e industrial. Por ejemplo, según la \textit{Normativa sobre los Derechos de Propiedad Industrial e Intelectual derivados de la actividad investigadora de la Universidad de Granada}\footnote{\url{https://www.ugr.es/sites/default/files/2017-08/NCG1151.pdf}} (aprobada en la sesión ordinaria del Consejo de Gobierno de 31 de enero de 2017), los derechos de tu TFG pueden ser tuyos, o compartidos con la persona que te dirige, dependiendo de su implicación.

En el artículo 6 de dicha normativa podemos leer lo siguiente:

\begin{itemize}
\begin{it}
\item b) En el caso de resultados de investigación generados por estudiantes de
la Universidad de Granada bajo la dirección, coordinación, colaboración
o tutorización efectiva de personal de esta Universidad, la titularidad y
propiedad de dichos resultados, así como de los derechos de propiedad
industrial e intelectual derivados de los mismos, corresponderá en
régimen de cotitularidad a la Universidad de Granada y al estudiante en
la proporción en que hubiese contribuido cada parte al resultado,
teniendo en cuenta tanto las aportaciones económicas como
intelectuales relevantes. En caso de imposibilidad de determinar la
contribución al resultado de investigación, se presumirá que la
titularidad corresponde al 50\% a cada uno de ellos.

\item c) En el caso de resultados de investigación generados por estudiantes de
la Universidad de Granada en los que el personal de dicha Universidad
se restrinja a su encargo y/o evaluación, la titularidad y propiedad de
dichos resultados, así como de los derechos de propiedad industrial e
intelectual derivados de los mismos, corresponderá exclusivamente al
estudiante o estudiantes generadores de los mismos. No obstante, y sin
perjuicio de lo anterior, la Universidad de Granada podrá reservarse un
derecho de uso no exclusivo, gratuito e intransferible con fines de
investigación y de docencia.
\end{it}
\end{itemize}

Y si nos vamos a la la normativa específica de TFG de la UGR\footnote{\url{https://www.ugr.es/universidad/normativa/ncg1872-reglamento-trabajo-proyecto-fin-grado-universidad-granada}}, en su artículo 14 (Propiedad industrial e intelectual del Trabajo fin de Grado) dice lo siguiente:

\begin{enumerate}
\begin{it}
    \item El estudiantado tendrá derecho al reconocimiento de la autoría de su Trabajo
fin de Grado y a la protección de la propiedad intelectual del mismo.
\item A los Trabajos fin de Grado les resultará de aplicación la legislación vigente en
materia de propiedad intelectual y propiedad industrial que pudiera afectar
tanto al estudiantado como a las personas responsables de la tutorización y
cotutorización, así como a las empresas u organismos que pudieran estar
involucrados en su elaboración.
\item La titularidad de los resultados de la actividad investigadora conducente a un
Trabajo fin de Grado, y en particular la titularidad de los derechos de
propiedad industrial e intelectual asociados, puede corresponder
exclusivamente al estudiante o la estudiante o bien ser compartida entre él o
ella y la Universidad de Granada, conforme a lo establecido en la Normativa
sobre los Derechos de Propiedad Industrial e Intelectual derivados de la
actividad investigadora de la Universidad de Granada.
\end{it}
\end{enumerate}

En resumen, la propiedad intelectual podría ser únicamente del estudiante, o compartida con la empresa colaboradora, la Universidad, o incluso con la persona que te tutorice dependiendo de su colaboración. Si el tutor o tutora no participan activamente en el desarrollo y simplemente evalúan sobre la escritura de la memoria, por ejemplo, la propiedad intelectual podría ser solamente tuya. En cualquier caso hay que dejar claro desde el principio el tipo de colaboración que se llevará a cabo durante el desarrollo para evitar futuros malentendidos. 


\section{¿Qué son las licencias?}

Una licencia no es más que un contrato que indica qué se puede y qué no se puede hacer con tu obra. Por ejemplo, quiénes son los autores, en qué territorios puede usarse, qué derechos das a las personas que van a usar tu obra, o incluso qué fechas de uso les puedes dar. 

En esta sección vamos a ver los tipos de licencias que existen sobre una obra, lo cual, en nuestro caso incluye software y recursos digitales (textos, imágenes, vídeos, código, etc.) que sean obras tuyas o de terceros. El reconocimiento de la propiedad intelectual de una obra es importante para tener argumentos que permitan ganar un litigio en caso de plagio, protegiendo a la persona que ha creado la obra, pero también otorgándoles derechos de uso a otras personas. Aunque haya licencias ya creadas y ampliamente extendidas (como CC o GNU/GPL), tú también puedes crear tu tipo de licencia desde cero, aunque te recomendamos que uses las que ya existen.

Comenzamos con los tipos de licencias existentes según los derechos de autor de una obra. Las licencias pueden ser abiertas o cerradas. De entre las licencias abiertas, se especifica principalmente si se pueden hacer cambios sobre el original y si se permite o no la distribución del original o copias (incluyendo copia y venta) por parte de otros. Así, las licencias abiertas pueden ser permisivas (se puede modificar el original y hay flexibilidad en la distribución), robustas débiles (se puede modificar el original pero no distribuirlo) o robustas fuertes (no se puede modificar el original ni distribuirlo, pero sí usar tal cual). Por ejemplo, el software de Apache tiene licencia abierta permisiva y el de Eclipse tiene licencia abierta robusta fuerte.

\section{Licencias para texto, conjuntos de datos e imágenes}

Las licencias Creative Commons (CC) \footnote{\url{http://creativecommons.org}} regulan bastante bien estos aspectos de las licencias abiertas y otros a la hora de crear recursos propios o usar recursos de terceros. Tienen en cuenta:
\begin{itemize}
    \item Reconocimiento al autor (BY): obligación de nombrar al autor cuando se use su obra.
    \item Compartir Igual (SA): las obras derivadas deben mantener la misma licencia que la original al ser distribuidas.
    \item Sin Obra Derivada (ND): no se puede transformar una obra original, por ejemplo, no se puede traducir y distribuir la traducción.
    \item NoComercial (NC): solo el autor puede distribuir la obra con fines comerciales.
    \item Dominio Público: el autor renuncia a sus derechos. Cualquiera puede copiar, modificar y distribuir la obra, incluso con fines comerciales, sin pedir permiso.
    \item Zero (0): puedes usar el original, generar una obra derivada, distribuirlo con uso comercial o no comercial. Es como el dominio público pero no tienes ni siquiera que mencionar al autor.
\end{itemize}

Todas las licencias de CC, excepto la \textit{zero}, tienen obligación BY, de reconocimiento del autor de la obra.

Vamos a ver los tipos más comunes de licencia CC que hay en función de la combinación de algunos de los aspectos de la lista.
\begin{itemize}
    \item (BY-NC) Reconocimiento - No Comercial: puedes generar obras derivadas siempre que no hagas un uso comercial de las mismas ni del original. 
    \item (BY-NC-SA) Reconocimiento - No Comercial - Compartir Igual: igual que BY-NC, pero además, la distribución de las obras derivadas se debe hacer con una licencia igual a la que regula la obra original. Por ejemplo, si alguien usa tu imagen con esta licencia para hacer un \textit{collage} o cambiar los colores, esta nueva imagen debe tener ese mismo tipo de licencia. Es frecuente que la licencia tenga un número de versión, que debes conservar en este caso, por ejemplo, CC BY-NC-SA 4.0.
    \item (BY-NC-ND) Reconocimiento - No Comercial - Sin Obra Derivada:  no puedes generar obras derivadas ni hacer un uso comercial de la obra original. Es la licencia más restrictiva de CC. Por ejemplo, si una imagen, esquema o \textit{dataset} tienen esta licencia, no puedes hacer cambios sobre ellos, pero sí usarlos tal cual están.
    \item (BY-SA) Reconocimiento - Compartir Igual: puedes crear obras derivadas y hacer uso comercial de éstas y del original, siempre que sea bajo el mismo tipo de licencia de la original.
    \item  (BY-ND) Reconocimiento - Sin Obra Derivada: no puedes generar obras derivadas, pero sí puedes hacer uso comercial del original. 
\end{itemize}

Existe otra licencia más restrictiva aún que es la del Copyright (c). Si una obra tiene una licencia con Copyright, no puedes usarla sin pedir permiso y no puedes generar obras derivadas, ni distribuirla, aunque sea para uso no comercial. Por ejemplo, no puedes poner en tu TFG una imagen con Copyright, aunque nombres al autor. 

Para que no infrinjas la ley, te recomendamos que si necesitas usar recursos digitales en tu memoria o código, busques en bancos de imágenes, sonidos y vídeos de dominio público o con licencia \textit{Creative Commons}. También tienes disponibles varios repositorios de recursos educativos abiertos (REA) que puedes usar para la introducción y estado del arte de tu TFG. En la tabla \ref{enlacesDescargas} te proporcionamos algunos enlaces a sitios desde donde puedes descargarlos usando filtros y palabras clave para buscar lo que deseas.

\begin{table}[t]
\begin{center}
\caption{Sitios con recursos digitales abiertos}

\resizebox{13cm}{!}{
\begin{tabular}{| l | c | c | c | c |}
\hline
Enlace & Imagen & Sonido & Vídeo & REA \\ \hline
\url{https://biblioteca.uoc.edu/es/biblioguias/biblioguia/Bancos-de-imagenes/} & x &  &  & \\ \hline

\url{https://biblioteca.uoc.edu/es/biblioguias/biblioguia/Bancos-de-audiovisuales/} &  &  & x & \\ \hline

\url{https://www.pexels.com/es-es/videos}&  &  & x & \\ \hline

\url{https://edpuzzle.com/}&  &  & x & \\ \hline

\url{http://recursostic.educacion.es} & x & x &  & \\ \hline

\url{http://stock.adobe.com} & x & x & & \\ \hline

\url{https://www.flaticon.es} & x &  &  & \\ \hline

\url{https://unsplash.com/es} & x &  &  & \\ \hline

\url{https://freesound.org} &  & x &  & \\ \hline

\url{cedec.intef.es}  &  &  &  & x\\ \hline

\url{procomun.educalab.es}   &  &  &  & x\\ \hline

\url{oerworldmap.org}    &  &  &  & x\\ \hline
\end{tabular}
}
\label{enlacesDescargas}
\end{center}
\end{table}


%En la OUC hacen un recopilatorio de webs que proporcionan imágenes de este tipo \footnote{\url{https://biblioteca.uoc.edu/es/biblioguias/biblioguia/Bancos-de-imagenes/}} y vídeos \footnote{\url{https://biblioteca.uoc.edu/es/biblioguias/biblioguia/Bancos-de-audiovisuales/}}. Otro sitio que con un buscador de imágenes o sonidos para usar en educación es Recursos TIC Educación\footnote{\url{http://recursostic.educacion.es}}. También puedes descargar imágenes o sonidos gratuitos desde Adobe Stock \footnote{\url{http://stock.adobe.com}}, buscando en aquellas colecciones que estén libres de derechos. Si necesitas iconos con licencia abierta, gratuitos, puedes descargarlos desde Flaticon \footnote{\url{https://www.flaticon.es}}. También hay imágenes gratuitas para descargar en Unsplash \footnote{\url{https://unsplash.com/es}}.
%También puedes descargar sonidos desde FreeSound \footnote{\url{https://freesound.org}}. Además, puedes encontrar recursos educativos  (materiales de enseñanza, aprendizaje e investigación) abiertos en repositorios como \href{cedec.intef.es}{Proyecto EDIA}, \href{procomun.educalab.es}{PROCOMUN} y \href{oerworldmap.org}{OER WORLD MAP}. 

Otra opción que tienes si necesitas una imagen, es la de usar el buscador de Google, seleccionando imágenes y luego en herramientas, en licencias de uso, elegir Creative Commons. Por ejemplo, si te bajas imágenes de Internet para usarlas en la memoria o el programa sin filtrar antes, verás que te indican con una marca superpuesta las que están bajo licencia. 


Si utilizas un generador automático de imágenes o texto con IA generativa, tú serás el autor o tendrás la propiedad intelectual de los \textit{prompts} o entradas, y del \textit{output} o salida, ya que los generadores no son personas que puedan poseer derechos de autor \cite{derechosautorIA}. Eso sí, si en los \textit{prompts} das información específica de fuentes con derechos de autor o utilizas contenido protegido, es tu responsabilidad el garantizar que cumples con las leyes de derechos de autor al citar adecuadamente esas fuentes.

Recuerda que aunque obtengas un recurso de forma gratuita, esto no quiere decir que lo puedas usar de forma abierta. Por ello, y a no ser que tenga licencia \textit{zero} (0), asegúrate siempre de mencionar en tu TFG al autor/a y/o la web desde donde te has descargado el recurso y el tipo de licencia que tiene. Tienes ejemplos de cómo citar la fuentes de imágenes originales o adaptadas con licencia CC en la web de Creative Commons \footnote{\url{https://wiki.creativecommons.org/wiki/Recommended_practices_for_attribution}}. Lo habitual es poner detrás del título de una imagen el título original, el autor/a y el tipo de licencia, así como indicar si has modificado la imagen y la nueva licencia que le atribuyes, si es el caso.

 


\section{Licencias de código fuente}

El código creado por ti durante el desarrollo del proyecto también podrías licenciarlo como CC, pero desde la versión 4.0, no se recomienda esta licencia para código fuente, así que lo más habitual es usar licencias específicas de software, que están pensadas para las especificidades del código (por ejemplo, también tratan su  ejecución). 

Si deseas licenciar código, bastará con que en las primeras líneas de cada fichero, dentro de un comentario, indiques el tipo de licencia que le atribuyes. Algunas licencias también requieren un fichero \textit{LICENSE} o \textit{README} con la licencia, que deberás compartir con tu código, por ejemplo, en el directorio raíz de tu proyecto.

Hay dos tipos de licencias de código: las privativas y las abiertas/libres. Respecto a las licencias de código abierto/libre hay dos corrientes: el Software Libre (\textit{Free Software}) y el Código Abierto (\textit{Open Source}). De hecho se refiere al conjunto de las dos como \textit{FLOSS} (\textit{Free/Libre Open Source Software}). El ``Libre'' del acrónimo es porque en inglés gratis y libre se escriben igual, así que le han añadido la palabra en español para que quede claro el componente de libertad, no de gratuidad (``\textit{Free as a bird, not as in beer}'').

Dependiendo de la comunidad de desarrollo, se prefieren algunos tipos de licencias a otras. Por ejemplo los paquetes de Node.js suelen tener licencias MIT o ISC, los \textit{crates} de Rust usan MIT o Licencia Apache, y los \textit{plugins} de Wordpress deben ser GNU, sobre todo para evitar problemas de compatibilidad.

Aunque no es obligatorio que liberes el código de tu TFG, puedes obtener una serie de beneficios si lo haces: te obliga a seguir buenas prácticas de desarrollo, sirve para crear comunidad, genera ejemplo, y no menos importante, te servirá como porfolio que te ayudará a encontrar un buen trabajo.

\subsection{Licencias de software libre}
La corriente del Software Libre es, digamos, más ``filosófica'', y está orientada a la Libertad de la persona que usa el software: libertad para usarlo, estudiarlo, distribuirlo y mejorarlo. Para conseguir estos objetivos es necesario tener acceso al código siempre, por lo que las licencias de Software Libre tienen como requisito compartir las modificaciones (como en CC-SA). La \textit{Free Software Foundation} (\textit{FSF})\footnote{\url{http://www.fsf.org}} es la entidad que se encarga de mantener licencias como la GNU. Su página web tiene muchísima información sobre cómo usar sus licencias explicada con un lenguaje claro \cite{FSFfaq}.

\begin{itemize}

\item \textbf{GNU General Public License (GPL)}. La última versión es la V3. Te permite el uso comercial del software que la tiene (es decir, venderlo y ganar dinero), distribuirlo, modificarlo, y las personas que contribuyen a él tienen derecho a participar en una posible patente (cumpliendo muchos requisitos). Como condiciones, si vas a distribuir el programa o una versión a partir de él (a un cliente, o ponerlo en Internet), tienes que revelar el código, mantener el \textit{copyright} original y el archivo de licencia, además de documentar los cambios respecto al original. Digamos que es una licencia vírica: puedes usar mi código para lo que quieras, pero tiene que seguir siendo GPL.

\item \textbf{GNU Affero General Public License (AGLP)}. En los últimos años ya no usamos el software en nuestro ordenador, a veces accedemos a él a través de Internet. Esta licencia en concreto extiende la GPL con una cláusula para que si el software ofrece servicios a través de una red, entonces también haya que liberar el código.

\item \textbf{GNU Lesser General Public License (LGLP)}. Esta es más flexible que la GPL. Si tu software se usa vía interfaces (por ejemplo, accediendo a los métodos de una librería), el software que llama a esas funciones puede usar cualquier otra licencia (incluso privativa), pero si modificas la librería entonces tienes que liberar el código modificado.

\end{itemize}

\subsection{Código abierto}
Podría decirse que el \textit{Open Source} está más orientado al mundo empresarial, y por eso sus licencias no exigen liberar los cambios. La entidad que promueve estas licencias es la \textit{Open Source Initiative (OSI)} \footnote{\url{https://opensource.org/}}. Las licencias de código abierto más extendidas son las siguientes (aunque hay más de cien en la web de la OSI):

\begin{itemize}
   
\item \textbf{Licencia Apache}. Tiene los mismos permisos que la GPL (uso comercial, distribución, modificación, derecho a patente), pero a diferencia de GPL no es necesario liberar el código ni usar la misma licencia, pero sí documentar los cambios que se han añadido.

\item \textbf{Licencia MIT}. Es mucho más permisiva que la anterior: no hace falta ni documentar el cambio, se pueden generar patentes de las ideas sin añadir a todas las personas que han contribuido, y no hay que documentar los cambios.

\end{itemize}

\section{Cómo aplicar una licencia a tu proyecto}

Casi todas las licencias software lo único que requieren para poder aplicarse es añadir un fichero LICENSE o README en el directorio raíz del código de tu proyecto con el texto de la licencia. De hecho, cuando creas un proyecto en Github o Gitlab, se te ofrece la opción de añadir automáticamente el fichero de licencia a elegir con un menú entre las opciones más utilizadas, como MIT, GNU/GPL, Apache y otras. Ten en cuenta que algunas licencias también requieren un fichero CHANGES donde indiques las modificaciones que se hayan hecho al código original.

Es muy importante que todas las personas y entidades que han contribuido a tu proyecto (como hemos explicado en la sección \ref{sec:propiedadinte}) aparezcan como autores en la licencia. Todas las licencias tienen una sección de autoría, normalmente junto a la palabra ``Copyright AÑO por'' (sí, incluso las licencias libres y abiertas usan la palabra Copyright).

Casi todas las licencias también requieren que añadas un comentario en la cabecera de cada fichero. En este caso lo más fácil es crear la plantilla en tu IDE que incrusta el texto cada vez que creas un fichero nuevo dentro de tu proyecto.

La memoria también puede tener licencia abierta. Sin embargo, al ser texto y no código, tienes que utilizar una licencia Creative Commons, o alguna licencia específica para documentación, como la GNU Free Documentation License\footnote{\url{https://www.gnu.org/licenses/fdl-1.3.html}}.

Por ejemplo, si deseas que tu memoria final tenga licencia CC, bastará con que uses el generador de licencias de CC\footnote{\url{https://chooser-beta.creativecommons.org/}}, para indicar qué tipos de licencia deseas asignar a tu TFG. Uno de los pasos en este proceso es escribir el nombre de los autores de la obra. Al final este generador te proporcionará un texto y un icono para insertar en la memoria, o un código HTML para incrustar todo en una web. Se suele insertar el texto e icono en la primera página. También puedes poner el texto legal completo en el que se basa la licencia (extrayéndolo de la página de CC). %Puedes optar entre (BY-NC), (BY-NC-SA) si deseas que alguien use tu memoria y haga adaptaciones de ella, o (BY-NC-ND) si das permiso para usarla tal cual, sin adaptaciones. También puedes añadir el enlace URL de tu memoria si está disponible de forma abierta en un blog, portal web o similar para facilitar la reutilización por parte de otros. 

Te aconsejamos que subas tu memoria a un repositorio oficial como Digibug UGR \footnote{\url{https://digibug.ugr.es}}, previa solicitud de claves. Al subir un documento a este repositorio, se licencia automáticamente como CC (BY-NC-ND), a no ser que desees que sea menos restrictiva, por ejemplo, solo BY-NC, permitiendo a otros que creen obras derivadas con tu trabajo. Si es así, debes indicarlo en la primera página del documento. 

Estas recomendaciones para licenciar la memoria y el código del TFG puedes aplicarlas a otros trabajos que realices dentro o fuera del ámbito universitario, y no solo en formato de texto/código, también para imágenes o vídeos creados por ti. CC facilita la descarga de elementos que identifiquen los diferentes tipos de licencias en los diferentes formatos, así, en la web oficial\footnote{\url{https://creativecommons.org/mission/downloads/}} puedes descargarte logos, iconos, \textit{stickers}, \textit{gifs} animados, e incluso \textit{bumpers} para colocar en tus vídeos.

\section{Código generado por IA o copiado de páginas web}

Por último, indicarte que si has usado alguna herramienta de inteligencia artificial generativa para obtener el código, o lo has tomado de foros como \textit{Stack Overflow} o \textit{Reddit}, por ejemplo, debes tener en cuenta también lo relativo a la LPDGDD y respetar las licencias que tienen los códigos originales que hayas copiado. Es decir, solo debes copiar o crear un código derivado como modificación del copiado si la licencia del original te lo permite. En caso de que no encuentres licencia asociada, por omisión tiene \textit{Copyright}, así que actúa en consecuencia.


\section{Otras consideraciones sobre las licencias y cómo obtenerlas}

En el caso de que utilices varios paquetes o librerías con licencia deberás tener mucho cuidado para que sean compatibles entre sí. Puedes usar webs como JoinUp\footnote{\url{https://joinup.ec.europa.eu/collection/eupl/soluti on/joinup-licensing-assistant/jla-compatibility-ch ecker}} para ver el tipo de licencia que tendrías que ponerle a tu proyecto si las combinas. Si no sabes qué licencia escoger, puedes usar el test Choose A License \footnote{\url{http://choosealicense.org}}.

Por último, nos gustaría comentarte otra opción que es la de obtener un registro de propiedad intelectual, que equivale a tener una licencia de Copyright, en la que el autor puede dar derechos a otras personas previa solicitud, por ejemplo, de uso o copia. Esto puede ser complementario a usar licencias abiertas y libres, es una especie de firma digital por si en un futuro tuvieras que demostrar tu autoría en un juicio, por ejemplo. Para obtener ese registro, debes ir físicamente a una oficina provincial de registro de propiedad intelectual o bien hacer el registro en una web oficial como la del Registro Territorial de Andalucía \footnote{\url{https://www.juntadeandalucia.es/organismos/turismoculturaydeporte/servicios/procedimientos/detalle/297.html}}. Para tal fin, debes completar unos formularios en los que se piden datos relativos a la obra y los autores, y pagar unas tasas que oscilan desde los quince euros si se hace de forma presencial, a los ocho euros si el registro es online. Al hacer un registro obtienes un documento que te reconoce como autor de la obra. %En España los derechos de autor expiran a los cincuenta años de la muerte del autor, en cuyo caso, la obra queda como dominio público.

Tanto con un registro de propiedad, como indicando tipo de licencia en tu memoria de TFG y código, si en el futuro alguien plagia tu TFG o parte de él, no te cita, o se lucra con tu trabajo sin tu autorización y sin respetar la licencia que has indicado, podrás tener un respaldo de tus derechos para denunciar.

Finalmente, recuerda que en la UGR puedes consultar con la OSL (Oficina de Software Libre)\footnote{\url{https://osl.ugr.es}} o con la OTRI (Oficina de Transferencia de Resultados de Investigación)\footnote{\url{https://otri.ugr.es}} para resolver dudas sobre licencias libres o normativa de transferencia y propiedad intelectual.

\include{anexoII_tipologias}
\chapter{Experiencias y consejos del estudiantado}
\label{anexo:Experiencias}

Queremos innovar en la filosofía de escritura de este libro y hacerlo un lugar donde el lector pueda convertirse también en partícipe del contenido.

Por tanto, te animamos a que, a toro pasado, nos cuentes cómo te ha ido y qué partes del libro te han resultado más interesantes, o cómo completarías tú algo que se nos haya quedado en el tintero.

Siéntete libre de comentar cualquier aspecto, pensamiento o emoción que te haya surgido durante la realización de tu TFG. 

Te lo agradecemos de manera anticipada :-)

\section{¡Incluye tu comentario o experiencia aquí!}

\clearpage
\thispagestyle{empty}
\null
\newpage

% Contraportada
\clearpage
\thispagestyle{empty} % Sin encabezado ni pie de página

\begin{titlepage}
    \begin{minipage}{0.9\textwidth}
 \textbf{Cómo escribir la memoria de tu TFG del Grado en Ingeniería Informática y presentarlo sin morir en el intento} \\ \\
        \small
      
       Alberto Guillén Perales\\
       Departamento de Ingeniería de Computadores, Automática y Robótica\\
       Universidad de Granada\\ 
       \vspace{1ex}\\
       Eugenio Martínez Cámara \\ 
       Departamento de Informática \\
       Universidad de Jaén\\
       \vspace{1ex}\\
       Juan Manuel Fernández Luna\\
       Departamento de Ciencias de la Computación e Inteligencia Artificial\\
       Universidad de Granada\\ 
       \vspace{1ex}\\
       Pablo García Sánchez\\
       Departamento de Ingeniería de Computadores, Automática y Robótica\\
       Universidad de Granada\\ 
       \vspace{1ex}\\
       Manuel Noguera García\\
       Departamento de Lenguajes y Sistemas Informáticos \\
       Universidad de Granada\\ 
       \vspace{1ex}\\
       María José Rodríguez Fórtiz\\
       Departamento de Lenguajes y Sistemas Informáticos \\
       Universidad de Granada\\ 
       \vspace{1ex} \\
       Rocío Romero Zaliz\\
       Departamento de Ciencias de la Computación e Inteligencia Artificial\\
       Universidad de Granada\\ 

       
    \end{minipage}

  %{\Large\bfseries Recursos Adicionales\par}
    \vspace{1cm}
    \begin{minipage}{0.9\textwidth}
        \centering
        \begin{minipage}{0.45\textwidth}
            \centering
            \includegraphics[width=0.9\textwidth]{images/qr-youtube.png}\\
            \small Canal de YouTube
        \end{minipage}
        \hfill
        \begin{minipage}{0.45\textwidth}
            \centering
            \includegraphics[width=0.9\textwidth]{images/qr-github.png}\\
            \small Repositorio de GitHub
        \end{minipage}
    \end{minipage}

\end{titlepage}

\newpage

\thispagestyle{empty} % Sin numeración
\includepdf[pages={1},width=1.45\textwidth]{images/contraportada.pdf}

\end{document}
