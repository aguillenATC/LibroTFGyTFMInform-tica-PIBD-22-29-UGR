\documentclass{book}
\usepackage{xcolor}
\usepackage{color}
\usepackage[spanish]{babel}
\usepackage{graphicx}
\usepackage{hyperref}
\usepackage{booktabs}
\usepackage{varwidth}
\usepackage{caption}
\usepackage{multirow}
\usepackage{textcomp}
\usepackage{url}
\usepackage{amsmath}
\usepackage[inline]{enumitem}
\usepackage{todonotes}
\usepackage{enumitem,amssymb}
\newlist{todolist}{itemize}{2}
\setlist[todolist]{label=$\square$}
\usepackage{color}
\usepackage{rotating}

\newcommand{\emc}[1]{\textcolor{blue}{#1}}
\newenvironment{naranja}{\par\color{orange}}{\par}

\begin{document}

\title{Cómo escribir la memoria de tu TFG del Grado de Ingeniería Informática y presentarlo sin morir en el intento (tentativo)}
\author{Tus Nombres}
\date{\today}
\maketitle

\tableofcontents
\listoftables
\listoffigures

% Incluye los archivos de los capítulos
\include{1.prologo}
\chapter{Introducción}
\label{cap:Introducción}
%Juanma
%Iniciativas previas: \url{https://drive.google.com/drive/folders/1WMPwX1v7IhG88bJYLXDgm4Zp6a_pCCHh}

Todavía no está terminado. Terminarlo al final de la revisión completa del libro.


Este libro queda dividido en los siguientes capítulos: en el capítulo \ref{cap:Recomendaciones} se ofrece, para empezar, información general sobre el proceso de asignación del TFG, contexto general que debes conocer antes de comenzar a trabajar, así como pautas y recomendaciones de trabajo para ayudarte en el desarrollo del proyecto. El capítulo \ref{cap:EstructuraMemoria} establece una propuesta de estructura general de la memoria, indicando cada una de las partes que la compone, para que tengas una idea general de cómo organizar este documento. El siguiente capítulo, el \ref{cap:IntroducciónTFG}, se centra en describir el contenido del capítulo de introducción, ofreciendo tanto una posible estructura del mismo como consejos para su elaboración. El capítulo \ref{cap:RevisionEstadoDelArte} pasa a describir cómo se debe hacer una revisión del estado del arte, parte fundamental en cualquier memoria de TFG. Un elemento importante en todo proyecto es la planificación y el presupuesto, y en el capítulo \ref{cap:PlanificacionPresupuesto} damos algunos consejos para su elaboración. Seguidamente, en el capítulo \ref{cap:Tipologías} se presentan los cuatro tipos de TFG principales: de desarrollo, experimental, investigación y de revisión. Recomendaciones sobre cómo abordar la confección de las conclusiones y los trabajos futuros se incluyen en el capítulo \ref{cap:Conclusiones}. El capítulo \ref{cap:bibliografia} aborda asuntos relacionados con la bibliografía y cómo referenciarla en la memoria y el siguiente, el capítulo \ref{cap:anexos} hace lo propio con los anexos que puedes incluir en la memora.  Una vez que la esta está terminada, en el capítulo \ref{cap:Revisión} se muestra una lista de comprobaciones que deberías hacer para estar seguros de que la memoria alcanza el mínimo de calidad exigible. Los dos capítulos siguientes, y últimos, el \ref{cap:elaboraciónPresentación} y \ref{cap:defensa}, se centran en la presentación y en la defensa, respectivamente, aconsejando sobre cómo montar la primera y cómo afrontar la segunda. Y como no podía ser de otra manera, este libro finaliza con un capítulo de anexos 
\chapter{Recomendaciones generales}
\label{cap:Recomendaciones}

% [Autores: María José Rodríguez Fórtiz, Juanma, Alberto]
%%%%%%%%%%%%%%%%%%%%%%%%%%%%%%%%%%%%%%%%%%%%%%
%%%%%%%%%%%%%%%%%%%%%%%%%%%%%%%%%%%%%%%%%%%%%%

\section{Sobre la elección del TFG}

%%%%%%%%%%%%%%%%%%%%%%%%%%%%%%%%%%%%%%%%%%%%%%
%%%%%%%%%%%%%%%%%%%%%%%%%%%%%%%%%%%%%%%%%%%%%%

Ya estás en cuarto del grado y en ese momento aparece una asignatura de segundo semestre que se denomina Trabajo Fin de Grado de la cual sabes poco salvo que tienes que realizar un proyecto. Te matriculas en ella y seguidamente se te vienen a la cabeza una serie de cuestiones, entre las que estarán seguro algunas de estas: ¿y ahora qué hago? ¿qué TFG elijo? ¿con qué profesor/a? Normalmente vienen seguidas de un tiempo de cierta incertidumbre y algo de ansiedad, hasta que por fin consigues tener un tutor o tutora y un proyecto en el que ponerte a trabajar. En esta sección vamos a hacerte algunas recomendaciones para ayudarte a responderlas y a que este proceso previo sea más llevadero.

Antes de comenzar, comentarte que existen dos fases en la asignación de TFGs. En la primera, el estudiante y el docente llegan a un acuerdo y el segundo hace una preasignación del TFG al estudiante. En la segunda, el centro publica una lista de TFGs que han propuesto los docentes y los estudiantes que no tienen TFGs preasignados en la primera fase, eligen por orden de preferencia propuestas de TFGs. La Escuela se encarga de realizar la asignación aplicando los criterios que estén vigentes en el momento (normalmente expediente académico de los estudiantes). Normalmente este proceso se realiza en octubre y en febrero o marzo.

La primera sugerencia que te hacemos es que seas proactivo en la elección de la temática del TFG. No esperes a la segunda fase y busca tu TFG en la primera. No seas tímido y propón cambios a la idea que esté propuesta por el profesor o, incluso, piensa en alternativas completamente alejadas. El profesorado puede no querer cambiar, pero por lo general, a la hora de proponer TFG hay bastante flexibilidad.

Respecto a cuando empezar, aunque es una asignatura de segundo semestre muchos estudiantes comienzan a buscar y trabajar en su TFG en el primer semestre o a final de tercero. Así, hay ocasiones en que los estudiantes tenéis muy claro qué quieres hacer en vuestro trabajo fin de carrera. Esto puede deberse a múltiples situaciones. Por ejemplo, imagina algunas de estas: desde algunos cursos antes tienes entre manos el desarrollo de un proyecto personal y consideras que el TFG puede ser un contexto idóneo para avanzarlo; o bien, lo que tienes es una idea de proyecto personal y esta asignatura te plantea una excusa perfecta para llevarlo a cabo. También puede ser que hayas aprendido alguna tecnología o aplicación en alguna asignatura y estás motivado para trabajar en ella. Pero no sólo tienen que ser proyectos personales, sino que también entra en juego tu mentalidad emprendedora y has detectado un nicho donde el producto o servicio que obtengas como salida de tu TFG es susceptible de ser comercializado. Y también están los proyectos profesionales. Es habitual que cuando llegues a cuarto estés haciendo tus prácticas en empresa o incluso que ya estés contratado. Y en ese entorno profesional pueden existir desafíos o temáticas en las que estás trabajando en tu empresa, o podrías trabajar, y que serían susceptibles de ser abordados en tu TFG.

Asumiendo que estás en una de estas situaciones, o en cualquier otra en la que ya tengas una idea clara de lo que quieres hacer, el primer paso que tienes que dar es sentarte tranquilamente y escribir una descripción del problema que deseas abordar, lo suficientemente completa para que un posible lector tenga una idea bastante clara de tu propuesta tras leerla. 

El siguiente paso será la búsqueda del docente que te dirigirá el trabajo. Aquí existen varias situaciones: la primera es que tienes cierta confianza con algún profesor o cierta afinidad personal, te cae bien o crees que puedes trabajar bien con él. En ese caso, concierta una tutoría y exponle tu idea y dile directamente que te gustaría que fuera la persona que te tutorice, explicándole las razones. Ese documento que habías hecho, házselo llegar previamente a la tutoría con objeto de que conozca tu idea y puedas explicársela con cierto conocimiento de la misma por su parte y resolver las dudas que te pueda plantear. Si te dice que sí, has triunfado y, a partir de ahí, os podéis poner a trabajar, seguramente en una fase inicial donde concretéis detalles del proyecto. Si te dice que no, busca otro profesor. Podría ser interesante que hicieras una lista con los docentes a los que plantearle la dirección, ordenada por preferencia. La otra alternativa para solicitar a un docente que sea la persona que te tutorice es que sea experto en la temática de tu idea. En este caso, busca qué profesores investigan en ese tema o imparten clases en asignaturas relacionadas, y procede de la misma forma explicada. Para ello, inspecciona las guías docentes de las asignaturas afines a la temática de tu idea, mira los sitios web de los departamentos y de los grupos de investigación para encontrar docentes que sean expertos en temas relacionados con tu idea. 

Si no tienes una idea clara, pero sí sabes con certeza con qué tutor querrías trabajar, ve a verlo y dile que quieres hacer el TFG con él y si existe la posibilidad de que te proponga algún tema que pueda ser de tu interés. Si dice que sí, y hay alguna propuesta que te guste, ya puedes empezar a trabajar. Si dice que no o ninguna de las propuestas te interesan, entonces echa mano del siguiente profesor de la lista.

Otra opción que existe cuando no se tiene una idea clara de lo que hacer y tampoco se tiene claro con quién hacerlo es acudir a la web donde se publican las propuestas de TFGs para ese curso. Los docentes suelen publicar sus propuesta con objeto de darlas a conocer y que los estudiantes las puedan elegir. Mira las que te gustan y cítate con los docentes para que te las expliquen con más detalle. Cuando encuentres una propuesta que te guste, solicítasela al profesor. 

Todo lo anteriormente indicado se aplica en la fase de preasignación. Si por cualquier causa (despiste, falta de iniciativa, dudas, etc.) no se te ha asignado un TFG en esta fase, pasarías a la segunda. Como ya hemos comentado, tendrás que ordenar los TFGs propuestos en orden decreciente de interés, y esperar que la Escuela te asigne alguno de los que has seleccionado. En este caso, no hagas la ordenación al tuntún ni introduzcas en la lista todos los TFGs, sino que selecciona los que verdaderamente te interesen. Habla con los tutores de los que te gustan y pídeles que te den detalles para que puedas hacer una ordenación informada según tus gustos, combinando tanto la temática del TFG como el docente que lo propone. Y a partir de ahí, a cruzar los dedos para que el que has puesto como primero, o alguno de los primeros puestos,  sea el que se te asigne. Corres el peligro te den uno situado al final de esa lista. Esto tiene al menos dos inconvenientes: puede ser que no te motive la propuesta o que la persona que te tutoriza que te ha tocado no sea santo de tu devoción. En ese caso, tendrás que plantearte si seguir para delante con lo que te ha tocado o dejar el TFG para el curso siguiente (no tiene que ser junio, también puede ser la convocatoria extraordinaria, normalmente en noviembre) y buscar o proponer tú el tema de tu TFG con tiempo.

Como podrás imaginar, siempre es mejor que tengas tú la iniciativa y buscar una proyecto que te motive y un tutor con el que estés a gusto trabajando. Piensa que las horas las tienes que trabajar igualmente, así que mucho mejor si lo haces persiguiendo un objetivo que te interese y motive.

También es importante, si optas por la la opción proactiva, es el momento en el que comienzas a moverte y a realizar contactos con el profesorado. Si lo dejas para muy tarde, seguramente éstos estén ya cargados de TFGs y puedas recibir varias negativas a tus propuestas o los propuestos que te interesan ya hayan sido preasignados. Esta es la razón por la que te recomendamos que comiences el proceso de búsqueda de TFG lo antes posible. Hay estudiantes que a final de tercero, antes de vacaciones, están buscando ya; otros justo a la vuelta, en septiembre y algunos lo dejan para la asignación de noviembre. En este último caso, las opciones quedarán reducidas considerablemente, así que, cuanto antes, mejor.

En esas reuniones previas a la preasignación donde finalmente hay un acuerdo entre estudiante y docente, un aspecto muy relevante es la discusión del alcance del TFG y sus objetivos. En ellas tenéis que concretar cuáles serán las funcionalidades de los desarrollos asociados al TFG y qué objetivos os marcáis para el mismo. Estas negociaciones son importantes porque determinarán hasta dónde tendrás que llegar en tu proyecto. En este sentido, cuando se te preasigne asegúrate que tengas claro este alcance. Procura que, al definir estos objetivos, que sean realistas, y que consideres para que así sean, el tiempo que tienes disponible para trabajar en él y cuándo quieres tenerlo listo para entregarlo. Puedes ver un resumen de todo este proceso en la Figura  \ref{fg_diagrama_proceso}.

Ya para concluir esta sección, comentarte que la realización de un TFG es un proceso que te está entrenando para el ámbito profesional. Vas a realizar un trabajo que se asemejará al que realizarás cuando hayas terminado y estés contratado en una empresa. Aprovecha esta oportunidad para aprender nuevas metodologías, tecnologías, lenguajes, etc. que no has visto en el grado y que te puedan servir en la nueva fase profesional que se te abrirá en pocos meses. Además, es una magnífica carta de presentación en las entrevistas ya que podrán ver tu capacidad de trabajo y de aprendizaje de nuevos conceptos, cosa que permitirá diferenciarte de otros candidatos. Por tanto, te sugerimos que aproveches la oportunidad de realizar un TFG en el que aprendas muchas cosas y en una temática que te guste. 

Como nota final, recuerda que para presentarlo en la convocatoria extraordinaria de noviembre tienes que haberlo solicitado en la sede.ugr.es tal y como indica la web de la Escuela \url{https://grados.ugr.es/informatica/pages/infoacademica/tfggestion2}. Hay unos plazos para dicha solicitud, no dejes que se te pase.

% SUGERENCIA -> INFOGRAFÏA CON DIAGRAMA DE FLUJO DEL PROCESO
\begin{figure}[!ht]
\centering
    \includegraphics[scale=0.5]{images/DecisionTFG.pdf}
    \caption{Diagrama de flujo del proceso.}\label{fg_diagrama_proceso}
\end{figure}

%%%%%%%%%%%%%%%%%%%%%%%%%%%%%%%%%%%%%%%%%%%%%%
%%%%%%%%%%%%%%%%%%%%%%%%%%%%%%%%%%%%%%%%%%%%%%

\section{Sobre el desarrollo del proyecto}

%%%%%%%%%%%%%%%%%%%%%%%%%%%%%%%%%%%%%%%%%%%%%%
%%%%%%%%%%%%%%%%%%%%%%%%%%%%%%%%%%%%%%%%%%%%%%

En esta sección vamos a comentar aspectos fundamentales para el éxito de cualquier proyecto. Comenzaremos con la comunicación con la persona que te tutoriza, posteriormente con la disciplina y la pauta de trabajo (no es lo mismo caminar 1Km cada día durante 100 días que caminar 100Km en un día...), y cuestiones menos técnicas como aspectos éticos y de propiedad intelectual.

%%%%%%%%%%%%%%%%%%%%%%%%%%%%%%%%%%%%%%%%%%%%%%
\subsection{Reuniones con la persona que te tutoriza}%(María José)
%%%%%%%%%%%%%%%%%%%%%%%%%%%%%%%%%%%%%%%%%%%%%%

Durante el desarrollo del TFG debes mantener varias reuniones con la persona que te tutoriza con el objetivo de organizar y revisar tu trabajo. Las reuniones pueden ser presenciales u online, dependerá de ti y de la persona que te tutoriza.

En las primeras reuniones con la persona que te tutoriza, tal y como se ha comentado antes, debes consensuar el alcance y objetivos del proyecto. Un TFG debe ser un trabajo original, claro, riguroso, coherente, relevante, y pertinente. Por eso, te insistimos en que concretes bien y sin divagar cuál va a ser la orientación y aportación del trabajo desde el principio, qué se va a mejorar respecto a lo existente. En las reuniones iniciales también se debe proponer y seleccionar la metodología a seguir y cuál será la estructura de la memoria. Aunque luego pueda haber cambios, es interesante hacer el ejercicio a priori para tener una idea de cómo se va a desarrollar el trabajo.

En las siguientes reuniones la persona que te tutoriza puede ayudarte a resolver dudas, ir validando el desarrollo realizado, y corregirte lo que vayas escribiendo de la memoria, por ejemplo, indicándote el grado de profundidad al exponer los temas, o las mejores herramientas de diseño a utilizar. También puede sugerirte bibliografía para completar el estado del arte o contexto del trabajo o facilitarte materiales para el desarrollo. 

Te aconsejamos que te reúnas con la persona que te tutoriza con frecuencia, aunque esta varíe según tus otras ocupaciones en otras asignaturas o en tu trabajo, si ya lo tienes. Lo habitual es quedar de forma sistemática con ella en diferentes momentos considerando la metodología de trabajo que estés siguiendo, al acabar un paso, fase o tarea y antes de empezar el siguiente. Eso te obligará a ser más constante en tu trabajo ya que sabes que tienes que rendir cuentas de lo que has hecho en ese tiempo. Así, por ejemplo, es frecuente planificar pasos, fases o tareas de dos semanas y reuniones con esa frecuencia. Si prevés que vas a estar un tiempo sin poder trabajar en el TFG, hazlo saber a la persona que te tutoriza para planificar reuniones a más largo plazo. Del mismo modo, si requieres reuniones más frecuentes, házselo saber para llegar a un acuerdo. 

Te aconsejamos también que planifiques bien cada reunión, qué quieres mostrarle y qué quieres consultar. Si es posible, envíale material resultante de tu trabajo antes de la reunión para que tenga tiempo de mirarlo. En ese caso, la reunión será más efectiva, centrándose en los comentarios que tenga sobre tu entrega, y en tus dudas. Otros consejos son que te vayas apuntando esas dudas que te surjan conforme vayas trabajando para llevarlas a la siguiente reunión, y que tomes acta (apuntes) lo que se hable en la reunión, o incluso la grabes si la persona que te tutoriza está de acuerdo, de ese modo podrás repasarlo después. 

De hecho, igual que recomendamos tener una bitácora para el desarrollo, también puedes incluir en ella estas reuniones (o crear una por separado). De ese modo, queda constancia del trabajo continuo a lo largo del tiempo y tanto tú, como el tutor, como el tribunal calificador pueden ver la evolución del trabajo y validar la planificación.

Las reuniones estrechan los lazos tutor-estudiante, ten confianza para exponer tus dudas, tu falta de tiempo o motivación, para que se puedan resolver los problemas cuanto antes. Si tienes desacuerdos con la persona que te tutoriza, justifica tu postura, eres tú el que estás realizando el trabajo y el que llegado un momento puede conocerlo mejor, así que haz un esfuerzo por argumentar tus decisiones. 

En el extraño caso de que haya una diferencia de opinión con tu tutor irresoluble, siempre desde la corrección y el respecto, puedes recurrir al coordinador de la titulación para que te asesore y ver cómo proceder. 

%%En el caso de que no llegues a tener esa confianza con la persona que te tutoriza o no estés recibiendo la supervisión que necesitas, te aconsejamos que lo hables y si no se le da solución, que busques apoyo en tus compañeros u otros profesores/as.

Por último, hacerte saber que las reuniones son la herramienta que la persona que te tutoriza tiene para evaluar tu trabajo y éste hará una evaluación continua de él. Recuerda que la persona que te tutoriza pone parte de la nota final, así que debes mantenerla informada de lo que vas haciendo, hacerle entregas y realizar mejoras sobre entregas previas teniendo en cuenta sus sugerencias. Si la persona que te tutoriza percibe que vas a las reuniones sin realizar el trabajo planificado, que si vas no estás atento o no tomas nota, que en las entregas no tienes en cuenta sus sugerencias o no argumentas porqué, no puedes esperar una gran calificación, aunque luego entregues una memoria final "estupenda".

%%%%%%%%%%%%%%%%%%%%%%%%%%%%%%%%%%%%%%%%%%%%%%
\subsection{Disciplina/pauta de trabajo}% (María José)
%%%%%%%%%%%%%%%%%%%%%%%%%%%%%%%%%%%%%%%%%%%%%%

El TFG se realiza durante varios meses, lo que requiere de una planificación del trabajo a corto y largo plazo. 

Si eres una persona disciplinada, no te costará trabajo hacer un calendario semanal en el que busques unos huecos para el TFG y para reuniones con la persona que te tutoriza, de tal forma que puedas hacer tu trabajo de forma progresiva. 

Sin embargo, si eres una persona con tendencia a la procrastinación (dejarlo todo para el último día para trabajar con presión), es el momento de que busques apoyos para evitar esto ya que te aseguramos de que así no podrás hacer el TFG. Te aconsejamos que te organices, que establezcas prioridades y que dividas tu trabajo en tareas más pequeñas que puedas realizar con menos esfuerzo. Ponte un horario planificando descansos y trabaja siempre en el mismo sitio, con una mesa de trabajo limpia y sin distractores. \textcolor{orange}{Utiliza aplicaciones de agenda y gestión del tiempo que te avisen de cuándo hay que hacer cada tarea, y programa un tiempo de finalización para que no te distraigas. Por ejemplo, aplicaciones como \href{https://trello.com/}{Trello} o \href{https://www.focustodo.cn/}{FocusToDo} ayudan a ello. Otra alternativa para combinar tanto la bitácora como un panel de tareas que se transforma en un Diagrama de Gantt en \url{https://www.notion.so/}. Si necesitas más apoyo aún, busca herramientas con "gamificación" que te impidan distraerte y te den premios al finalizar tu trabajo, como \href{https://www.forestapp.cc/}{Forest}, o dátelos tú mismo/a para tener un refuerzo positivo que te ayude a seguir adelante y mantener esa dinámica. }

%Alberto -> revisar
Para cuantificar el tiempo que le dedicas, medir tu avance y llegar a las 300 horas que, como mínimo, se deben invertir en el TFG puedes cronometrar el tiempo dedicado mediante aplicaciones como \url{https://clockify.me/es/} y también puedes guardar un log o bitácora donde puedes incluir un párrafo resumiendo lo avanzado en ese intervalo de tiempo que has dedicado. El ejercicio de resumir lo que has avanzado te ayudará a avanzar e incluso a resolver algún problema que te haya surgido. No te desanimes si, en ocasiones, no tienes mucho que escribir, el poder transmitir conocimiento no es sencillo y puede requerir que discurra un tiempo hasta que puedas hacerlo. Lo que está claro es que si no te sientas a escribir esa bitácora, muy probablemente nunca llegues a desarrollar esa capacidad, en palabras atribuidas, entre otros a Picasso: {\it "Si llegan las musas, que te pillen trabajando”} \footnote{\url{https://www.elcorreoweb.es/economia/2020/07/02/musa-pilla-trabajando-104593995.html} 

En el cómputo de las horas dedicadas, incluye también el tiempo invertido con el tutor desde la primera reunión hasta el posible ensayo de la presentación, el tiempo de buscar bibliografía, el tiempo que has dedicado a leer cosas relacionadas con el proyecto así como cualquier aspecto del desarrollo.



%%%%%%%%%%%%%%%%%%%%%%%%%%%%%%%%%%%%%%%%%%%%%%
\subsection{Planificación y herramientas} %(María José)
%%%%%%%%%%%%%%%%%%%%%%%%%%%%%%%%%%%%%%%%%%%%%%
Una vez consensuados con la persona que te tutoriza cuáles van a ser los objetivos del TFG, lo primero que se debe hacer es planificar las tareas a realizar durante todo el proyecto, teniendo en cuenta una metodología de trabajo en la que habrá varios pasos o fases.

En tu TFG seguramente tendrás que realizar tareas de varios tipos, que podrán variar dependiendo el tipo de proyecto que estés realizando. Estos tipos principales de tareas son: revisión bibliográfica, aprendizaje de herramientas y tecnologías, desarrollo, reuniones, redacción de la memoria, y evaluación. Además, el propio desarrollo requiere también de una metodología de trabajo que variará según éste, pero que incluirá sus propias tareas.

Te aconsejamos que conociendo tu disponibilidad de tiempo, seas tú el que hagas la primera propuesta de planificación temporal que muestres al tutor, incluyendo tareas de todos los tipos mencionados arriba. La distribución de tareas la realizarás, como hemos indicado, teniendo en cuenta las metodologías a seguir, pero dependerá de ti y de la persona que te tutoriza el que alternes tareas más de desarrollo con otro tipo de tareas de aprendizaje, lectura de bibliografía y redacción. Esta alternancia puede ayudar a hacer el trabajo menos tedioso, a evitar frustraciones si nos atrancamos en algún punto y nos permitimos pasar a otro tema mientras tanto, así como también puede ayudarnos a tener una conciencia de que estamos progresando un poco en todo. 

Respecto a la distribución de tareas, no aconsejamos que se haga el desarrollo al principio y se escriba la memoria después, sino que se haga al mismo tiempo, que conforme se van concretando las especificaciones y diseños, éstos vayan formando parte de la memoria. Tampoco aconsejamos empezar el desarrollo sin haber hecho una buena revisión de tecnologías y posibles herramientas a utilizar, y sin revisar aplicaciones o trabajos similares. Estas revisiones ayudarán a elegir las mejores herramientas y tecnologías, así como nos darán mejores ideas sobre cuál puede ser nuestra aportación sobre el estado del arte existente. 

Para hacer una planificación debes cuantificar el tiempo que vas a dedicar a cada tarea. Cuanto más pequeña sea la tarea te será más fácil es de calcular cuánto tiempo le debes dedicar, por lo que te aconsejamos llegues a la granularidad más fina que puedas. Organiza las tareas en un calendario semanal en el que también tengas programadas el resto de tus actividades, incluidas las personales. Procura que las tareas del TFG no interfieran en el resto de tus actividades y viceversa, manteniendo un equilibrio. No te pongas tareas del TFG en tiempo de exámenes, de entregas de otros trabajos, o de vacaciones. Planifica con holgura, considerando que puedas siempre tardar un poco más de lo que estimas, porque muchas veces surgen complicaciones e imprevistos y es mejor contar con esa holgura por si acaso.

A lo largo del desarrollo del proyecto, procura seguir el cronograma que has fijado. Revisa tu ajuste a las fechas con frecuencia y en el caso de que haya desviaciones frente a la estimación inicial, haz cambios en la planificación inmediatamente. Los motivos más frecuentes de desvío pueden ser que no se haya hecho una buena estimación inicial, que haya cambios en el alcance (requisitos u objetivos) del proyecto, que haya problemas con las tecnologías o herramientas elegidas para el desarrollo, que no se sea tan productivo como se pensaba, etc. Es muy raro que en algún proyecto no haya que hacer ajustes y por ello hay que considerarlos realizando actuaciones que mitiguen el retraso. Los ajustes pueden ser temporales (p.ej.: trabajar más las próximas semanas) o de alcance del proyecto (p.ej.: replanificar o quitar objetivos o tareas, o cambiar prioridades). Consensúa con la persona que te tutoriza qué hacer en función del motivo del desajuste y el estado del TFG. Esto es importante, ya que si no tomas consciencia a tiempo de que vas con retraso en tu trabajo y no tomas medidas, ese retraso puede hacer que no termines el TFG a tiempo de su entrega, pese a haber dedicado mucho tiempo a él. 

Para realizar tu planificación temporal puedes usar herramientas como los diagramas de Gantt. Hay muchas aplicaciones gratuitas que te facilitan su realización, y muchas de ellas te ayudan a hacer ajustes y cambios, o incluso simulaciones sobre una línea base, para ayudarte a tomar decisiones sobre las mejores opciones. 

% Alberto -> me lo he llevado a Pauta de trabajo. Aparte de esto, te aconsejamos que tomes nota de forma sistemática del tiempo que dedicas al trabajo y en qué tareas. Esto te ayudará a hacer un cálculo final del presupuesto del proyecto, cuantificando el coste de tu trabajo por hora. También te permitirá ver cómo es tu productividad y comparar tu dedicación a diferentes tipos de tareas. Existen herramientas online como \href{https://clockify.me/}{Clockify} que ayudan a apuntar el tiempo dedicado a las tareas y generan varios tipos de informes, aunque también puedes usar una simple hoja de cálculo si lo prefieres.

%%%%%%%%%%%%%%%%%%%%%%%%%%%%%%%%%%%%%%%%%%%%%%
\subsection{Cuestiones éticas}% María José
%%%%%%%%%%%%%%%%%%%%%%%%%%%%%%%%%%%%%%%%%%%%%%

A la hora de preparar el TFG y de redactar la memoria hay que tener en cuenta varios aspectos éticos.
Si en tu TFG vas a hacer algún experimento docente o de salud que implique seres humanos, los principales principios éticos a considerar son los de garantizar el respeto de opiniones, causar beneficio y no daño o perjuicio, y procurar justicia. El TFG es corresponsabilidad del estudiante y la persona que tutoriza, luego ambos debéis garantizar que se cumplan esos principios \cite{EticaCantabria}. 
Si en el TFG se va a hacer un estudio con datos sensibles de personas como los datos clínicos, se van a recoger muestras biológicas, o se va a trabajar datos que les puedan identificar, es necesario que el estudio esté aprobado por el Comité de Ética de la universidad, (\url{https://investigacion.ugr.es/informacion/presentacion/apoyo/comite-etica/evaluacion}), solicitando una evaluación y preparando un consentimiento informado. El consentimiento informado es un documento que firma el participante (o un tutor si el participante es menor de edad) en el que se describe la naturaleza del estudio, se precisa el tipo de participación de las personas que formarán parte del estudio, y que el uso de los resultados será solo para fines docentes, y/o de investigación en su caso. Además, es obligatorio indicar que la participación es voluntaria y que se puede revocar en cualquier momento, sin tener que dar explicación. También se debe asegurar el anonimato de los participantes y la custodia de datos, todo ello recogido por la (\href{https://www.boe.es/buscar/act.php?id=BOE-A-2018-16673}{Ley Orgánica de Protección de Datos Personales y Garantía de Derechos Digitales- LPDGDD}). 
Según la LPDGDD, los datos sensibles de una persona, aparte de los datos médicos, son aquellos que permiten identificar su ideología, afiliación sindical, religión, orientación sexual, creencias u origen racial o étnico. También son sensibles los datos de contacto de las personas, siempre y cuando no sean los de su localización profesional. Si tu TFG recoge algunos de estos datos sensibles, debes seguir las indicaciones de la LPDGDD. Por ejemplo, si vas a acceder a historias clínicas, aseguraos que los datos estén ya anonimizados y que son facilitados por personal sanitario acreditado, que actuará como supervisor. Si en tu TFG vas a usar grabaciones de vídeo o audio de personas, ten en cuenta que también son datos de carácter personal, luego también hay que pedir un consentimiento informado en estos casos. Si la recogida de datos se hace un centro no universitario, debes contar además con autorización de ese otro centro para realizar el estudio.
Si tu TFG ha recogido datos sensibles como los mencionados, y tienes previsto difundir información de él, debes hacer esa difusión exclusivamente a través de los servicios de comunicación de la universidad, así como contar también con la autorización del centro donde se realice el estudio, si es externo a la universidad \cite{EticaUGR}.

La UGR también considera que en un TFG hay otros aspectos éticos a considerar que son los siguientes \cite{EticaUGR}:
\begin{itemize}
    \item Valorar el impacto social y medioambiental de las soluciones técnicas que se proponen, comprendiendo la responsabilidad ética y social.
    \item Usar mecanismos que fomenten la igualdad y participación, rompiendo la brecha digital.
    \item Fomentar el espíritu crítico y transdisciplinar.
\end{itemize}

Si en tu TFG has tenido en cuenta algunos de esos aspectos, debes mencionarlos en la memoria, bien cuando planteas tu solución de diseño o arquitectura, o bien en las conclusiones.

Centrándonos en la ética informática, y según \cite{bynum2000}, ésta identifica y analiza el impacto de la tecnología en los valores sociales y humanos, por ser personas las que manipulan la tecnología. Las asociaciones de profesionales de informática y algunas empresas han elaborado códigos de conducta profesional (muchos de los cuales también están recogidos por la LPDGDD) para regular las nuevas tecnologías y orientar sobre principios específicos relacionados con la ética de su profesión, y que tú, como ya casi profesional de la informática, debes considerar tales como garantizar:  
\begin{itemize}
    \item Privacidad y disposición de información, gestionando el consentimiento en el uso de los datos por parte del participante, con derechos de acceso, rectificación, supresión,Limitación del tratamiento, portabilidad y posición.
    \item Transparencia, seguridad y protección en la recolección, uso, procesamiento, transmisión y comunicación de información: exactitud, anonimización, minimización, 
    \item Uso de las alternativas tecnológicas menos invasivas para no interferir en los derechos de las personas
    \item Responsabilidad sobre el trabajo realizado si se incumple alguna legislación, principalmente para no causar perjuicio económico, moral o social.
    \item Protección y respeto a los derechos de autor y propiedad intelectual, no plagiando y citando el trabajo de otros
\end{itemize}

En resumen, habría que seguir os mandamientos de ética que ya sugería \cite{EticaUCM} en 1992 y que siguen vigentes: No usar el ordenador para dañar a otras personas, robar, dar falso testimonio, plagiar o apoderase del trabajo de otros; Pensar en las consecuencias del programa o sistema que se desarrolla (sobre todo si el sistema toma decisiones, realiza predicciones/clasificaciones, o genera datos, de forma automática); y Garantizar siempre la consideración y el respeto hacia los semejantes.

Asegúrate de que tu TFG respete esos mandamientos y tenga en cuenta la legislación vigente.

%%%%%%%%%%%%%%%%%%%%%%%%%%%%%%%%%%%%%%%%%%%%%%
\section{Sobre la propiedad Intelectual} %Esta sección completa puede ser un anexo
%%%%%%%%%%%%%%%%%%%%%%%%%%%%%%%%%%%%%%%%%%%%%%

En esta sección comentaremos algunos aspectos importantes de cara a llegar a explotar el producto o servicio desarrollado en el TFG. Es especialmente útil para aquellos TFG que se quieran desarrollar con la colaboración de empresas o terceros.

\subsection{Normativa de la universidad}

Cada universidad tiene su propia normativa respecto a la propiedad intelectual e industrial. Por ejemplo, según la Normativa sobre los Derechos de Propiedad Industrial e Intelectual derivados de la actividad investigadora de la Universidad de Granada\footnote{\url{https://www.ugr.es/sites/default/files/2017-08/NCG1151.pdf}} (aprobado en la sesión ordinaria del Consejo de Gobierno de 31 de enero de 2017), los derechos de tu TFG pueden ser tuyos, o compartidos con la persona que te dirige, dependiendo de su implicación.

En el artículo 6 de dicha normativa podemos leer lo siguiente:

\begin{itemize}
\begin{it}
\item b) En el caso de resultados de investigación generados por estudiantes de
la Universidad de Granada bajo la dirección, coordinación, colaboración
o tutorización efectiva de personal de esta Universidad, la titularidad y
propiedad de dichos resultados, así como de los derechos de propiedad
industrial e intelectual derivados de los mismos, corresponderá en
régimen de cotitularidad a la Universidad de Granada y al estudiante en
la proporción en que hubiese contribuido cada parte al resultado,
teniendo en cuenta tanto las aportaciones económicas como
intelectuales relevantes. En caso de imposibilidad de determinar la
contribución al resultado de investigación, se presumirá que la
titularidad corresponde al 50\% a cada uno de ellos.

\item c) En el caso de resultados de investigación generados por estudiantes de
la Universidad de Granada en los que el personal de dicha Universidad
se restrinja a su encargo y/o evaluación, la titularidad y propiedad de
dichos resultados, así como de los derechos de propiedad industrial e
intelectual derivados de los mismos, corresponderá exclusivamente al
estudiante o estudiantes generadores de los mismos. No obstante, y sin
perjuicio de lo anterior, la Universidad de Granada podrá reservarse un
derecho de uso no exclusivo, gratuito e intransferible con fines de
investigación y de docencia.
\end{it}
\end{itemize}

Es decir, en este caso depende del grado de colaboración con la persona que te dirige, por lo que puede ser relevante establecer esta colaboración mediante la aplicación de licencias. Para más información sobre las licencias, por favor, no dejes de consultar el Anexo \ref{anexo:licencias}.



%%%%%%%%%%%%%%%%%%%%%%%%%%%%%%%%%%%%%%%%%%%%%%
%%%%%%%%%%%%%%%%%%%%%%%%%%%%%%%%%%%%%%%%%%%%%%

\section{Sobre la redacción de la memoria} %Alberto

%%%%%%%%%%%%%%%%%%%%%%%%%%%%%%%%%%%%%%%%%%%%%%
%%%%%%%%%%%%%%%%%%%%%%%%%%%%%%%%%%%%%%%%%%%%%%

% añadido por María José en forma más o menos telegráfica, por completar por Alberto:
% - Resumen, debe ocupar una media página. Debe incluir varios párrafos que resuman:  el contexto o motivación, sus objetivos generales, la metodología seguida, las tecnologías o herramientas aplicadas, los resultados obtenidos y una conclusión sobre su aportación. Cuidado, que muchos resúmenes se quedan solo en el contexto y objetivos. Debe ser completo. El resumen debe hacerse al final. 
% - Índice, mejor hacerlo automático al final. Usando estilos para las secciones y párrafos. Numerado. 
% - Agradecimientos: a instituciones y personas
% - Introducción: motivación académica y científica de la pertinencia del TFG: dónde está su innovación, originalidad y utilidad.
% - Conclusiones basadas en revisión de alcance de objetivos, resumiendo cómo e indicando en qué lugar de la memoria puede verse.evidenciar resultados cualitativos y cuantitativos, apoyándose en tablas y gráficos. Interpretar los resultados y comparar con trabajos previos para visibilizar la contribución del trabajo. Argumentar posibles limitaciones del trabajo.

Un gran desarrollo y diseño se origina en la mente del ingeniero, pero uno de los modos más eficaces y eficientes para transmitir y explicar cómo se ha hecho es mediante la redacción de una memoria. 

La memoria suele tener un conjunto de secciones más o menos establecidas y sobre el que nos moveremos entorno a las recomendaciones. No obstante, como ya hemos indicado en este texto, no existe una estructura fija e inamovible, aunque hay ciertos elemento que no deben faltar:

\begin{itemize}
    \item \textbf{Resumen}: con una extensión de una página como máximo debe reflejar de manera concisa y concentrada el motivo del trabajo, sus objetivos iniciales y las conclusiones a las que se llega. No olvides incluir algunos términos claves que ayuden a posicionarlo en una búsqueda.
    \item \textbf{Índice}: es fundamental que haya un índice de contenidos. De manera opcional también es bastante recomendable incluir un índice de figuras y de tablas así como un índice de términos, abreviaturas y glosario que facilite la lectura del documento. La recomendación es saber usar adecuadamente el editor/procesador de texto para que se genere de manera automática incluyendo la numeración y evitando errores.
    \item \textbf{Agradecimientos}: {\it Es de bien nacidos ser agradecidos...} dice el refranero popular, aquí tenemos una clara oportunidad para mencionar a aquellas personas que nos hayan apoyado a lo largo del desarrollo del TFG o de los estudios de Grado. 
    \item \textbf{Objetivos}: Deben estar enunciados en infinitivo y ser concisos. Se han acuñado las siglas: SMART (\textit{Specific, Measurable, Achievable, Realistic, Time-bound}) \footnote{Doran, G. T. (1981). `There's a S.M.A.R.T. way to write management's goals and objectives'' (PDF). Management Review. 70 (11): 35–36.}. Podríamos desarrollar más estas ideas, pero cada palabra realmente tiene un significado propio y un objetivo enunciado debe cumplirlas todas. Normalmente, es sensato tener un objetivo general menos concreto que se puede descomponer en objetivos específicos. Es una buena práctica nombrarlos y numerarlos p.ej. OG1, OG2, OE1,OE2,... La ubicación de los objetivos en la memoria debe ser al principio, hay quien prefiere numerarlos como la sección 0 y hay quien prefiere incluirlos como una subsección de la Introducción o la Motivación. En cualquier caso, es obvio que deben estar al principio de la memoria para que el lector tenga claro cómo y por qué se desarrollan las siguientes secciones que encuentre.

    \item \textbf{Introducción}: Es la primera sección de nuestra memoria y donde tenemos más flexibilidad para desarrollarla según nuestros intereses. Aquí puedes contar cómo surgió la idea del trabajo, así como consolidar por escrito el conocimiento del dominio del problema adquirido durante su desarrollo. Es bastante adecuado incluir un estado del problem (\textit{State-of-the-art)} donde se haga gala de haber revisado la cuestión y los distintos enfoques y soluciones ya propuestos. No pasa nada si alguien ya ha tratado la cuestión, en el mundo académico, ser capaz de reproducir algo ya es un avance y no ser capaz de reproducirlo también da señales de que hay que seguir trabajando en esa dirección. No podemos subir alto sin apoyarnos en hombros de gigantes, pero hay que reconocerles el mérito que han tenido.

    \item \textbf{Conclusiones}: Escribirlas bien es todo un reto porque deben reflejar en unos párrafos el problema, la solución alcanzada y los resultados... como primera aproximación, lee los objetivos de nuevo, comenta si se han alcanzado y el porqué no en caso de no haber llegado. Maximiza la información y evita cualquier palabra o frase innecesaria. No es el lugar para discutir los resultados, es el lugar para concluir el resultado de esa discusión. 

    \item \textbf{Bibilografía}: Aquí listamos esos hombros de gigantes que nos han servido de inspiración y apoyo para llegar a buen puerto. Cuidemos que su reconocimiento es correcto y que están todos los datos sin errores. En el caso de tener enlaces web, es recomendable incluir la última fecha de consulta. Respecto a los textos, no descuides las referencias básicas de la disciplina (aunque parezcan obsoletas) y procura tener referencias recientes para demostrar haber revisado bien la cuestión tratada en tu memoria. No se recorren 100 Km en día, pero recorrer 1Km al día durante 100 días es más asequible. Por cada texto o web consultada, crea tu referencia con un minipárrafo sobre lo que te ha gustado más y al final del proceso tendrás el camino hecho sin apenas esfuerzo.
    
\end{itemize}

Partiendo de ese esqueleto más lo comentado para cada tipo de TFG, tenemos un esqueleto sólido que sustente nuestras preferencias personales.

%%%%%%%%%%%%%%%%%%%%%%%%%%%%%%%%%%%%%%%%%%%%%%
\subsection{Estilo de redacción} %Alberto 
%%%%%%%%%%%%%%%%%%%%%%%%%%%%%%%%%%%%%%%%%%%%%%
% añadido por María José en forma más o menos telegráfica, por completar por Alberto:

% Ver y comentar: 
%  https://recursoslinguisticosblog.files.wordpress.com/2019/11/registro-de-marcas-lingc3bcc3adsticas-para-pc3a1gina-final-protegido.pdf : aunque este último es para tesis, puede ayudar a redactar partes de la memoria, ya que sugiere cómo, qué palabras o trozos frases utilizar en la redacción. 

%  - Mejor redactar en impersonal ("se observa") o en tercera persona del plural ("observamos"). No se aconseja redactar en primera persona ("observo").

Aunque lo ideal es predicar con el ejemplo, este libro que estás leyendo no es un TFG y por eso hemos considerado que tutearte puede ser el estilo de redacción más adecuado para que te sientas interpelado y el mensaje te llegue con más claridad. 

En el caso de un TFG, debemos considerar que es un documento académico y, por tanto, debemos procurar ser correctos y precisos en su redacción, cuidando los estilos de redacción. Estos textos, normalmente, suelen escribirse de manera impersonal, p.ej. se han hecho estos experimentos, se ha considerado esta solución, se han evaluado este método, etc. También es común usar la primera personal del plural: hemos desarrollado, etc. que no entra en conflicto con tener múltiples personalidades porque la persona que te tutoriza está implicada en el TFG. 

Desde nuestra perspectiva, te recomendamos usar siempre el impersonal, y coincidimos con otras universidades al respecto \cite{UPC-guia}. Te puede parecer un poco forzado a veces, pero enfócalo desde la perspectiva de ejercicio académico y considera que la estética en el lenguaje también es algo que el humano ha desarrollado y es capaz de apreciar.

Otras recomendaciones comunes que facilian la lectura de cualquier texto, en general son:
\begin{itemize}
    \item Usa frases cortas y párrafos breves. Existen los puntos y las comas, úsalos.
    \item Usa listas (como esta) para presentar ideas, luego siempre puedes desarrollarlas.
    \item Procura que cada párrafo aporte información, y evita que haya repeticiones salvo que quieras remarcarlo explícitamente o dar diferentes puntos de vista o formas de explicarlo.  
    \item Cuida las faltas de ortografía y los errores gramaticales (tildes, comas, mayúsculas, etc.)
    \item Incluye las definiciones de los acrónimos y siglas la primera vez que los usas. Puedes añadir, además, un glosario de términos.
    \item Salvo excepciones, procura no usar expresiones coloquiales.
    \item Intenta ser lo más objetivo posible, puedes expresar opiniones personales pero siempre apoyado de referencias y procura usar el condicional. Ejemplo: "Python es el lenguaje más usado como indica el índice TIOBE", en lugar de "Python es el lenguaje más usado". Otro ejemplo: "Las redes neuronales son una técnica muy usada en la actualidad" en lugar de "Las redes neuronales son la técnica más usada en la actualidad".
    \item Cada párrafo debe aportar algo (sí, lo repetimos porque es importante) pero también debe ir enlazado con el anterior y el posterior. No solo a nivel de contenido, sino también a nivel de redacción.
\end{itemize}

%CONSULTAR (pedido a biblioteca)

 %El español académico : guía práctica para la elaboración de textos académicos
%Regueiro Rodríguez, María Luisa
%Madrid : Arco Libros, 2013
%Bibliografía recomendada

%%%%%%%%%%%%%%%%%%%%%%%%%%%%%%%%%%%%%%%%%%%%%%
\section{Sobre la maquetación}  %Creo que Alberto es el que debe hacer esta sección.
%%%%%%%%%%%%%%%%%%%%%%%%%%%%%%%%%%%%%%%%%%%%%%
%añadido por María José por si és útil: ver y mencionar si procede
%https://sistema-artext.com/medicina/trabajo-de-fin-de-grado-(tfg)
%Numeración de figuras y tablas poniendo un primer número el del capítulo en el que se está, para facilitar la renumeración, el añadir nuevas y categorizarlas mejor.
%Pablo: recomendar uso de Latex, también por la gestión de bibliografía



Aunque pueda parecer mucho más complejo, para un estudiante de grado TIC, no debería suponer un problema el usar \LaTeX. Una vez superado el miedo y la posible frustración inicial, usando una plantilla y centrándote en el contenido, el resultado será un documento maquetado perfectamente.

El utilizar otro tipo de procesador de texto es posible pero, a la larga, va a complicar bastante la maquetación y, aunque el contenido es lo más importante, el cómo esté presentado también hay que cuidarlo.

En primer lugar, debemos incluir una portada que contenga, como mínimo, los siguientes elementos:
\begin{itemize}
    \item Título del TFG
    \item Nombre y apellidos del autor
    \item Nombre del tutor
    \item Nombre del grado o titulación
    \item Nombre de la universidad, del centro y del departamento
    \item Fecha de convocatoria
    \item Curso académico
\end{itemize}

En la portada, además, se suele incluir el logo de la universidad y, en algunos casos, el logo del departamento o centro.

A partir de ahí también es interesante incluir unos agradecidos o dedicatorias a la gente que te ha apoyado en el desarrollo del TFG y de la carrera. Como dice el refranero, {\it de bien nacidos es ser agradecidos}.

Según la normativa, se suele requerir un resumen donde se debe incluir los objetivos, la metodología, las tecnologías y herramientas usadas, los resultados y las conclusiones. Este resumen debe ser claro y conciso, y no debe superar una página de extensión y debemos acompañarlo con su traducción al inglés.

Posteriormente, debemos incluir un índice de contenidos, que debe estar numerado incluyendo al menos los capítulos y secciones (aunque también podemos incluir subsecciones). También es recomendable incluir un índice de figuras y otro para las tablas que hayamos incluido en el documento.

Respecto a cómo dividir las estructura de la memoria, debemos agrupar la información en capítulos, secciones, etc. como se desarrolla en el siguiente capítulo \ref{cap:EstructuraMemoria} pero aquí debemos mencionar que es necesario establecer un mismo formato para los títulos de cada capítulo, sección y subsección.

Respecto a las figuras y las tablas, deben ser referenciadas durante el texto y numeradas. Además, debemos incluir una leyenda que explique qué representa la figura. Lo normal es ubicarlas centradas respecto al texto y evitar que la leyenda esté en otra página. Si ves que no va a caber, divide la imagen o la tabla en subimágenes y subtablas.

Todo lo mencionado hasta aquí se gestiona automáticamente usando \LaTeX sin apenas esfuerzo y con un resultado de calidad profesional.

\subsection{Presentación}

La elaboración de la presentación está desarrollada en el capítulo \ref{cap:elaboraciónPresentación}, pero, como recomendaciones generales de cara a la presentación considera que es un apoyo a tu discurso, no algo que debas leer. Piensa en ella como un apoyo al que te escucha también, el tribunal puede distraerse con una idea del discurso y hay que facilitar que se enganche de nuevo. No se exige belleza ni diseño, pero es innegable que algo bien presentado resulta más atractivo y también puede ser un indicador de la capacidad de esfuerzo y trabajo del estudiante. Pero, recuerda, lo más importante es que se entienda el mensaje, si la belleza o el diseño dificultan la legibilidad o la claridad de la exposición, ve a lo práctico. Quizá un enfoque iterativo sea una aproximación correcta. Termina una presentación muy simple y sencilla y luego la enriqueces.

%%%%%%%%%%%%%%%%%%%%%%%%%%%%%%%%%%%%%%%%%%%%%%
\section{La finalización del proyecto}
%%%%%%%%%%%%%%%%%%%%%%%%%%%%%%%%%%%%%%%%%%%%%%

Llega el momento en el que tienes que tener terminado ya tu trabajo para entregarlo y defenderlo, pero... ¿lo tienes terminado? ¿cumples todos los objetivos que os habéis propuesto? Si es así, entonces seguramente sólo te quedará cerrar algunos flecos de la memoria y del producto a desarrollar y lo podrás entregar en tiempo y forma en la convocatoria que te habías marcado como objetivo.

¿Pero qué ocurre si se da alguna de las situaciones anteriores? Claramente, tendrás que valorar el posponer a la siguiente convocatoria la entrega. Si te falta todavía la consecución de algunos objetivos, habla con la persona que te tutoriza y determina si esos que faltan son realmente necesarios y si lo son, entonces márcate un nuevo límite de entrega y a seguir trabajando. Podría darse el caso que esos últimos objetivos sean menores y su no consecución no dañara las bases de un TFG de calidad. En ese caso, seguramente tendrás el visto bueno de la persona que te tutoriza para entregarlo. 

Si tú quieres entregarlo pero la persona que te tutoriza no lo ve claro porque considere que no está en condiciones, aunque tú eres soberano para hacerlo cuando lo consideres oportuno y a pesar del criterio del docente, siempre es bueno llegar a acuerdos y consensos y no tomar tus propias decisiones. Las que son unilaterales sólo son perjudiciales para ti, sobre todo en términos de nota. A veces es mejor renunciar a una convocatoria de forma voluntaria que hacerlo con mal ambiente o con un suspenso debajo del brazo al haber forzado la situación en contra del criterio de la persona que te tutoriza. 

Ese paso atrás cuando no está totalmente terminado también te puede servir para ganar unos meses hasta la siguiente convocatoria y poder finalizarlo y obtener una nota mucho más merecedora del trabajo y del esfuerzo realizado. 

Si ya lo has entregado y defendido y has tenido una buena nota, no lo dudes: celébralo. Has trabajado duro y la calificación lo refleja. Si la nota no es tan buena como tú esperabas o desearías, debes aceptarlo y aprender de los comentarios que te hayan indicado los profesores.  Y si aún así no estás de acuerdo, siempre tienes a tu disposición los diferentes canales que  te ofrece la universidad para revisar los exámenes, que se aplican igualmente para el caso de un TFG. Sé profesional y nunca te lo tomes como algo personal y procura mantener la corrección en las formas con los miembros de la comisión, tampoco están felices con el resultado, solo están haciendo su trabajo lo mejor posible. 

%%%%%%%%%%%%%%%%%%%%%%%%%%%%%%%%%
\subsection{Lecciones aprendidas}
%%%%%%%%%%%%%%%%%%%%%%%%%%%%%%%%%

Y ya, una vez terminado el TFG, defendido y con la nota puesta, llegas a tu casa y te sientas tranquilamente (perdido, sabiendo que te falta algo, desorientado, sin saber qué hacer...). Pues bien, es buen momento para hacer una reflexión general sobre todo el proceso en todas y cada una de sus facetas y sobre lo que has aprendido. Esta reflexión te permitirá analizar tu rendimiento , detectar debilidades y puntos fuertes, experiencias positivas y negativas, y sobre todo, mejorar como persona y profesional estando en mejores condiciones para afrontar exitosamente el futuro cercano donde comenzarás tu carrera profesional.

Algunas cosas que te puedes plantear son las siguientes:

\begin{itemize}
    \item Sobre el desarrollo del proyecto:
        \begin{itemize}
            \item Organización: ¿Crees que te has organizado bien al trabajar?¿Deberías haber dedicado menos tiempo al TFG o haber empezado antes? Si lo tuvieras que hacer de nuevo (¡menos mal que no!), ¿organizarías tu tiempo de la misma manera? Las respuestas a estas preguntas te indicarán qué has aprendido de organización temporal de tareas en un proyecto, lo cual podrás aplicar al ámbito profesional.
            \item Metodología: ¿Has estado cómodo/a con la metodología seguida en el proyecto?¿Has usado los métodos y herramientas adecuados para conseguir los objetivos de tu TFG?¿Te ha ayudado el TFG a aprender nuevos métodos, y herramientas, incluidos lenguajes de programación, etc?¿Te sientes más capacitado para afrontar el mundo laboral después de hacer el TFG? Si respondes que sí a estas preguntas, enhorabuena. En caso contrario, plantea qué deberías mejorar para futuros proyectos.
            \item Resultados: ¿Has podido alcanzar todos los objetivos planteados de forma satisfactoria para ti? ¿Crees que has hecho un buen TFG o que podrías haber obtenido mejores resultados? Si tus respuestas son positivas, estupendo. Si son negativas, debes revisar qué ha fallado: por ejemplo, la viabilidad de los objetivos, tu rendimiento, tu formación, tu organización, o la comunicación con la persona que te tutoriza. Esta reflexión debe ayudarte a mejorar los aspectos que han fallado, de tal forma en proyectos futuros te sientas mejor al realizarlos, y tus resultados sean mejores.
        \end{itemize}
    \item Sobre la relación con la persona que te tutoriza:
        \begin{itemize}
            \item ¿Has trabajado de forma conjunta con la persona que te tutoriza o, por el contrario, ha sido de una manera aislada? La respuesta puede darte pistas, salvando las distancias, sobre si eres una persona que trabaja bien o no en equipo. Ten presente que en tu futuro trabajo casi con un 99\% de probabilidad vas a trabajar en un equipo y debes saber cómo relacionarte con tus compañeros de forma efectiva. 
            \item ¿Eres capaz de defender tus ideas y convencer de que son buenas o por el contrario asumes las que te indican de forma directa? ¿Eres capaz de realizar críticas a tus ideas o planteamientos así como a los de los demás? Estas preguntas te puede dar una idea de la capacidad de convicción que tienes y de crítica (constructiva siempre), capacidades que te resultarán muy útiles en tu trabajo.
        \end{itemize}
    \item Sobre la escritura de la memoria:
        \begin{itemize}
            \item ¿Te ha costado mucho escribir la memoria? ¿Por qué? ¿Te cuesta trabajo plasmar las ideas por escrito? Si es así, entonces debes plantearte mejorar esta faceta o alguno de sus aspectos. Quizá en los ratos libres podrías inscribirte en algún curso de escritura. Esto es fundamental para un ingeniero pues es una forma de expresar ideas y hechos muy importante. Y también lee más, no sólo manuales técnicos o páginas web, sino novela, poesía, etc. De esta forma, además de pasarlo bien, estarás mejorando tu capacidad de expresión.
        \end{itemize}
    \item Sobre la presentación y la defensa:
        \begin{itemize}
            \item ¿Has necesitado apoyarte mucho en recursos a la hora de presentar o has sido capaz de describir tu trabajo sin apenas apoyo?
            \item ¿Te has expresado correctamente?
            \item ¿Eres capaz de contar mis ideas de forma breve y concisa?
            \item ¿Respondes correctamente a las preguntas sin irte por las ramas?¿Te falta seguridad para responder a las preguntas?
            \item ¿Te ajustas a los tiempos establecidos?
            \item ¿Aceptas de buen grado las críticas y sugerencias?
            \item En las respuestas a estas cuestiones podrás evaluar tu capacidad de comunicación oral y de síntesis. En el futuro tendrás que hacer presentaciones y el objetivo será convencer de que lo que planteas es bueno. La comunicación oral se configurará como un aliado para conseguir ese objetivo. Si consideras que tienes que mejorar, inscríbete en cursos de comunicación verbal.
        \end{itemize}
\end{itemize}

%%%%%%%%%%%%%%%%%%%%%%%%%%%%%%%%%
\subsection{Ingeniería Curricular} 
%%%%%%%%%%%%%%%%%%%%%%%%%%%%%%%%%

Hay veces que no es cuestión de \textit{sobretrabajar}, sino de trabajar con inteligencia. Otra forma de decirlo es que {\it cantidad no es calidad}. En este sentido, conviene conocer y tener presente las rúbricas de evaluación que utilizarán tutores y comisiones (cuando las haya). Estas rúbricas habrán sido consensuadas por el profesorado y, aunque en algunos términos puede ser ambiguas para dar cabida a todas las casuísticas, normalmente ofrecen, un norte claro con el que guiarse.  

Por ejemplo, la rúbrica de evaluación para tutores de la ETSIIT de Granada está disponible en \url{https://grados.ugr.es/informatica/pages/infoacademica/tfg/evaluacion/rubrica_informe_tutores_tfg/! } y la de la comisión en \url{https://grados.ugr.es/informatica/pages/infoacademica/tfg/evaluacion/rubrica_comision_evaluadora_tfg/!}. Ambas son similares y cubren aspectos ya tratados en este documento, pero de manera mucho más resumida.

En la Tabla \ref{tab:rubricas} te damos recomendaciones para que se visibilice tu trabajo, de tal forma que puntúe alto en cada ítem de las rúbricas.

\begin{sidewaystable}[]
    \centering
\resizebox{22cm}{!}{
\begin{tabular}{|p{3cm} |p{5cm}| p{15cm}|}
\hline
Ítem & Descripción breve de aspectos a evaluar & Recomendaciones \\ \hline

Búsqueda y tratamiento de la información· & Calidad, cantidad y variedad de las fuentes. Análisis y comprensión de la información
 & Bibliografía actualizada, a ser posible de fuentes serias y avaladas científicamente como revistas o congresos. Uso de tablas para sintetizar, analizar y comparar. Cuadros o figuras para resaltar tus contribuciones. Negrita o resaltar texto en conclusiones o valoraciones.  \\ \hline
 
Autonomía e iniciativa & Capacidad en la toma de decisiones. Planteamiento de propuestas propias & Resaltar opciones iniciales y decisiones tomadas en introducción y al empezar con la propuesta. Uso de tablas que incluyan la propuesta en detrimento de otras opciones \\ \hline

Planificación  & Análisis de requisitos, identificación de tareas y recursos, cumplimiento de objetivos &  Inclusión de al menos un diagrama de Gantt con la planificación final, que incluya recursos utilizados. Revisión en las conclusiones del cumplimiento de los objetivos específicos.\\ \hline

Análisis y propuestas & Analiza y propone alternativas al problema planteado. La solución conceptual propuesta es correcta. Argumenta y analiza los resultados obtenidos. Las conclusiones son adecuadas para el trabajo desarrollado. & Uso de tablas comparativas que revisen alternativas tecnológicas para usar en el estado del arte. Inclusión en el capítulo de la propuesta de una sección o al menos una explicación sobre cuáles son tecnologías elegidas para el desarrollo y porqué. Inclusión en el capítulo de la propuesta de un diagrama del sistema, con una explicación general de la arquitectura de la solución. Resaltar los resultados y explicar porqué son buenos. \\  \hline

Calidad técnica de la solución & Calidad técnica de la documentación y de la solución &  Respecto a la documentación, revisa bien la memoria, que esté bien escrita y bien estructurada, con un formato y aspecto cuidados. Pide ayuda a familia o amigos para ello. En cuanto a la solución técnica, indica explícitamente porqué tiene calidad, diciendo por ejemplo si las tecnologías usadas son nuevas, actuales, las más adecuadas para el problema planteado, etc. Busca y cita fuentes que las hayan usado previamente de forma satisfactoria, y/o que indiquen cuáles son sus ventajas.\\ \hline

Conocimientos, habilidades y competencias &	Idiomas. Requisitos éticos y legales. Aplica correctamente los conocimientos, competencias y habilidades adquiridos en el grado, o contiene conocimiento actual o de vanguardia. & Incluye referencias en inglés. Incluye referencias a legislación considerada o a aspectos éticos, si es el caso. Haz referencia a conocimiento adquirido en asignaturas previas o a trabajos científicos, en la sección de estado del arte o en tu propuesta. Resalta en el capítulo de conclusiones qué conocimientos, habilidades y competencias has adquirido. Revisa listas existentes de habilidades blandas y competencias para inspirarte.  \\ \hline

Comunicación escrita &	Corrección en ortografía y gramática, vocabulario, expresión de ideas y argumentación. Formato y presentación de la memoria correctos. &  Lo mismo que en Calidad técnica de la solución. \\ \hline

Comunicación oral &	Expresión oral correcta, comunicativa. Responde adecuadamente a las preguntas y capacidad para el debate. & Prepara bien tu discurso, ajústate al tiempo, ensaya varias veces, prepara posibles preguntas y respuestas, y confía en tus conocimientos y en tu trabajo. Haz una exposición seria y profesional. \\ \hline

Otras cuestiones opcionales a valorar & Capacidad emprendedora, comercial o innovadora. Manejo de nuevas tecnologías, no vistas en el Grado. Alineamiento con ODS (objetivos de desarrollo sostenible). & Añade en la introducción y conclusiones de la memoria si tienes intención de emprender con este TFG o si ya lo has hecho. Añade en tus objetivos y en la propuesta que vas a usar tecnologías no vistas en el grado y resalta y en las conclusiones el trabajo que te ha supuesto y lo que has aprendido. Revisa los ODS para ver si cumples con alguno. Si es así, plantéalo como objetivo, que forme parte de alguna tarea, y resáltalo en las conclusiones. \\  \hline
\end{tabular}
}
    \caption{Recomendaciones para cubrir las rúbricas}
    \label{tab:recomendaciones}
\end{sidewaystable}


%%%%%%%%%%%%%%%%%%%%%%%%%%%%%%%%%
\subsection{Recomendaciones generales para el uso de la IA} % TODOS
%Revisado por: Rocio
%Revisado por:
%%%%%%%%%%%%%%%%%%%%%%%%%%%%%%%%%
%Lo que hay por ahora ha sido completado por MJ
La memoria del TFG debes escribirla tú. La persona que te tutoriza la supervisará y evaluará tu trabajo. Finalmente, un tribunal la leerá y evaluará, junto con el resto de tu trabajo, lo que hayas desarrollado o investigado. Las herramientas de Inteligencia Artificial (IA) Generativa pueden ayudarte a mejorar varios aspectos de tu memoria para que sea más pertinente, más completa y mejor evaluada, pero no te aconsejamos que la IA redacte tu memoria o parte de ella. Sí te aconsejamos que la uses para mejorar los textos redactados previamente por ti (consulta esta y más ideas en \cite{IAGenerativaUNED}), por ejemplo, puedes pedirle que a partir de un texto: 
\begin{itemize}  
    \item detecte errores
    \item busque puntos débiles
    \item mejore la gramática
    \item cambie el tono para que sea más formal, académico o conciso
    \item traduzca 
    \item elabore un esquema
    \item haga un resumen
    \item extraiga palabras clave
    \item genere preguntas para autoevaluación o ayuda a la revisión 
\end{itemize} 

Si te bloqueas y solo ves una página en blanco, puedes pedir que te haga una lluvia de ideas sobre un tema, o que te sugiera mejoras o ampliaciones desde otras perspectivas, por ejemplo, histórica, legal, económica, tecnológica, etc. Fíjate que no le estamos pidiendo que nos escriba el contenido, solo que nos ayude a avanzar.

La IA generativa también puede ayudarte a hacer un primer calendario de trabajo, si le indicas bien las tareas a realizar y el tiempo del que dispones.

Hay algunas herramientas de IA generativa que te indican cuáles son las fuentes en las que se han basado, de tal forma que puedas usarlas en tu memoria. Pero ¡cuidado!, eres responsable de tu memoria, por lo cual, debes comprobar que sean correctas, ya que muchas veces estas herramientas pueden no ser correctas o inventadas. Si desconoces las fuentes, puedes estar incurriendo en incumplimiento de la LPDGDD que hemos mencionado antes, y en plagio, por eso, debes leer muy bien lo que se genera y verificarlo antes de incluirlo en tu memoria.

De manera muy esquemática te compartimos nuestras reflexiones sobre un tema complejo para que, mientras se regula, puedas avanzar más rápido y con seguridad:

\begin{itemize}  %Completado por MJ
    \item Si usas herramientas de IA Generativa del tipo ChatGPT, Copilot, Perplexity, etc., debes indicarlo en la memoria y referenciarlo así, por ejemplo:
    \textit{\\https://chat.openai.com/chat}.
    \item Incluye (como Anexo) los prompts que has utilizado para hacer consultas.
    \item Contextualiza bien tus preguntas, no le digas que se ponga en el papel de alguien que no eres tú, como un gurú tecnológico o un profesional experimentado en el dominio de aplicación de tu trabajo, ya que tanto la redacción como los contenidos de la respuesta serán difíciles de entender y no encajarán en la memoria.
    \item No incluyas nunca en los prompts información personal ni confidencial, ya que esa información puede ser usada por los sistemas de generación de conocimiento.
    \item No incluyas en los prompts textos o enlaces a material protegido por derechos de autor, con Copyright, ya que no tienes autorización para usarlo en ningún sitio.
    \item Revisa bien los textos generados, porque suelen ser superficiales y poco fiables. En ocasiones hay partes inventadas, llamadas \textit{alucinaciones}, que tendrás que identificar para eliminar.
    \item Asegúrate de que entiendes lo que se ha generado, en primer lugar para poder identificar las alucinaciones mencionadas, y en segundo lugar para poder integrar lo generado en la memoria, como parte del conocimiento que revisas o usas.
    %\item 
\end{itemize}

%Como anécdota te compartimos las recomendaciones de Co-pilot de Bing sobre su uso:

%\begin{figure}
%    \centering
%    \includegraphics[scale=0.5]{images/Bing.png}
%    \caption{Respuesta (1/3) de Co-pilot de Bing ante la pregunta: ¿Qué recomendaciones tienes para usar la IA generativa en la realización de un Trabajo Fin de Grado de Ingeniería Informática? }
%    \label{fig:Bing}
%\end{figure} 

%\begin{figure}
%    \centering
%    \includegraphics[scale=0.5]{images/Bing2.png}
%    \caption{Respuesta (2/3) de Co-pilot de Bing ante la pregunta: ¿Qué recomendaciones tienes para usar la IA generativa en la realización de un Trabajo Fin de Grado de Ingeniería Informática? }
%    \label{fig:Bing1}
%\end{figure} 

%\begin{figure}
%    \centering
%    \includegraphics[scale=0.5]{images/Bing3.png}
%    \caption{Respuesta (3/3) de Co-pilot de Bing ante la pregunta: ¿Qué recomendaciones tienes para usar la IA generativa en la realización de un Trabajo Fin de Grado de Ingeniería Informática? }
%    \label{fig:Bing2}
%\end{figure} 

Como última recomendación, pregunta y revisa las directrices que indique tu Universidad, tu centro o tu coordinador del plan de estudios.


% Consejos sobre el proceso de desarrollo al estilo de lo que pone María José en sus diapositivas.
% Sobre el código (control de versiones, licencias…) (Aquí puedo ayudar yo (Pablo))
% Sobre cuestiones éticas de código y datos (Alberto).
% Qué cosas hacer y qué no hacer con ejemplos.


%%%%%%%%%%%%%%%%%%%%%%%%%%%%%%%%%%%%%%%%%%%%%%%%%%%%%%%%%%%%%%%%%%%%%%%%%%%%%%
%%%%%%%%%%%%%%%%%%%%%%%%%%%%%%%%%%%%%%%%%%%%%%%%%%%%%%%%%%%%%%%%%%%%%%%%%%%%%%
%Lo consensuado en la reunión del 21/02/24
%%%%%%%%%%%%%%%%%%%%%%%%%%%%%%%%%%%%%%%%%%%%%%%%%%%%%%%%%%%%%%%%%%%%%%%%%%%%%%
%%%%%%%%%%%%%%%%%%%%%%%%%%%%%%%%%%%%%%%%%%%%%%%%%%%%%%%%%%%%%%%%%%%%%%%%%%%%%%

%%%USAR https://docs.google.com/document/d/115O_20YW4Mjx1iJBBxxxDCDWJOBEL_rTzbeHVmI8Cak/edit#heading=h.n5bem9dnws3z



\chapter{La estructura general de la memoria}
\label{cap:EstructuraMemoria}
% [Autores: Pablo]
% Introducción general a cada una de las partes de la memoria.
% (este quizás sea de los últimos capítulos en escribir)
%Pablo revisa/copia/quita lo que vea oportuno de las recomendaciones generales -> Sobre la redacción de la memoria
% Lo siguiente añadido por María José
La estructura general de la memoria la debes consensuar con la persona que te tutoriza. Aquí te damos una sugerencia de una posible organización de capítulos y secciones dentro de cada capítulo.


\begin{enumerate}
    \item Introducción
        \begin{itemize}
            \item Contexto/Antecedentes
            \item Justificación/Motivación
            \item Objetivos/Hipótesis
            \item Planificación temporal
            \item Presupuesto
            \item Estructura de la memoria
        \end{itemize}
    \item Estado del arte
        \begin{itemize}
            \item Descripción de dominio del problema
            \item Metodologías potenciales a aplicar
            \item Tecnologías potenciales para usar
            \item Trabajos relacionados
        \end{itemize}
    \item Propuesta
    \begin{itemize}
            \item Descripción de la propuesta
            \item Metodología
            \item Secciones según la metodología y tipo de proyecto
        \end{itemize}
    \item Conclusiones y trabajos futuros
    \begin{itemize}
            \item Conclusiones
            \item Trabajos Futuros
        \end{itemize}
\end{enumerate}

En el capítulo del estado del arte se pueden incluir tantas secciones como se desee para agrupar bien los tipos de revisiones realizadas. Si la envergadura de estas secciones es muy grande, pueden separarse en capítulos aparte.

Tal y como se indica en el capítulo 6, \ref{cap:RevisionEstadoDelArte}, cada sección del estado del arte que se desee incluir deberá constar primero de una introducción explicando la metodología seguida para la revisión, luego la revisión concreta y al final unas conclusiones. Es muy aconsejable incluir tablas con aspectos comparativos, ya que son más visuales y sirven de resumen. Por ejemplo, se recomienda incluir comparativas entre las potenciales tecnologías o metodologías, para luego en el capítulo de la propuesta justificar cuáles de ellas son las elegidas para la solución. También es común comparar los trabajos relacionados entre sí y según nuestros objetivos o requisitos.

Las secciones del capítulo en el que describimos nuestra propuesta están muy relacionadas con el tipo de proyecto y metodología. Vamos a dar unas guías de cómo podría estructurarse según ello, aunque tienes información más detallada en un capítulo aparte para cada tipo de proyecto..

\begin{itemize}
    \item \textbf{Proyectos de investigación. (Ver \section{TFG de investigación}
\label{appendix:investigacion}

% (Por si puede ser de interés: https://bpb-us-w2.wpmucdn.com/portfolio.newschool.edu/dist/2/14941/files/2017/06/Judith_Bell_Doing_Your_Research_Project-xhunbu.pdf)
% Se podría dar un layout genérico para seguir en un TFG de este tipo.

% \subsection{Introducción}
%OJOJOJOJOJO ¿¿¿¿¿METER ALGUNA REFERENCIA BIBLIOGRÁFICA??????
La Ingeniería Informática se centra en la aplicación de conocimientos teóricos-prácticos para ofrecer la mejor solución posible a un problema, teniendo en cuenta las dimensiones temporales, personales, materiales y económicas. La Ingeniería Informática actual cuenta con desafíos científicos de gran envergadura en cada una de sus especialidades, destacándose a continuación algunos de ellos:
\begin{itemize}
    \item los retos electrónicos de aumentar el nivel de integración de las placas de unidades de procesamiento, ya sean de datos (CPU) o especializadas en gráficos (GPU), fundamentales para continuar ampliando la capacidad de cómputo de los sistemas informáticos;
    \item el amplio y complejo reto computacional y social que representa la inteligencia artificial;
    \item el continuo empeño en mejorar todo lo relativo al procesamiento geométrico para impulsar el avance de la representación gráfica;
    \item el objetivo de conseguir una informática sostenible a través de la mejora de la aplicación de los conceptos de la informática teórica para el diseño y desarrollo de algoritmos eficientes en tiempo y en espacio;
    \item la responsabilidad de mejorar las metodologías y métodos de desarrollo para la programación de sistemas informáticos seguros, robustos y respetuosos con la privacidad de los datos.
\end{itemize}
%Sin embargo, la aplicación trasversal de la informática hace que la Ingeniería Informática esté presente en el trabajo científico diario de un amplio espectro de disciplinas, sobresaliendo la investigación en medicina, química o biología. Los retos científicos propios y ajenos a los que se tiene que enfrentar la Ingeniería Informática, obliga al graduado a disponer de unas mínimas habilidades científicas, que le permitan aplicar los principios de la Ingeniería Informática al método científico, y contribuir, de esta forma, al progreso científico de cualquier disciplina.

La Ingeniería Informática es una disciplina muy transversal, por tanto, tenemos el reto de ser excelentes dentro de nuestro propio campo y, además, debemos aprender y conocer el dominio del problema donde aplicaremos la solución. En el contexto de un trabajo de investigación, adicionalmente, tendremos que respetar las pautas del método científico. 

%La Ingeniería Informática, al igual que otras ingenierías, trasciende las fronteras clásicas de la ingeniería. E, es decir, a la aplicación de conocimientos teórico-prácticos para ofrecer la mejor solución posible a un problema con unas restricciones temporales, personales, materiales y económicas. La Ingeniería Informática actual cuenta con desafíos científicos de gran envergadura en cada una de sus especialidades, destacándose a continuación algunos de ellos: los retos electrónicos de aumentar el nivel de integración de las placas de unidades de procesamiento, ya sean de datos (CPU) o especializadas en gráficos (GPU), fundamentales para continuar ampliando la capacidad de cómputo de los sistemas informáticos; el amplio, ilusionante y complejo reto computacional y social que representa la inteligencia artificial; el continuo empeño en mejorar todo lo relativo al procesamiento geométrico para impulsar el avance de la representación gráfica; el objetivo de conseguir una informática sostenible a través de la mejora de la aplicación de los conceptos de la informática teórica para el diseño y desarrollo de algoritmos eficientes en tiempo y en espacio; la responsabilidad de mejorar las metodologías y métodos de desarrollo para la programación de sistemas informáticos seguros, robustos y respetuosos con la privacidad de los datos; o el desafío de mejorar las metodologías de trabajo con el fin de que la Informática continúe avanzando con una verdadera ingeniería. Sin embargo, la aplicación trasversal de la informática, hace que la Ingeniería Informática esté presente en el trabajo científico diario de un amplio espectro de disciplinas, sobresaliendo la investigación en medicina, química o biología. Los retos científicos propios y ajenos a los que se tiene que enfrentar la Ingeniería Informática, obliga al graduado en Ingeniería Informática a disponer de unas mínimas habilidades científicas, que le permitan aplicar los principios de la Ingeniería Informática al método científico, y contribuir de esta forma al progreso científico de cualquier disciplina.

La tipología de {\it TFG de Experimentación Científica} representa la oportunidad de aplicar los conocimientos adquiridos durante el grado a la resolución de un problema científico en el que interviene una metodología informática o sistema computacional, y por tanto demostrar y ampliar el desarrollo de las habilidades científicas del futuro graduado en Ingeniería Informática.
%La tipología de TFG de experimentación científica representa la oportunidad de aplicar los conocimientos adquiridos durante el grado a la resolución de un problema científico en el que interviene una metodología informática o un sistema computacional, y por tanto de mostrar y ampliar el desarrollo de las habilidades científicas del futuro graduado en Ingeniería Informática.

Aunque pudiera parecer que en este tipo de TFG sólo se deben aplicar recomendaciones o prácticas propias de la actividad científica, no se debe olvidar que este TFG es indispensable para la obtención de un título de graduado en ingeniería. Por tanto, además de aplicar procedimientos científicos, se deben seguir y emplear principios de ingeniería, y sobre todo los propios de la Ingeniería Informática estudiados durante la carrera.
%Aunque pudiera parecer que en este tipo de TFG solo se deben aplicar recomendaciones o prácticas propias de la actividad científica, no se debe olvidar que este TFG es indispensable para la obtención de un título de graduado en ingeniería. Por tanto, además de aplicar procedimientos científicos, se deben seguir y emplear principios de ingeniería, y sobre todo los propios de la Ingeniería Informática estudiados durante la carrera.

Las recomendaciones sobre el tipo de TFG de Experimentación Científica se expondrán de la siguiente manera: primero se presentará el \textit{método científico}, es decir, elemento diferenciador de este tipo de TFG y que debe guiar su elaboración (véase la sección \ref{tfx_inv_s_met_cientifico}); posteriormente se expondrá cómo estructurar el trabajo y la memoria del TFG buscando siempre una correspondencia con el método científico (Sección  \ref{tfx_inv_s_est_trabajo_memoria}) y por último se enunciarán una serie de recomendaciones (Sección \ref{tfx_inves_ss_recomendaiones}).
%Las recomendaciones sobre el tipo de  TFG de experimentación científica se expondrán de la siguiente manera: primero se presentará el \textit{método científico}, es decir, elemento diferenciador de este tipo de TFG y que debe guiar su elaboración (v. sección \ref{tfx_inv_s_met_cientifico}); posteriormente se expondrá cómo estructurar el trabajo y la memoria del TFG buscando siempre una correspondencia con el método científico (v. sección \ref{}) y por último se enunciarán una serie de recomendaciones (v. sección \ref{}).}

\subsection{El Método Científico}\label{tfx_inv_s_met_cientifico}

La investigación es un actividad intelectual y experimental realizada de modo sistemático con el propósito de aumentar los conocimientos sobre una determinada materia\footnote{Real Academia Española - \url{http://rae.es}}. En otras palabras, la investigación busca adquirir nuevos conocimientos y/o resolver problemas teóricos o prácticos mediante una actividad metódica y \underline {reproducible}. La consecución de los objetivos de la investigación precisa de la aplicación de una determinada estrategia de trabajo, que en este caso se conoce como \textbf{método científico}. Este está constituido por una serie de actividades, que aunque se resumen en la Figura \ref{fg_met_cientifico}, se detallan a continuación:

\begin{figure}[!t]
    \centering
    \usetikzlibrary {shapes.geometric}
    % Define block styles
    \tikzstyle{decision} = [diamond, draw, fill=black!10, text width=4.5em, text badly centered, node distance=3cm, inner sep=0pt]
    \tikzstyle{block} = [rectangle, draw, fill=blue!10, minimum width=7em, text centered, rounded corners, minimum height=4em]
    \tikzstyle{arrow}=[draw, -latex]
    %\tikzstyle{cloud} = [draw, ellipse,fill=red!20, node distance=3cm, minimum height=2em]
    
    \begin{tikzpicture}[node distance = 2cm, auto]
        % Place nodes
        \node [block, align=center] (A) {Hacer una pregunta};
        \node [block, below of=A, align=center] (B) {Estudiar el estado\\del arte};
        \node [block, below of=B, align=center] (C) {Construir una hipótesis};
        \node [block, below of=C, align=center] (D) {Evaluar la hipótesis};
        \node [block, below of=D, align=center] (E) {Analizar los resultados};
        \node [block, right of=C, align=center, xshift=6em] (F) {Replantear\\hipótesis};
        \node [decision, below of=E, align=center, yshift=1.8em] (G) {¿Hipótesis cierta?};
        \node [block, below of=G, align=center, yshift=-1em] (J) {Concluir};
        % Draw edges
        \path [arrow] (A) -> (B);
        \path [arrow] (B) -> (C);
        \path [arrow] (C) -> (D);
        \path [arrow] (D) -> (E);
        \path [arrow] (F) -> (C);
        \path [arrow] (E) -> (G);
        \path [arrow] (G) -> node [text width=2.5cm, midway, right] {Sí} (J);
        \path [arrow] (G) -| node [text width=2.5cm, midway, above] {No} (F);
    \end{tikzpicture}

    %\includegraphics[width=.8\textwidth]{images/Método_científico_2021.jpg}
    \caption{Diagrama que representa el método científico.}\label{fg_met_cientifico}% \textcolor{red}{DEBE GENERARSE CON TIKZ. ESTO ES UN BORRADOR. Fuente: \url{https://es.wikipedia.org/wiki/M\%C3\%A9todo_cient\%C3\%ADfico\#/media/Archivo:M\%C3\%A9todo_cient\%C3\%ADfico_2021.jpg}}
\end{figure}

\begin{itemize}
    \item \textbf{Hacer una pregunta}. Todo trabajo científico comienza con la identificación de un problema, de una necesidad, de una carencia, en definitiva, de un ¿a qué se debe esto? o ¿cómo puedo resolver esto?, o lo que anglosajones denominan \textit{gap}. Si no existe esta pregunta, problema o necesidad, deja de tener todo sentido el invertir tiempo y esfuerzo, y por ende dinero, en la realización de una experimentación cuyo fin no está asociado a ofrecer una respuesta útil.
    %\textbf{Hacer una pregunta}. Todo trabajo científico comienza con la identificación de un problema, de una necesidad, de una carencia, en definitiva, de un ¿a qué se debe esto? o ¿cómo puedo resolver esto?, o lo que anglosajones denominan \textit{gap}. Si no existe esta pregunta, problema o necesidad, deja de tener todo sentido el invertir tiempo y esfuerzo, y por ende dinero, en la realización de una experimentación cuyo fin no está asociado a ofrecer una respuesta útil.

    \item \textbf{Estudiar el estado del arte}. La identificación de una necesidad o el surgimiento de una pregunta científica, no implica que ésta sea novedosa o que no tenga ya una respuesta. Por consiguiente, las acciones que intervienen en esta fase son: \begin{enumerate*}[label=(\arabic*)] \item identificar la disciplina científica donde se encuadra la pregunta a responder; \item analizar la novedad de la pregunta identificada, así como si ha sido ya respondida; y \item en caso de que el reto sí sea novedoso, estudiar la investigación relacionada con el ánimo de aprender los métodos ya usados, y emplearlos como inspiración en el diseño de la metodología o método original que de respuesta al desafío en estudio.\end{enumerate*}
    %\textbf{Estudiar el estado del arte}. La identificación de una necesidad o el alumbramiento de una pregunta científica, no implica que esta sea novedosa o que no tenga ya una respuesta. Por consiguiente, las acciones que intervienen en esta fase son: \begin{enumerate*}[label=(\arabic*)] \item identificar la disciplina científica donde se encuadra la pregunta a responder; \item analizar la novedad de la pregunta identificada, así como si ha sido ya respondida; y \item en caso de que el reto sí sea novedoso, estudiar la investigación relacionada con el ánimo de aprender los métodos ya usados, y emplearlos como inspiración en el diseño de la metodología o método original que de respuesta al desafío en estudio.\end{enumerate*}

    \item \textbf{Construir una hipótesis}. Una vez estudiado el estado del arte relacionado con el problema, se debe definir una hipótesis, sobre la cual se diseñará la solución y se evaluará si se confirma, y por tanto resolverá el reto, o si se desecha, y en consecuencia tener que definir otra hipótesis.
    %\textbf{Construir una hipótesis}. Una vez estudiado el estado del arte relacionado con el problema, se debe definir una hipótesis, sobre la cual se diseñará la solución y se evaluará si se confirma, y por tanto resolverá el reto, o si se desecha, y en consecuencia tener que definir otra hipótesis.

    \item \textbf{Evaluar la hipótesis}. Esta fase se compone de las siguientes acciones: \begin{enumerate*}[label=(\arabic*)]\item diseñar y desarrollar una metodología o método acorde a la hipótesis; \item diseñar un conjunto de experimentos que permitan evaluar la metodología o método desarrollado; y \item evaluar con métricas de evaluación estándares y propias del área donde se circunscribe el problema de investigación con el fin de ofrecer una evaluación objetiva y comparable. Así mismo, se recomienda ofrecer todos los detalles de desarrollo y de evaluación, con el ánimo de que la experimentación sea reproducible por cualquier investigador. Este recomendación contribuye a confiar en la experimentación realizada, en el método o metodología propuesta, a la transferencia de conocimiento y al progreso científico.\end{enumerate*}
    %\textbf{Evaluar la hipótesis}. Esta fase se compone de las siguientes acciones: \begin{enumerate*}[label=(\arabic*)]\item diseñar y desarrollar una metodología o método acorde a la hipótesis; \item diseñar un conjunto de experimentos que permitan evaluar la metodología o método desarrollado; y \item evaluar con métricas de evaluación estándares y propias del área donde se circunscribe el problema de investigación con el fin de ofrecer una evaluación objetiva y comparable. Así mismo, se recomienda ofrecer todos los detalles de desarrollo y de evaluación, con el ánimo de que la experimentación sea reproducible por cualquier investigador. Este recomendación contribuye a confiar en la experimentación realizada, en el método o metodología propuesta, a la transferencia de conocimiento y al progreso científico.\end{enumerate*}

    \item \textbf{Analizar los resultados}. Esta fase es la más determinante, porque en función de su resultado, se podrá decir que se acepta la hipótesis, y por tanto se concluye que el método o metodología desarrollados resuelven el desafío objeto de estudio, o por el contrario se rechaza la hipótesis. Este rechazo implica que se debe volver a la definición de la hipótesis, o dicho de otro modo, a modificar la hipótesis y desarrollar otro método o metodología acorde a la nueva hipótesis.
    %\textbf{Analizar los resultados}. Esta fase es la más determinante, porque en función de su resultado, se podrá decir qu se acepta la hipótesis, y por tanto se concluye que el método o metodología desarrollados resuelven el desafío objeto de estudio, o por el contrario se rechaza la hipótesis. Este rechazo implica que se debe volver a la definición de la hipótesis,  o dicho de otro modo, a modificar la hipótesis y desarrollar otro método o metodología acorde a la nueva hipótesis.
\end{itemize}

Al igual que el método científico guía la labor de los investigadores de cualquier área, también debe ser la base de un TFG de Experimentación Científica.

% ------------------------------------------------------------

%Definición de investigación.
%La investigación es un actividad intelectual y experimental realizada de modo sistemático con el propósito de aumentar los conocimientos sobre una determinada materia\footnote{Real Academia Española - \url{http://rae.es}}. En otras palabras, la investigación busca adquirir nuevos conocimientos y/o resolver problemas teóricos o prácticos mediante una actividad metódica y \underline {reproducible}. Para poder llevar a cabo una investigación exitosa es necesario tener en cuenta los siguientes elementos: 
%\begin{itemize}
%    \item Recopilar toda la información relevante sobre el problema a tratar utilizando fuentes heterogéneas y fiables.
%    \item Indagar sobre otros estudios ya realizados y publicados en la literatura científica sobre la misma problemática, analizando sus beneficios y sus limitaciones.
%    \item Seguir una metodología científica para poder así desarrollar el trabajo de manera organizada y coherente.
%    \item Mostrar los resultados obtenidos y valorarlos de forma objetiva sin omisiones.
%    \item Asegurar la reproducibilidad del trabajo, permitiendo que el tribunal o cualquier otra persona interesada sea capaz de verificarlo y replicarlo.
%\end{itemize}

%Conocimiento previo sobre el estado de la cuestión.
%Estructura del marco teórico y conceptual.
%De esta lista de elementos haremos especial hincapié en proveer de un marco teórico y conceptual que permita al tribunal, y a cualquier otro potencial lector, de la información básica necesaria para comprender la investigación realizada. De todas las tipologías de TFG de esta sección, la tipología de TFG de investigación es, probablemente, la que lleve más páginas dedicadas al estado del arte. Por tanto, también la bibliografía utilizada será más extensa. Debe de quedar claro que comprendes la temática y conoces qué cosas se han hecho ya y qué falta por hacer.

%Importancia de partir de una pregunta de partida.
%Otro punto importante a destacar en la memoria de los TFGs de investigación es la motivación del trabajo. No hay que confundir el apartado de motivación con una motivación personal. La motivación del trabajo, en este tipo de tipología de TFG, se refiere a identificar una brecha en el conocimiento actual del tema a tratar y de cómo la investigación en ese punto puede ayudar a contribuir en al área de estudio. Por tanto la motivación del trabajo debe incluirse en la memoria luego del estudio del estado del arte, ya que es muy posible que debas hacer referencia a algún punto ya tratado.

%Una vez detallado el problema de estudio y la motivación del TFG, es momento de hablar de los objetivos del trabajo. Para poder definir correctamente estos objetivos es necesario primero definir las hipótesis de partida, ya que la verificación o no de estas hipótesis serán justamente los objetivos principales del TFG. En el proceso de formulación de hipótesis debes plantear posibles respuestas a las preguntas surgidas durante tu análisis del estado del arte. Recuerda que las hipótesis deben poder ser verificables y falsables, es decir, susceptibles de ser refutadas.
   
En el siguiente apartado veremos en más detalle como llevar a cabo un TFG de Experimentación Científica siguiendo una serie de pasos que nos aseguren que los resultados obtenidos sean de fiables y de calidad.

\subsection{Estructura del Trabajo y de la Memoria}\label{tfx_inv_s_est_trabajo_memoria}

La estructura de la memoria no debe por qué distar mucho del resto de tipologías, pero sí debe recoger de forma adecuada el desarrollo científico realizado. Esto obliga a que el trabajo científico siga unas pautas coherentes que dirijan, al menos, la labor experimental de una forma ordenada y dirigida a la evaluación con éxito de la hipótesis. El método científico, presentado en la sección \ref{tfx_inv_s_met_cientifico}, es una guía muy recomendable para orientar todo la actividad experimental y la redacción de la memoria. Por ende, la memoria debe reflejar: \begin{enumerate*}[label=(\arabic*)]\item la existencia de una necesidad de investigación, \item que esta no haya sido aún resuelta, o por lo menos no completamente, \item la definición de una hipótesis, \item el análisis, diseño e implementación del método o metodología que lleve a cabo la hipótesis y \item la evaluación de esta.\end{enumerate*} A continuación, se presentará una correspondencia entre el método científico y la estructura de la memoria, que a su vez ordena todo el trabajo relacionado con esta tipología de TFG.
%La estructura de la memoria no debe por qué distar mucho del resto de tipologías, pero sí debe recoger de forma adecuada el desarrollo científico realizado. Esto obliga a que el trabajo científico siga unas pautas coherentes que dirijan, al menos, la labor experimental de una forma ordenada y dirigida a la evaluación con éxito de la hipótesis. El método científico, presentado en la sección \ref{tfx_inv_s_met_cientifico}, es una guía muy recomendable para orientar todo la actividad experimental y la redacción de la memoria. Por ende, la memoria debe reflejar: \begin{enumerate*}[label=(\arabic*)]\item la existencia de una necesidad de investigación, \item que esta no haya sido aún resuelta, o por lo menos no completamente, \item la definición de una hipótesis, \item el análisis, diseño e implementación del método o metodología que lleve a cabo la hipótesis y \item la evaluación de esta.\end{enumerate*} A continuación, se presentará una correspondencia entre el método científico y la estructura de la memoria, que a su vez ordena todo el trabajo relacionado con esta tipología de TFG.

\paragraph{Motivación - Pregunta de investigación} El primer capítulo o el capítulo de introducción del TFG debe recoger la motivación del trabajo, es decir, la razón por la cual se realiza y merece la pena esforzarse en él durante los meses que involucre su realización. Esto que, en un inicio parece complicado, consiste en encontrar un reto que tenga asociado una pregunta científica, como podría ser \textit{¿es posible la generación de texto coherente y precisa para ofrecer información sobre trastornos de la conducta alimenticia?} Si se ha llegado a esa pregunta, es que existe una necesidad ligada a una problema, que a su vez tiene un contexto. Esta conjunción de contexto, problema, necesidad y pregunta científica es lo que constituye la motivación del TFG, y la cual lo convierte en atractivo para cualquier lector, y sobre todo para la comunidad científica.
%\paragraph{Motivación - Pregunta de investigación} El primer capítulo o capítulo de introducción del TFG debe recoger la motivación del trabajo, es decir, la razón por la cual se realiza, merece la pena esforzarse en él durante los meses que involucre su realización, es interesante para cualquier lector, y representa un avance científico. Esto que parece complicado consiste en encontrar un reto que tenga asociado una pregunta científica, como podría ser \textit{¿es posible la generación de texto coherente y precisa para ofrecer información sobre trastornos de la conducta alimenticia?} Si se ha llegado a esa pregunta, es que existe una necesidad ligada a una problema, que a su vez tiene un contexto. Esta conjunción de contexto, problema, necesidad y pregunta científica es lo que constituye la motivación del TFG, y la cual lo convierte en atractivo para cualquier lector, y sobre todo para la comunidad científica.

\paragraph{Investigación relacionada - Estudiar el estado del arte} Una vez que ya se ha fijado el problema o pregunta a resolver, se deben realizar las siguientes acciones:

\begin{enumerate}
    \item \textbf{Determinar el área de investigación}. La Ingeniería Informática abarca distintas áreas científicas, por lo que lo primero es saber si la pregunta se responde desde, por ejemplo, la informática teórica, la informática gráfica, la inteligencia artificial, la electrónica, o incluso si precisa de conocimiento externo a la Ingeniería Informática, como podría ser una investigación en un dominio interdisciplinar (medicina, biología, química, etc.). Atendiendo a la pregunta anterior, si se está cuestionando sobre la posibilidad de generar automáticamente lenguaje, parece que se trataría de un problema de inteligencia artificial, dado que se pide imitar un comportamiento característico de la inteligencia humana. En este ejemplo, una vez fijada que el área es la de la inteligencia artificial, el siguiente paso sería identificar la disciplina especializada en el desafío en cuestión, que en este ejemplo sería el procesamiento del lenguaje natural, ya que es la disciplina de inteligencia artificial encargada del desarrollo de métodos y metodologías computacionales orientados a la comprensión y generación de lenguaje por parte de un ordenador.
    %\textbf{Determinar el área de investigación}. La Ingeniería Informática abarca distintas áreas científicas, por lo que lo primero es saber si la pregunta se responde desde por ejemplo la informática teórica, la informática gráfica, la inteligencia artificial, la electrónica, o incluso si precisa de conocimiento externo a la Ingeniería Informática, como podría ser una investigación en el dominio de la biología. Atendiendo a le pregunta anterior, si se está cuestionando sobre la posibilidad de generar automáticamente lenguaje, parece que se trataría de un problema de inteligencia artificial, dado que se pide imitar un comportamiento característico de la inteligencia humana como es la generación de lenguaje. Una vez fijada que el área es la de la inteligencia artificial, el siguiente paso sería identificar la disciplina especializada en el desafío en cuestión, que en este caso sería el procesamiento del lenguaje natural, ya que es la disciplina de inteligencia artificial encargada del desarrollo de métodos y metodologías computacionales orientados a la comprensión y generación de lenguaje por parte de un ordenador.

    \item \textbf{Búsqueda de los trabajos más relevantes}. Conocida el área y disciplina de investigación, es momento de encontrar los trabajos relacionados con la pregunta de investigación. En el caso particular que estamos tratando, serían modelos de generación de lenguaje. Con este dato, ya es momento de estudiar todo lo relacionado con la problemática, y seleccionar los avances más recientes. En este caso se corresponderían con los grandes modelos de lenguaje.
    %\textbf{Búsqueda de los trabajos más relevantes}. Conocida el área y disciplina de investigación, es momento de encontrar los trabajos relacionados con la pregunta de investigación. En el caso particular que estamos tratando, serían modelos de generación de lenguaje. Con este dato, ya es momento de estudiar todo lo relacionado con la problemática, y seleccionar los avances más recientes. En este caso se corresponderían con los grandes modelos de lenguaje.

    \item \textbf{Evaluación de si el reto está resuelto}. Tras la selección de los artículos más importantes es momento de evaluar si el desafío planteado está o no resuelto. En caso de que así sea, se deberá revaluar la pregunta de investigación, hasta encontrar aquella que aún no lo esté. Por contra, si no lo está, la pregunta representa un verdadero reto sobre el que merece la pena realizar el TFG.
    %\textbf{Evaluación de si el reto está resuelto}. Tras la selección de los artículos más importantes es momento de evaluar si el desafío planteado está resuelto. En caso de que así sea, se deberá revaluar la pregunta de investigación, hasta encontrar aquella que aún no lo esté. Por contra, si no lo está, la pregunta representa un verdadero reto sobre el que merece la pena trabajar.
    
    Pudiera parecer que el trabajo de esta fase termina aquí, pero no es así, porque ahora es el momento de determinar las estrategias seguidas hasta la fecha, con el fin de realizar una propuesta novedosa que pueda compararse con lo existente, y que ofrezca mejores resultados.
    %Pudiera parecer que el trabajo de esta fase termina aquí, pero no es así, porque ahora es momento de determinar las estrategias seguidas hasta la fecha con el fin de realizar una propuesta novedosa que pueda compararse con lo existente, y que ofrezca unos mejores resultados.
\end{enumerate}

El resultado de este estudio deberá reflejarse en un capítulo específico sobre el estado de la investigación asociada al proyecto. Algunos ejemplos de nombres que puede tener este capítulo son: contexto, estado de arte, estado de la cuestión, trabajos relacionados o el nombre específico de la disciplina de investigación asociada al desafío científico del TFG.
%El resultado de este estudio deberá reflejarse en un capítulo específico sobre el estado de la investigación asociada al proyecto. Algunos ejemplos de nombres que puede tener este capítulo son: contexto, estado de arte, estado de la cuestión, trabajos relacionados o el nombre específico de la disciplina de investigación asociada al desafío científico del TFG.

\paragraph{Hipótesis} Un conocimiento profundo del problema científico asociado prepara para la enunciación de la hipótesis del TFG. Esta es una suposición de estrategia, metodología o modelo a desarrollar para resolver el desafío científico en el que se está trabajando. Así mismo, si la evaluación de la hipótesis la confirma, conlleva haber propuesto una solución válida al reto científico del TFG. Automáticamente se podría pensar que un resultado negativo implica el fracaso del trabajo científico desarrollado. Esto no tiene por qué ser así, dado que de los resultados negativos también se pueden obtener conclusiones positivas, que incluso pueden servir de base para posteriores avances científicos.
%\paragraph{Hipótesis} Un conocimiento profundo del problema científico asociado prepara para la enunciación de la hipótesis del TFG. Esta es una suposición de estrategia, metodología o modelo a desarrollar para resolver el desafío científico en el que se está trabajando. Así mismo, si la evaluación de la hipótesis la confirma, conlleva haber propuesto una solución válida al reto científico del TFG. Automáticamente se podría pensar que un resultado negativo implica el fracaso del trabajo científico desarrollado. Esto no tiene por qué ser así, dado que de los resultados negativos también se pueden obtener conclusiones muy positivas, que incluso pueden servir de base para posteriores avances científicos.

La hipótesis se recomienda que también se indique en el capítulo de introducción, ya que va asociada a la pregunta de investigación. Así mismo, una vez formulada la hipótesis, ya se puede establecer los objetivos e hitos del TFG, los cuales son una forma de concretar el trabajo a realizar.
%La hipótesis se recomienda que también se indique en el capítulo de introducción, porque va asociada a la pregunta de investigación. Así mismo, una vez formulada la hipótesis, ya se puede establecer los objetivos e hitos del TFG, los cuales son una forma de concretizar el trabajo a realizar.

\paragraph{Desarrollo - Evaluación de la hipótesis} Para poder probar la validez de la hipótesis, antes hay que desarrollar la metodología o método subyacente a la misma. Por tanto, se debe comenzar primeramente con el diseño del marco experimental. En la mayoría de los casos el marco experimental va a seguir los cánones de la investigación cuantitativa, la cual requiere de una evaluación empírica. Esto obliga a seleccionar el conjunto o conjuntos de datos sobre los que realizar la evaluación, y a determinar las medidas de evaluación a utilizar
%\paragraph{Desarrollo - Evaluación de la hipótesis} Para poder probar la validez de la hipótesis, antes hay que desarrollar la metodología o método subyacente a la misma. Por tanto, se debe comenzar primeramente con el diseño del marco experimental. En la mayoría de los casos, el marco experimental va a seguir los cánones de la investigación cuantitativa, la cual requiere de una evaluación empírica. Esto obliga a seleccionar el conjunto o conjuntos de datos sobre los que realizar la evaluación, y a determinar las medidas de evaluación a usar.

La elección del conjunto de datos es una decisión muy importante del proceso de evaluación, porque si estos no son de calidad, o no son representativos, la calidad y credibilidad del marco experimental se verá resentido. En cualquier caso, siempre hay que describir los datos elegidos, y en caso de que no sean propios hacer referencia a su fuente. A continuación te mostramos algunos ejemplos de conjuntos de datos que se pueden usar:
%La elección del conjunto de datos es una decisión muy importante del proceso de evaluación, porque si estos no son de calidad o no son representativos, la calidad y credibilidad del marco experimental se verá resentido. En cualquier caso, siempre hay que describir los datos elegidos, y en caso de que no sean propios hacer referencia a su fuente. A continuación algunos ejemplos de conjuntos de datos que se pueden usar:

\begin{itemize}
    \item {\bf Conjuntos de datos externos}: Si trabajas con conjuntos de datos públicos y ampliamente conocidos por la comunidad científica es sencillo hacer referencia a ellos en la memoria. Se recomienda utilizar una tabla para mostrar comparativamente los distintos conjuntos de datos empleados, como en el ejemplo de la Tabla \ref{tab:uci}. Puedes incorporar a la tabla las columnas que sean necesarios y que incluyan toda la información sobre la cual haces referencia en el texto principal de la memoria. Presta atención a el texto que acompaña a la tabla en donde se indica tanto el origen de los datos como una cita o referencia on-line a ellos. 

\begin{table}[!ht]
\centering
\begin{tabular}{cccccc}
\hline
 & \textbf{No. de} & \textbf{No. de} & \multicolumn{3}{c}{\textbf{No. de Atributos}} \\
{\it \textbf{Dataset}} & \textbf{Casos} & \textbf{Clases} & \textbf{Nominal} & \textbf{Numérico} & \textbf{Perdidos}\\ \hline
Anneal & 898 & 6 & 32 & 6 & 29 \\
Breast Cancer & 286 & 2 & 9 & 0 & 2 \\
Diabetes & 768 & 2 & 8 & 0 & 8 \\
Heart-C & 303 & 5 & 7 & 6 & 2 \\
Hepatitis & 155 & 2 & 13 & 6 & 15 \\
House Votes & 435 & 2 & 16 & 0 & 16 \\
Iris & 150 & 3 & 0 & 4 & 0 \\
Lymphography & 148 & 4 & 15 & 3 & 0 \\
Vowel & 990 & 11 & 3 & 10 & 0 \\
Wine & 178 & 3 & 0 & 13 & 0 \\
Zoo & 101 & 7 & 16 & 1 & 0 \\
\hline
\end{tabular}
\caption{Conjuntos de datos utilizados en este trabajo provenientes del repositorio UCI (\url{https://archive.ics.uci.edu/}).\label{tab:uci}}
\end{table}

    \item {\bf Conjuntos de datos propios}: si dentro de tu propuesta de trabajo incluyes la generación de conjuntos de datos artificiales propios, entonces necesitas crear un apartado en la memoria para detallar todo el proceso. Si por otro lado, el conjunto de datos te lo ha dado tu tutor, entonces necesitas detallar su origen y características. Si es posible incluye el o los conjuntos de datos al entregar la memoria. Si por alguna razón los datos son privados o no tienes permiso para difundirlos debes entonces explicarlo detalladamente. Recuerda que para trabajar con datos con información de carácter personal, antes debes anonimizarlos. 
    \item {\bf Conjuntos de datos artificiales}: si cuentas con conjuntos de datos creados artificialmente, también llamados datos sintéticos, debes explicar cómo se han generado. Si el procedimiento de generación de estos datos es propuesta tuya, entonces descríbelo con detalle como se explica en el apartado anterior. En caso de ser generados por terceras partes entonces alcanza con referenciar ese trabajo y explicar brevemente cómo se realiza este proceso. Igualmente para ambos casos necesitas mostrar en una tabla una descripción de los conjuntos de datos, como puedes ver en la Tabla \ref{tab:fake}. Recuerda que tu trabajo debe de poder ser reproducible por cualquier persona, por tanto debes incluir en el material a entregar los conjuntos de datos que hayas generado tú o un enlace a aquellos generados por terceras personas.

\begin{table}[!ht]
\centering
\label{tab:results}
\begin{tabular}{c|ccc}
\hline
{\it \textbf{Dataset}} & \textbf{Edificio} & \textbf{Árbol} & \textbf{Persona} \\
\hline
Sintético A & 5 & 10 & 4 \\
Sintético B & 0 & 12 & 11 \\
Sintético C & 1 & 2 & 2 \\
Sintético D & 3 & 2 & 3 \\
Sintético E & 4 & 5 & 1 \\
\hline
\end{tabular}
\caption{Elementos incluidos en los distintos conjuntos de datos generados artificialmente.\label{tab:fake}}
\end{table}
\end{itemize}

El siguiente paso es \textbf{diseñar e implementar la metodología o modelo}. Se podría decir que esta parte del TFG es para la que está más preparado el estudiante, por ser la de mayor contenido técnico. El diseño y desarrollo deben seguir los principios de ingeniería aprendidos durante la carrera, siendo recomendable que el alumno tome decisiones en el ámbito de elección de lenguaje de programación, de bibliotecas a usar y de entorno de programación a utilizar.
%El siguiente paso es \textbf{diseñar e implementar la metodología o modelo}. Se podría decir que esta parte del TFG es para la que está más preparado el alumno, por ser la de mayor contenido técnico. El diseño y desarrollo deben seguir los principios de ingeniería aprendidos durante la carrera, siendo recomendable que el alumno tome decisiones en el ámbito de elección de lenguaje de programación, de librerías a usar y de entorno de programación a utilizar.

Antes de proseguir, nos gustaría hacerte la recomendación de que este tipo de TFG estuvieran acompañados de un demostrador, es decir, de integrar la propuesta científica en un sistema informático que muestre su utilidad. Si se opta por ello, el desarrollo del demostrador debe seguir las recomendaciones de la ingeniería del software, lo que implica: \begin{enumerate*}[label=(\arabic*)]\item elección de metodología de trabajo, \item realización de fase de análisis, \item diseñar la solución, \item implementar el diseño y \item validar y evaluar el software desarrollado.\end{enumerate*} Es muy recomendable la realización de un demostrador, porque permite que el alumno muestre los conocimientos adquiridos durante la carrera, sobre todo los relativos a los procesos y principios de la ingeniería del software.
%Ante de proseguir, quisiera hacerse la recomendación de que este tipo de TFG estuvieran acompañados de un demostrador, es decir, de integrar la propuesta científica en un sistema informático que muestre su utilidad. Si se opta por ello, el desarrollo del demostrador debe seguir las recomendaciones de la ingeniería del software, lo que implica: \begin{enumerate*}[label=(\arabic*)]\item elección de metodología de trabajo, \item realización de fase de análisis, \item diseñar la solución, \item implementar el diseño y \item validar y evaluar el software desarrollado.\end{enumerate*} Es muy recomendable la realización de un demostrador, porque permite que el alumno muestre los conocimientos adquiridos durante la carrera, sobre todo los relativos a los procesos y principios de la ingeniería del software.

Una vez concluida la implementación de la metodología o modelo pasamos a la fase de evaluación. Ésta debe realizarse con las medidas de evaluación usadas en los trabajos de investigación relacionados. Esto es relevante porque permite comparar la propuesta del TFG con la literatura, y así poder saber si la propuesta representa realmente un avance o no. Además, se recomienda facilitar la comparación con el estado del arte por medio de una tabla comparativa con los modelos más relevantes. No olvidar que para poder decir, por ejemplo, que un algoritmo es mejor que otro, es necesario no solo realizar múltiples experimentos, sino también someter los resultados obtenidos a análisis estadísticos. Recuerda que los experimentos que realizas deben de poder reproducirse, por tanto no olvides fijar una semilla para que, en caso de usar un generador de números aleatorios, tu trabajo sea reproducible.
%Una vez concluida la implementación de la metodología o modelo se debe evaluar. La evaluación debe realizarse con las medidas de evaluación usadas en los trabajos de investigación relacionados. Esto es relevante porque permite comparar la propuesta del TFG con la literatura, y así poder saber si la propuesta representa realmente un avance o no. Además, se recomienda facilitar la comparación con el estado del arte por medio de una tabla comparativa con los modelos más relevantes.

En cuanto a la organización de la evaluación de la hipótesis en varias secciones o capítulos, dependerá de su extensión y de si se realiza un sistema demostrador. En el caso de que se haga un sistema informático, se recomienda separar en capítulos las distintas fases del proceso de desarrollo de software. En caso contrario, puede que solo sea necesario elaborar un capítulo describiendo todo lo relativo a la elección de datos, implementación y evaluación.
%En cuanto a la organización de la evaluación de la hipótesis en capítulos, todo va a depender de su extensión y de si se realiza un sistema demostrador. En el caso de que se haga un sistema informático, se recomienda separar en capítulos las distintas fases del proceso de desarrollo de software. En caso contrario, puede que solo sea necesario elaborar un capítulo describiendo todo lo relativo a la elección de datos, implementación y evaluación.

\paragraph{Análisis de los resultados.} El análisis de los resultados y la comparación con el estado del arte es el momento decisivo de esta tipología de TFG, debido que determina si la hipótesis se acepta o no. En el caso de que se acepte, el trabajo no concluye aquí, ya que se recomienda realizar un análisis tanto de los aciertos, como de los errores que se hayan producido, dado que estos permitirán realmente entender cómo el modelo o metodología desarrollados están dando respuesta al reto de investigación.
%\paragraph{Análisis de los resultados} El análisis de los resultados y la comparación con el estado del arte es el momento decisivo de esta tipología de TFG, debido que determina si la hipótesis se acepta o no. En el caso de que se acepte, el trabajo no concluye aquí, ya que se recomienda realizar un análisis tanto de los aciertos, como de los errores que se hayan producido, dado que estos permitirán realmente entender cómo el modelo o metodología desarrollados están dando respuesta al reto de investigación.

En caso de que la hipótesis sea rechazada, se debe evaluar si un análisis profundo de los resultados puede ser interesante para la comunidad investigadora, y por tanto constituir una contribución importante para el TFG. Este análisis puede concluir que se deben elegir otros datos, que el reto ha llegado a un escenario de resultados tope, que la estrategia algorítmica no es la adecuada y que se deben seguir otras, o que la elección de las características y los parámetros de configuración del algoritmo no han sido los óptimos. Esto que pudiera parecer una fracaso, puede ser la base de futuros progresos porque ofrece a la comunidad científica un conjunto de lecciones aprendidas a tener en cuenta para posteriores desarrollos. Ahora bien, se remarca que cuando se ha obtenido un resultado negativo, el análisis de resultados debe ser profundo y con conclusiones interesantes para considerarlo como una contribución y como un TFG válido.
%En caso de que la hipótesis sea rechazada, se debe evaluar si un análisis profundo de los resultados puede ser interesante para la comunidad investigadora, y por tanto constituir una contribución importante para el TFG. Este análisis puede concluir que se deben elegir otros datos, que el reto ha llegado a un escenario de resultados tope, que la estrategia algorítmica no es la adecuada y que se deben seguir otras o que la elección de las características y los parámetros de configuración del algoritmo no han sido los óptimos. Esto que pudiera parecer una fracaso, puede ser la base de futuros progresos porque ofrece a la comunidad científica un conjunto de lecciones a tener en cuenta para posteriores desarrollos. Ahora bien, se remarca que cuando se ha obtenido un resultado negativo, el análisis de resultados debe ser profuso y con conclusiones interesantes para considerarlo como una contribución y como TFG válido.

También es posible que los experimentos realizados tengan ciertas limitaciones que afecten a la interpretación de los resultados obtenidos. Debes, por tanto, abordar este tema de manera transparente. Las limitaciones son características del diseño o metodología que influyen en la validez, aplicabilidad o generalización de los hallazgos. Pueden derivarse del diseño inicial del estudio, del método utilizado o de desafíos imprevistos durante el proyecto. Existen distintos tipos de limitaciones, como las relacionadas con el diseño o la metodología del estudio, o bien de los datos en sí, que pueden presentar problemas de calidad, disponibilidad o fiabilidad. Reconocer estas limitaciones te da la oportunidad para proponer formas de abordar estas limitaciones y demuestra un pensamiento crítico.

\paragraph{Conclusiones} El último capítulo debe ser el de conclusiones, que no debe ser una mera recapitulación del TFG, sino un lugar donde destacar si la propuesta permite aceptar la hipótesis, y los principales resultados del análisis a modo de lecciones aprendidas. En resumen, unas conclusiones que aporten valor al TFG. Adicionalmente puede ser útil contar con una lista de trabajos futuros en donde el estudiante pueda demostrar que ha comprendido el problema sobre el que ha trabajado y proponga líneas de trabajo futuras a partir de sus conclusiones.
%\paragraph{Conclusiones} El último capítulo debe ser el de conclusiones, que no debe ser una mera recapitulación del TFG, sino un lugar donde destacar si la propuesta permite aceptar la hipótesis, y los principales resultados del análisis a modo de lecciones aprendidas. En resumen, unas conclusiones que aporten valor al TFG.


%Muchos de los algoritmos que se utilizan para minería de datos tienen una componente aleatoria. Esto quiere decir que dependiendo de la semilla del generador de números aleatorios utilizado puede darnos una respuesta diferente. Para evitar tener este sesgo es necesario ejecutar varias veces el mismo algoritmo utilizando distintas semillas. Normalmente se suele ejecutar unas 5 o 10 veces y luego tomar para cada métrica a medir su media y desviación estándar. De esta manera puedes deducir si el algoritmo es robusto, si tiene mucha variabilidad, etc. Por tanto, a la hora de reportar los resultados debes incluir tanto la media como la desviación estándar en la tabla de resultados, como puedes ver en la Tabla \ref{tab:sd}.

%\begin{table}[htbp]
%\centering
%\label{tab:cityscapes}
%\begin{tabular}{c|c|c|c}
%\hline
%\textbf{Método} & \textbf{Iris} & \textbf{Diabetes} & \textbf{Ionosphere} \\ 
%\hline
%K-means & 214 (289) & 2628 (175) & 805 (202) \\
%LKM & 169 (10) & 1691 (47) & 908 (34) \\ 
%DE-LKM &  1938 (151) & 7425 (214) & 9449 (55) \\
%LC-LKM & 3764 (2431) & 12797 (44) & 17653 (2298) \\
%EM & 255 (139) & 3033 (226) & 1688 (1301) \\
%\hline
%\end{tabular}
%\caption{Tiempo de ejecución en ms. (media y desviación estándar entre paréntesis) de las 10 ejecuciones de cada %algoritmo por conjunto de datos.\label{tab:sd}}
%\end{table}

%También es necesario realizar un proceso de entrenamiento o {\it training} y uno de prueba o {\it test} posterior. Para ello se puede dividir el conjunto de datos en una parte para training y otra para test. Sin embargo, la opción más recomendada es utilizar algún tipo de validación cruzada que divide el conjunto de datos original en varias partes. Luego se utilizan todas menos una para entrenar y la restante para validar. Esto se realiza tantas veces como divisiones del conjunto de datos original hemos hecho. Finalmente se reporta la media y la desviación estándar de las métricas evaluadas. En caso de utilizar particiones ya realizadas por terceros indícalo e incluye una referencia en la memoria y/o los ficheros mismos al entregarla.

\subsection{Recomendaciones}\label{tfx_inves_ss_recomendaiones}
%Sugerencias para futuros investigadores.
%Áreas de mejora o posibles extensiones del estudio.

A modo de buenas prácticas para la realización de este tipo de TFG, se proponen a continuación una seria de consejos que esperamos que sean de utilidad:
%A modo de buenas prácticas para la realización de este tipo de TFG, se proponen una seria de consejos que se estiman que allanarán su elaboración.

\begin{itemize}
    \item \textbf{Sistema demostrador}. Como ya se ha indicado, se recomienda realizar un sistema informático que integre la propuesta científica, con el fin de que el alumno aplique los principios de ingeniería de software adquiridos durante la carrera. 
    
    \item \textbf{Manual de instalación y uso}. En caso de no realizar un sistema demostrador, es esencial proveer de toda la información necesaria para asegurar la reproducibilidad de los resultados. Eso incluye un manual de instalación y uso, el código fuente utilizado, etc.

    \item \textbf{Experimentación y redacción en paralelo}. Es práctica común de los estudiantes el realizar primero la experimentación y posteriormente la redacción de la memoria. Esto suele conllevar agobio y hastío, dado que suele apetecer más el programar un sistema informático que redactar una memoria. Por tanto, para facilitar el proceso de redacción, y también para que este sea más rico, se recomienda realizar la tarea de experimentación y redacción en paralelo.

    \item \textbf{Registrar todo paso de la experimentación}. Es importante registrar todo paso del proceso científico que se realice, dado que facilitará luego su desarrollo en la memoria. Esto incluye el uso de sistemas de control de versiones, llevar un \textit{log} donde se detallen los distintos experimentos realizados, etc.

    \item \textbf{Continua interacción con la persona que te tutoriza}. Teniendo en cuenta que la metodología de investigación te puede resultar novedosa, es muy importante que tengas reuniones con tu tutor o tutora frecuentes con objeto de que te vaya guiando paso a paso por ella y en las que tendrás que aprovechar para consultar todas las dudas que te vayan saliendo.

    \item \textbf{Trabajos futuros}. La investigación es un proceso continuo donde cada investigador aporta su pequeño grano de arena. A medida que uno se sumerge dentro del área de estudio, van surgiendo nuevas ideas e hipótesis que por la extensión y cantidad de horas para realizar el TFG no pueden ser acometidas. Es importante por tanto dejarlas por escrito, en el apartado de conclusiones o uno propio, de forma que cualquier lector pueda retomar el trabajo realizado y extenderlo desde el punto en donde ha acabado el TFG.

    \item \textbf{Reproducibilidad}. Aunque ya se ha comentado en esta sección el tema de la reproducibilidad, es importante hacer hincapié en ella. Es muy importante que describas tu investigación con tal detalle que cualquier otra persona pueda reproducir de manera fehaciente todo el proceso que has realizado y obtener los mismos resultados. De esta forma podrá comparar sus propios resultados con los tuyos. Por desgracia, aunque es verdad que cada vez menos, te encontrarás con situaciones en las que te interesaría evaluar un modelo que ya ha sido publicado pero no tienes suficientes detalles para implementarlo tú o no aportan información de valores de parámetros con los que han obtenido los resultados que se han publicado. Estas situaciones son bastante frustrantes pues, aunque el modelo sea interesante, similar al tuyo, utilice el mismo enfoque, o cualquier otra causa, no podrás emplearlo para comparar resultados. Por tanto, describe tus modelos y entornos experimentales con todo detalles. Incluso pon a disposición de la comunidad científica el código y los conjuntos de datos (siempre que puedas).

    \item \textbf{Un mal resultado también es un resultado}. La investigación puede llegar a ser a veces un proceso ingrato pues no se suelen obtener normalmente los resultados magníficos que uno espera. Muchas veces repetimos el ciclo de la Figura \ref{fg_met_cientifico} varias veces y los resultados obtenidos no son los que deseamos. ¿Esto implica que debemos continuar \textit{sine die} con nuestro trabajo hasta conseguir buenos resultados? ¿Y si no los conseguimos significa que debemos abandonar el problema entre manos y cambiar de temática de TFG? La respuesta en ambas cuestiones es la misma y muy contundente: no. Si no conseguimos en un plazo razonable resultados que consideremos que realizan una contribución científica, tenemos que tener en cuenta que científicamente mostrar el camino por donde no se lleva a nada positivo es también una aportación, que será útil para otros investigadores (``por aquí, aunque parecía prometedor no podemos ir porque no conduce a nada''). Y para ti, pues desde el punto de vista de tu TFG estarás mostrando que eres capaz de realizar una investigación científica completa y eso en sí es suficiente para tu TFG. Ya tendrás tiempo en tu TFM o en tu tesis doctoral de continuar y encontrar alternativas que mejoren el estado del arte del problema entre manos.

    \item \textbf{Publica los resultados}. Una vez que tengas hecho todo el trabajo científico, ¿por qué no publicar los resultados en un congreso o en una revista y hacer partícipe a la comunidad científica nacional o internacional de la investigación que has realizado y de las conclusiones obtenidas? De esta forma publicitarás tu trabajo y otros investigadores podrán apoyarse en él para seguir avanzando. Además, que tu TFG esté avalado por una publicación científica siempre será una magnífica carta de presentación para la evaluación de tu TFG, sin contar con el hecho de que estarás empezando ha hacerte un currículum muy útil para una posterior fase profesional si te gusta el mundo de la investigación.
\end{itemize}

)}
    \begin{itemize}
                \item \textbf{Preguntas de investigación}: identificación de problemas o carencias que se van a abordar en el proyecto 
                \item \textbf{Hipótesis}: enunciados a demostrar o probar
                \item \textbf{Evaluación de las hipótesis}: Elección de un método de investigación, determinación de experimentaciones a realizar, elección de instrumentos de recogida de datos y de herramientas análisis de datos y de evaluación.
                \item \textbf{Resultados}: Variables y valores. Tablas de análisis, matrices de correlación, ...
                \item \textbf{Discusión}: Interpretación de resultados, referenciando estado del arte e hipótesis u objetivos.
    \end{itemize}
    
    \item \textbf{Proyectos de desarrollo según la metodología seguida. (Ver \section{TFG de Desarrollo de Software}
\label{apendice:desarrollo}
% María José

Existe una modalidad de TFG que es la de desarrollo de software, cuyo objetido será la creación de una aplicación o biblioteca en el contexto de un problema determinado. El objetivo de esta sección será explicar brevemente cómo llevar a cabo la elaboración de la memoria si tu TFG es de este tipo.

Lo primero que debes hacer es decidir qué metodología de desarrollo o ciclo de vida seguirás. Tienes varias opciones que pasamos a describir brevemente, pues seguro las has visto con más detalle en una o varias asignaturas de la carrera: 
\begin{itemize}
    \item \textbf{Ciclo de vida clásico o en cascada}: los requisitos se establecen al inicio y no cambian, se realizan secuencialmente las fases de diseño, implementación y pruebas, y los resultados se ven cuando el proyecto está ya avanzado. 
     \item \textbf{Metodología ágil}: se establecen iteraciones en las que se van se van agregando nuevas funcionalidades a la aplicación final. En cada iteración se realizan tareas de especificación, diseño, implementación y pruebas. El usuario final puede ir viendo y probando los resultados al final de cada iteración e intervenir para mejorar. Ejemplo: SCRUM.
     \item \textbf{Espiral o de Riesgos}: consiste en cuatro etapas: planificación, análisis de riesgo, desarrollo de prototipo y evaluación del cliente. Se puede volver de cada etapa a las anteriores en caso de que haya que modificar algo y antes de seguir con las siguientes etapas. Conviene para proyectos en los que se han previsto inicialmente riesgos económicos o técnicos, como la seguridad, que hay que gestionar si se quiere que el proyecto llegue a buen término.
     \item \textbf{Dirigida por pruebas}: primero se diseñan las pruebas, pensando en los requisitos que puede ser más difícil de abordar, principalmente los no funcionales, luego se completan los requisitos y se escribe un código. A continuación se realizan las pruebas planificadas y después se refactoriza el código para mejorarlo. Todo este proceso se hace de forma iterativa, abordando en cada iteración diferentes pruebas y requisitos. Ejemplo: Test Driven Development (TDD).

\end{itemize}

Existen otras metodología más específicas, según el tipo de desarrollo que vas a realizar, como la DevOps, en la que los  equipos de desarrollo  de software y los equipos de operaciones de la empresa (como el de marketing, contabilidad o gestión de almacén, por ejemplo) trabajen juntos, facilitando la comunicación e integrando mejor las tareas de ambos. También hay metodologías específicas para el diseño de videojuegos, que incluyen fases para el diseño de \textit{storyboards}, y diferencian la creación de personajes, escenas, narrativa, etc. Igual ocurre con metodologías que impliquen el diseño o uso de hardware (como tarjetas, placas, dispositivos del tipo Blackberry Pi, o sensores), se suelen incluir algunas fases de diseño, construcción y prueba del hardware. 


Sea cual sea la metodología que escojas, debes justificar su elección en la memoria, bien en el capítulo del estado del arte,  revisando y comparando varias, o bien en el capítulo de tu propuesta, antes de empezar a dar detalles del desarrollo. En la carrera has visto varias asignaturas relacionadas con ingeniería del software y has aprendido varias metodologías y herramientas. Es el momento de aplicarlas en un proyecto completo. Te aconsejamos que las valores, y desde el principio escojas una, y llegues a un acuerdo sobre ella con la persona que te tutoriza. La planificación temporal del proyecto debe tener en cuenta esta metodología en la parte del desarrollo para asignar tiempos a todas sus fases o iteraciones. 

En el capítulo de tu propuesta te aconsejamos que incluyas una sección para cada una de las fases del ciclo de vida que sigas. En el caso de metodologías ágiles, incluye una primera sección con el \textit{Product Backlog} (la lista de historias de usuario priorizadas y organizadas en iteraciones) y luego una sección por cada iteración, explicando en cada una las tareas de especificación, diseño, implementación y pruebas que llevas a cabo.

Es muy importante que en la memoria utilices herramientas de ingeniería del software como las que te han enseñado en la carrera, que ayudan a visualizar diversos aspectos del desarrollo, del tipo diagramas de casos de uso, plantillas de casos de uso o de historias de usuario, diagramas de arquitectura del sistema, de clases, de secuencia, de actividad, de la estructura de la base de datos, etc. También puedes presentar esquemas o listas de cómo has estructurado el software: paquetes, tipos de ficheros, localización en la arquitectura del software final, etc. Si tu software tiene una interfaz que has diseñado, incluye los bocetos que has hecho, como fotografías de los dibujos en papel, o capturas de pantalla de herramientas de diseño de interfaces). 

Evita incluir código en la memoria cuando abordes la implementación, a no ser que un objetivo de tu proyecto sea la propuesta de un algoritmo específico o la realización de cambios en algoritmos existentes para su mejora. En algunos casos, puede ser interesante incluir pseudocódigo. En cualquier caso tu código debería estar en un repositorio \textit{online} (como Github o Gitlab) enlazado desde el proyecto.

Debes destacar qué tipo de pruebas realizas sobre el software: unidad, integración, rendimiento, usabilidad, etc. Da detalles sobre las pruebas que planificas, y sobre qué métodos y herramientas utilizas para realizarlas, y también sobre sus resultados. Explica las mejoras realizadas si los resultados no son los esperados. 

En algunos proyectos también se hacen pruebas finales para validar el desarrollo realizado con expertos o usuarios finales. En ese caso, incluye una sección dando detalles de estas pruebas: objetivos, participantes, procedimiento, instrumentos de evaluación, resultados y valoración final.

El profesor JJ Merelo ha escrito una serie de artículos muy interesantes sobre cómo aplicar buenas prácticas de desarrollo ágil en tu TFG \cite{TFGs2024JJ}. Échales un vistazo porque pueden serte de mucha utilidad. 

Para finalizar, si durante el desarrollo has encontrado problemas que has tenido que solucionar, dale visibilidad a ese trabajo describiendo las alternativas de solución que has explorado y explicando la solución escogida para que otra persona que lea tu memoria pueda beneficiarse de ella. Si ha quedado algún problema por resolver, indica el motivo. Puede ser interesante que al final del capítulo de propuesta incluyas una sección sobre esto para mostrar tus capacidades de resolución de problemas y toma de decisiones como ingeniero o ingeniera. 

Y ten en cuenta que tu TFG puede ser una gran excusa para aprender cosas nuevas. Si en el grado has estudiado una metodología muy bien y la has practicado pero no has visto otra que consideres interesante y quizá útil profesionalmente y te interesaría aprender, no te quedes en tu zona de confort y aprovecha la oportunidad para aprenderla y ponerla en práctica. Tu TFG y todo lo que hayas aprendido durante el proceso de desarrollo del mismo será una magnífica carta de presentación para tu próxima etapa profesional.
)}
        \begin{itemize}
            \item \textbf{Metodologías ágiles}
                \begin{itemize}
                    \item \textbf{Product Backlog}: listado de historias de usuario priorizadas y agrupadas en iteraciones o sprints
                    \item \textbf{Iteraciones}: Una sección por cada iteración o sprint describiendo las tareas y pruebas de cada iteración, e incluyendo gestión de riesgos si procede
                \end{itemize}
            \item \textbf{Ciclo de vida clásico}
                \begin{itemize}
                    \item Requisitos
                    \item Diseño
                    \item Implementación
                    \item Pruebas
                \end{itemize}
            \item \textbf{Otros ciclos de vida}
                \begin{itemize}
                    \item Secciones según las fases del ciclo
                \end{itemize}
        \end{itemize}
        
     \item \textbf{Proyectos de revisión de estado del arte. (Ver \section{TFG de revisión del estado del arte}
\label{appendix:revisionestado}

Existe una modalidad de TFG que es en sí una revisión del estado del arte. Para que nos entendamos, es como la sección de revisión del estado del arte vista en el capítulo \ref{cap:RevisionEstadoDelArte}, pero a lo grande, en la que la revisión ocupa toda la memoria del TFG.

Si vas a realizar este tipo de TFG, en primer lugar te recomendamos que leas ese capítulo con objeto de obtener una idea general de qué es una revisión del estado del arte y, por supuesto, el capítulo \ref{cap:bibliografia} con el objetivo de tener claro cómo gestionar la bibliografía del TFG, pues, si ya es importante este asunto, en este tipo de proyecto fin de carrera, las referencias bibliográficas se configuran como algo vital.

El objetivo de esta sección será explicar brevemente cómo llevar a cabo la elaboración de la memoria de un trabajo de revisión del estado del arte. En este caso, hay mucho material publicado que puede serte de utilidad, por lo que haremos una revisión poco somera del mismo con objeto de ofrecerte algún material de inicio para que, al menos, comiences la tarea. De cualquier forma, ponte de acuerdo con la persona que te tutoriza sobre la forma de enfocar el trabajo, pues es un tipo muy especial de TFG que necesita ser definido y desarrollado de forma muy precisa.

\subsection{Tipos de revisiones}

Lluis Codina en \cite{codina2024lluis} establece dos grandes grupos de tipos de revisiones: las \textit{tradicionales} o \textit{narrativas} y las de tipo sistemático. Las primeras son más bien ensayos y carecen de validez científica; las segundas, sí que tienen esta validez, y se clasifican en \textit{sistemáticas}, que se emplean para determinar el impacto de intervenciones, y \textit{de alcance}, usadas para describir hasta dónde (alcance) llega un conocimiento dado. En este trabajo, el autor explica claramente cuándo se debería elegir un tipo u otro de revisión.

Pero esta es una simplificación que el profesor Codina ha realizado porque en realidad existe una gran cantidad de tipos diferentes de revisiones tal y como Grant y Booth indican en \cite{grant2009maria}, la mayoría ampliamente usados en el campo de las ciencias de la salud y con un componente estadístico muy fuerte:

\begin{itemize}
    \item Revisiones tradicionales. 
    \item Síntesis de conocimiento: de forma genérica se podrían definir como aquellas revisiones que contextualizan e integran los hallazgos de investigación de estudios individuales dentro del cuerpo más amplio de conocimiento sobre el tema que tengas entre manos. Algunos de los más conocidos y usados pueden ser los siguientes:
    \begin{itemize}
        \item Revisiones sistemáticas: identifican, evalúan y sintetizan todas las pruebas empíricas que cumplen unos criterios de elegibilidad previamente especificados. Las revisiones sistemáticas deben ser lo más exhaustivas e imparciales posibles.
        
        \item Metanálisis: subconjunto de revisiones sistemáticas que combina estadísticamente los resultados de estudios cuantitativos encontrados, con objeto de ofrecer un efecto más preciso de los resultados.
        
        \item Revisiones de alcance: abordan una pregunta de investigación exploratoria destinada a extraer conceptos clave, tipos de evidencia y nichos en la investigación relacionada con un área o campo definido, mediante la búsqueda sistemática, selección y síntesis del conocimiento existente.
        
        \item Revisiones rápidas: un tipo de síntesis de conocimiento en el cual los procesos de revisión sistemática se aceleran y los métodos se simplifican para completar la revisión más rápidamente que en el caso de las revisiones sistemáticas típicas, que vienen a tener un tiempo de realización de un año, reduciendo el tiempo de confección de cinco a doce semanas. Se emplean cuando no se dispone de mucho tiempo para realizarlas.
        
        \item Revisiones realistas: comprenden y desentrañan los mecanismos por los que una intervención funciona (o no funciona), proporcionando así una explicación, en lugar de un juicio sobre cómo funciona.

        \item Revisiones cualitativas: aquellas que integran o comparan los hallazgos de estudios cualitativos.

        \item Revisiones mixtas: combinación de los hallazgos de estudios cualitativos y cuantitativos dentro de una sola revisión sistemática para abordar las mismas preguntas de revisión superpuestas o complementarias.
        
        \item Síntesis narrativas: se basan en el uso de palabras y texto para resumir y explicar los hallazgos de la síntesis más que en resultados estadísticos. Básicamente usan la palabra para ``contar la historia'' de los hallazgos en los estudios incluidos.

        \item Revisiones tipo paraguas: se refiere a una revisión que recopila evidencia de múltiples revisiones en un documento accesible y utilizable. 
        
    \end{itemize}
\end{itemize}

En la web de la {biblioteca de la Universidad de Melbourne}\footnote{\url{https://unimelb.libguides.com/whichreview}} dispones de una definición muy detallada de los diferentes tipos de revisiones de literatura así como bibliografía de cada una de ellas para que las consultes. Si estás pensando realizar tu TFG en este contexto, te recomendamos que conozcas previamente los diferentes tipos de revisión existentes y que, junto a quien te dirige el trabajo, decidáis cuál es la que mejor se ajusta a los objetivos del mismo.

\subsection{Revisiones sistemáticas}

Una de las más ampliamente usadas es la revisión sistemática, sobre todo en el ámbito de la salud, aunque su uso se ha generalizado a todos los campos del conocimiento. Tal y como indica Codina en \cite{codina2018lluis} habría que diferenciar entre las revisiones sistemáticas y las sistematizadas, ya que estas primeras están centradas en conocer la eficacia de una intervención basándose en el análisis de estudios científicos que se han realizado sobre ella, como hemos dicho en el campo de la salud, y las segundas están enfocadas en explorar campos de conocimiento e investigación específicos, identificando  tendencias y corrientes dominantes, y detectando vacíos y posibles oportunidades para futuras investigaciones. Esta segunda definición sí que podría ser aplicada a cualquier campo de conocimiento y, por tanto, si somos estrictos con el lenguaje, en el campo de la informática deberíamos realizar una revisión sistematizada. Pero... por abuso del lenguaje se habla de forma general, independientemente del campo, de revisión sistemática. 

De cualquier forma una revisión (sistemática o sistematizada) de este tipo está compuesta de varias fases muy bien pautadas, que pasamos a describir seguidamente, cuyas descripciones detalladas podrás encontrar en  \cite{booth2021a}, un clásico en el campo de las revisiones de literatura:

\begin{enumerate}
\item Formulación de la pregunta(s) de investigación: define claramente el objetivo de la revisión y las preguntas de investigación que se abordarán. Es importante que estas preguntas sean específicas, claras y relevantes para el tema de estudio.

\item Búsqueda de literatura: se realiza una búsqueda exhaustiva y sistemática de la literatura relevante utilizando bases de datos académicas, bibliotecas digitales y otros recursos. En nuestro campo de la informática también puede interesarte realizar búsquedas en repositorios de código abierto, por ejemplo.

\item Selección de estudios (en inglés, \textit{screening}): se aplican criterios de inclusión y exclusión para seleccionar los estudios que cumplen con los criterios de la revisión. Esta selección se suele realizar en varias etapas, comenzando con la revisión de títulos y resúmenes para realizar un primer filtrado, seguida de la revisión de los textos completos de los estudios potencialmente relevantes que han pasado esta primera criba. Por ejemplo, se pueden descartar los estudios que lleguen a conclusiones con pocos datos o que usen métodos o tecnologías no actuales o no apropiados. 

\item Extracción de datos: se recopilan los datos relevantes de cada estudio seleccionado, como características del estudio, métodos utilizados y resultados obtenidos. En nuestro caso, información específica sobre tecnologías, detalles técnicos, prestaciones de aplicaciones, algoritmos empleados, lenguajes, etc.

\item Evaluación de la calidad de los estudios: se realiza una evaluación crítica de la calidad metodológica de los estudios incluidos en la revisión. 

\item Análisis y síntesis de los datos: se analizan los datos extraídos de los estudios y se realiza una síntesis para identificar patrones, tendencias, inconsistencias o nichos.

\item Interpretación de los resultados: se interpretan los hallazgos de la revisión en el contexto de la pregunta de investigación y se discuten sus implicaciones.

\item Escritura de la revisión: se redacta un informe detallado que describe el proceso de revisión, los métodos utilizados, los resultados obtenidos y las conclusiones alcanzadas. La memoria de tu TFG podría seguir una estructura estándar en este tipo de trabajos, que podría ser algo así:

\begin{enumerate}
\item Introducción: contextualización del tema y justificación de su importancia. Establecimiento de los objetivos.
\item Metodología: descripción de los métodos usados (estrategia de búsqueda, bases de datos, criterios de inclusión y exclusión, procedimientos de selección de estudios, evaluación de la calidad de los mismos, técnicas de análisis de datos, si corresponde).
\item Resultados: exposición de los resultados conseguidos. Tablas y gráficas ayudarán a visibilizarlos.
\item Discusión: interpretación de los resultados teniendo en cuenta los objetivos de la revisión, implicaciones prácticas o teóricas, y también es importante que se establezcan las limitaciones del proceso de revisión. Finalmente, la exposición de los resultados finales y recomendaciones. 
\item Conclusiones: resumen de los principales hallazgos, conclusiones finales y recomendaciones en base a los resultados obtenidos. También es habitual meter aquí una serie de líneas de trabajo futuras. 
\end{enumerate}
\end{enumerate}

Ten en cuenta que este proceso no es estático en el sentido de que en cualquier momento vas a poder obtener nuevos trabajos y tendrás que decidir si son relevantes para tu estudio e incorporarlos al mismo en caso afirmativo, con los cambios que conllevará en el análisis que has realizado hasta el momento. Con la persona que te tutoriza tendréis que decidir cuándo parar de incorporar más estudios para revisar.

Ni que decir tiene que todas las referencias deben estar correctamente citadas y dispuestas en una sección final de bibliografía.

El aporte realmente relevante de tu revisión vendrá de la mano del contenido de la sección de discusión, pues es ahí donde vas a mostrar tu capacidad de análisis y descubrimiento de hallazgos a partir de los trabajos analizados. La calidad de la interpretación de estos y las conclusiones harán o no valioso tu trabajo de revisión. Es por esto que te recomendamos que te esfuerces especialmente en esta parte de tu TFG. 

Para finalizar, indicarte que en las referencias \cite{carrera2022angela,kofod2022anders,silva2016rodrigo} tienes algunos ejemplos en los que los autores realizan una adaptación de las revisiones sistemáticas al campo de la informática. Échales un vistazo porque pueden serte de mucha utilidad.

)
                \begin{itemize}
                    \item En estos proyectos se prescindiría del capítulo de propuesta, siendo el capítulo de revisión de estado del arte de mayor envergadura. Se debe describir la metodología seguida para la revisión (Dónde buscar, qué palabras clave utilizar, cómo filtrar la búsqueda, qué características revisar/comparar, ...)
                \end{itemize}
\end{itemize}

 Las metodologías de proyectos no son exclusivas entre sí, por ejemplo, un proyecto de tipo desarrollo podría incluir una parte de validación que implique aplicar la metodología de investigación. Igualmente, un proyecto de investigación puede necesitar del desarrollo de software. Por ello, se pueden añadir secciones de un tipo en otro tipo.



\include{5.introduccion_tfg}
\chapter{La revisión del estado del arte} \label{cap:RevisionEstadoDelArte}

% [Autores: Pablo, Juanma]

\section{Introducción}
¿Qué es una revisión el estado del arte? Básicamente un análisis del conocimiento actual sobre un tema de tal forma que te permita identificar qué hay hecho sobre el mismo (qué se ha investigado o desarrollado -- visión general del conocimiento) y qué lagunas existen. Es el paso previo para que tú puedas realizar una propuesta que mejore lo que hay ya hecho y, sobre todo, que sitúes tú trabajo en un contexto. La idea es que busques recursos bibliográficos (fundamentalmente literatura relevante en forma de artículos científicos o técnicos) y los estudies, comprendiendo el problema y las soluciones propuestas para resolverlo, entendiendo los métodos aplicados y que, como consecuencia de ese estudio, seas capaz de tener una imagen general y clara del tema y de su evolución e identifiques situaciones o elementos de mejora. En lo que se refiere al TFG no es resumir los recursos que has encontrado, sino analizarlos, sintetizarlos, clasificarlos, contextualizarlos y evaluarlos de forma crítica para obtener una imagen clara del estado del conocimiento.

Por tanto, ¿para qué te sirve realizar una revisión del estado del arte en tu TFG? Es una de las primeras tareas que tienes que llevar a cabo una vez asignado el TFG ya que te permitirá inicialmente familiarizarte con el problema y con todo lo que se ha hecho en el campo de tu TFG. Y también de forma práctica, como se aborda en una fase inicial, para evitar que hagas algo que ya está hecho. También sirve para que identifiques limitaciones, lagunas o brechas en el conocimiento o problemas no resueltos donde tu TFG puede aportar. El que la memoria de tu TFG conste de una buena sección del estado del arte es importante ya que está demostrando que conoces lo que hay hecho en el contexto de tu TFG y también muestra tu capacidad de síntesis y análisis. Además, te va a permitir desarrollar un marco teórico y metodológico para tu proyecto y posicionarte con respecto a las metodologías, teorías o desarrollos existentes, con objeto de diferenciar tu trabajo. 

Veamos un par de ejemplos. En el primero, tienes entre mano un TFG en el que deseas aplicar inteligencia artificial a la gestión de invernaderos. Lo primero que tienes que hacer, por tanto, es preguntarte qué se ha hecho en este campo, qué problemas se han abordado, qué técnicas se han aplicado, qué resultados se han obtenido. Y esto sólo puedes hacerlo mediante una búsqueda bibliográfica, leyendo todos los recursos que obtengas y analizándolos. De esta forma puedes darte cuenta que el campo en general está muy trillado, pues es algo donde se ha trabajado mucho, se han producido muchas aportaciones y con buenos resultados, y quizá que tu aportación sería insignificante o poco relevante. Pero con este análisis, además de llegar a conocer bien el tema y de ser capaz de describir cuáles han sido sus avances, te has dado cuenta que en el área del riego inteligente por goteo los métodos que se ha aplicado no funcionan del todo bien y no se adaptan correctamente a las circunstancias específicas de los invernaderos (fundamentalmente por los diferentes tipos de suelo, de plantaciones y condiciones ambientales, por ejemplo). En este caso acabas de detectar una posible línea de trabajo y, por tanto, un lugar por donde poder orientar tu TFG y realizar una aportación metodológica o práctica: la aplicación de algoritmos genéticos a la gestión eficiente del riego (por decir algo). 

Imagina una segunda situación en la que vas a desarrollar una aplicación móvil para ayudar a las personas mayores en caso de necesidad. ¿Qué otras aplicaciones hay en el mercado que estén en este ámbito? En este caso, los recursos bibliográficos no sólo serán artículos científicos donde se presentan esas aplicaciones y se evalúan, sino también  sitios web de aplicaciones de este estilo, en la que cuentan sus prestaciones y funcionalidades. En este caso debes hacer una búsqueda exhaustiva y obtener todas las aplicaciones, analizar su funcionamiento en el caso de que te las puedas descargar y probar, o estudiar las prestaciones publicadas y clasificarlas por funcionalidades. Así conocerás qué hace la ``competencia'' y qué puntos fuertes y débiles tiene cada una. Con esta información establecerás las prestaciones que tendrá que tener tu aplicación (las habituales que todas tienen, por ejemplo, el botón de petición de ayuda) y podrás contribuir con aquellas novedosas que no has visto en ninguna (la conexión automática con familiares en caso de necesidad o la monitorización de constantes vitales y el envío de un médico de forma automática cuando se detecta un problema en estas, por decir algo) y, por tanto, que aporten un valor añadido a la tuya.

El objetivo de este capítulo es dar unas consideraciones generales que te permitan realizar un estudio del estado del arte en tu TFG y plasmarlo correctamente en la memoria. Para tal fin, este capítulo tiene como objetivo ayudarte a conocer cómo realizar y organizar tu revisión del estado del arte en la memoria del TFG. Así, en primer lugar tendrás que buscar información, realizar un análisis y plasmarlo en la memoria. En las secciones siguientes te indicamos cómo llevar a cabo este proceso.

\section{Búsqueda de información}

Una revisión del estado del arte tiene una primera fase que es la búsqueda de los recursos que posteriormente pasarás a analizar. Veamos algunos elementos importantes en esta etapa.

\subsection{Identificar las preguntas}

 %añadido por María José en forma más o menos telegráfica (obtenido del curso de la UNED de redacción de TFGs):  Para hacer una revisión del estado del arte, antes tenemos que delimitar el tema del TFG y en base a ello hacer una revisión bibiográfica con la que  podemos ver qué hay publicado o realizado sobre el mismo tema o temas cercanos, qué aspectos se han analizado en esas publicaciones, qué discusiones o polémicas han suscitado, etc. 

A estas alturas de la película tienes una idea más o menos clara sobre de qué va a tratar tu proyecto. La idea ahora es ver qué hace tu proyecto especial y no ser la enésima aplicación CRUD. Pero también puedes aprender de lo que ya existe. ¿Te suena la expresión ``a hombros de gigantes''? Pues vamos a subirnos a los hombros de los gigantes y hacernos una serie de preguntas:

\begin{itemize}
    \item \textit{¿Alguien ha hecho antes lo mismo que yo?}
    
Posiblemente tu idea no sea 100\% original. Y no pasa nada. De hecho, es muy raro que lo sea: para empezar cada año se leen cientos de TFGs solo en tu escuela y cada vez es más difícil realizar algo totalmente novedoso. Pero es importante saber qué han hecho los demás para no caer en los mismos errores, sino también encontrar sus fortalezas.

\item \textit{¿Qué han hecho los demás que puede serme útil?}

Mira las funcionalidades que ofrecen. ¿Qué te parece interesante? ¿Qué te parece que no aporta nada? ¿Qué te parece que está bien pero se podría mejorar?

\item \textit{¿Cómo lo han hecho?} 

¿Qué tecnologías / metodologías han usado? ¿Por qué esas y no otras? ¿Qué les han ofrecido a sus creadores? Podemos incluso enfocarnos un poco en estas preguntas en caso de un TFG de desarrollo: ¿Es mejor una aplicación web o una aplicación de escritorio? ¿Es mejor usar C o Python? ¿Es mejor un algoritmo evolutivo o una red neuronal para este problema?

\item \textit{¿Qué ofrece mi proyecto que no ofrece el resto?} 

Esta es quizás la pregunta más importante a resolver y que hará que tu proyecto brille sobre el resto. Quizás tu proyecto es el primero que es Software Libre, y con ello ayudarás a la comunidad. Quizás tu proyecto sea el primero para Android. O quizás tu proyecto sea el primero que utiliza un algoritmo de explicabilidad. Por pequeña que sea tu propuesta, ya habrás hecho algo nuevo y mejorado el estado del arte.

\end{itemize}

Para resolver estas preguntas obviamente necesitas ver y entender lo que ha hecho el resto. Y para ello necesitas información de calidad.

\subsection{Creación de la consulta}

Antes de comenzar a buscar recursos debes definir el alcance y los criterios de la revisión. 

En primer lugar debes establecer una pregunta que resuma la intención que tienes con la revisión. El título de tu TFG te dará una pista bastante importante para tal fin. 

El siguiente paso será determinar las palabras que vas a emplear para realizar la consulta, o términos de búsqueda. Puedes usar las de la pregunta anterior o incluso las palabras que aparecen en el título de tu TFG o las palabras clave del mismo. Asegúrate que usas los términos adecuados y que te ayudarán a encontrar material relevante para poder responder a la pregunta de la revisión. Parte de ahí y refina la consulta, añadiendo o quitando términos, hasta que creas que cubre el ámbito que quieres alcanzar. Este proceso lo deberás hacer previamente a la búsqueda en sí e iterativamente después de consultar, analizando los resultados, porque a la luz de los recursos que obtengas puede ser que cambies los términos empleados o añadas algunos más que centren mejor la consulta sobre lo que quieres buscar.

\subsection{Selección de las fuentes y recursos}

Las fuentes concretas dependerán de cada proyecto pero de forma genérica sí podemos concluir que las principales fuentes bibliográficas son bases de datos académicas. Estos repositorios nos permitirán acceder a una gran cantidad de material de forma sencilla y eficiente. 

Por supuesto la web de la biblioteca de tu universidad (por ejemplo, \url{http://biblioteca.ugr.es}, para la de la UGR) es una fuente magnífica en la que apoyarte para hacer la búsqueda y encontrar recursos relevantes. Suelen tener convenios con editoriales que nos permiten acceder a recursos a los cuales de otra forma no podríamos acceder sin tener que pagar). Utilízala, selecciona recursos interesantes y no tengas miedo a irte a la biblioteca de tu centro y consultar físicamente el libro, por ejemplo. Las bibliotecas están llenas de recursos inesperados. No todo lo encontrarás en la Web. Estamos acostumbrados a hacerlo todo digitalmente y, muchas veces, hacerlo de forma presencial nos puede aportar sensaciones diferentes. No hay nada como coger un libro en una sala de la biblioteca y consultarlo con tranquilidad. 

Algunas fuentes interesantes que puedes emplear para comenzar son las siguientes:

\begin{itemize}
    \item \href{www.scopus.com}{Scopus}
    \item \href{https://scholar.google.es}{Google Scholar}
    \item \href{https://www.microsoft.com/en-us/research/project/academic}{Microsoft Academic}
\end{itemize}

En estas bases de datos podrás consultar mediante un formulario, obtener una lista de resultados, normalmente ordenados por relevancia a tu consulta y descargar los recursos, sus metadatos y referencias bibliográficas.

También podrás emplear otras bases de datos bibliográficas específicas según la temática de tu TFG. Por ejemplo, en informática podrás usar la \href{https://dl.acm.org}{biblioteca digital de la ACM}, de la \href{https://ieeexplore.ieee.org}{IEEE}, entre otros, o en temas biomédicos, el conocido \href{https://pubmed.ncbi.nlm.nih.gov/}{PubMed}, por ejemplo. Editoriales como \href{https://link.springer.com}{Springer} o \href{https://www.sciencedirect.com}{Elsevier}, por citar dos de las más importantes, tienen también sus propias bibliotecas digitales para la búsqueda de artículos en sus revistas y congresos.

Otra fuente de información pueden ser las bases de datos de patentes, como la \href{https://worldwide.espacenet.com}{europea} y la \href{https://ppubs.uspto.gov/pubwebapp/static/pages/landing.html}{estadounidense}.

Bases de datos como \href{https://www.educacion.gob.es/teseo}{Teseo}, de tesis doctorales, también pueden ser una fuente interesante para que la consideres según el tipo de TFG que estés haciendo.

Pero no sólo debes tirar de los recursos encontrados en las bases de datos bibliográficas, sino que también debes de tener en cuenta las referencias bibliográficas citadas en un recurso, ya que son una fuente de información muy preciada para obtener información relevante del tema. Por tanto, analiza las referencias de los recursos para ampliar la búsqueda.

Y ahora llega la pregunta del millón: ¿uso la Wikipedia como fuente? Y la respuesta es clara y contundente: no. ¿Por qué? La primera razón es que es está abierta a que cualquier persona sea una autora de sus artículos, sea experta o no de un tema, y por tanto los artículos pueden incorporar información incorrecta o poco precisa. La segunda es que puede ofrecer información no actual sobre una temática. Si quieres usarla, puedes emplearla como punto de partida para tener una idea general sobre tu temática o como punto de partida siguiendo las referencias de los artículos, pero nunca para citarla en tu revisión del estado del arte.

En relación a los recursos que obtienes de estas bases de datos, los principales a considerar son publicaciones académicas: artículos en congresos, revistas, libros, capítulos de libros, memorias de tesis doctorales, de trabajos fin de máster o de grado, informes de proyectos, etc. 

Los recursos deben ser creíbles y confiables. ¿Y cómo detectarlo? Puedes ver si el lugar de publicación (editorial, revista, congreso, sitio web) son reputados en el campo y centrarte seguidamente en estudiar a los autores del artículo y determinar si son expertos en el tema y con una trayectoria avalada. También en si se plasma en ella información objetiva, contrastada y demostrada, o simplemente opiniones personales. Descarta entonces aquellas que sean subjetivas y de personas poco conocidas o con poca experiencia en el tema entre manos.  Un recurso tiende a ser fiable si ha sido publicado en las actas de congresos o en revistas nacionales o internacionales de prestigio. Se supone que el propio proceso de evaluación por pares de las publicaciones es una garantía de que el artículo cumple unos estándares mínimos de calidad. Aún así tendrás que filtrar recursos porque consideres que no son de calidad (por ejemplo, porque no aportan detalles para reproducir los experimentos, o el análisis de resultados es poco somero, las conclusiones muy vagas, etc.). Ten cuidado con las revistas o congresos que publican todo lo que les cae en las manos y no tienen procesos de calidad que aseguren que lo publicado tiene unos mínimos asegurados de rigor.

También se pueden considerar publicaciones en la web. En ese caso deberás tener en cuenta si el autor está cualificado para hablar sobre el tema y si está publicado bajo un dominio de una institución reconocida (universidad, institución, organización, etc.). También tenemos los casos de artículos publicados en sitios como \href{www.medium.com}{Medium} o \href{www.researchgate.net}{ResearchGate}, entre otros muchos. De nuevo debes comprobar la reputación del autor o si se apoya en referencias bibliográficas, por ejemplo. Lo que está claro que no debes citar ninguna respuesta de sitios como StackOverflow o Reddit porque son opiniones y respuestas a preguntas, que aunque pueden ser correctas muchas, sólo deben servir para ayudarte a solucionar algún inconveniente surgido (una cosa es que introduzcas una referencia a la solución de un problema que has tenido en el proceso de desarrollo de la aplicación de tu TFG y que has tomado de alguna entrada de estos sitios y otra es que la emplees como recurso para el estado del arte).

Hay que tener especial cuidado con las publicaciones web que vienen de empresas o compañías pues pueden estar sesgadas por temas comerciales y publicitarios. Aún así, no es lo mismo una descripción de un software en el sitio web de la compañía que lo desarrolla, que en alguna otra página de alguien que no conoces, por ejemplo, y que simplemente está dando opiniones sin fundamento. Ten cuidado con estas cosas.  

Otro aspecto importante es el temporal, ya que en ocasiones querrás seleccionar aquellos recursos que sean actuales y en otras te sirvan todos los recursos que encuentres, independientemente del momento en que hayan sido escritos. 

Los criterios de inclusión/exclusión se suelen aplicar en esta fase. Son condiciones que deben cumplir los recursos para poder considerarlos en las fases siguientes o descartarlos, respectivamente. Suelen estar compuestos por periodos temporales, idiomas, restricciones en cuanto a metodologías usadas o tipos de publicaciones. Todos los recursos que no los cumplan no serán tenidos en cuenta para su análisis y los que sí pasarán a la siguiente fase de análisis.

\subsection{Evaluación de la calidad de la información, análisis y síntesis}

En la sección anterior has visto que existen unas fuentes más fiables que otras. Un artículo reciente revisado por pares en una revista de calidad puede que sea más importante más que un post de Reddit de hace 10 años (ojo, no caigamos en falacias de autoridad tampoco). Tampoco te fíes de qué te dice ChatGPT.

Para hacer un buen TFG debes convertirte en la persona más experta en el tema que vas a desarrollar. O por lo menos, acercarse.

Lo primero que recomendamos es que puedas separar lo importante de lo que no lo es. ¿Y qué es lo importante?  Pues resulta que no lo sabrás hasta que no hayas leído lo suficiente y empieces a ser consciente de ello. Es decir, vas a empezar a trabajar sin saber en qué centrarte. En este momento comenzarás a cribar recursos, eliminando los que consideres que no son interesantes para tu trabajo y seleccionando los relevantes para su estudio en profundidad.

Empieza a tomar notas de todo lo que leas y te parezca interesante. Pero cuidado, no te interesa tener un montón de notas enormes una detrás de otra con un montón de datos vomitados, porque quizás la mitad no sirva de nada. Pero tampoco algo muy críptico, porque cuando la cabeza te haga ``clic'' y descubras qué es Lo Importante\texttrademark  te va a tocar releer cosas que ya pensabas que habías revisado. 

Quizás lo suyo es un término medio. Un truco que nos funciona muy bien es abrir una hoja de cálculo en tu suite favorita (LibreOffice, MS Office, Notion, Obsidian...) y crear las siguientes columnas (para empezar):
\begin{itemize}
    \item Año,
    \item título de la referencia,
    \item autores,
    \item enlace,
    \item justificación para el proyecto.
\end{itemize}

Las primeras 4 columnas son autodescriptivas y objetivas, datos de la fuente, pero la última es lo que la referencia aporta a tu estado del arte. Intenta resumirlo a una o dos frases como mucho.

Mientras vayas leyendo a lo mejor descubres que tienes que crear una nueva columna. Por ejemplo, puede interesarte una columna ``Algoritmo'', ``Lenguaje usado'', ``Framework'', ``Sistema operativo compatible'' o ``Licencia''. A lo mejor, si tu trabajo es de investigación en redes neuronales, la quinta referencia que estás leyendo de repente te inspira para crear ``Librería usada'', ``Número de capas'' y ``Dataset utilizado''. Vaya, ahora vas a tener que releer las cuatro referencias anteriores que ya habías anotado antes de crear la columna y completar sus huecos. Pero no pasa nada, eso es que te estás haciendo una persona más experta en el tema ahora que cuando leías esas cuatro primeras referencias. Y no solo eso, sino que también conforme vayas rellenando esa tabla podrás agrupar las referencias según algunos criterios relevantes para tí: metodologías, algoritmos, problemas, soluciones, etc.

Cuando termines tendrás una visión general muy fácil de comprender de un sólo vistazo. Quizás veas que la gran mayoría de software parecido al tuyo solo es compatible con Windows. O que casi todo el mundo utiliza el algoritmo de explicabilidad de Shapley. O que la mayoría de proyectos como el tuyo se han desarrollado en los últimos dos años. A partir de toda esta información podrás extraer conocimiento que podrás plasmar en la memoria de una manera más fácil: tu proyecto es el primero en ser multiplataforma, tu proyecto es el primero en ofrecer esta funcionalidad, tu proyecto es el primero en aplicar este dataset, tu proyecto es el primero en aplicar esta metodología... Tu proyecto es el primero en algo.

De hecho, es muy buena idea poner una tabla-resumen en tu TFG. Permitirá a la persona que lea tu capítulo tener también una visión muy rápida de cómo está el asunto.

Actualmente con el desarrollo que están teniendo los LLMs, existen herramientas como \href{https://notebooklm.google}{NoteBookLM}, de Google, que te pueden ayudar a realizar un estudio comparativo de varios recursos que subas a esta aplicación. Utiliza este recurso si lo ves conveniente, pero no olvides revisar la salida y entender bien lo que te dicen.

\section{Cómo plasmarlo en la memoria}

Una vez que haz hecho la búsqueda, la selección de material relevante, su lectura y análisis, queda plasmarlo en la memora. Esta sección de la revisión del estado del arte podría tener tres partes claramente diferenciadas:

\begin{itemize}
    \item Introducción. Tras dar una breve descripción del contexto entre manos o del problema para el cual vas a hacer la revisión, debes seguir con la exposición de los objetivos de la misma.  Tienes que dejar claro por qué es necesario hacer una revisión y qué pretendes conseguir con ella. En esta sección también tienes que explicar la metodología seguida: cuál es tu pregunta, cómo has generado la consulta, qué fuentes bibliográficas has consultado, qué criterios son los de aceptación y de exclusión y por qué y cualquier otra cosa que consideres relevante para describir el proceso de búsqueda y confección de la revisión.

    \item La revisión propiamente dicha. Esta parte es donde debes plasmar la revisión que has realizado. Podrías hacerlo mediante estas estrategias:

    \begin{itemize}
        \item Cronológicamente: simplemente ir mostrando la evolución en la temática a lo largo del tiempo. Pero esto no es simplemente un listado, sino que tienes que indicar qué se va aportando en cada fuente analizada, es decir, cuál es el avance.

        \item Temáticamente: como habrás identificado ciertas temáticas, conceptos, enfoques o áreas, puedes organizar la revisión entorno a esos campos, quizá en forma de secciones, y dentro de cada una la presentación de los recursos y su discusión.

        \item Metodológicamente: también puedes haber distinguido diferentes metodologías o aportaciones teóricas. En ese caso, al igual que el punto anterior, puedes estructurar tu análisis agrupando los recursos según las que apliquen y analizando la forma de aplicarlas y los resultados obtenidos.
    \end{itemize}
    
    También puedes combinar estas estrategias. Por ejemplo, puedes hacer una organización temática de primer nivel y seguidamente dentro de cada sección hacer el análisis cronológico.

    Es útil, para resumir y ofrecer información relevante de un vistazo, que incluyas una tabla con las referencias en las filas y en las columnas características de interés para tu estudio y los diferentes valores que aporta cada recurso analizado, al estilo de lo explicado en el apartado anterior.

    De cualquier forma, recuerda que debes ofrecer los elementos principales de cada recurso, interpretarlos y discutir sus aportaciones de forma individual, pero también de forma global. Y todo con tus palabras. Realiza una labor crítica estableciendo ventajas, inconvenientes, puntos fuertes o débiles y situaciones de mejora. También es muy importante que como resultado de este análisis encuentres lagunas de conocimiento, que pueden manifestarse como áreas poco estudiadas o aspectos que han recibido poca atención en la investigación previa, preguntas sin respuesta o temas que requieren una exploración más profunda. Identificar y señalar estos vacíos en la literatura es importante porque puede orientar investigaciones futuras y proporcionar oportunidades para contribuir de manera significativa al campo de estudio.

    Todos los recursos deben estar citados convenientemente y sus detalles bibliográficos deben aparecer en la sección de bibliografía, tal y como se indica en el Capítulo \ref{cap:bibliografia}.

    \item Conclusión. Es la sección donde debes resumir los hallazgos de ese proceso de revisión crítica de los recursos que has encontrado y analizado. Debes dar importancia a dichos hallazgos y sobre todo enfatizar las lagunas que has encontrado y cómo tu propuesta de proyecto se enmarca en alguna de ellas y cómo puede ayudar a mejorar la situación actual de conocimiento.
\end{itemize}

\section{Recomendaciones generales}

Seguidamente te damos algunas recomendaciones generales para realizar la revisión del estado del arte y confeccionar la sección correspondiente en la memoria de tu TFG:

\begin{itemize}
    \item Haz la revisión al principio. No lo dejes para el final pues puedes llevarte sorpresas de que lo que tú hayas hecho en tu proyecto ya esté hecho.
    \item Dale importancia a la escritura de la revisión pues es una magnífica carta de presentación tuya ya que estás mostrando tu capacidad de comprensión, análisis y síntesis.
    \item Como ya hemos dicho, no copies texto. Eso es plagio. Entiéndelo y plásmalo con tus palabras.
    \item Un argumento es como una katana, no puedes sacarla sin hacer sangre. Así que cualquier argumento que escribas debería ir citado. Y no solo deberías añadir citas en el capítulo del estado del arte, sino a lo largo de toda la memoria.
    \item No copies texto literalmente de un recurso. Eso es plagio. Entiéndelo y redáctalo con tus propias palabras.
    \item Analiza la documentación de que dispones y plasma ese análisis. No pongas simples resúmenes.
    \item Compara enfoques y aproximaciones, establece ventajas e inconvenientes, puntos fuertes y débiles.
    \item Habla con la persona que te tutoriza sobre la orientación del estudio. 
    \item Discute el estudio la persona que te supervisa conforme lo vas realizando.
    \item Busca lagunas y limitaciones de lo que hay hecho. Ahí es donde podrás enmarcar y construir tu propuesta.
    \item Dedícale tiempo a buscar y a leer. Hazlo con tranquilidad y con tiempo suficiente.  
    \item Incluye figuras que ilustren la metodogía que has seguido en el estudio.
    \item Incluye figuras y gráficos que, de forma visual, describan en análisis que has hecho.
    \item Incluye tablas comparativas.
    \item Ve al grano. No es necesario que escribas la biblia en verso en este capítulo. Recuerda que lo bueno si breve dos veces bueno.
\end{itemize}

\chapter{La planificación y el presupuesto}
\label{cap:PlanificacionPresupuesto}
% [Autores: María José Rodríguez Fórtiz]
% Se puede meter también un plan de contingencias: hecho

Cualquier proyecto que se precie debe indicar cómo se va a desarrollar en el tiempo, con objeto de mostrar su duración y cómo se van a ir realizando las tareas que lo componen, y cuánto va a costar, con objeto de determinar su viabilidad económica, tanto para un cliente que esté interesado en su realización como para nosotros mismos, o nuestra empresa, si somos quienes lo llevaremos a cabo. Y el proyecto de tu TFG no será menos. Por tanto, la memoria de tu trabajo deberá contener estos dos elementos. En este capítulo te damos algunos consejos sobre la elaboración de la planificación y del presupuesto de tu proyecto.


\section{Planificación temporal}
El objetivo de esta sección de la memoria de tu TFG es mostrar temporalmente cómo se van a organizar las diferentes tareas de tu TFG. Para ello, lo habitual es incluir un diagrama de Gantt con el cronograma final de realización del TFG. También puedes incluir el cronograma de planificación inicial y hacer una comparativa entre ambos, explicando sus diferencias y justificando los motivos de cambio.

Un diagrama de Gantt en una tabla en la que las columnas son unidades de tiempo (días, semanas o meses) y las filas son tareas en las que se descompone el proyecto. Habitualmente las tareas se agrupan en paquetes de trabajo, según su funcionalidad o las etapas o fases del proyecto en el que éste se haya dividido. En las celdas de la tabla se marcará qué tareas se han hecho en esas unidades de tiempo, de tal forma que tenemos información gráfica del comienzo, duración de tareas, y relación con otras. Ocasionalmente se pueden incluir vínculos entre tareas, para forzar relaciones de fin a comienzo. Por ejemplo, no puedo empezar una tarea B hasta que no haya terminado una tarea A. También puede haber tareas con ejecución intermitente. Por ejemplo, puedes poner una tarea de ``Reuniones con mi tutor o tutora" que tenga asignados varias franjas temporales, como un día cada dos semanas.

En la figura \ref{fig:gantt} se muestra un ejemplo de un diagrama de Gantt básico para un proyecto.

\begin{figure}[!t]
    \centering
    \includegraphics[width=.8\textwidth]{images/EjemploGantt.pdf}
    \caption{Diagrama de Gantt\label{fig:gantt}}
\end{figure}

El diagrama de Gantt debe incluir todas las tareas que has realizado (o vas a realizar, pues es se hace antes de comenzar a realizar las tareas) vinculadas con los objetivos del TFG. Ten en cuenta que en los objetivos de tu TFG no debe haber solo objetivos de desarrollo, sino también en muchos casos de aprendizaje, porque te plantees aprender nuevas tecnologías o conocer más del dominio de aplicación del problema que vas a abordar. También tendrás objetivos relacionados con la organización de tu trabajo como reuniones, revisiones y redacción de la memoria. 

Teniendo en cuenta esto, debes planificar todas las tareas necesarias para cubrir los objetivos planteados, y por ello, en el cronograma debe haber tareas de aprendizaje, de desarrollo, y de organización. En cuanto al desarrollo, si vas a seguir una metodología específica, esta tiene que verse reflejada en el cronograma. Por ejemplo, si sigues una metodología ágil con iteraciones cada dos semanas, en tu cronograma debería aparecer una tarea con duración de dos semanas por cada iteración. Si tu ciclo de vida no es iterativo sino clásico, deberían aparecer paquetes de trabajo para especificación, diseño, implementación, pruebas, etc, con tareas específicas dentro de cada uno de ellos.

Cuanto mayor sea el nivel de detalle del cronograma, mejor, ya que darás más información de las tareas y tiempo dedicados. Te aconsejamos que la unidad temporal usada en el cronograma sea semanal, aunque también puedes usar meses. 

OJOJOJOJOJOJOJJOJOJOJOJ ESTO NO LO VEO -> DEBE HACERSE A PRIORI
Te será muy fácil hacer el cronograma si has sido metódico/a, apuntando cada día de trabajo en el TFG el número de horas dedicadas al TFG y a qué tarea específica las has dedicado. Solo recuerda que todo el tiempo de tu vida que dediques al proyecto debe estar reflejado en el cronograma final. 

Existen varias aplicaciones informáticas de escritorio o en línea para hacer diagramas de Gantt y gestionar cambios sobre estos. Muchas de estas herramientas permiten fijar una línea base (como una fotografía del diagrama en un momento) para comparar con cambios posteriores, de tal forma que también se pueden hacer simulaciones. Otra funcionalidad que ofrecen estas herramientas es poder asignar recursos humanos y materiales a las tareas, con costes asociados, lo cual es muy útil para hacer un presupuesto del proyecto. Algunas de estas aplicaciones te permiten trabajar en equipo haciendo un uso compartido entre varios usuarios. En tu caso podrías compartir el diagrama de planificación con la persona que te tutoriza.

A día de hoy, te podemos sugerir algunas herramientas gratuitas y sencillas para hacer y gestionar diagramas de Gantt, como son: \href{https://www.ganttproject.biz/}{GanttProject}, \href{https://www.projectlibre.com/}{Project Libre}, 
 \href{https://www.monday.com}{Monday.com}, \href{https://app.clickup.com/}{Clickup}. 

También puedes hacer tu diagrama de Gantt usando \href{https://www.canva.com/}{Canva}, eligiendo una de las plantillas de este tipo que se proveen, o con una hoja de cálculo, pero en estos casos no tienes tantas facilidades para editar los cambios temporales, ni funcionalidades como las apuntadas arriba.

Algunas recomendaciones sobre esta sección son que describas brevemente los tipos de tareas de tu diagrama y cuál es tu metodología de desarrollo para que se pueda comprender mejor. En cuanto a su visualización dentro de la memoria, debes asegurarte de que el diagrama se vea bien, para lo cual te sugerimos que lo pongas en apaisado a página completa, o que lo subdividas en varios diagramas, por ejemplo, uno por cada paquete de trabajo. El diagrama deberá formar parte de la presentación final, así que es mejor que uses colores o tramas para diferenciar mejor los tipos de tareas y para que sea más atractivo visualmente. 

\section{Plan de gestión de riesgos y contingencias}

En algunos TFG, por su temática o tecnologías usadas, o porque la persona que te tutoriza lo vea conveniente, se puede presentar un plan de gestión de riesgos. El plan es una lista de posibles riesgos que pueden surgir durante el desarrollo del TFG, cada uno de ellos con acciones para evitarlos y/o mitigarlos.


La lista de riesgos suele priorizarse según su impacto en el proyecto. Para ello, hay que hacer un estudio previo en el que valoremos si la ocurrencia del riesgo afecta al alcance del TFG (cambios en requisitos), al tiempo (retrasos, cambio de orden y tiempo asignado a tareas), al coste (incremento de gastos), a los recursos (cambios en tecnología usada), etc. Una vez valorado esto, asignaremos más prioridad a aquellos que tengan mayor impacto. 

Teniendo en cuenta el impacto de cada riesgo, lo siguiente es planificar una acción de prevención para evitar que ocurra, si es posible. Se deben planificar acciones de mitigación si no podemos evitar que el riesgo ocurra o si el plan de prevención fallara.

Por ejemplo, un riesgo podría ser que la tecnología a usar para el desarrollo del TFG sea muy nueva, lo que conlleva que no haya apenas manuales de uso y poca gente la conozca, así que no podrás tener mucha ayuda en foros, y  si tienes algún problema con ella puede que no puedas seguir adelante y te quedes bloqueado. El impacto de ese riesgo podría afectar al alcance del proyecto, pero también al tiempo y por supuesto a los recursos. Como plan de prevención, podrías apuntarte a un curso de formación existente (aunque incremente los costes del proyecto), y como plan de mitigación, la acción propuesta podría ser dejar el uso de esa tecnología solo para una parte del desarrollo menos importante, no para todo, de tal forma que no se vea afectado el proyecto completo.

Tienes una lista de diversos tipos de riesgos y más información sobre el plan de gestión de riesgos en el capítulo de Estimación de riesgos, de \cite{guerin2018gestion}, disponible en línea en la biblioteca de la UGR.

Una vez planteado el plan de mitigación, durante el proyecto debe revisarse para decidir si se realiza alguna de las acciones previstas y en esta sección de la memoria, explicar con detalle qué acciones se han realizado y sus resultados, incluyendo comentarios sobre los costes asociados.

Si vas a incluir un plan de gestión de riesgos, consensúa con la persona que te tutoriza en qué capítulo incluirlo, ya que una opción es que sea una sección del capítulo de la propuesta, para que esté más cercano a la explicación de  cómo se han gestionado los riesgos durante el desarrollo.

\section{Presupuesto}

Con esta sección se indica al lector de la memoria cuál sería el coste de desarrollo del proyecto realizado dentro del TFG en el caso de que este proyecto se ejecutara en la vida real. Esta sección ayuda al lector valorar el tiempo dedicado por tu parte al proyecto, tus decisiones respecto a la tecnología elegida, y el coste de todo ello. Como ingeniero, debes demostrar que sabes hacer un proyecto coherente  y sensato, buscando las mejores alternativas. 

El presupuesto debe tener dos tipos de conceptos principales: costes de personal y costes de ejecución.

El personal que realiza el proyecto eres tú, por lo que el coste se calcula multiplicando el número de horas que tú has dedicado al TFG por el coste por hora estimado. Si para el desarrollo de tu proyecto has tenido que aprender tecnologías y revisar un estado del arte, ese tiempo también debe cuantificarse. Es decir, en el coste final debes contar también las horas que no son exclusivamente de desarrollo. 

Las reuniones que mantengas con tu tutor también deben computarse como tiempo dedicado y debes computar en el coste total las horas dedicadas por tutor (al correspondiente precio/hora).

Para el cálculo total de horas ya te hemos recomendado que lleves un diario donde las anotes. Tiene que haber coherencia entre tu planificación temporal y número de horas dedicado al proyecto, y el número de horas que pones en esta subsección.

¿Cómo puedes estimar el coste por hora? Te recomendamos que consultes en páginas especializadas, mirando los anuncios, o usando, si las proporcionan, herramientas de cálculo de salarios brutos anuales según ciudad, como las siguientes: \href{https://www.tecnoempleo.com/ofertas-trabajo/}{TecnoEmpleo} o \href{https://www.tecnoempleo.com/ofertas-trabajo/}{InfoJobs}, o en el cluster de empresas tecnologícas de Granada, \href{https://www.ontechinnovation.com/bolsa-de-trabajo/}{On Granada Tech City}. Cuando hagas la consulta, ten en cuenta que no todos los informáticos cobran lo mismo, ni en todas las ciudades. Tu categoría será posiblemente \textit{junior}, sin experiencia previa en trabajo. 

Respecto a los costes de ejecución, aquí debe ir una línea por cada uno de los gastos hardware, software, viajes y dietas que hayas tenido durante el desarrollo del TFG. 

En cuanto al hardware, puedes incluir el coste de dispositivos usados como tu ordenador, tablet, teléfono, etc. pero no el coste íntegro. Piensa que esos dispositivos los has usado para otros proyectos, incluidos los personales, por lo que debes prorratear teniendo en cuenta su tiempo de vida útil, el coste actual del dispositivo en el estado en el que está y el tiempo de vida de uso exclusivo para el proyecto. Por ejemplo, si un portátil te costó 600 euros en su día, y hoy valdría 400 euros, y le quedan dos años de vida útil, pero tú lo has usado solo 6 meses para este proyecto, podrías anotar en el coste de ejecución asociado a equipamiento unos 100 euros como mucho.  

Por otro lado, si para tu TFG has necesitado material hardware aparte que hayas tenido que adquirir, o te haya facilitado la persona que te tutoriza, también debes incluirlo (con su prorrateo si es el caso). Si has tenido que contratar un servidor, añade también los costes según los meses que lo has usado, y prevé el coste un uso de un año más para dar soporte al proyecto y su mantenimiento. Lo mismo si has necesitado adquirir o contratar software.

Si necesitas viajar para realizar tu TFG, también puedes incluir como gastos de ejecución los costes de dietas y desplazamiento. Esto suele incluirse en TFGs de investigación o más aplicados, que requieren reuniones con colaboradores como entidades o empresas que han solicitado que se realice un TFG con ellas y que actúan como nuestros clientes, dando requisitos, evaluando diseños y haciendo pruebas. 

Por último, puedes añadir costes indirectos. Son aquellos que se comparten entre varios proyectos, como la electricidad o cuota de Internet. Normalmente se calcula un 10\% del total del presupuesto para costes indirectos.

El presupuesto lo puedes hacer en una tabla pero te aconsejamos que lo hagas mejor con una herramienta como una hoja de cálculo, o una de gestión de proyectos que haga cálculo con los costes de recursos y planificación temporal, como las mencionadas en la subsección anterior.

Como consejo, no añadas únicamente la tabla del presupuesto a la memoria, añade un texto breve explicativo, incluyendo alguna justificación sobre los gastos como en qué te has basado para el cálculo de coste por hora, o porqué se ha adquirido y usado cierta tecnología que mencionas en el presupuesto. Si durante el proyecto has cambiado de tecnología y ésto ha influido en los costes, menciónalo también.

\subsection{Ejemplo de presupuesto}

\subsection{Coste de desarrollo}

\subsubsection{Coste de personal}

\subsubsection{Coste de material}

\subsection{Coste de despliegue y mantenimiento}
\chapter{Las conclusiones y los trabajos futuros}
\label{cap:Conclusiones}

% [Autores: María José Rodríguez Fórtiz]
El capítulo de \textit{Conclusiones y trabajos futuros} es muy importante pues recoge qué se ha realizado en el TFG, los principales resultados y qué puede hacerse a partir de este momento. Muchas personas leen la introducción y luego las conclusiones antes de leerse el resto de la memoria, para así conocer bien la motivación, objetivos y los resultados del trabajo. Eso significa que debes tener especial cuidado al redactar este capítulo para que quede muy claro y sea muy completo.

Habitualmente incluye dos secciones, la de conclusiones y la de trabajos futuros.
 
 \section{Conclusiones}
Debes empezar las conclusiones con una frase inicial a modo de resumen sobre los resultados de tu trabajo, con una valoración positiva sobre ellos.  A continuación debes hacer un repaso uno a uno de los objetivos específicos, indicando en él (1) el porcentaje de realización, (2) un resumen de lo que se ha hecho para cumplir ese objetivo (dos o tres líneas explicando las tareas realizadas asociadas a ese objetivo, y los resultados obtenidos deben bastar), y (3) una indicación de dónde pueden verse las evidencias de ese objetivo en la memoria, en qué capítulo o sección.

También puedes mencionar en esta parte, cómo tu formación previa en materias concretas del grado te ha sido de ayuda para el TFG y qué has tenido que aprender para concluir exitosamente el proyecto.

 En cuanto a la redacción de este repaso de objetivos, te ponemos un ejemplo. Suponiendo que estás abordando un objetivo específico que has redactado como ``Revisar aplicaciones similares para comparar con la propuesta'', puedes indicar que ese objetivo se ha cumplido completamente, explicando por ejemplo que has revisado 6 aplicaciones similares y que has realizado una tabla comparando 8 características básicas de cada una, la cual puede consultarse en el capítulo o sección X de la memoria. También puedes añadir que al elaborar esta tabla se demuestran tus capacidades de análisis y síntesis de información. Si este objetivo no se hubiera cumplido completamente, porque, por ejemplo solo hayas revisado 2 aplicaciones y tenías previsto revisar más, pues dices lo que sí has hecho pero solo un 30\%, y argumentas porqué es insuficiente, por ejemplo, porque solo hay 2 de libre acceso que has podido consultar con profundidad, o porque has priorizado terminar la tarea X, que habéis considerado que era más importante para el TFG. 

 De cara a la redacción de esta sección puedes tener en cuenta el registro de marcas propuesto en \cite{meza2019comunicacion}, que sugiere verbos que puedes utilizar (en este caso para explicar en las conclusiones cuál ha sido el alcance de cada objetivo), como son: ``ha abordado'', ``hemos hecho un recorrido por'', ``podemos afirmar que'', ``esto evidencia que'', ``confirmamos que'', ``confirma nuestras hipótesis/ideas'',``hemos propuesto, obtenido, identificado, revisado, observado, descubierto, utilizado, demostrado, explicado, desarrollado, etc.'',  ``no hemos podido demostrar, confirmar, revisar, identificar ... porque ...'', etc. Como marcadores discursivos, podemos usar conectores como los siguientes: ``por tanto'', ``sin embargo'', ``en consecuencia'', ``por el contrario'', ``a pesar de'', ``gracias a'', ``entendemos que'', o ``de acuerdo/según todo lo anterior''.
 
Es importante que en las conclusiones añadas un párrafo final como valoración personal, escrito esta vez en primera persona. En esa valoración debes mencionar cómo te has sentido al realizar el TFG y en base a sus resultados. Puedes indicar que te sientes orgulloso/a, contento/a, satisfecho/a, encantado/a, etc. por lo que has aprendido, por cómo te has organizado en el tiempo, por cómo has redactado la memoria, por la calidad del código desarrollado, por cómo te has comunicado con tu tutor, etc. Si tienes alguna valoración negativa, debes mencionarla también, pero te recomendamos que la redactes de forma positiva, aportando que has aprendido de ello. Por ejemplo, `No estoy satisfecho/a con cómo he organizado el trabajo temporalmente porque he dejado muchas tareas para el último mes y eso me ha saturado, con lo cual he aprendido que en un futuro debo hacer una mejor planificación temporal desde el principio.". En tu valoración personal, y si no lo has hecho al revisar los objetivos, también puedes mencionar cómo has aplicado y mejorado tus habilidades blandas o \textit{soft skills}, como son organización de trabajo, pensamiento crítico, creatividad, adaptación, resolución de problemas y comunicación. 

 \section{Trabajo futuro}

 En esta sección se enumeran:
 \begin{itemize}
     \item Tareas que que estaban previstas y no se han hecho o han quedado incompletas, de las mencionadas en las conclusiones.
     \item Requisitos de desarrollo que tenías previsto abordar pero que al final no has tratado.
     \item Nuevos requisitos que hayan surgido durante el desarrollo del TFG, que no se habían previsto y por tanto no se han planificado ni abordado.
     \item Nuevos objetivos e ideas para dar continuidad al TFG en futuros TFGs, desarrollos o investigaciones.
 \end{itemize} 

 Para cada una de estas tareas, requisitos u objetivos conviene añadir un pequeño párrafo que explique porqué se propone y cómo se abordaría, de forma muy resumida. Por ejemplo: ``En un futuro se puede desarrollar una versión en iOS del prototipo realizado en el TFG. Esto ayudaría a que más personas pudieran utilizar la aplicación. Para ello, se podría utilizar un \textit{framework} de desarrollo como Flutter o Ionic, que permiten esta portabilidad y el desarrollo híbrido de aplicaciones móviles. Habría que valorar si el código actual o parte de éste puede reutilizarse''. Otro ejemplo de párrafo: ``Sería necesario completar la gestión de usuarios en la aplicación desarrollada, ya que por el momento solo pueden hacerse altas y modificaciones. Bastaría para ello diseñar e incluir funciones e interfaces para el borrado de usuarios de la misma forma que se ha hecho para las otras operaciones. Esto no supondría ningún cambio en la base de datos''. 

 En el caso de ser una tarea del primer tipo, justifica bien la razón por la que no se ha podido realizar íntegra o parcialmente. 

 

 
\include{10.bibliografia_tfg}
\include{11.anexostfg}
\chapter{La revisión del proyecto y la memoria}
\label{cap:Revisión}

% Comentarios:
% ¿Meter una lista de ítems a comprobar?
% ¿Aconsejar que se la confeccione el estudiante?
% Esto lo tendremos que ordenar por grupos.

\begin{itemize}
  \item Sobre la estructura o formato de la memoria:

  \begin{todolist}
    \item Seguir la normativa de portadas (logos, colores, etc.) y prólogos (autorización, resumen, etc.).
    \item Los índices de contenidos, tablas y figuras deben estar actualizados, generándolos automáticamente.
    \item No quedan tablas cortadas al final de una página.
    \item Si hay títulos de secciones al final de una página, debajo de ellos debe haber al menos un párrafo.
    \item Utiliza los mismos estilos y tipo de letra en toda la memoria, a no ser que quieras resaltar algo como citas literales, fórmulas, ecuaciones o código.
    \item Las páginas en blanco que dejes deben ser intencionales, por ejemplo para que cada capítulo empiece en página impar.
  \end{todolist}
  
  \item Sobre la bibliografía:

  \begin{todolist}
    \item Todas las referencias de la bibliografía deberían aparecer citadas en el texto.
    \item Todas las citas bibliográficas del texto deben tener una referencia asociada.
    \item Las referencias de recursos de Internet deben indicar también la fecha de la última consulta.
  \end{todolist}

  \item Sobre las figuras y tablas:

  \begin{todolist}
    \item Todas las figuras y tablas deben estar numeradas secuencialmente y referenciadas/citadas en el texto.
    \item Todas las figuras y tablas deben tener un título descriptivo.
    \item Los títulos de figuras y tablas deben estar en la misma página de la figura o tabla.
    \item Si se usan figuras hechas por otra persona deberían ser citadas/usadas correctamente, por ejemplo en el pie de figura "Extraído de [x]" o "Fuente: [x].
    \item Las imágenes y las tablas están dentro del espacio del cuerpo de la página y ninguna se desborda hacia los márgenes. 
  \end{todolist}

  \item Por último, no olvides:

  \begin{todolist}
    \item Revisa la ortografía. 
  \end{todolist}

\end{itemize}
\chapter{La elaboración de la presentación en la defensa} \label{cap:elaboraciónPresentación}

% [Autores: Pablo, Alberto]

¡Ya queda menos para poder presentar el proyecto! Has llegado a los objetivos que te habías propuesto (tu programa ya funciona, tus resultados experimentales son geniales...) y ya estás listo para enseñar tus avances al mundo.

Después de todo el esfuerzo, de todas las horas, de todos los cabezazos contra la pantalla, resulta que solo tienes 20-25 minutos ante un tribunal para contarles lo que has hecho. Además, seamos sinceros, muchas veces el tribunal se va a leer la memoria de pasada mientras estas haciendo la presentación, así que te juegas mucho dependiendo de cómo lo hagas en esos 20 minutos y lo que enseñes y no enseñes. En este capítulo te enseñaremos cómo preparar la presentación, es decir el archivo (el PDF, el ODP, el PPTX) que proyectarás en el aula, mientras que en el siguiente capítulo nos centraremos en cómo preparar la defensa de esa presentación.

\section{Centrarte en lo importante}

El primer consejo que te vamos a dar es que el tribunal tiene que mirarte a ti, no a las transparencias. Tú eres la estrella de la función, y la presentación es un apoyo a lo que estás diciendo. El segundo consejo, y más importante todavía es: \textbf{no te pases del tiempo}. Alguien del tribunal tiene un cronómetro encendido y no hay nada que quede peor en una defensa que decirle al estudiante ``lo siento, tienes que cortar''. Así que no prepares 50 transparencias y quieras contarlo todo con miles de pelos y señales. Por ejemplo, puedes ahorrarte transparencias de bibliografía, que aportan poco en la defensa.

Aunque dependerá del tipo de TFG que realices, una posible estructura puede ser la siguiente:
\begin{itemize}
\item Título del proyecto, fecha, tu nombre y tu correo electrónico y el nombre de tu director o directora. Logos de la Universidad y Escuela, para dejarlo más profesional.
\item La segunda transparencia debe ser un índice numerado de las secciones de la presentación.
\item Una o dos transparencias de introducción y contexto.
\item Una transparencia definiendo los objetivos.
\item Una transparencia resumiendo el estado del arte
\item Una o dos transparencias de planificación y metodología
\item Si tu trabajo es de desarrollo
    \begin{itemize}
    \item Dos o tres transparencias del diseño e implementación, diagramas sobre todo.
    \item Una transparencia hablando de pruebas.
    \item Capturas de la UI, aunque mejor haz una demo.
    \end{itemize}
\item Si tu trabajo es experimentación o investigación:
    \begin{itemize}
        \item Dos o tres transparencias describiendo el método y mostrando los resultados (gráficas sobre todo). Dependiendo del número de experimentos que realices quizás necesites más. Pero recuerda, casi siempre menos es más.
    \end{itemize}
\item Una transparencia de conclusiones, incluyendo enlace a repositorio del código, si lo hubiera.
\item Una transparencia de despedida, con un texto parecido a ``Muchas gracias por su atención''.
\end{itemize}

Esto nos lleva a una presentación con 15 a 20 transparencias aproximadamente. Si te centras un minuto en cada una, y créenos, un minuto pasa muy rápido, ya lo tienes listo.

Respecto a los títulos de las transparencias, mucha gente cambia el título de cada una para que sea como un titular de periódico, que le da un toque más innovador.  Es decir, en vez de poner títulos genéricos como ``Introducción (I)'' o ``Introducción (II)'' puedes poner una transparencia (que no usaremos en el conteo) con el texto centrado ``Introducción'' y pasar rápidamente a las siguientes tituladas ``El problema de los tres cuerpos es muy difícil de resolver'' y ``Se han utilizado algunas cosas sin éxito''. Fíjate como parecen titulares de periódico, pero seguimos siendo conscientes de que estamos en la introducción. De hecho, como curiosidad, fíjate también en los nombres de las secciones de este capítulo, no hace falta leer el texto para sacar la idea principal de cada una.

\section{La presentación no es un karaoke}

Si vas a leer lo que pone en las transparencias envíala por correo y nos ahorramos tiempo. Y recuerda que menos es más. Así que quita la broza. En tu memoria quizás hayas escrito algo como 

``\textit{El objetivo de este proyecto es demostrar que, bajo ciertas condiciones, utilizar el algoritmo Williamsito, creado por Williams [11] permite obtener mejor rendimiento para resolver el problema de los tres cuerpos [12] reduciendo el tiempo de computación}''. 

Pues eso, en la presentación quita la broza: 

``\textit{Williamsito reduce el tiempo para resolver el problema de los tres cuerpos}''. 

Ya está, dicen exactamente lo mismo, pero más rápido. El tribunal lo habrá pillado enseguida y podemos pasar a otra cosa.

Especialmente importante será la transparencia de conclusiones. Es la última carta que tienes en la manga para que el tribunal vea que has hecho un buen trabajo.

\section{El poder de lo visual}
Vamos a mejorar el texto anterior: 

``\textit{Williamsito \textbf{reduce el tiempo} para resolver el problema de los tres cuerpos}''. 

Fíjate cómo hemos usado un componente visual (la negrita) para ir directamente a la idea de la frase. No te cortes en usar técnicas tipográficas para facilitar la lectura (negrita, cursiva, color), pero tampoco te pases. Resaltar de una a tres palabras por frase es más que suficiente.

A veces incluso se puede sustituir el texto con una imagen o un pictograma que represente la idea. Por ejemplo, en vez de poner una lista de los componentes de tu sistema y lo que hacen, utiliza un diagrama. De un solo vistazo vemos que tiene 5 componentes y se comunican usando MQTT. Perfecto, todo claro. 

Pero igual que antes, no pongas diagramas que no aportan y te pongas a explicarlos. El diagrama de clases o el de E/R, por ejemplo, generalmente no aportan absolutamente nada en la presentación, ya has aprobado las asignaturas que te lo evaluaban. Tampoco pongas código fuente, a menos que sea realmente necesario (que es casi nunca).

Y ojo con los colores. Utiliza paletas ya establecidas. Existen webs que te permiten coger un grupo de 4 colores que combinan bien. Por ejemplo Coolors.co \footnote{\url{https://coolors.co}}, pero hay otras muchas. A menos que sepas de diseño gráfico y teoría del color no te fíes de tu criterio artístico. Esto es especialmente importante si estás visualizando datos. Además, puede haber personas daltónicas en el tribunal.

Además, texto oscuro en fondo claro hace que el público mire a la transparencia, pero texto claro sobre fondo oscuro hace que miren al orador.

Y para terminar un consejo muy fácil de aplicar y que queda muy bien: añade el número de transparencia actual y el total de transparencias en la esquina (Ej: \textit{4 de 20}). De este modo el tribunal podrá orientarse y saber cuánto te queda. Si por ejemplo ven que te pasas un poco de tiempo pero solo te queda una transparencia quizás no te interrumpan. También servirá para que en el turno de preguntas puedas moverte a una transparencia concreta si te lo piden.

\section{Show. Don't tell.}

Esto es un aforismo que se usa mucho en el guión cinematográfico. En vez de escribir qué hace tu aplicación prepara una demo de dos o tres minutos en la que veamos cómo funciona. No hace falta que muestres todo, por ejemplo, como crear usuarios, que es algo trivial, sino la parte interesante.

Si tienes miedo de que algo falle (por ejemplo si el servidor está en tu casa), prepara un vídeo, pero no grabes tu voz explicándolo, da la explicación en directo mientras que visualizas el vídeo.

Consensúa esta presentación con la persona que te tutoriza el TFG hasta que estéis ambos contentos con ella.

\include{14.defensa}
\include{15.agradecimientos_libro}

\bibliographystyle{plain}
\bibliography{bibliografia_libro}


%\appendix
\appendix
\chapter{Aquí va el primer apéndice si hace falta}
Primer apéndice
\chapter{Las tipologías de TFG y su desarrollo}
\label{cap:Tipologías}

% -> Alberto: 
% Sugiero sacarlo del texto y meterlo como Anexo (o incluso ni eso) lo hablamos en reunión

% María José: lo he puesto directamente en el capítulo 4. Sugiero quitar este capítulo.

% Comentarios: Meter una introducción a las diferentes topologías, antes de entrar en profundidad a cada una de ellas, indicando también que existen otras tipologías menos comunes incluso híbridas.
% Descripción detallada de cada uno de los elementos que tienen que aparecer en la documentación.
% Ejemplo para un desarrollo clásico de software. Variaciones y desarrollo ágil (por ejemplo, siguiendo Scrum).

\section{TFG de desarrollo de software} %Maria Jose
\section{TFG de Experimentación Científica} % Eugenio y Rocio
\section{TFG de investigación}
\label{appendix:investigacion}

% (Por si puede ser de interés: https://bpb-us-w2.wpmucdn.com/portfolio.newschool.edu/dist/2/14941/files/2017/06/Judith_Bell_Doing_Your_Research_Project-xhunbu.pdf)
% Se podría dar un layout genérico para seguir en un TFG de este tipo.

% \subsection{Introducción}
%OJOJOJOJOJO ¿¿¿¿¿METER ALGUNA REFERENCIA BIBLIOGRÁFICA??????
La Ingeniería Informática se centra en la aplicación de conocimientos teóricos-prácticos para ofrecer la mejor solución posible a un problema, teniendo en cuenta las dimensiones temporales, personales, materiales y económicas. La Ingeniería Informática actual cuenta con desafíos científicos de gran envergadura en cada una de sus especialidades, destacándose a continuación algunos de ellos:
\begin{itemize}
    \item los retos electrónicos de aumentar el nivel de integración de las placas de unidades de procesamiento, ya sean de datos (CPU) o especializadas en gráficos (GPU), fundamentales para continuar ampliando la capacidad de cómputo de los sistemas informáticos;
    \item el amplio y complejo reto computacional y social que representa la inteligencia artificial;
    \item el continuo empeño en mejorar todo lo relativo al procesamiento geométrico para impulsar el avance de la representación gráfica;
    \item el objetivo de conseguir una informática sostenible a través de la mejora de la aplicación de los conceptos de la informática teórica para el diseño y desarrollo de algoritmos eficientes en tiempo y en espacio;
    \item la responsabilidad de mejorar las metodologías y métodos de desarrollo para la programación de sistemas informáticos seguros, robustos y respetuosos con la privacidad de los datos.
\end{itemize}
%Sin embargo, la aplicación trasversal de la informática hace que la Ingeniería Informática esté presente en el trabajo científico diario de un amplio espectro de disciplinas, sobresaliendo la investigación en medicina, química o biología. Los retos científicos propios y ajenos a los que se tiene que enfrentar la Ingeniería Informática, obliga al graduado a disponer de unas mínimas habilidades científicas, que le permitan aplicar los principios de la Ingeniería Informática al método científico, y contribuir, de esta forma, al progreso científico de cualquier disciplina.

La Ingeniería Informática es una disciplina muy transversal, por tanto, tenemos el reto de ser excelentes dentro de nuestro propio campo y, además, debemos aprender y conocer el dominio del problema donde aplicaremos la solución. En el contexto de un trabajo de investigación, adicionalmente, tendremos que respetar las pautas del método científico. 

%La Ingeniería Informática, al igual que otras ingenierías, trasciende las fronteras clásicas de la ingeniería. E, es decir, a la aplicación de conocimientos teórico-prácticos para ofrecer la mejor solución posible a un problema con unas restricciones temporales, personales, materiales y económicas. La Ingeniería Informática actual cuenta con desafíos científicos de gran envergadura en cada una de sus especialidades, destacándose a continuación algunos de ellos: los retos electrónicos de aumentar el nivel de integración de las placas de unidades de procesamiento, ya sean de datos (CPU) o especializadas en gráficos (GPU), fundamentales para continuar ampliando la capacidad de cómputo de los sistemas informáticos; el amplio, ilusionante y complejo reto computacional y social que representa la inteligencia artificial; el continuo empeño en mejorar todo lo relativo al procesamiento geométrico para impulsar el avance de la representación gráfica; el objetivo de conseguir una informática sostenible a través de la mejora de la aplicación de los conceptos de la informática teórica para el diseño y desarrollo de algoritmos eficientes en tiempo y en espacio; la responsabilidad de mejorar las metodologías y métodos de desarrollo para la programación de sistemas informáticos seguros, robustos y respetuosos con la privacidad de los datos; o el desafío de mejorar las metodologías de trabajo con el fin de que la Informática continúe avanzando con una verdadera ingeniería. Sin embargo, la aplicación trasversal de la informática, hace que la Ingeniería Informática esté presente en el trabajo científico diario de un amplio espectro de disciplinas, sobresaliendo la investigación en medicina, química o biología. Los retos científicos propios y ajenos a los que se tiene que enfrentar la Ingeniería Informática, obliga al graduado en Ingeniería Informática a disponer de unas mínimas habilidades científicas, que le permitan aplicar los principios de la Ingeniería Informática al método científico, y contribuir de esta forma al progreso científico de cualquier disciplina.

La tipología de {\it TFG de Experimentación Científica} representa la oportunidad de aplicar los conocimientos adquiridos durante el grado a la resolución de un problema científico en el que interviene una metodología informática o sistema computacional, y por tanto demostrar y ampliar el desarrollo de las habilidades científicas del futuro graduado en Ingeniería Informática.
%La tipología de TFG de experimentación científica representa la oportunidad de aplicar los conocimientos adquiridos durante el grado a la resolución de un problema científico en el que interviene una metodología informática o un sistema computacional, y por tanto de mostrar y ampliar el desarrollo de las habilidades científicas del futuro graduado en Ingeniería Informática.

Aunque pudiera parecer que en este tipo de TFG sólo se deben aplicar recomendaciones o prácticas propias de la actividad científica, no se debe olvidar que este TFG es indispensable para la obtención de un título de graduado en ingeniería. Por tanto, además de aplicar procedimientos científicos, se deben seguir y emplear principios de ingeniería, y sobre todo los propios de la Ingeniería Informática estudiados durante la carrera.
%Aunque pudiera parecer que en este tipo de TFG solo se deben aplicar recomendaciones o prácticas propias de la actividad científica, no se debe olvidar que este TFG es indispensable para la obtención de un título de graduado en ingeniería. Por tanto, además de aplicar procedimientos científicos, se deben seguir y emplear principios de ingeniería, y sobre todo los propios de la Ingeniería Informática estudiados durante la carrera.

Las recomendaciones sobre el tipo de TFG de Experimentación Científica se expondrán de la siguiente manera: primero se presentará el \textit{método científico}, es decir, elemento diferenciador de este tipo de TFG y que debe guiar su elaboración (véase la sección \ref{tfx_inv_s_met_cientifico}); posteriormente se expondrá cómo estructurar el trabajo y la memoria del TFG buscando siempre una correspondencia con el método científico (Sección  \ref{tfx_inv_s_est_trabajo_memoria}) y por último se enunciarán una serie de recomendaciones (Sección \ref{tfx_inves_ss_recomendaiones}).
%Las recomendaciones sobre el tipo de  TFG de experimentación científica se expondrán de la siguiente manera: primero se presentará el \textit{método científico}, es decir, elemento diferenciador de este tipo de TFG y que debe guiar su elaboración (v. sección \ref{tfx_inv_s_met_cientifico}); posteriormente se expondrá cómo estructurar el trabajo y la memoria del TFG buscando siempre una correspondencia con el método científico (v. sección \ref{}) y por último se enunciarán una serie de recomendaciones (v. sección \ref{}).}

\subsection{El Método Científico}\label{tfx_inv_s_met_cientifico}

La investigación es un actividad intelectual y experimental realizada de modo sistemático con el propósito de aumentar los conocimientos sobre una determinada materia\footnote{Real Academia Española - \url{http://rae.es}}. En otras palabras, la investigación busca adquirir nuevos conocimientos y/o resolver problemas teóricos o prácticos mediante una actividad metódica y \underline {reproducible}. La consecución de los objetivos de la investigación precisa de la aplicación de una determinada estrategia de trabajo, que en este caso se conoce como \textbf{método científico}. Este está constituido por una serie de actividades, que aunque se resumen en la Figura \ref{fg_met_cientifico}, se detallan a continuación:

\begin{figure}[!t]
    \centering
    \usetikzlibrary {shapes.geometric}
    % Define block styles
    \tikzstyle{decision} = [diamond, draw, fill=black!10, text width=4.5em, text badly centered, node distance=3cm, inner sep=0pt]
    \tikzstyle{block} = [rectangle, draw, fill=blue!10, minimum width=7em, text centered, rounded corners, minimum height=4em]
    \tikzstyle{arrow}=[draw, -latex]
    %\tikzstyle{cloud} = [draw, ellipse,fill=red!20, node distance=3cm, minimum height=2em]
    
    \begin{tikzpicture}[node distance = 2cm, auto]
        % Place nodes
        \node [block, align=center] (A) {Hacer una pregunta};
        \node [block, below of=A, align=center] (B) {Estudiar el estado\\del arte};
        \node [block, below of=B, align=center] (C) {Construir una hipótesis};
        \node [block, below of=C, align=center] (D) {Evaluar la hipótesis};
        \node [block, below of=D, align=center] (E) {Analizar los resultados};
        \node [block, right of=C, align=center, xshift=6em] (F) {Replantear\\hipótesis};
        \node [decision, below of=E, align=center, yshift=1.8em] (G) {¿Hipótesis cierta?};
        \node [block, below of=G, align=center, yshift=-1em] (J) {Concluir};
        % Draw edges
        \path [arrow] (A) -> (B);
        \path [arrow] (B) -> (C);
        \path [arrow] (C) -> (D);
        \path [arrow] (D) -> (E);
        \path [arrow] (F) -> (C);
        \path [arrow] (E) -> (G);
        \path [arrow] (G) -> node [text width=2.5cm, midway, right] {Sí} (J);
        \path [arrow] (G) -| node [text width=2.5cm, midway, above] {No} (F);
    \end{tikzpicture}

    %\includegraphics[width=.8\textwidth]{images/Método_científico_2021.jpg}
    \caption{Diagrama que representa el método científico.}\label{fg_met_cientifico}% \textcolor{red}{DEBE GENERARSE CON TIKZ. ESTO ES UN BORRADOR. Fuente: \url{https://es.wikipedia.org/wiki/M\%C3\%A9todo_cient\%C3\%ADfico\#/media/Archivo:M\%C3\%A9todo_cient\%C3\%ADfico_2021.jpg}}
\end{figure}

\begin{itemize}
    \item \textbf{Hacer una pregunta}. Todo trabajo científico comienza con la identificación de un problema, de una necesidad, de una carencia, en definitiva, de un ¿a qué se debe esto? o ¿cómo puedo resolver esto?, o lo que anglosajones denominan \textit{gap}. Si no existe esta pregunta, problema o necesidad, deja de tener todo sentido el invertir tiempo y esfuerzo, y por ende dinero, en la realización de una experimentación cuyo fin no está asociado a ofrecer una respuesta útil.
    %\textbf{Hacer una pregunta}. Todo trabajo científico comienza con la identificación de un problema, de una necesidad, de una carencia, en definitiva, de un ¿a qué se debe esto? o ¿cómo puedo resolver esto?, o lo que anglosajones denominan \textit{gap}. Si no existe esta pregunta, problema o necesidad, deja de tener todo sentido el invertir tiempo y esfuerzo, y por ende dinero, en la realización de una experimentación cuyo fin no está asociado a ofrecer una respuesta útil.

    \item \textbf{Estudiar el estado del arte}. La identificación de una necesidad o el surgimiento de una pregunta científica, no implica que ésta sea novedosa o que no tenga ya una respuesta. Por consiguiente, las acciones que intervienen en esta fase son: \begin{enumerate*}[label=(\arabic*)] \item identificar la disciplina científica donde se encuadra la pregunta a responder; \item analizar la novedad de la pregunta identificada, así como si ha sido ya respondida; y \item en caso de que el reto sí sea novedoso, estudiar la investigación relacionada con el ánimo de aprender los métodos ya usados, y emplearlos como inspiración en el diseño de la metodología o método original que de respuesta al desafío en estudio.\end{enumerate*}
    %\textbf{Estudiar el estado del arte}. La identificación de una necesidad o el alumbramiento de una pregunta científica, no implica que esta sea novedosa o que no tenga ya una respuesta. Por consiguiente, las acciones que intervienen en esta fase son: \begin{enumerate*}[label=(\arabic*)] \item identificar la disciplina científica donde se encuadra la pregunta a responder; \item analizar la novedad de la pregunta identificada, así como si ha sido ya respondida; y \item en caso de que el reto sí sea novedoso, estudiar la investigación relacionada con el ánimo de aprender los métodos ya usados, y emplearlos como inspiración en el diseño de la metodología o método original que de respuesta al desafío en estudio.\end{enumerate*}

    \item \textbf{Construir una hipótesis}. Una vez estudiado el estado del arte relacionado con el problema, se debe definir una hipótesis, sobre la cual se diseñará la solución y se evaluará si se confirma, y por tanto resolverá el reto, o si se desecha, y en consecuencia tener que definir otra hipótesis.
    %\textbf{Construir una hipótesis}. Una vez estudiado el estado del arte relacionado con el problema, se debe definir una hipótesis, sobre la cual se diseñará la solución y se evaluará si se confirma, y por tanto resolverá el reto, o si se desecha, y en consecuencia tener que definir otra hipótesis.

    \item \textbf{Evaluar la hipótesis}. Esta fase se compone de las siguientes acciones: \begin{enumerate*}[label=(\arabic*)]\item diseñar y desarrollar una metodología o método acorde a la hipótesis; \item diseñar un conjunto de experimentos que permitan evaluar la metodología o método desarrollado; y \item evaluar con métricas de evaluación estándares y propias del área donde se circunscribe el problema de investigación con el fin de ofrecer una evaluación objetiva y comparable. Así mismo, se recomienda ofrecer todos los detalles de desarrollo y de evaluación, con el ánimo de que la experimentación sea reproducible por cualquier investigador. Este recomendación contribuye a confiar en la experimentación realizada, en el método o metodología propuesta, a la transferencia de conocimiento y al progreso científico.\end{enumerate*}
    %\textbf{Evaluar la hipótesis}. Esta fase se compone de las siguientes acciones: \begin{enumerate*}[label=(\arabic*)]\item diseñar y desarrollar una metodología o método acorde a la hipótesis; \item diseñar un conjunto de experimentos que permitan evaluar la metodología o método desarrollado; y \item evaluar con métricas de evaluación estándares y propias del área donde se circunscribe el problema de investigación con el fin de ofrecer una evaluación objetiva y comparable. Así mismo, se recomienda ofrecer todos los detalles de desarrollo y de evaluación, con el ánimo de que la experimentación sea reproducible por cualquier investigador. Este recomendación contribuye a confiar en la experimentación realizada, en el método o metodología propuesta, a la transferencia de conocimiento y al progreso científico.\end{enumerate*}

    \item \textbf{Analizar los resultados}. Esta fase es la más determinante, porque en función de su resultado, se podrá decir que se acepta la hipótesis, y por tanto se concluye que el método o metodología desarrollados resuelven el desafío objeto de estudio, o por el contrario se rechaza la hipótesis. Este rechazo implica que se debe volver a la definición de la hipótesis, o dicho de otro modo, a modificar la hipótesis y desarrollar otro método o metodología acorde a la nueva hipótesis.
    %\textbf{Analizar los resultados}. Esta fase es la más determinante, porque en función de su resultado, se podrá decir qu se acepta la hipótesis, y por tanto se concluye que el método o metodología desarrollados resuelven el desafío objeto de estudio, o por el contrario se rechaza la hipótesis. Este rechazo implica que se debe volver a la definición de la hipótesis,  o dicho de otro modo, a modificar la hipótesis y desarrollar otro método o metodología acorde a la nueva hipótesis.
\end{itemize}

Al igual que el método científico guía la labor de los investigadores de cualquier área, también debe ser la base de un TFG de Experimentación Científica.

% ------------------------------------------------------------

%Definición de investigación.
%La investigación es un actividad intelectual y experimental realizada de modo sistemático con el propósito de aumentar los conocimientos sobre una determinada materia\footnote{Real Academia Española - \url{http://rae.es}}. En otras palabras, la investigación busca adquirir nuevos conocimientos y/o resolver problemas teóricos o prácticos mediante una actividad metódica y \underline {reproducible}. Para poder llevar a cabo una investigación exitosa es necesario tener en cuenta los siguientes elementos: 
%\begin{itemize}
%    \item Recopilar toda la información relevante sobre el problema a tratar utilizando fuentes heterogéneas y fiables.
%    \item Indagar sobre otros estudios ya realizados y publicados en la literatura científica sobre la misma problemática, analizando sus beneficios y sus limitaciones.
%    \item Seguir una metodología científica para poder así desarrollar el trabajo de manera organizada y coherente.
%    \item Mostrar los resultados obtenidos y valorarlos de forma objetiva sin omisiones.
%    \item Asegurar la reproducibilidad del trabajo, permitiendo que el tribunal o cualquier otra persona interesada sea capaz de verificarlo y replicarlo.
%\end{itemize}

%Conocimiento previo sobre el estado de la cuestión.
%Estructura del marco teórico y conceptual.
%De esta lista de elementos haremos especial hincapié en proveer de un marco teórico y conceptual que permita al tribunal, y a cualquier otro potencial lector, de la información básica necesaria para comprender la investigación realizada. De todas las tipologías de TFG de esta sección, la tipología de TFG de investigación es, probablemente, la que lleve más páginas dedicadas al estado del arte. Por tanto, también la bibliografía utilizada será más extensa. Debe de quedar claro que comprendes la temática y conoces qué cosas se han hecho ya y qué falta por hacer.

%Importancia de partir de una pregunta de partida.
%Otro punto importante a destacar en la memoria de los TFGs de investigación es la motivación del trabajo. No hay que confundir el apartado de motivación con una motivación personal. La motivación del trabajo, en este tipo de tipología de TFG, se refiere a identificar una brecha en el conocimiento actual del tema a tratar y de cómo la investigación en ese punto puede ayudar a contribuir en al área de estudio. Por tanto la motivación del trabajo debe incluirse en la memoria luego del estudio del estado del arte, ya que es muy posible que debas hacer referencia a algún punto ya tratado.

%Una vez detallado el problema de estudio y la motivación del TFG, es momento de hablar de los objetivos del trabajo. Para poder definir correctamente estos objetivos es necesario primero definir las hipótesis de partida, ya que la verificación o no de estas hipótesis serán justamente los objetivos principales del TFG. En el proceso de formulación de hipótesis debes plantear posibles respuestas a las preguntas surgidas durante tu análisis del estado del arte. Recuerda que las hipótesis deben poder ser verificables y falsables, es decir, susceptibles de ser refutadas.
   
En el siguiente apartado veremos en más detalle como llevar a cabo un TFG de Experimentación Científica siguiendo una serie de pasos que nos aseguren que los resultados obtenidos sean de fiables y de calidad.

\subsection{Estructura del Trabajo y de la Memoria}\label{tfx_inv_s_est_trabajo_memoria}

La estructura de la memoria no debe por qué distar mucho del resto de tipologías, pero sí debe recoger de forma adecuada el desarrollo científico realizado. Esto obliga a que el trabajo científico siga unas pautas coherentes que dirijan, al menos, la labor experimental de una forma ordenada y dirigida a la evaluación con éxito de la hipótesis. El método científico, presentado en la sección \ref{tfx_inv_s_met_cientifico}, es una guía muy recomendable para orientar todo la actividad experimental y la redacción de la memoria. Por ende, la memoria debe reflejar: \begin{enumerate*}[label=(\arabic*)]\item la existencia de una necesidad de investigación, \item que esta no haya sido aún resuelta, o por lo menos no completamente, \item la definición de una hipótesis, \item el análisis, diseño e implementación del método o metodología que lleve a cabo la hipótesis y \item la evaluación de esta.\end{enumerate*} A continuación, se presentará una correspondencia entre el método científico y la estructura de la memoria, que a su vez ordena todo el trabajo relacionado con esta tipología de TFG.
%La estructura de la memoria no debe por qué distar mucho del resto de tipologías, pero sí debe recoger de forma adecuada el desarrollo científico realizado. Esto obliga a que el trabajo científico siga unas pautas coherentes que dirijan, al menos, la labor experimental de una forma ordenada y dirigida a la evaluación con éxito de la hipótesis. El método científico, presentado en la sección \ref{tfx_inv_s_met_cientifico}, es una guía muy recomendable para orientar todo la actividad experimental y la redacción de la memoria. Por ende, la memoria debe reflejar: \begin{enumerate*}[label=(\arabic*)]\item la existencia de una necesidad de investigación, \item que esta no haya sido aún resuelta, o por lo menos no completamente, \item la definición de una hipótesis, \item el análisis, diseño e implementación del método o metodología que lleve a cabo la hipótesis y \item la evaluación de esta.\end{enumerate*} A continuación, se presentará una correspondencia entre el método científico y la estructura de la memoria, que a su vez ordena todo el trabajo relacionado con esta tipología de TFG.

\paragraph{Motivación - Pregunta de investigación} El primer capítulo o el capítulo de introducción del TFG debe recoger la motivación del trabajo, es decir, la razón por la cual se realiza y merece la pena esforzarse en él durante los meses que involucre su realización. Esto que, en un inicio parece complicado, consiste en encontrar un reto que tenga asociado una pregunta científica, como podría ser \textit{¿es posible la generación de texto coherente y precisa para ofrecer información sobre trastornos de la conducta alimenticia?} Si se ha llegado a esa pregunta, es que existe una necesidad ligada a una problema, que a su vez tiene un contexto. Esta conjunción de contexto, problema, necesidad y pregunta científica es lo que constituye la motivación del TFG, y la cual lo convierte en atractivo para cualquier lector, y sobre todo para la comunidad científica.
%\paragraph{Motivación - Pregunta de investigación} El primer capítulo o capítulo de introducción del TFG debe recoger la motivación del trabajo, es decir, la razón por la cual se realiza, merece la pena esforzarse en él durante los meses que involucre su realización, es interesante para cualquier lector, y representa un avance científico. Esto que parece complicado consiste en encontrar un reto que tenga asociado una pregunta científica, como podría ser \textit{¿es posible la generación de texto coherente y precisa para ofrecer información sobre trastornos de la conducta alimenticia?} Si se ha llegado a esa pregunta, es que existe una necesidad ligada a una problema, que a su vez tiene un contexto. Esta conjunción de contexto, problema, necesidad y pregunta científica es lo que constituye la motivación del TFG, y la cual lo convierte en atractivo para cualquier lector, y sobre todo para la comunidad científica.

\paragraph{Investigación relacionada - Estudiar el estado del arte} Una vez que ya se ha fijado el problema o pregunta a resolver, se deben realizar las siguientes acciones:

\begin{enumerate}
    \item \textbf{Determinar el área de investigación}. La Ingeniería Informática abarca distintas áreas científicas, por lo que lo primero es saber si la pregunta se responde desde, por ejemplo, la informática teórica, la informática gráfica, la inteligencia artificial, la electrónica, o incluso si precisa de conocimiento externo a la Ingeniería Informática, como podría ser una investigación en un dominio interdisciplinar (medicina, biología, química, etc.). Atendiendo a la pregunta anterior, si se está cuestionando sobre la posibilidad de generar automáticamente lenguaje, parece que se trataría de un problema de inteligencia artificial, dado que se pide imitar un comportamiento característico de la inteligencia humana. En este ejemplo, una vez fijada que el área es la de la inteligencia artificial, el siguiente paso sería identificar la disciplina especializada en el desafío en cuestión, que en este ejemplo sería el procesamiento del lenguaje natural, ya que es la disciplina de inteligencia artificial encargada del desarrollo de métodos y metodologías computacionales orientados a la comprensión y generación de lenguaje por parte de un ordenador.
    %\textbf{Determinar el área de investigación}. La Ingeniería Informática abarca distintas áreas científicas, por lo que lo primero es saber si la pregunta se responde desde por ejemplo la informática teórica, la informática gráfica, la inteligencia artificial, la electrónica, o incluso si precisa de conocimiento externo a la Ingeniería Informática, como podría ser una investigación en el dominio de la biología. Atendiendo a le pregunta anterior, si se está cuestionando sobre la posibilidad de generar automáticamente lenguaje, parece que se trataría de un problema de inteligencia artificial, dado que se pide imitar un comportamiento característico de la inteligencia humana como es la generación de lenguaje. Una vez fijada que el área es la de la inteligencia artificial, el siguiente paso sería identificar la disciplina especializada en el desafío en cuestión, que en este caso sería el procesamiento del lenguaje natural, ya que es la disciplina de inteligencia artificial encargada del desarrollo de métodos y metodologías computacionales orientados a la comprensión y generación de lenguaje por parte de un ordenador.

    \item \textbf{Búsqueda de los trabajos más relevantes}. Conocida el área y disciplina de investigación, es momento de encontrar los trabajos relacionados con la pregunta de investigación. En el caso particular que estamos tratando, serían modelos de generación de lenguaje. Con este dato, ya es momento de estudiar todo lo relacionado con la problemática, y seleccionar los avances más recientes. En este caso se corresponderían con los grandes modelos de lenguaje.
    %\textbf{Búsqueda de los trabajos más relevantes}. Conocida el área y disciplina de investigación, es momento de encontrar los trabajos relacionados con la pregunta de investigación. En el caso particular que estamos tratando, serían modelos de generación de lenguaje. Con este dato, ya es momento de estudiar todo lo relacionado con la problemática, y seleccionar los avances más recientes. En este caso se corresponderían con los grandes modelos de lenguaje.

    \item \textbf{Evaluación de si el reto está resuelto}. Tras la selección de los artículos más importantes es momento de evaluar si el desafío planteado está o no resuelto. En caso de que así sea, se deberá revaluar la pregunta de investigación, hasta encontrar aquella que aún no lo esté. Por contra, si no lo está, la pregunta representa un verdadero reto sobre el que merece la pena realizar el TFG.
    %\textbf{Evaluación de si el reto está resuelto}. Tras la selección de los artículos más importantes es momento de evaluar si el desafío planteado está resuelto. En caso de que así sea, se deberá revaluar la pregunta de investigación, hasta encontrar aquella que aún no lo esté. Por contra, si no lo está, la pregunta representa un verdadero reto sobre el que merece la pena trabajar.
    
    Pudiera parecer que el trabajo de esta fase termina aquí, pero no es así, porque ahora es el momento de determinar las estrategias seguidas hasta la fecha, con el fin de realizar una propuesta novedosa que pueda compararse con lo existente, y que ofrezca mejores resultados.
    %Pudiera parecer que el trabajo de esta fase termina aquí, pero no es así, porque ahora es momento de determinar las estrategias seguidas hasta la fecha con el fin de realizar una propuesta novedosa que pueda compararse con lo existente, y que ofrezca unos mejores resultados.
\end{enumerate}

El resultado de este estudio deberá reflejarse en un capítulo específico sobre el estado de la investigación asociada al proyecto. Algunos ejemplos de nombres que puede tener este capítulo son: contexto, estado de arte, estado de la cuestión, trabajos relacionados o el nombre específico de la disciplina de investigación asociada al desafío científico del TFG.
%El resultado de este estudio deberá reflejarse en un capítulo específico sobre el estado de la investigación asociada al proyecto. Algunos ejemplos de nombres que puede tener este capítulo son: contexto, estado de arte, estado de la cuestión, trabajos relacionados o el nombre específico de la disciplina de investigación asociada al desafío científico del TFG.

\paragraph{Hipótesis} Un conocimiento profundo del problema científico asociado prepara para la enunciación de la hipótesis del TFG. Esta es una suposición de estrategia, metodología o modelo a desarrollar para resolver el desafío científico en el que se está trabajando. Así mismo, si la evaluación de la hipótesis la confirma, conlleva haber propuesto una solución válida al reto científico del TFG. Automáticamente se podría pensar que un resultado negativo implica el fracaso del trabajo científico desarrollado. Esto no tiene por qué ser así, dado que de los resultados negativos también se pueden obtener conclusiones positivas, que incluso pueden servir de base para posteriores avances científicos.
%\paragraph{Hipótesis} Un conocimiento profundo del problema científico asociado prepara para la enunciación de la hipótesis del TFG. Esta es una suposición de estrategia, metodología o modelo a desarrollar para resolver el desafío científico en el que se está trabajando. Así mismo, si la evaluación de la hipótesis la confirma, conlleva haber propuesto una solución válida al reto científico del TFG. Automáticamente se podría pensar que un resultado negativo implica el fracaso del trabajo científico desarrollado. Esto no tiene por qué ser así, dado que de los resultados negativos también se pueden obtener conclusiones muy positivas, que incluso pueden servir de base para posteriores avances científicos.

La hipótesis se recomienda que también se indique en el capítulo de introducción, ya que va asociada a la pregunta de investigación. Así mismo, una vez formulada la hipótesis, ya se puede establecer los objetivos e hitos del TFG, los cuales son una forma de concretar el trabajo a realizar.
%La hipótesis se recomienda que también se indique en el capítulo de introducción, porque va asociada a la pregunta de investigación. Así mismo, una vez formulada la hipótesis, ya se puede establecer los objetivos e hitos del TFG, los cuales son una forma de concretizar el trabajo a realizar.

\paragraph{Desarrollo - Evaluación de la hipótesis} Para poder probar la validez de la hipótesis, antes hay que desarrollar la metodología o método subyacente a la misma. Por tanto, se debe comenzar primeramente con el diseño del marco experimental. En la mayoría de los casos el marco experimental va a seguir los cánones de la investigación cuantitativa, la cual requiere de una evaluación empírica. Esto obliga a seleccionar el conjunto o conjuntos de datos sobre los que realizar la evaluación, y a determinar las medidas de evaluación a utilizar
%\paragraph{Desarrollo - Evaluación de la hipótesis} Para poder probar la validez de la hipótesis, antes hay que desarrollar la metodología o método subyacente a la misma. Por tanto, se debe comenzar primeramente con el diseño del marco experimental. En la mayoría de los casos, el marco experimental va a seguir los cánones de la investigación cuantitativa, la cual requiere de una evaluación empírica. Esto obliga a seleccionar el conjunto o conjuntos de datos sobre los que realizar la evaluación, y a determinar las medidas de evaluación a usar.

La elección del conjunto de datos es una decisión muy importante del proceso de evaluación, porque si estos no son de calidad, o no son representativos, la calidad y credibilidad del marco experimental se verá resentido. En cualquier caso, siempre hay que describir los datos elegidos, y en caso de que no sean propios hacer referencia a su fuente. A continuación te mostramos algunos ejemplos de conjuntos de datos que se pueden usar:
%La elección del conjunto de datos es una decisión muy importante del proceso de evaluación, porque si estos no son de calidad o no son representativos, la calidad y credibilidad del marco experimental se verá resentido. En cualquier caso, siempre hay que describir los datos elegidos, y en caso de que no sean propios hacer referencia a su fuente. A continuación algunos ejemplos de conjuntos de datos que se pueden usar:

\begin{itemize}
    \item {\bf Conjuntos de datos externos}: Si trabajas con conjuntos de datos públicos y ampliamente conocidos por la comunidad científica es sencillo hacer referencia a ellos en la memoria. Se recomienda utilizar una tabla para mostrar comparativamente los distintos conjuntos de datos empleados, como en el ejemplo de la Tabla \ref{tab:uci}. Puedes incorporar a la tabla las columnas que sean necesarios y que incluyan toda la información sobre la cual haces referencia en el texto principal de la memoria. Presta atención a el texto que acompaña a la tabla en donde se indica tanto el origen de los datos como una cita o referencia on-line a ellos. 

\begin{table}[!ht]
\centering
\begin{tabular}{cccccc}
\hline
 & \textbf{No. de} & \textbf{No. de} & \multicolumn{3}{c}{\textbf{No. de Atributos}} \\
{\it \textbf{Dataset}} & \textbf{Casos} & \textbf{Clases} & \textbf{Nominal} & \textbf{Numérico} & \textbf{Perdidos}\\ \hline
Anneal & 898 & 6 & 32 & 6 & 29 \\
Breast Cancer & 286 & 2 & 9 & 0 & 2 \\
Diabetes & 768 & 2 & 8 & 0 & 8 \\
Heart-C & 303 & 5 & 7 & 6 & 2 \\
Hepatitis & 155 & 2 & 13 & 6 & 15 \\
House Votes & 435 & 2 & 16 & 0 & 16 \\
Iris & 150 & 3 & 0 & 4 & 0 \\
Lymphography & 148 & 4 & 15 & 3 & 0 \\
Vowel & 990 & 11 & 3 & 10 & 0 \\
Wine & 178 & 3 & 0 & 13 & 0 \\
Zoo & 101 & 7 & 16 & 1 & 0 \\
\hline
\end{tabular}
\caption{Conjuntos de datos utilizados en este trabajo provenientes del repositorio UCI (\url{https://archive.ics.uci.edu/}).\label{tab:uci}}
\end{table}

    \item {\bf Conjuntos de datos propios}: si dentro de tu propuesta de trabajo incluyes la generación de conjuntos de datos artificiales propios, entonces necesitas crear un apartado en la memoria para detallar todo el proceso. Si por otro lado, el conjunto de datos te lo ha dado tu tutor, entonces necesitas detallar su origen y características. Si es posible incluye el o los conjuntos de datos al entregar la memoria. Si por alguna razón los datos son privados o no tienes permiso para difundirlos debes entonces explicarlo detalladamente. Recuerda que para trabajar con datos con información de carácter personal, antes debes anonimizarlos. 
    \item {\bf Conjuntos de datos artificiales}: si cuentas con conjuntos de datos creados artificialmente, también llamados datos sintéticos, debes explicar cómo se han generado. Si el procedimiento de generación de estos datos es propuesta tuya, entonces descríbelo con detalle como se explica en el apartado anterior. En caso de ser generados por terceras partes entonces alcanza con referenciar ese trabajo y explicar brevemente cómo se realiza este proceso. Igualmente para ambos casos necesitas mostrar en una tabla una descripción de los conjuntos de datos, como puedes ver en la Tabla \ref{tab:fake}. Recuerda que tu trabajo debe de poder ser reproducible por cualquier persona, por tanto debes incluir en el material a entregar los conjuntos de datos que hayas generado tú o un enlace a aquellos generados por terceras personas.

\begin{table}[!ht]
\centering
\label{tab:results}
\begin{tabular}{c|ccc}
\hline
{\it \textbf{Dataset}} & \textbf{Edificio} & \textbf{Árbol} & \textbf{Persona} \\
\hline
Sintético A & 5 & 10 & 4 \\
Sintético B & 0 & 12 & 11 \\
Sintético C & 1 & 2 & 2 \\
Sintético D & 3 & 2 & 3 \\
Sintético E & 4 & 5 & 1 \\
\hline
\end{tabular}
\caption{Elementos incluidos en los distintos conjuntos de datos generados artificialmente.\label{tab:fake}}
\end{table}
\end{itemize}

El siguiente paso es \textbf{diseñar e implementar la metodología o modelo}. Se podría decir que esta parte del TFG es para la que está más preparado el estudiante, por ser la de mayor contenido técnico. El diseño y desarrollo deben seguir los principios de ingeniería aprendidos durante la carrera, siendo recomendable que el alumno tome decisiones en el ámbito de elección de lenguaje de programación, de bibliotecas a usar y de entorno de programación a utilizar.
%El siguiente paso es \textbf{diseñar e implementar la metodología o modelo}. Se podría decir que esta parte del TFG es para la que está más preparado el alumno, por ser la de mayor contenido técnico. El diseño y desarrollo deben seguir los principios de ingeniería aprendidos durante la carrera, siendo recomendable que el alumno tome decisiones en el ámbito de elección de lenguaje de programación, de librerías a usar y de entorno de programación a utilizar.

Antes de proseguir, nos gustaría hacerte la recomendación de que este tipo de TFG estuvieran acompañados de un demostrador, es decir, de integrar la propuesta científica en un sistema informático que muestre su utilidad. Si se opta por ello, el desarrollo del demostrador debe seguir las recomendaciones de la ingeniería del software, lo que implica: \begin{enumerate*}[label=(\arabic*)]\item elección de metodología de trabajo, \item realización de fase de análisis, \item diseñar la solución, \item implementar el diseño y \item validar y evaluar el software desarrollado.\end{enumerate*} Es muy recomendable la realización de un demostrador, porque permite que el alumno muestre los conocimientos adquiridos durante la carrera, sobre todo los relativos a los procesos y principios de la ingeniería del software.
%Ante de proseguir, quisiera hacerse la recomendación de que este tipo de TFG estuvieran acompañados de un demostrador, es decir, de integrar la propuesta científica en un sistema informático que muestre su utilidad. Si se opta por ello, el desarrollo del demostrador debe seguir las recomendaciones de la ingeniería del software, lo que implica: \begin{enumerate*}[label=(\arabic*)]\item elección de metodología de trabajo, \item realización de fase de análisis, \item diseñar la solución, \item implementar el diseño y \item validar y evaluar el software desarrollado.\end{enumerate*} Es muy recomendable la realización de un demostrador, porque permite que el alumno muestre los conocimientos adquiridos durante la carrera, sobre todo los relativos a los procesos y principios de la ingeniería del software.

Una vez concluida la implementación de la metodología o modelo pasamos a la fase de evaluación. Ésta debe realizarse con las medidas de evaluación usadas en los trabajos de investigación relacionados. Esto es relevante porque permite comparar la propuesta del TFG con la literatura, y así poder saber si la propuesta representa realmente un avance o no. Además, se recomienda facilitar la comparación con el estado del arte por medio de una tabla comparativa con los modelos más relevantes. No olvidar que para poder decir, por ejemplo, que un algoritmo es mejor que otro, es necesario no solo realizar múltiples experimentos, sino también someter los resultados obtenidos a análisis estadísticos. Recuerda que los experimentos que realizas deben de poder reproducirse, por tanto no olvides fijar una semilla para que, en caso de usar un generador de números aleatorios, tu trabajo sea reproducible.
%Una vez concluida la implementación de la metodología o modelo se debe evaluar. La evaluación debe realizarse con las medidas de evaluación usadas en los trabajos de investigación relacionados. Esto es relevante porque permite comparar la propuesta del TFG con la literatura, y así poder saber si la propuesta representa realmente un avance o no. Además, se recomienda facilitar la comparación con el estado del arte por medio de una tabla comparativa con los modelos más relevantes.

En cuanto a la organización de la evaluación de la hipótesis en varias secciones o capítulos, dependerá de su extensión y de si se realiza un sistema demostrador. En el caso de que se haga un sistema informático, se recomienda separar en capítulos las distintas fases del proceso de desarrollo de software. En caso contrario, puede que solo sea necesario elaborar un capítulo describiendo todo lo relativo a la elección de datos, implementación y evaluación.
%En cuanto a la organización de la evaluación de la hipótesis en capítulos, todo va a depender de su extensión y de si se realiza un sistema demostrador. En el caso de que se haga un sistema informático, se recomienda separar en capítulos las distintas fases del proceso de desarrollo de software. En caso contrario, puede que solo sea necesario elaborar un capítulo describiendo todo lo relativo a la elección de datos, implementación y evaluación.

\paragraph{Análisis de los resultados.} El análisis de los resultados y la comparación con el estado del arte es el momento decisivo de esta tipología de TFG, debido que determina si la hipótesis se acepta o no. En el caso de que se acepte, el trabajo no concluye aquí, ya que se recomienda realizar un análisis tanto de los aciertos, como de los errores que se hayan producido, dado que estos permitirán realmente entender cómo el modelo o metodología desarrollados están dando respuesta al reto de investigación.
%\paragraph{Análisis de los resultados} El análisis de los resultados y la comparación con el estado del arte es el momento decisivo de esta tipología de TFG, debido que determina si la hipótesis se acepta o no. En el caso de que se acepte, el trabajo no concluye aquí, ya que se recomienda realizar un análisis tanto de los aciertos, como de los errores que se hayan producido, dado que estos permitirán realmente entender cómo el modelo o metodología desarrollados están dando respuesta al reto de investigación.

En caso de que la hipótesis sea rechazada, se debe evaluar si un análisis profundo de los resultados puede ser interesante para la comunidad investigadora, y por tanto constituir una contribución importante para el TFG. Este análisis puede concluir que se deben elegir otros datos, que el reto ha llegado a un escenario de resultados tope, que la estrategia algorítmica no es la adecuada y que se deben seguir otras, o que la elección de las características y los parámetros de configuración del algoritmo no han sido los óptimos. Esto que pudiera parecer una fracaso, puede ser la base de futuros progresos porque ofrece a la comunidad científica un conjunto de lecciones aprendidas a tener en cuenta para posteriores desarrollos. Ahora bien, se remarca que cuando se ha obtenido un resultado negativo, el análisis de resultados debe ser profundo y con conclusiones interesantes para considerarlo como una contribución y como un TFG válido.
%En caso de que la hipótesis sea rechazada, se debe evaluar si un análisis profundo de los resultados puede ser interesante para la comunidad investigadora, y por tanto constituir una contribución importante para el TFG. Este análisis puede concluir que se deben elegir otros datos, que el reto ha llegado a un escenario de resultados tope, que la estrategia algorítmica no es la adecuada y que se deben seguir otras o que la elección de las características y los parámetros de configuración del algoritmo no han sido los óptimos. Esto que pudiera parecer una fracaso, puede ser la base de futuros progresos porque ofrece a la comunidad científica un conjunto de lecciones a tener en cuenta para posteriores desarrollos. Ahora bien, se remarca que cuando se ha obtenido un resultado negativo, el análisis de resultados debe ser profuso y con conclusiones interesantes para considerarlo como una contribución y como TFG válido.

También es posible que los experimentos realizados tengan ciertas limitaciones que afecten a la interpretación de los resultados obtenidos. Debes, por tanto, abordar este tema de manera transparente. Las limitaciones son características del diseño o metodología que influyen en la validez, aplicabilidad o generalización de los hallazgos. Pueden derivarse del diseño inicial del estudio, del método utilizado o de desafíos imprevistos durante el proyecto. Existen distintos tipos de limitaciones, como las relacionadas con el diseño o la metodología del estudio, o bien de los datos en sí, que pueden presentar problemas de calidad, disponibilidad o fiabilidad. Reconocer estas limitaciones te da la oportunidad para proponer formas de abordar estas limitaciones y demuestra un pensamiento crítico.

\paragraph{Conclusiones} El último capítulo debe ser el de conclusiones, que no debe ser una mera recapitulación del TFG, sino un lugar donde destacar si la propuesta permite aceptar la hipótesis, y los principales resultados del análisis a modo de lecciones aprendidas. En resumen, unas conclusiones que aporten valor al TFG. Adicionalmente puede ser útil contar con una lista de trabajos futuros en donde el estudiante pueda demostrar que ha comprendido el problema sobre el que ha trabajado y proponga líneas de trabajo futuras a partir de sus conclusiones.
%\paragraph{Conclusiones} El último capítulo debe ser el de conclusiones, que no debe ser una mera recapitulación del TFG, sino un lugar donde destacar si la propuesta permite aceptar la hipótesis, y los principales resultados del análisis a modo de lecciones aprendidas. En resumen, unas conclusiones que aporten valor al TFG.


%Muchos de los algoritmos que se utilizan para minería de datos tienen una componente aleatoria. Esto quiere decir que dependiendo de la semilla del generador de números aleatorios utilizado puede darnos una respuesta diferente. Para evitar tener este sesgo es necesario ejecutar varias veces el mismo algoritmo utilizando distintas semillas. Normalmente se suele ejecutar unas 5 o 10 veces y luego tomar para cada métrica a medir su media y desviación estándar. De esta manera puedes deducir si el algoritmo es robusto, si tiene mucha variabilidad, etc. Por tanto, a la hora de reportar los resultados debes incluir tanto la media como la desviación estándar en la tabla de resultados, como puedes ver en la Tabla \ref{tab:sd}.

%\begin{table}[htbp]
%\centering
%\label{tab:cityscapes}
%\begin{tabular}{c|c|c|c}
%\hline
%\textbf{Método} & \textbf{Iris} & \textbf{Diabetes} & \textbf{Ionosphere} \\ 
%\hline
%K-means & 214 (289) & 2628 (175) & 805 (202) \\
%LKM & 169 (10) & 1691 (47) & 908 (34) \\ 
%DE-LKM &  1938 (151) & 7425 (214) & 9449 (55) \\
%LC-LKM & 3764 (2431) & 12797 (44) & 17653 (2298) \\
%EM & 255 (139) & 3033 (226) & 1688 (1301) \\
%\hline
%\end{tabular}
%\caption{Tiempo de ejecución en ms. (media y desviación estándar entre paréntesis) de las 10 ejecuciones de cada %algoritmo por conjunto de datos.\label{tab:sd}}
%\end{table}

%También es necesario realizar un proceso de entrenamiento o {\it training} y uno de prueba o {\it test} posterior. Para ello se puede dividir el conjunto de datos en una parte para training y otra para test. Sin embargo, la opción más recomendada es utilizar algún tipo de validación cruzada que divide el conjunto de datos original en varias partes. Luego se utilizan todas menos una para entrenar y la restante para validar. Esto se realiza tantas veces como divisiones del conjunto de datos original hemos hecho. Finalmente se reporta la media y la desviación estándar de las métricas evaluadas. En caso de utilizar particiones ya realizadas por terceros indícalo e incluye una referencia en la memoria y/o los ficheros mismos al entregarla.

\subsection{Recomendaciones}\label{tfx_inves_ss_recomendaiones}
%Sugerencias para futuros investigadores.
%Áreas de mejora o posibles extensiones del estudio.

A modo de buenas prácticas para la realización de este tipo de TFG, se proponen a continuación una seria de consejos que esperamos que sean de utilidad:
%A modo de buenas prácticas para la realización de este tipo de TFG, se proponen una seria de consejos que se estiman que allanarán su elaboración.

\begin{itemize}
    \item \textbf{Sistema demostrador}. Como ya se ha indicado, se recomienda realizar un sistema informático que integre la propuesta científica, con el fin de que el alumno aplique los principios de ingeniería de software adquiridos durante la carrera. 
    
    \item \textbf{Manual de instalación y uso}. En caso de no realizar un sistema demostrador, es esencial proveer de toda la información necesaria para asegurar la reproducibilidad de los resultados. Eso incluye un manual de instalación y uso, el código fuente utilizado, etc.

    \item \textbf{Experimentación y redacción en paralelo}. Es práctica común de los estudiantes el realizar primero la experimentación y posteriormente la redacción de la memoria. Esto suele conllevar agobio y hastío, dado que suele apetecer más el programar un sistema informático que redactar una memoria. Por tanto, para facilitar el proceso de redacción, y también para que este sea más rico, se recomienda realizar la tarea de experimentación y redacción en paralelo.

    \item \textbf{Registrar todo paso de la experimentación}. Es importante registrar todo paso del proceso científico que se realice, dado que facilitará luego su desarrollo en la memoria. Esto incluye el uso de sistemas de control de versiones, llevar un \textit{log} donde se detallen los distintos experimentos realizados, etc.

    \item \textbf{Continua interacción con la persona que te tutoriza}. Teniendo en cuenta que la metodología de investigación te puede resultar novedosa, es muy importante que tengas reuniones con tu tutor o tutora frecuentes con objeto de que te vaya guiando paso a paso por ella y en las que tendrás que aprovechar para consultar todas las dudas que te vayan saliendo.

    \item \textbf{Trabajos futuros}. La investigación es un proceso continuo donde cada investigador aporta su pequeño grano de arena. A medida que uno se sumerge dentro del área de estudio, van surgiendo nuevas ideas e hipótesis que por la extensión y cantidad de horas para realizar el TFG no pueden ser acometidas. Es importante por tanto dejarlas por escrito, en el apartado de conclusiones o uno propio, de forma que cualquier lector pueda retomar el trabajo realizado y extenderlo desde el punto en donde ha acabado el TFG.

    \item \textbf{Reproducibilidad}. Aunque ya se ha comentado en esta sección el tema de la reproducibilidad, es importante hacer hincapié en ella. Es muy importante que describas tu investigación con tal detalle que cualquier otra persona pueda reproducir de manera fehaciente todo el proceso que has realizado y obtener los mismos resultados. De esta forma podrá comparar sus propios resultados con los tuyos. Por desgracia, aunque es verdad que cada vez menos, te encontrarás con situaciones en las que te interesaría evaluar un modelo que ya ha sido publicado pero no tienes suficientes detalles para implementarlo tú o no aportan información de valores de parámetros con los que han obtenido los resultados que se han publicado. Estas situaciones son bastante frustrantes pues, aunque el modelo sea interesante, similar al tuyo, utilice el mismo enfoque, o cualquier otra causa, no podrás emplearlo para comparar resultados. Por tanto, describe tus modelos y entornos experimentales con todo detalles. Incluso pon a disposición de la comunidad científica el código y los conjuntos de datos (siempre que puedas).

    \item \textbf{Un mal resultado también es un resultado}. La investigación puede llegar a ser a veces un proceso ingrato pues no se suelen obtener normalmente los resultados magníficos que uno espera. Muchas veces repetimos el ciclo de la Figura \ref{fg_met_cientifico} varias veces y los resultados obtenidos no son los que deseamos. ¿Esto implica que debemos continuar \textit{sine die} con nuestro trabajo hasta conseguir buenos resultados? ¿Y si no los conseguimos significa que debemos abandonar el problema entre manos y cambiar de temática de TFG? La respuesta en ambas cuestiones es la misma y muy contundente: no. Si no conseguimos en un plazo razonable resultados que consideremos que realizan una contribución científica, tenemos que tener en cuenta que científicamente mostrar el camino por donde no se lleva a nada positivo es también una aportación, que será útil para otros investigadores (``por aquí, aunque parecía prometedor no podemos ir porque no conduce a nada''). Y para ti, pues desde el punto de vista de tu TFG estarás mostrando que eres capaz de realizar una investigación científica completa y eso en sí es suficiente para tu TFG. Ya tendrás tiempo en tu TFM o en tu tesis doctoral de continuar y encontrar alternativas que mejoren el estado del arte del problema entre manos.

    \item \textbf{Publica los resultados}. Una vez que tengas hecho todo el trabajo científico, ¿por qué no publicar los resultados en un congreso o en una revista y hacer partícipe a la comunidad científica nacional o internacional de la investigación que has realizado y de las conclusiones obtenidas? De esta forma publicitarás tu trabajo y otros investigadores podrán apoyarse en él para seguir avanzando. Además, que tu TFG esté avalado por una publicación científica siempre será una magnífica carta de presentación para la evaluación de tu TFG, sin contar con el hecho de que estarás empezando ha hacerte un currículum muy útil para una posterior fase profesional si te gusta el mundo de la investigación.
\end{itemize}


\section{TFG de revisión del estado del arte}
\label{appendix:revisionestado}

Existe una modalidad de TFG que es en sí una revisión del estado del arte. Para que nos entendamos, es como la sección de revisión del estado del arte vista en el capítulo \ref{cap:RevisionEstadoDelArte}, pero a lo grande, en la que la revisión ocupa toda la memoria del TFG.

Si vas a realizar este tipo de TFG, en primer lugar te recomendamos que leas ese capítulo con objeto de obtener una idea general de qué es una revisión del estado del arte y, por supuesto, el capítulo \ref{cap:bibliografia} con el objetivo de tener claro cómo gestionar la bibliografía del TFG, pues, si ya es importante este asunto, en este tipo de proyecto fin de carrera, las referencias bibliográficas se configuran como algo vital.

El objetivo de esta sección será explicar brevemente cómo llevar a cabo la elaboración de la memoria de un trabajo de revisión del estado del arte. En este caso, hay mucho material publicado que puede serte de utilidad, por lo que haremos una revisión poco somera del mismo con objeto de ofrecerte algún material de inicio para que, al menos, comiences la tarea. De cualquier forma, ponte de acuerdo con la persona que te tutoriza sobre la forma de enfocar el trabajo, pues es un tipo muy especial de TFG que necesita ser definido y desarrollado de forma muy precisa.

\subsection{Tipos de revisiones}

Lluis Codina en \cite{codina2024lluis} establece dos grandes grupos de tipos de revisiones: las \textit{tradicionales} o \textit{narrativas} y las de tipo sistemático. Las primeras son más bien ensayos y carecen de validez científica; las segundas, sí que tienen esta validez, y se clasifican en \textit{sistemáticas}, que se emplean para determinar el impacto de intervenciones, y \textit{de alcance}, usadas para describir hasta dónde (alcance) llega un conocimiento dado. En este trabajo, el autor explica claramente cuándo se debería elegir un tipo u otro de revisión.

Pero esta es una simplificación que el profesor Codina ha realizado porque en realidad existe una gran cantidad de tipos diferentes de revisiones tal y como Grant y Booth indican en \cite{grant2009maria}, la mayoría ampliamente usados en el campo de las ciencias de la salud y con un componente estadístico muy fuerte:

\begin{itemize}
    \item Revisiones tradicionales. 
    \item Síntesis de conocimiento: de forma genérica se podrían definir como aquellas revisiones que contextualizan e integran los hallazgos de investigación de estudios individuales dentro del cuerpo más amplio de conocimiento sobre el tema que tengas entre manos. Algunos de los más conocidos y usados pueden ser los siguientes:
    \begin{itemize}
        \item Revisiones sistemáticas: identifican, evalúan y sintetizan todas las pruebas empíricas que cumplen unos criterios de elegibilidad previamente especificados. Las revisiones sistemáticas deben ser lo más exhaustivas e imparciales posibles.
        
        \item Metanálisis: subconjunto de revisiones sistemáticas que combina estadísticamente los resultados de estudios cuantitativos encontrados, con objeto de ofrecer un efecto más preciso de los resultados.
        
        \item Revisiones de alcance: abordan una pregunta de investigación exploratoria destinada a extraer conceptos clave, tipos de evidencia y nichos en la investigación relacionada con un área o campo definido, mediante la búsqueda sistemática, selección y síntesis del conocimiento existente.
        
        \item Revisiones rápidas: un tipo de síntesis de conocimiento en el cual los procesos de revisión sistemática se aceleran y los métodos se simplifican para completar la revisión más rápidamente que en el caso de las revisiones sistemáticas típicas, que vienen a tener un tiempo de realización de un año, reduciendo el tiempo de confección de cinco a doce semanas. Se emplean cuando no se dispone de mucho tiempo para realizarlas.
        
        \item Revisiones realistas: comprenden y desentrañan los mecanismos por los que una intervención funciona (o no funciona), proporcionando así una explicación, en lugar de un juicio sobre cómo funciona.

        \item Revisiones cualitativas: aquellas que integran o comparan los hallazgos de estudios cualitativos.

        \item Revisiones mixtas: combinación de los hallazgos de estudios cualitativos y cuantitativos dentro de una sola revisión sistemática para abordar las mismas preguntas de revisión superpuestas o complementarias.
        
        \item Síntesis narrativas: se basan en el uso de palabras y texto para resumir y explicar los hallazgos de la síntesis más que en resultados estadísticos. Básicamente usan la palabra para ``contar la historia'' de los hallazgos en los estudios incluidos.

        \item Revisiones tipo paraguas: se refiere a una revisión que recopila evidencia de múltiples revisiones en un documento accesible y utilizable. 
        
    \end{itemize}
\end{itemize}

En la web de la {biblioteca de la Universidad de Melbourne}\footnote{\url{https://unimelb.libguides.com/whichreview}} dispones de una definición muy detallada de los diferentes tipos de revisiones de literatura así como bibliografía de cada una de ellas para que las consultes. Si estás pensando realizar tu TFG en este contexto, te recomendamos que conozcas previamente los diferentes tipos de revisión existentes y que, junto a quien te dirige el trabajo, decidáis cuál es la que mejor se ajusta a los objetivos del mismo.

\subsection{Revisiones sistemáticas}

Una de las más ampliamente usadas es la revisión sistemática, sobre todo en el ámbito de la salud, aunque su uso se ha generalizado a todos los campos del conocimiento. Tal y como indica Codina en \cite{codina2018lluis} habría que diferenciar entre las revisiones sistemáticas y las sistematizadas, ya que estas primeras están centradas en conocer la eficacia de una intervención basándose en el análisis de estudios científicos que se han realizado sobre ella, como hemos dicho en el campo de la salud, y las segundas están enfocadas en explorar campos de conocimiento e investigación específicos, identificando  tendencias y corrientes dominantes, y detectando vacíos y posibles oportunidades para futuras investigaciones. Esta segunda definición sí que podría ser aplicada a cualquier campo de conocimiento y, por tanto, si somos estrictos con el lenguaje, en el campo de la informática deberíamos realizar una revisión sistematizada. Pero... por abuso del lenguaje se habla de forma general, independientemente del campo, de revisión sistemática. 

De cualquier forma una revisión (sistemática o sistematizada) de este tipo está compuesta de varias fases muy bien pautadas, que pasamos a describir seguidamente, cuyas descripciones detalladas podrás encontrar en  \cite{booth2021a}, un clásico en el campo de las revisiones de literatura:

\begin{enumerate}
\item Formulación de la pregunta(s) de investigación: define claramente el objetivo de la revisión y las preguntas de investigación que se abordarán. Es importante que estas preguntas sean específicas, claras y relevantes para el tema de estudio.

\item Búsqueda de literatura: se realiza una búsqueda exhaustiva y sistemática de la literatura relevante utilizando bases de datos académicas, bibliotecas digitales y otros recursos. En nuestro campo de la informática también puede interesarte realizar búsquedas en repositorios de código abierto, por ejemplo.

\item Selección de estudios (en inglés, \textit{screening}): se aplican criterios de inclusión y exclusión para seleccionar los estudios que cumplen con los criterios de la revisión. Esta selección se suele realizar en varias etapas, comenzando con la revisión de títulos y resúmenes para realizar un primer filtrado, seguida de la revisión de los textos completos de los estudios potencialmente relevantes que han pasado esta primera criba. Por ejemplo, se pueden descartar los estudios que lleguen a conclusiones con pocos datos o que usen métodos o tecnologías no actuales o no apropiados. 

\item Extracción de datos: se recopilan los datos relevantes de cada estudio seleccionado, como características del estudio, métodos utilizados y resultados obtenidos. En nuestro caso, información específica sobre tecnologías, detalles técnicos, prestaciones de aplicaciones, algoritmos empleados, lenguajes, etc.

\item Evaluación de la calidad de los estudios: se realiza una evaluación crítica de la calidad metodológica de los estudios incluidos en la revisión. 

\item Análisis y síntesis de los datos: se analizan los datos extraídos de los estudios y se realiza una síntesis para identificar patrones, tendencias, inconsistencias o nichos.

\item Interpretación de los resultados: se interpretan los hallazgos de la revisión en el contexto de la pregunta de investigación y se discuten sus implicaciones.

\item Escritura de la revisión: se redacta un informe detallado que describe el proceso de revisión, los métodos utilizados, los resultados obtenidos y las conclusiones alcanzadas. La memoria de tu TFG podría seguir una estructura estándar en este tipo de trabajos, que podría ser algo así:

\begin{enumerate}
\item Introducción: contextualización del tema y justificación de su importancia. Establecimiento de los objetivos.
\item Metodología: descripción de los métodos usados (estrategia de búsqueda, bases de datos, criterios de inclusión y exclusión, procedimientos de selección de estudios, evaluación de la calidad de los mismos, técnicas de análisis de datos, si corresponde).
\item Resultados: exposición de los resultados conseguidos. Tablas y gráficas ayudarán a visibilizarlos.
\item Discusión: interpretación de los resultados teniendo en cuenta los objetivos de la revisión, implicaciones prácticas o teóricas, y también es importante que se establezcan las limitaciones del proceso de revisión. Finalmente, la exposición de los resultados finales y recomendaciones. 
\item Conclusiones: resumen de los principales hallazgos, conclusiones finales y recomendaciones en base a los resultados obtenidos. También es habitual meter aquí una serie de líneas de trabajo futuras. 
\end{enumerate}
\end{enumerate}

Ten en cuenta que este proceso no es estático en el sentido de que en cualquier momento vas a poder obtener nuevos trabajos y tendrás que decidir si son relevantes para tu estudio e incorporarlos al mismo en caso afirmativo, con los cambios que conllevará en el análisis que has realizado hasta el momento. Con la persona que te tutoriza tendréis que decidir cuándo parar de incorporar más estudios para revisar.

Ni que decir tiene que todas las referencias deben estar correctamente citadas y dispuestas en una sección final de bibliografía.

El aporte realmente relevante de tu revisión vendrá de la mano del contenido de la sección de discusión, pues es ahí donde vas a mostrar tu capacidad de análisis y descubrimiento de hallazgos a partir de los trabajos analizados. La calidad de la interpretación de estos y las conclusiones harán o no valioso tu trabajo de revisión. Es por esto que te recomendamos que te esfuerces especialmente en esta parte de tu TFG. 

Para finalizar, indicarte que en las referencias \cite{carrera2022angela,kofod2022anders,silva2016rodrigo} tienes algunos ejemplos en los que los autores realizan una adaptación de las revisiones sistemáticas al campo de la informática. Échales un vistazo porque pueden serte de mucha utilidad.




\end{document}
