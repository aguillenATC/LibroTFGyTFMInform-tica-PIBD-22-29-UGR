\chapter{La estructura general de la memoria}
\label{cap:EstructuraMemoria}
% [Autores: Pablo]
% Introducción general a cada una de las partes de la memoria.
% (este quizás sea de los últimos capítulos en escribir)
%Pablo revisa/copia/quita lo que vea oportuno de las recomendaciones generales -> Sobre la redacción de la memoria
% Lo siguiente añadido por María José

En este capítulo te mostramos de forma breve cómo estructurar la memoria de tu TFG. Esta estructura debería ser lo primero que deberías abordar, y dependerá del tipo de TFG/TFM que realices.

Es muy importante que la estructura general de la memoria esté consensuada con la persona que te tutoriza. Aquí te damos una sugerencia de una posible organización de capítulos y secciones dentro de cada capítulo. 


\begin{enumerate}
    \item Introducción
        \begin{itemize}
            \item Contexto/Antecedentes
            \item Justificación/Motivación
            \item Objetivos/Hipótesis
            \item Estructura de la memoria
        \end{itemize}
    \item Estado del arte
        \begin{itemize}
            \item Descripción de dominio del problema
            \item Metodologías potenciales a aplicar
            \item Tecnologías potenciales para usar
            \item Trabajos relacionados
        \end{itemize}
    \item Propuesta
    \begin{itemize}
            \item Descripción de la propuesta
            \item Metodología
            \item Planificación temporal
            \item Presupuesto
            \item Otras secciones según la metodología y tipo de proyecto
        \end{itemize}
    \item Conclusiones y trabajos futuros
    \begin{itemize}
            \item Conclusiones
            \item Trabajos Futuros
        \end{itemize}
    \item Bibliografía
    \item Anexos
\end{enumerate}


En el capítulo del Estado del Arte se pueden incluir tantas secciones como se desee para agrupar bien los tipos de revisiones realizadas. Si la envergadura de estas secciones es muy grande, pueden incluso separarse en capítulos aparte, aunque te recomendamos que el EDA.

Tal y como se indica en el capítulo 6, \ref{cap:RevisionEstadoDelArte}, cada sección del estado del arte que se desee incluir deberá constar primero de una introducción explicando la metodología seguida para la revisión, luego la revisión concreta y al final unas conclusiones. Es muy aconsejable incluir tablas con aspectos comparativos, ya que son más visuales y sirven de resumen. Por ejemplo, se recomienda incluir comparativas entre las potenciales tecnologías o metodologías, para luego en el capítulo de la propuesta justificar cuáles de ellas son las elegidas para la solución. También es común comparar los trabajos relacionados entre sí y según nuestros objetivos o requisitos.

Las secciones del capítulo en el que describimos nuestra propuesta están muy relacionadas con el tipo de proyecto y metodología. Vamos a dar unas guías de cómo podría estructurarse según ello, aunque tienes información más detallada en un capítulo aparte para cada tipo de proyecto..

\begin{itemize}
    \item \textbf{Proyectos de investigación. (Ver \ref{appendix:investigacion})}
    \begin{itemize}
                \item \textbf{Preguntas de investigación}: identificación de problemas o carencias que se van a abordar en el proyecto 
                \item \textbf{Hipótesis}: enunciados a demostrar o probar
                \item \textbf{Evaluación de las hipótesis}: Elección de un método de investigación, determinación de experimentaciones a realizar, elección de instrumentos de recogida de datos y de herramientas análisis de datos y de evaluación.
                \item \textbf{Resultados}: Variables y valores. Tablas de análisis, matrices de correlación, ...
                \item \textbf{Discusión}: Interpretación de resultados, referenciando estado del arte e hipótesis u objetivos.
    \end{itemize}
    
    \item \textbf{Proyectos de desarrollo según la metodología seguida. (Ver \ref{appendix:desarrollo})}
        \begin{itemize}
            \item \textbf{Metodologías ágiles}
                \begin{itemize}
                    \item \textbf{Product Backlog}: listado de historias de usuario priorizadas y agrupadas en iteraciones o sprints
                    \item \textbf{Iteraciones}: Una sección por cada iteración o sprint describiendo las tareas y pruebas de cada iteración, e incluyendo gestión de riesgos si procede
                \end{itemize}
            \item \textbf{Ciclo de vida clásico}
                \begin{itemize}
                    \item Requisitos
                    \item Diseño
                    \item Implementación
                    \item Pruebas
                \end{itemize}
            \item \textbf{Otros ciclos de vida}
                \begin{itemize}
                    \item Secciones según las fases del ciclo
                \end{itemize}
        \end{itemize}
        
     \item \textbf{Proyectos de revisión de estado del arte. (Ver \ref{appendix:revisionestado})}
                \begin{itemize}
                    \item En estos proyectos se prescindiría del capítulo de propuesta, siendo el capítulo de revisión de estado del arte de mayor envergadura. Se debe describir la metodología seguida para la revisión (Dónde buscar, qué palabras clave utilizar, cómo filtrar la búsqueda, qué características revisar/comparar, ...)
                \end{itemize}
\end{itemize}

 Las metodologías de proyectos no son exclusivas entre sí, por ejemplo, un proyecto de tipo desarrollo podría incluir una parte de validación que implique aplicar la metodología de investigación. Igualmente, un proyecto de investigación puede necesitar del desarrollo de software. Por ello, se pueden añadir secciones de un tipo en otro tipo.


