\chapter{Prólogo}



% [Autores: lo podemos hacer al final todos]
\textit{Aviso 1}: en este libro se refleja la  “opinión” de un conjunto de profesores a partir de su experiencia y conocimientos académicos, así como su experiencia, tanto como tutores como evaluadores. Es un conjunto de recomendaciones, sugerencias y buenas prácticas que en ningún momento garantizan que, tras seguirlas, se obtengan buenos resultados y además puede haber efectos secundarios ;-) 

\textit{Aviso 2}: al igual que las distribuciones normales tienen una media representativa, las colas de la distribución siguen siendo valores correctos de la distribución. En este documento hemos intentado comentar la media con una amplia desviación, no obstante, la tupla formada por (TFG, alumno, profesor, tema) puede caer en los extremos de la distribución y seguir siendo normal. En otras palabras, si consideras que tu TFG no se acomoda a lo aquí presentado, no te preocupes, y si sabes responder el porqué, seguro que estará en el extremo de los excelentes. 

\textit{Aviso 3}: este libro ha intentado realizar un uso no sexista del lenguaje, haciendo uso indistinto de  tutor/tutora, profesor/profesora y alumno/alumna, empleando sustantivos genéricos como estudiante, estudiantado, alumnado o profesorado, pronombres sin marcas de género (quienes), entre otras alternativas al lenguaje no sexista. Si algo se nos ha pasado, pedimos disculpas.

