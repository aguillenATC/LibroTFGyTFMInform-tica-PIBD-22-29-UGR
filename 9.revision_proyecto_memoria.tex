\chapter{La revisión del proyecto y la memoria}
\label{cap:Revisión}

% Comentarios:
% ¿Meter una lista de ítems a comprobar?
% ¿Aconsejar que se la confeccione el estudiante?
% Esto lo tendremos que ordenar por grupos.

Una vez que la memoria está terminada es el momento de comprobar que está correcta y que contiene todo lo que debe contener. ¿Por qué? Porque sencillamente se suele ir con prisas en esta parte final y es necesario revisarla para asegurarnos de que no nos falta nada. Para tal fin, te aconsejamos que hagas una lista de ítems ({\it checklist}) para comprobar y que determines si cumples con todos ellos. Nosotros te proponemos aquí algunas cosas que deberías comprobar para que tú y la persona que te tutoriza añadáis lo que consideréis pertinente con objeto de no dejar nada relevante.


\begin{itemize}
  \item Sobre el formato de la memoria:

  \begin{todolist}
    \item Se sigue la normativa de portadas (logos, colores, etc.) y prólogos (autorización, resumen, etc.).
    \item La portada contiene el título de TFG, el nombre autor y el de la persona que te tutoriza, así como la fecha de entrega o, al menos, el curso académico en el que se entrega.
    \item Los índices de contenidos para capítulos, secciones, tablas y figuras deben estar actualizados. Lo más sencillo es que los generes de forma automática.
    \item No quedan tablas cortadas al final de una página.
    \item Si hay títulos de secciones al final de una página, debajo de ellos debe haber al menos un párrafo.
    \item Se utilizan los mismos estilos y tipo de letra en toda la memoria, a no ser que quieras resaltar algo como citas literales, fórmulas, ecuaciones o código.
    \item Todos los márgenes deben tener las mismas dimensiones, salvo casos excepcionales.
    \item Las páginas en blanco que dejes deben ser intencionales, por ejemplo, para que cada capítulo empiece en página impar.
    \item Las páginas están numeradas (correctamente).
    \item Existen encabezados y pies de páginas y están correctos.
 \end{todolist}

  \item Sobre la estructura:
  \begin{todolist}
    \item Están todas las secciones que, según el tipo de TFG, deben aparecer. 
    \item El resumen claramente describe de forma concisa qué se ha realizado en el TFG.
    \item La introducción establece claramente el contexto y la motivación necesarias para entender el problema entre manos.
    \item Están los objetivos. Son claros y concisos.
    \item Están las conclusiones, donde se muestran los resultados y contribuciones del TFG, y los trabajos futuros.
    \item Está la bibliografía.
    \item Están todos los anexos necesarios.
    \item En caso de que en el TFG se utilice código fuente, éste está enlazado en la memoria (por ejemplo, GitHub, GitLab).
    \item Es interesante incluir un anexo con el glosario de términos y abreviaturas empleados en la memoria.
  \end{todolist}
  
  \item Sobre la bibliografía:

  \begin{todolist}
    \item Todas las referencias de la bibliografía deberían aparecer citadas en el texto.
    \item Todas las citas bibliográficas del texto deben tener una referencia asociada. No puede haber referencias sin citar.
    \item Las referencias bibliográficas deben estar completas y también todas en el mismo formato.
    \item Las referencias de recursos de Internet deben indicar también la fecha de la última consulta.
  \end{todolist}

  \item Sobre las figuras y tablas:

  \begin{todolist}
    \item Todas las figuras y tablas deben estar numeradas secuencialmente y referenciadas/citadas en el texto.
    \item Todas las figuras y tablas deben tener un título descriptivo ({\it{caption}}), que en las primeras suele situado ir debajo y en las segundas arriba.
    \item Los títulos de figuras y tablas deben estar en la misma página de la figura o tabla.
    \item Si se usan figuras hechas por otra persona deberían ser citadas/usadas correctamente, por ejemplo en el pie de figura ``Tomado de [x]'',``Extraído de [x]'' o ``Fuente: [x]'', si no, indicar ``Elaboración propia''.
    \item Las imágenes y las tablas están dentro del espacio del cuerpo de la página y ninguna se desborda hacia los márgenes. 
  \end{todolist}

  \item Sobre las licencias:

  \begin{todolist}
    \item He incluido en la portada o primera página el tipo de licencia que deseo para la memoria.
    \item He incluido en el código realizado por mí el tipo de licencia que deseo.
    \item Si he usado figuras, código o documentos multimedia de otros, he citado las fuentes y me he asegurado de podía usarlas con su licencia original.
  \end{todolist}

  \item Y por último:

  \begin{todolist}
    \item Uso de un lenguaje adecuado y 
    \item que no haya errores ortográficos ni gramaticales, por lo que más quieras (¡¡¡Pásale el corrector ortográfico, por favor!!!).
  \end{todolist}

\end{itemize}

Además, también te recomendamos que te descargues las rúbricas de evaluación del tutor y de la comisión y compruebes que tratas, de una u otra forma, todos los puntos que aparecen en ella. Esto es importante porque si te dejas alguno sin tratar, los miembros de la comisión pueden preguntarte por él. Quien evita la ocasión, evita el peligro ;-)