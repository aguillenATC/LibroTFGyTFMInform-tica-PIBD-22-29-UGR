\chapter{Introducción}
\label{cap:Introducción}
%Juanma
%Iniciativas previas: \url{https://drive.google.com/drive/folders/1WMPwX1v7IhG88bJYLXDgm4Zp6a_pCCHh}

Estás cursando el cuarto curso del Grado en Ingeniería Informática o a punto de hacerlo. Si estás leyendo esta introducción, es porque el desarrollo del Trabajo Fin de Grado (TFG) es algo que te ronda la cabeza o ya lo tienes entre manos. 

Estás terminando tus estudios de grado y aparece esta asignatura obligatoria de doce créditos en el segundo semestre de cuarto, que constituye un desafío fundamental y monumental en tu proceso formativo como estudiante de Ingeniería Informática. En ella, pondrás en práctica muchos de los conocimientos adquiridos en el grado y otros muchos que obtendrás por cuenta propia; te enfrentarás a retos técnicos y/o metodológicos que no imaginarías. Al igual que la obra maestra gremial, el TFG es un proyecto culminatorio que cada estudiante debe desarrollar de forma individual tras años de aprendizaje y estudio en una disciplina. El TFG demanda la aplicación de los conocimientos y habilidades adquiridas durante los estudios, así como, capacidad de investigación, análisis, pensamiento crítico y creatividad dentro de un marco académico. Y lo mejor es que, aunque todavía no lo sabes y casi ni te lo imaginas, porque ahora estás cag... eh, perdón, temeroso/a, vas a salir airoso/a de ellos proponiendo soluciones creativas y mostrando tus habilidades y capacidades. En definitiva, es un paso más en tu formación como ingeniero/a y debes aprovecharlo para convertirte en un mejor profesional.

Normalmente, en este tiempo aplicarás metodologías de desarrollo y escribirás mucho código. Como informáticos, nos gusta esta última tarea más que cualquier otra cosa, especialmente, e infinitamente, más que escribir informes y memorias. Disfrutamos programando y nuestro humor cambia radicalmente cuando se nos dice que hay que explicar lo que hemos hecho por escrito. ¿Por qué? Claramente porque nos suele costar trabajo: bien porque no nos gusta esta tarea, bien porque no sabemos cómo hacerla. Y normalmente no le damos la importancia que se merece. Piensa que la memoria de un TFG no es sólo un requisito administrativo para conseguir tu título de graduado, sino que es el medio principal que emplearás para comunicar y justificar el (buen) trabajo que has llevado a cabo en el tiempo que has dedicado al TFG. Es, de alguna forma, como una especie de carta de presentación ante un tribunal evaluador que describe todo lo que has hecho y por qué lo has hecho. También es una carta de presentación para potenciales empleadores, ya que podrán conocer tus conocimientos y habilidades de forma práctica. Piensa también que lo habitual en cualquier trabajo es describir tus tareas para que los compañeros y compañeras entiendan lo que has hecho y para que tú te acuerdes unos meses más tarde. Por tanto, la confección de la memoria del TFG no es más que un ejercicio de entrenamiento en este sentido.

Este libro tiene como objetivo acompañarte en el desarrollo de tu TFG, de forma general, pero muy especialmente en una de las actividades más importantes, y precisamente en la que más dudas y miedos aparecen: la escritura de la memoria de tu proyecto. En este manual nos enfocamos en guiarte a través del desafiante, y a la vez gratificante, proceso de elaborar una memoria de TFG en Informática. Para ello, te proporcionamos herramientas prácticas, consejos y ejemplos concretos que abarcan, desde la elección de la temática, hasta la defensa, pasando por todo el proceso de escritura de la memoria. Somos conscientes de que cada (estudiante, tutor/a, proyecto) es único, por lo que este libro no pretende ser una guía rígida, sino un recurso flexible y totalmente adaptable por ti y la persona que te tutoriza, que facilite la comunicación de lo que has conseguido en tu TFG.

¿Y por qué este libro? Los autores somos profesores de Informática y llevamos mucho tiempo tutorizando a estudiantes y evaluando TFG. Hemos visto que la tarea que más os cuesta en todo el desarrollo del TFG es la escritura de la memoria. Si bien durante la carrera habéis ejecutado numerosos proyectos en las prácticas de las asignaturas y redactado informes técnicos de los mismos, partimos del hecho de que no se os ha formado en redacción, ni general ni técnica, en vuestro paso por la Universidad y que no tenéis la experiencia necesaria para escribir una memoria de este calibre porque nunca habéis documentado un proyecto de esta envergadura. Así las cosas, cuántos casos existen de estudiantes que hacen un buen TFG, pero que se bloquean con la memoria, que escriben un informe final muy escueto o hecho con prisas por dejarlo para el final, emborronado, sin una organización adecuada, escrito con desgana, con problemas de redacción (gramaticales y ortográficos), sin saber cómo presentar las ideas o resultados, sin apoyarlo en referencias bibliográficas, etc. Podréis tener el mejor proyecto del mundo, pero si lo acompañáis de una mala memoria, el fracaso está garantizado. Y eso es lo que queremos evitar. O dicho de otro modo: una memoria bien estructurada y redactada puede marcar la diferencia entre un TFG notable y uno mediocre, independientemente de la calidad técnica del proyecto desarrollado.

Este documento es el resultado de un proyecto de innovación docente (PIBD 22-29 -- ``Cómo escribir tu TFG o TFM de Ingeniería Informática y no morir en el intento: dificultades, retos y elaboración de materiales docentes'') de la Universidad de Granada, donde los participantes nos propusimos crear un recurso flexible para que tengáis apoyo en esta fase de escritura del informe final del TFG y procurar evitar caer en todos los problemas citados anteriormente. Con él queremos animarte a ver la redacción de la memoria como una parte integral de tu proyecto, tan importante como el análisis, el diseño o la implementación, y que tengas herramientas para conseguir un producto de calidad. Pero el objetivo no es sólo ofrecer un material de apoyo finalista, sino que con los consejos que te damos, su lectura y práctica te permitan afrontar cualquier otra redacción técnica con mucha mayor claridad y soltura, ya sea de un hipotético Trabajo Fin de Máster (TFM)\footnote{Podríamos decir que todo lo indicado en este libro sobre la redacción de la memoria sería extensible a la escritura del informe final de un TFM.} o de encargos profesionales.

También queremos ofrecer este recurso a la comunidad de docentes que tutorizan TFG. Muchas veces somos nosotros, los docentes, los que también necesitamos algo de orientación para poder, a su vez, orientar a los estudiantes. Y disponer de recursos de referencia nos resulta muy útil en nuestra labor de enseñanza. Por tanto, este libro pretende mostrar a nuestras compañeras y compañeros cómo los autores pensamos que podría ser una memoria de TFG y el proceso para su confección. Cada cual que tome lo que vea interesante para el ejercicio de su tutoría.

Llegado este punto de la introducción, donde hemos expresado nuestra motivación e intenciones, creemos que es el momento de hacer dos descargos de responsabilidad:

\begin{itemize}
    \item \textit{Disclaimer} I: este libro refleja la ``opinión'' consensuada de un conjunto de profesores y profesoras de Universidad a partir de su experiencia y conocimientos académicos, así como su experiencia en el rol de tutores y de evaluadores de TFG. Es un conjunto de recomendaciones, sugerencias y buenas prácticas que en ningún momento garantizan que, tras seguirlas, se obtengan buenos resultados y además puede haber efectos secundarios ;-) 

    \item \textit{Disclaimer} II: al igual que las distribuciones normales matemáticas tienen una media representativa, las colas de la distribución siguen siendo valores correctos de la distribución. En este documento hemos intentado comentar la media con una amplia desviación, no obstante, la tupla formada por (TFG, alumno, profesor, tema) puede caer en los extremos de la distribución y seguir siendo normal. En otras palabras, si consideras que tu TFG no se acomoda a lo aquí presentado, no te preocupes, y si sabes responder el porqué, seguro que estará en el extremo de los excelentes. 
    
\end{itemize}

Conocedores de cómo el estudiantado va evolucionando en cuanto a la forma en que realiza su aprendizaje, y con objeto de facilitárselo, hemos grabado varios vídeos cortos de cada capítulo con el objetivo de ofrecer las ideas principales de los mismos en un formato complementario y resumido al libro. Los vídeos están alojados en nuestro canal de YouTube \footnote{\url{https://www.youtube.com/@ComoescribirtuTFGenInformatica}}.También podéis consultar una presentación \footnote{\url{https://hdl.handle.net/10481/76907}}y vídeo \footnote{\url{https://drive.google.com/file/d/1hlWHa9p0BbWpWT8Xw1SWQllKvGTCZQ6j/view?usp=drive_link}} previos a este libro, con el mismo propósito, realizado por María José Rodríguez Fórtiz en el año 2020:

Asimismo, pretendemos que este texto sea un documento vivo, en el que todos los miembros de la comunidad académica del Grado en Ingeniería Informática puedan participar de forma colaborativa en su evolución y mejora. Por ello, hemos subido a nuestro repositorio de  GitHub\footnote{\url{https://github.com/aguillenATC/LibroTFGyTFMInform-tica-PIBD-22-29-UGR}} los fuentes del libro y te invitamos a que hagas aportaciones al texto, detectando erratas, añadiendo cosas que se nos han pasado y que consideras importantes, proponiendo nuevos temas sobre los que hablar, etc. También puedes crear ``issues'' con sugerencias o comentarios. Por supuesto, también puedes enviar agradecimientos o comentarios positivos, si consideras que te hemos ayudado. Ello nos podrá indicar si hemos conseguido nuestro propósito de ayudarte a ti a escribir el TFG sin morir en el intento y también a más estudiantes que quieran conocer opiniones sobre el trabajo.

Este libro ha intentado realizar un uso no sexista del lenguaje, haciendo uso indistinto de tutor/tutora, profesor/profesora y alumno/alumna, empleando sustantivos genéricos como estudiante, estudiantado, alumnado o profesorado, docente, pronombres sin marcas de género (quienes), entre otras alternativas. Si algo se nos ha pasado, pedimos disculpas.

Y por último, vamos a describirte cómo hemos organizado el libro, su estructura y contenidos, para que conozcas los diferentes temas que vamos a tratar en este texto:

\begin{itemize}
\item Todo lo que siempre quisiste saber sobre el TFG y que nunca te atreviste a preguntar. $\rightarrow$ Capítulo \ref{cap:Recomendaciones}. Recomendaciones generales.
\item Sobre la construcción de un castillo de naipes sin que se te derrumbe. $\rightarrow$ Capítulo \ref{cap:EstructuraMemoria}. La estructura general de la memoria.
\item Ese agujero negro donde un estudiante pasa tres días delante de una página en blanco para justificar su proyecto. $\rightarrow$ Capítulo \ref{cap:IntroducciónTFG}. La introducción del TFG.
\item Donde descubres que tu brillante idea ya se hizo en los años 80. $\rightarrow$ Capítulo \ref{cap:RevisionEstadoDelArte}. Revisión del estado del arte.
\item Cómo no demostrar mis dotes de ciencia ficción (I) con un diagrama de Gantt que parece sacado de un universo paralelo con días de 48 horas y presupuestos donde mi sueldo como 'ingeniero junior' es tan optimista que hasta mi madre se ha reído al verlo. $\rightarrow$ Capítulo \ref{cap:PlanificacionPresupuesto}. La planificación y el presupuesto.
\item Cómo no demostrar mis dotes de ciencia ficción (II) resumiendo todo como si hubiera sido planeado y proponiendo trabajos futuros que, sinceramente, esperas que haga otro. $\rightarrow$ Capítulo \ref{cap:Conclusiones}. Las conclusiones y los trabajos futuros.
\item Cómo organizar un cajón con un millón de calcetines desparejados. $\rightarrow$ Capítulo \ref{cap:bibliografia}. La bibliografía.
\item Cómo intentar darle los últimos retoques a una obra de arte mientras el reloj corre y el público espera ansioso su presentación. $\rightarrow$ Capítulo \ref{cap:Revisión}. La revisión del proyecto y la memoria.
\item Cómo teniendo una buena moto ser capaz de venderla (I). $\rightarrow$ Capítulo \ref{cap:elaboraciónPresentación}. La elaboración de la presentación de la defensa.
\item Cómo teniendo una buena moto ser capaz de venderla (II). $\rightarrow$ Capítulo \ref{cap:Defensa}. La defensa. 
\item Cómo ser un mago que protege tu obra con un hechizo, y permitir que otros la disfruten sin que se rompa. $\rightarrow$ Apéndice \ref{anexo:licencias}. Las licencias. 
\item Cómo organizar los diferentes mundos del videojuego con un poco de orden. $\rightarrow$ Apéndice \ref{anexo:Tipologías}. Las tipologías de TFG y su desarrollo.
\item La experiencia de otros/as compañeros/as $\rightarrow$ Apéndice \ref{anexo:Experiencias}. Experiencias y consejos del estudiantado. 
 \end{itemize}

\subsubsection*{Manual de uso}

Te recomendamos las siguientes formas de interactuar con este libro y sus vídeos:

\begin{itemize} \item Ha llegado el verano de tercero, tengo mucho tiempo, estoy aburrido y no se me va de la cabeza el TFG $\rightarrow$ entonces aprovecha el tiempo y léete de tirón el libro.

\item Ha llegado el verano de tercero, tengo mucho tiempo, estoy aburrido y no se me va de la cabeza el TFG, pero me da pereza ponerme a leer en las vacaciones $\rightarrow$ entonces échale un vistazo a los vídeos entre baño y baño.

\item Estoy en el comienzo de cuarto y sigue sin írseme de la cabeza el tema del TFG $\rightarrow$ léete el capítulo \ref{cap:Recomendaciones} de recomendaciones generales para saber cómo proceder y tener una idea general del TFG.

\item Ya tengo tema y tutor y estoy empezando a trabajar en el TFG $\rightarrow$ visualiza cada vídeo conforme te haga falta durante el desarrollo de tu TFG y léete cada capítulo correspondiente si quieres profundizar en algo concreto.

\item Usa el libro y los vídeos como te venga en gana, pero... ¡úsalos! 

\item ¡Por fin he terminado el TFG! $\rightarrow$ sé generoso y comparte tu experiencia y comentarios, que será muy útil para quien viene detrás tuya.

\end{itemize}

Y por último, agradecerte la lectura del libro, esperando que te sea útil ya seas tanto estudiante como docente. Esperamos tus comentarios.

%Este libro queda dividido en los siguientes capítulos: en el capítulo \ref{cap:Recomendaciones} se ofrece, para empezar, información general sobre el proceso de asignación del TFG, contexto general que debes conocer antes de comenzar a trabajar, así como pautas y recomendaciones de trabajo para ayudarte en el desarrollo del proyecto. El capítulo \ref{cap:EstructuraMemoria} establece una propuesta de estructura general de la memoria, indicando cada una de las partes que la compone, para que tengas una idea general de cómo organizar este documento. El siguiente capítulo, el \ref{cap:IntroducciónTFG}, se centra en describir el contenido del capítulo de introducción, ofreciendo tanto una posible estructura del mismo como consejos para su elaboración. El capítulo \ref{cap:RevisionEstadoDelArte} pasa a describir cómo se debe hacer una revisión del estado del arte, parte fundamental en cualquier memoria de TFG. Un elemento importante en todo proyecto es la planificación y el presupuesto, y en el capítulo \ref{cap:PlanificacionPresupuesto} damos algunos consejos para su elaboración. Seguidamente, en el capítulo \ref{cap:Tipologías} se presentan los cuatro tipos de TFG principales: de desarrollo, experimental, investigación y de revisión. Recomendaciones sobre cómo abordar la confección de las conclusiones y los trabajos futuros se incluyen en el capítulo \ref{cap:Conclusiones}. El capítulo \ref{cap:bibliografia} aborda asuntos relacionados con la bibliografía y cómo referenciarla en la memoria y el siguiente, el capítulo \ref{cap:anexos} hace lo propio con los anexos que puedes incluir en la memora.  Una vez que la esta está terminada, en el capítulo \ref{cap:Revisión} se muestra una lista de comprobaciones que deberías hacer para estar seguros de que la memoria alcanza el mínimo de calidad exigible. Los dos capítulos siguientes, y últimos, el \ref{cap:elaboraciónPresentación} y \ref{cap:Defensa}, se centran en la presentación y en la defensa, respectivamente, aconsejando sobre cómo montar la primera y cómo afrontar la segunda. Y como no podía ser de otra manera, este libro finaliza con un capítulo de anexos 