\chapter{La defensa}
\label{cap:Defensa}
% [Autores: Juanma, Alberto]

\section{La estructura del acto}

Una vez que finalice el periodo de entrega del TFG, el centro publicará las comisiones de evaluación y los TFG asignados a cada una. La comisión está compuesta por tres miembros. A saber: un presidente, un secretario y un vocal. Una vez nombrada, el presidente, normalmente, convocará a los estudiantes, informándoles de la fecha, hora y lugar del acto de defensa. Este se divide, como bien hemos indicado anteriormente, en dos fases: la presentación del trabajo y la discusión con los miembros de la comisión.

El acto es público, por lo que puedes entrar a ver a tus compañeros. El que lo hagas o no dependerá de ti, de si te pones nervioso viendo a otros o no, o de si quieres estar tranquilo y concentrado antes. Si te quedas fuera de la sala, quédate cerca de ella. El secretario saldrá a llamarte cuando sea tu turno. En ese momento, entras en la sala, saludas al tribunal (buenos días, buenas tardes, hola) y te diriges al lugar donde expondrás, conectas tu portátil (no, no hay disponible ninguno para tu uso, tienes que llevarte el tuyo) al proyector y preparas la presentación. 

El presidente, en ese momento, dirá algo así como que se va a dar comienzo a la defensa del TFG titulado X realizado por el estudiante Y. Y te informará de la estructura del acto, indicándote explícitamente el tiempo que tienes disponible para la exposición (normalmente veinte minutos) y el tiempo, aproximado éste, de la ronda de preguntas y comentarios (otros veinte minutos habitualmente). Tras la presentación, que comenzarás sólo cuando el presidente te dé la palabra, éste dará paso al secretario y al vocal, los cuales te plantearán sus cuestiones, participando finalmente el propio presidente en dicha ronda. Una vez finalizada la segunda fase, éste te dirá que se ha concluido el acto, te dirá que se te comunicará la nota en uno o dos días y, finalmente, te dará las gracias. Lo habitual es que tú también des las gracias. Y... se acabó lo que se daba. Ya has terminado, recoges y abandonas la sala, o bien te quedas a oír a los compañeros siguientes.

Como ya hemos indicado, es un acto público, por lo que también pueden asistir tus familiares y amigos. El que lo hagan o no, dependerá también de ti: si tus padres y hermanos o amigos quieren ir a verte y tú no te pones nervioso con su presencia, es un momento bonito para que te vean en acción y lo compartan contigo. A veces te puedes sentir más tranquilo/a con su presencia. Si no es así, mejor les dices que prefieres que no vayan y no hay ningún problema. De cualquier forma, el público en general y los familiares y amigos en particular, deben permanecer en silencio, sin intervenir en ningún momento y sin realizar comentarios de ningún tipo a las preguntas de la comisión ni a tus respuestas (vamos que no se lleven pancartas ni pompones, ni pongan caretos ni resoplen cuando algún docente haga algún comentario). Tampoco es normal que se aplauda cuando finalices. Los abrazos y felicitaciones déjalos para cuando estés fuera de la sala.

\section{La importancia de la defensa}

%https://www.youtube.com/watch?v=AK_xGgGSdCo

Una vez entregado el TFG, la siguiente y última fase en el proceso es la defensa de tu trabajo. Es conveniente que descanses y desconectes uno o dos días antes de ponerte a trabajar en la confección del material que usarás en la defensa. Te permitirá hacer esas tareas mucho más tranquilamente, con más concentración, perspectiva y ganas. Has trabajado muy duro durante todo el curso y en especial en las últimas semanas previas a la entrega. El cansancio y el estrés se acumulan. Y es oportuno hacerlo porque este acto es el momento de la verdad donde tu TFG será evaluado y tienes que estar fresco/a y en plenitud de condiciones.

La defensa y su preparación tienes que tomártelas en serio. Es el principal acto académico que vas a realizar en tus estudios y de su resultado dependerá la nota de tu proyecto. Tu trabajo ya está terminado y en este acto lo que tienes que hacer es convencer a los miembros de la comisión de que es un buen trabajo. Puedes haber realizado un trabajo excepcional pero si no lo ``vendes'' bien, la nota que tengas no será la que verdaderamente refleje la calidad de tu TFG (y esto es un hecho que, lamentablemente ocurre muy frecuentemente). Imagina que eres un vendedor de motos y tienes la mejor del mercado. La moto no se vende por sí sola, tienes que hacerle ver al cliente todas las magníficas prestaciones que hacen que sea una de las mejores del mercado, si no la mejor, y conseguir que la compre. Si no lo haces así, el cliente no se va a enterar y no vas a poder ``venderle la moto''. Por tanto, tan importante es, al elaborar un buen TFG, el producto y la memoria, como comunicarlo correctamente a los miembros de la comisión. 

Pero ojo, ``vender la moto'' no es ``vender humo''. Una cosa es que emplees todas las técnicas de comunicación a tu alcance para mostrar la calidad del producto y el trabajo desarrollado, y otra es que las utilices para tratar de engañar a la comisión, exagerando o engordando tu trabajo. Ten cuidado porque estas cosas son muy fáciles de detectar y te pueden poner en un aprieto, sobre todo porque los miembros de la comisión han leído tu memoria y saben lo que has hecho y además disponen de un informe del tutor sobre tu trabajo.

Por tanto, tu capacidad de comunicación jugará un papel importante en esta fase. Mediante una comunicación efectiva serás capaz de expresar de forma clara, concisa y efectiva, tus ideas, facilitando su comprensión por parte de la comisión. Las habilidades de comunicación podemos dividirlas en dos: la verbal y la no verbal. La primera implica al lenguaje hablado (uso de vocabulario, gramática, volumen y tono de voz); la segunda, tiene que ver con la forma de comunicar sin hablar, es decir, la parte más física: las posturas, gestos, miradas, etc. Estos dos tipos los describiremos con más profundidad en la siguiente sección, pero sé consciente que ambos son importantes en la defensa. Esta comunicación la apoyarás en una presentación clara, concisa y bien estructurada, como se ha indicado en el capítulo anterior, en la fase de exposición de tu trabajo. También tendrás que hacerlo sin apoyo en la fase de discusión con la comisión, en la que te realizarán preguntas y comentarios sobre tu trabajo, y tendrás que responder de forma clara, concisa y convincente.

Por tanto, si comunicas de forma efectiva, no vas a tener problema en hacer ver el trabajo que has realizado y, por tanto, la comisión será capaz de entender lo bueno que es. 

Algunos consejos para preparar la defensa son los siguientes:

\begin{itemize}

    \item Prepara la defensa con la persona que te tutoriza. Esto es un proceso iterativo en el cual inicialmente te indicará las líneas generales con las que debes hacer la presentación. Seguidamente, prepara un primer borrador y se lo pasas al docente para que lo evalúe. Te hará los comentarios pertinentes. Si tienes dudas sobre ellos, habladlo y llegad a acuerdos. Refléjalos en la presentación y vuelve a enviárselos. Este proceso se repetirá hasta que haya una estabilidad en la presentación y no haya cambios. En ese momento puedes decir que tienes lista la presentación y puedes comenzar a ensayar.

    \item No sólo prepara una presentación sino también una demostración, si el tipo de TFG que has desarrollado se presta a esto. Puedes hacer un vídeo mostrando las prestaciones de tu software o una demostración en vivo del mismo. De cualquier forma, acuerda con la persona que te tutoriza lo más relevante de esta demostración, el tiempo que le vas a dedicar y el momento en que lo harás, y prepárala concienzudamente también. 

    \item Normalmente afrontamos este acto con inseguridad y miedo. Para evitarlo, la única receta que hay es ensayar la presentación una y otra vez hasta que tengas seguridad y confianza, y sepas qué decir en cada momento. También imagina, visualiza, cómo va a ser la defensa, la sala, los miembros, dónde estarás tú situado, los gestos que harás, cómo te moverás. Este entrenamiento te dará confianza en ti mismo.

    \item Cronométrate en los ensayos. No puedes pasarte del tiempo indicado para la exposición. Si lo haces, el presidente te podrá decir que te has excedido del tiempo asignado y retirarte la palabra. Esto implica que puede haber cosas importantes que te dejes sin comentar, con el consiguiente problema para que los docentes que te evalúen comprendan la dimensión, dificultad, alcance, etc. de tu trabajo. Además, aunque te dejaran más tiempo, dejarías mala impresión en los miembros de la comisión y hay un ítem de la rúbrica que evalúa esto. Por tanto, cronométrate y ajústate al tiempo. Esto puede implicar que elimines diapositivas, que las hagas más breves, que contengan menos texto, que simplemente pases por encima de algunas diciendo apenas una frase. Cualquier cosa para ajustarte al tiempo pero, ojo, sin que el mensaje pierda contenido importante.

    \item Piensa en posibles preguntas que los miembros de la comisión podrían hacerte y prepara las respuestas. Estas suelen ser del tipo:

    \begin{itemize}
        \item ¿Qué es lo que has aprendido?
        \item ¿Por qué has hecho esto de esa forma (y no de esta otra)?
        \item ¿Por qué no has tenido en cuenta esto?
        \item ¿Cuáles son las limitaciones de tu trabajo?
        \item ¿Qué aporta tu trabajo?
        \item ¿Por qué has elegido esta opción y no esta otra?
        \item ¿Por qué no has liberado tu código? 
        \item ¿Por qué has escogido esta licencia y no otra?
        \item ¿Por qué no has incluido X en tu revisión del estado del arte?
        \item ¿Por qué no has mencionado ni valorado el uso de esta tecnología, metodología, método o herramienta?
        \item ¿Cómo abordarías la propuesta X de tus trabajos futuros? 
        \item ¿Qué es lo que más te ha costado?
        \item ¿Qué desafíos técnicos encontraste y cómo los superaste?
        \item ¿Qué métricas utilizaste para medir el éxito de tu proyecto?
        \item ¿Cómo podría aplicarse tu proyecto en un entorno real?
        \item ¿Qué escalabilidad tendría el software que has desarrollado?
       
    \end{itemize}

    Ten en cuenta que te podrán preguntar sobre cualquier cosa que no esté clara en la memoria o no hayas incluido. Por eso es tan importante revisarla bien antes de entregarla.

     \item También te pueden hacer comentarios sobre la memoria y presentación del tipo: `No está bien explicado el propósito del TFM''. ``Faltan objetivos relacionados con tu formación''. `Deberías haber añadido una tabla comparativa en el estado del arte''. ``Hubiera sido mejor utilizar la tecnología X'', `En el diagrama de clases hay estos errores...''. En esos casos, dales la razón si la tienen e indica que lo tendrás en cuenta para futuros trabajos. Si te dan pie para responder, explica o completa lo que te están indicando. Por ejemplo, `La tecnología X también es óptima pero he escogido Y porque además me permite hacer Z, cosa que con X es más costoso, o porque estoy más familiarizado con ella''.  

    \item Como ya se ha explicado en el capítulo anterior, la presentación es un mero elemento de apoyo para presentar tu proyecto. Es un guion para que tú sepas qué tienes que decir. Por tanto, no leas su contenido. Y para que puedas decir todo lo que debes decir, cuando diseñes la presentación hazte un documento en el que escribas las cosas que tienes que comunicar en cada diapositiva y apréndetelo. Pero en la exposición no hables como un loro que suelta el texto de memoria, ve con tranquilidad, haciendo pausas, dirigiéndote a los miembros de la comisión, mirándolos, y explicando las cosas. No las recites. Lo de aprenderse todo lo que tienes que decir es para evitar dejarte algo que sea importante. 

    \item Practica la exposición con la persona que te tutoriza, si es posible. Él o ella te indicará aspectos a mejorar de la presentación y de la comunicación. Además podréis simular el turno de preguntas y te hará algunas que son habituales en las comisiones y otras específicas que pueden surgir tras exponer la presentación. Y, sabiendo de antemano quiénes serán los examinadores, también te podrá indicar algunas típicas de estos profesores. También te hará comentarios sobre las respuestas que das, tanto en forma como en contenido.

    \item Practica la exposición con tus familiares y amigos. Ellos no van a entender la parte técnica, pero sí que te podrán dar una realimentación muy valiosa sobre la forma de expresarte, moverte, gestos, muletillas o cualquier otra cosa que les llame la atención. Haz caso a sus comentarios y ensaya de nuevo, ya sin ellos, procurando evitar las situaciones negativas que te han indicado. Si no tienes posibilidad de tener espectadores en los ensayos, grábate tú con el móvil y luego analiza minuciosamente toda la exposición para ver qué cosas tienes que mejorar.

    \item Si es posible, por las fechas y horarios, asiste a alguna otra defensa. De esta manera podrás ver en directo todo el proceso, la exposición que hace el estudiante, las preguntas y cómo responde. Esto te servirá para entender el acto y para quitarte el miedo, pues es una situación que, aunque importante y algo más relevante, ya habrás realizado en múltiples ocasiones en tu grado.

    \item Procura preparar la presentación, llevar a cabo las correspondientes modificaciones, y los ensayos pertinentes con tiempo, de tal manera que no necesites nada más que dar un pequeño repaso a la misma el día anterior de tu defensa. Estarás más tranquilo/a ya que llegar al día de antes sin tenerla bien preparada sólo es una fuente de nervios y tensión que te perjudicará. Ese día descansa y come especialmente bien. 
    
\end{itemize}

Y algunos otros para el momento de la defensa:

\begin{itemize}
    \item Imaginamos que en el momento de comenzar tendrás nervios. Eso es normal. No te preocupes. Siempre es mejor tener un poco de tensión que ir súper relajado. Esos nervios se irán disipando conforme avances. Al comenzar, respira hondo y empieza con tu exposición. Céntrate en contar qué y cómo lo has hecho de una manera lo más efectiva posible.

    \item Ha sido un periodo largo de trabajo duro y ahora llega el momento que lo culmina. Nadie sabe más que tú del trabajo que has hecho, del cual debes sentirte orgulloso/a, cosa que debe infundirte seguridad, por tanto, disfruta de la presentación. 

    \item Intenta captar continuamente la atención de los miembros de la comisión empleando tanto un buen diseño de las diapositivas, tal y como se ha explicado en el capítulo \ref{cap:elaboraciónPresentación}, como técnicas verbales, como, por ejemplo, usando preguntas que lanzas a la comisión, pero que no esperas que respondan, cambios de volumen y tono, etc.

    \item Llévate un guión que será de ayuda si en algún momento te quedas en blanco o no te acuerdas de por dónde seguir, pero no lo tengas en la mano. Déjalo en la mesa a un lado.

    \item Si los miembros del tribunal no te están mostrando en algún momento atención, no te preocupes. No significa que no le interese tu presentación ni que hayan dejado de escucharte. Simplemente estarán mirando algo en la memoria o anotando alguna cuestión a hacerte. Tú no te pongas nervioso/a ni pienses que no les está gustando tu presentación y sigue con ella, mirándolos como si ellos te estuvieran mirando y escuchando también.

    \item Tanto en la presentación como en las respuestas sé claro/a y ve al grano. No hay mucho tiempo y si te vas por las ramas te vas a quedar sin tiempo para explicar cosas importantes. Y sé honesto/a y sincero/a. Si algo no lo has hecho, por ejemplo, y te preguntan, di que no lo has hecho y el porqué. No te inventes nada porque los miembros de la comisión se darán cuenta y te meterás en un lío. 

    \item Ve con tranquilidad al acto de defensa. En la presentación los nervios de durarán el primer minuto, luego disfruta del momento. En la ronda de preguntas, también debes estar tranquilo/a porque nadie sabe más que tú de tu proyecto, porque lo has hecho tú.

    \item Pon el móvil en modo avión y no lo sitúes a la vista. La única excepción a esto último es que lo quieras emplear como temporizador para ver el tiempo de exposición, aunque si la tienes bien ensayada, tampoco es necesario el móvil y ofreces una mejor imagen que si estás consultándolo continuamente para ver cuánto tiempo te queda.
\end{itemize}

En \cite{vallejo2009defensa} tienes algunos otros consejos que te podrán ser útiles en este acto de la defensa.

\section{La comunicación con la comisión}

La forma de interactuar con los miembros de la comisión es también muy importante en el acto de la defensa. En esta sección te hacemos varias recomendaciones sobre este asunto.

Si el presidente de la comisión no te ha presentado, entonces, tras darte la palabra y tu agradecerla, preséntate tú mismo e indica seguidamente que vas a presentar tu trabajo fin de grado, con el título correspondiente. Si ya te ha presentado, entonces da las gracias y di que comienzas con la presentación de tu TFG (y no vuelvas a presentarte). 

Cuando finalices tu presentación, puedes decir algo así como ``y con esto finalizo la presentación de este TFG y quedo a disposición de la comisión para resolver cualquier duda o comentario que deseen realizar'', y esperas a que tome la palabra el presidente para organizar la segunda fase del acto. En ese momento comenzará la ronda de preguntas. 

Lo habitual es dirigirse a los miembros de la comisión de usted y siempre con el máximo respeto. Invócalos mediante la denominación de profesor y su apellido (profesora García). Seguramente conoces a los docentes que estén en la comisión. Alguno te habrá dado clase en alguna asignatura del grado y tienes cierta confianza. En ese caso, puedes bajar algo el nivel de formalidad a la hora de dirigirte a ellos aunque nunca entrando en el colegueo. En ese caso puedes llamarlos por su nombre de pila. Si el presidente te habla de Usted, lo normal es que hables tú también en ese grado. Si lo hace de tú, decide si hablarles también de tú o de Usted. 

En relación a las preguntas y comentarios, como ya hemos indicado anteriormente, responde siempre yendo al grano y de forma clara. No des rodeos ni respondas otra cosa si no sabes qué decir o no quieres decir algo. Habla con sinceridad. Si no entiendes una pregunta di que no has comprendido qué quieren decirte y que, por favor, te la repitan. Si no estás de acuerdo con algún comentario, siempre con educación y de forma cortés, rebátelo y ofrece las razones por las que discrepas. No tengas miedo a hacerlo así. Eso es una discusión académica y es un punto a tu favor, ya que estás dejando claro a la comisión que tienes los conocimientos y la capacidad para defender lo que has hecho y el porqué.

Sé receptivo/a a los comentarios, críticas y sugerencias. No te los tomes a mal ni de forma personal. En algunas situaciones puedes responder que gracias por el comentario y que lo tendrás en cuenta para mejorar la aplicación, por ejemplo. 

%Y si en las preguntas aparece una donde la respuesta implica que ha sido la persona que te tutoriza la que ha tomado una decisión concreta, dilo también sin problema alguno. 

Si hay alguna respuesta que pueda encajar con ``porque me lo dijo o indicó mi tutor'', no tengas reparo en expresarlo de ese modo. No obstante, muestra tu opinión personal fundamentada al respecto, tanto si coincide, como si no, con lo que indicó la persona que te tutoriza. 

No tengas miedo de preguntar cualquier cuestión o pedirles ayuda en algún tema (por ejemplo, si al conectar el ordenador al proyector hay algún problema). Los docentes de la comisión te responderán con toda amabilidad y te ayudarán en la medida de sus posibilidades. También son conscientes de que se pasan algunos nervios en este tipo de actos y, por tanto, serán comprensivos y te intentarán tranquilizar.

Ten también presente que los miembros del tribunal no van a preguntarte con mala intención, simplemente están haciendo su trabajo y deben plantear ciertas cuestiones para entender detalles que no les han quedado claros o conocer tu capacidad para desenvolverte en estas situaciones. En definitiva, tienen que tener toda la información disponible para poder evaluar objetivamente tu trabajo.

\section{Cuestiones de protocolo}

La principal cuestión en este apartado es la vestimenta. ¿Qué me pongo? ¿Voy muy arreglado/a? ¿Voy como normalmente visto? La respuesta es sencilla pero a la vez complicada: irás vestido/a de la forma adecuada para ofrecer la imagen que quieres dar. Por simplificar, recomendamos evitar camisetas, tirantes, bermudas, pantalones cortos y ropa deportiva, ya que es un atuendo demasiado informal para una presentación pública de este calibre. ¿Irías como vistes normalmente a una entrevista de trabajo? No, ¿verdad? Pues esa es la idea en este acto. 

Uno de los aspectos que da más información sobre ti es tu forma de vestir, ya que refleja tu personalidad y esta también permite a los demás construir una primera impresión sobre ti en muy poco tiempo. Y lo que quieres es caer bien a los miembros de la comisión y que se lleven una primera impresión de que eres  profesional.

En general, simplemente debes ir vestido/a acorde al acto en el que vas a participar. Si fuera el de defensa de una tesis doctoral, claramente deberías ir de traje, pero esta presentación pública, aunque importante, podríamos decir que no tiene la categoría de la defensa de un trabajo doctoral. Por tanto, debes ir arreglado/a pero no es necesario súper arreglado. Un pantalón de vestir y una camisa podrían ser suficientes. Evalúa también una falda y una blusa o camisa, o un vestido como posibilidades. En definitiva algo con lo que te sientas cómodo/a y que sea uno o dos puntos más formal de la forma en que normalmente vistes. Procura no usar ropa llamativa o accesorios que puedan distraerte a ti o a los miembros del tribunal. Y ni qué decir tiene que debes acudir bien aseado/a.




%0) La importancia de la defensa -> Juanma y Alberto
%1) Estructura del acto -> Juanma	
%   mencionar que es un acto público (jugador 12? o motivo de nervios extra)

% https://www.researchgate.net/publication/301587876_Writing_and_Presenting_a_Dissertation_on_Linguistics_Applied_Linguistics_and_Culture_Studies_for_Undergraduates_and_Graduates_in_Spain

%2) Comunicación con el tribunal -> Juanma
%   “intro, si no la hay la haces tú”
%   Turno de preguntas -> cómo responder a las preguntas
% 3) Comunicación oral  -> Alberto
%   lectura de hojas vs memorización, etc.
% 4) Comunicación no verbal -> Alberto
% 5) Duración -> respetar los límites -> ALberto
% 6) Demos y vídeos -> postura común de demo + vídeo -> Alberto	
%   comentar brevemente ventajas/desventajas de demo + vídeo vs vídeo + demo
%  7) Cuestiones de protocolo -> Juanma
%   vestimenta adecuada, aseo, [qué imagen proyectar de nosotros]

%https://ucm.es/data/cont/media/www/pag-135806/Defensa%20del%20Trabajo%20Acade%CC%81mico.pdf
